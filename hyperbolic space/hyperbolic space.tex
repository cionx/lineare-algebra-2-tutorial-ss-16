%\documentclass[a4paper,10pt]{article}
\documentclass[a4paper,10pt,numbers=noenddot]{scrartcl}

\usepackage{../generalstyle}
\usepackage{specificstyle}

\title{Lösungen zu Zettel 11}
\author{Jendrik Stelzner}
\date{\today}

\begin{document}
\maketitle


Wir erinnern zunächst an das folgende Lemma, das im Tutorium schon einmal gezeigt wurde:


\begin{lemma}\label{lem: existence of spacelike vectors}
  Es seien $u, v \in V$ zwei linear unabhängige Vektoren, die jeweils lichtartig oder zeitartig sind (d.h.\ es ist $\beta(u,u) \leq 0$ und $\beta(v,v) \leq 0$).
  Dann enthält die Ebene $\Ell(u,v)$ einen raumartigen Vektor.
\end{lemma}
\begin{proof}
  Es sei $\mc{B} = (e_1, e_2, e_3)$ eine Sylvesterbasis von $V$ mit
  \[
    \Mat_\mc{B}(\beta)
    =
    \begin{pmatrix}
      1 &   &     \\
        & 1 &     \\
        &   & -1
    \end{pmatrix},
  \]
  und es seien $u = u_1 e_1 + u_2 e_2 + u_3 e_3$ und $v = v_1 e_1 + v_2 e_2 + v_3 e_3$ mit $u_1, u_2, u_3, v_1, v_2, v_3 \in \Rbb$.
  Da $v \neq 0$ gibt es ein $i \in \{1, 2, 3\}$ mit $v_i \neq 0$; da
  \[
      v_1^2 + v_2^2 - v_3^2
    = \beta(v,v)
    \leq 0
  \]
  muss deswegen $v_3 \neq 0$.
  Es sei 
  \[
              w
    \coloneqq u - \frac{u_3}{v_3} v
    \in       \Ell(u, v).
  \]
  Es ist $w = w_1 e_1 + w_2 e_2 + w_3 e_3$ mit $w_i = u_i - (u_3/v_3) \cdot v_i$ für alle $i = 1, 2, 3$; insbesondere ist $w_3 = 0$.
  Da $u$ und $v$ linear unabhängig sind, ist aber auch $w \neq 0$, und somit $w_1 \neq 0$ oder $w_2 \neq 0$.
  Damit haben wir
  \[
      \beta(w,w)
    = w_1^2 + w_2^2 - w_3^2
    = w_1^2 + w_2^2
    > 0,
  \]
  weshalb $w$ raumartig ist.
\end{proof}










\addtocounter{section}{2}










\section{}





\subsection{}


Wir bemerken zunächst, dass die Bedingung, dass $\beta|_{X \times X}$ nichtentartet ist, unnötig ist:


\begin{lemma}
  Es sei $X \subseteq V$ ein zweidimensionaler Untervektorraum, so dass $\beta|_{X \times X}$ vom Typ $(1,1)$ ist.
  Dann ist $\beta|_{X \times X}$ nichtentartet.
\end{lemma}
\begin{proof}
  Da $\beta|_{X \times X}$ vom Typ $(1,1)$ ist, und $1 + 1 = 2 = \dim X$, gibt es eine Sylvesterbasis $\mc{B} = (b_1, b_2)$ von $X$, so dass
  \[
    \Mat_\mc{B}(\beta|_{X \times X})
    =
    \begin{pmatrix*}[r]
      1 &  0  \\
      0 & -1
    \end{pmatrix*}.
  \]
  Da diese Matrix keine Nullen auf der Diagonalen hat, ist $\beta|_{X \times X}$ nichtentartet.
\end{proof}


Wir zeigen im Folgenden, dass für einen zweidimensionalen Untervektorraum $X \subseteq V$ genau dann $\hyperbolic \cap X \neq \emptyset$, wenn $\beta|_{X \times X}$ den Typ $(1,1)$ hat.

Wenn $\beta|_{X \times X}$ den Typ $(1,1)$ hat, dann gibt es eine Sylvesterbasis $(b_1, b_2)$ von $X$ mit $\beta(b_1, b_1) = 1$ und $\beta(b_2, b_2) = -1$ (sowie $\beta(b_1, b_2) = 0$).
Insbesondere gilt $b_2 \in \hyperbolic_{\pm}$ und somit entweder $b_2 \in \hyperbolic$ oder $-b_2 \in \hyperbolic$, also $b_2 \in \hyperbolic \cap X$ oder $-b_2 \in \hyperbolic \cap X$.
Auf jeden Fall gilt $\hyperbolic \cap X \neq \emptyset$.

Angenommen, es ist $\hyperbolic \cap X \neq \emptyset$.
Dann gibt es ein $b \in \hyperbolic \cap X$, und da $b \in \hyperbolic_{\pm 1}$ ist $\beta(b,b) = -1$.
Für den Typen von $\beta|_{X \times X}$ gibt es wegen der Zweidimensionalität von $X$ a priori sechs Möglichkeiten: $(0,0)$, $(1,0)$, $(0,1)$, $(2,0)$, $(1,1)$ oder $(0,2)$.

Die Fälle $(0,0)$, $(1,0)$ und $(2,0)$ können wir ausschließen, denn sonst wäre $\beta|_{X \times X}$ positiv semidefinit, was $\beta(b,b) = -1$ widerspricht.

In den Fällen $(0,1)$ und $(0,2)$ wäre $\beta|_{X \times X}$ negativ semidefinit; eine beliebige Basis $\mc{B}$ von $X$ würde dann aus licht- oder zeitartigen Vektoren bestehen, weshalb $X$ nach Lemma~\ref{lem: existence of spacelike vectors} einen raumartigen Vektor enthälten müsste.
Dies stünde dann aber im Widerspruch zur negativen Semidefinitheit von $\beta|_{X \times X}$.

Es bleibt also nur noch die Möglichkeit $(1,1)$ übrig.





\subsection{}

Die entscheidende Beobachtung ist, dass $x$ und $y$ linear unabhängig sind:


\begin{lemma}\label{lem: elements of hyperbolic space are linear independent}
  Je zwei verschiedene Elemente $u, v \in \hyperbolic$ sind linear unabhängig.
\end{lemma}
\begin{proof}
Nach Aufgabe 2 ist $\beta(u,v) < -1$, weshalb
\[
    \beta(u,v)^2
  > (-1)^2
  = (-1)(-1)
  = \beta(u,u)\beta(v,v).
\]
Da $u$ und $v$ beide zeitartig sind, folgt damit die lineare Unabhängigkeit aus Aufgabe 1.
\end{proof}

Ist $\hyperbolicline \subseteq \hyperbolic$ eine Gerade, die $x$ und $y$ enthält, so gibt es einen zweidimensionalen Untervektorraum $X \subseteq V$ mit $\hyperbolicline = \hyperbolic \cap X$.
Da die zweidimensionale Ebene $X$ die beiden linear unabhängigen Vektoren $x$ und $y$ enthält, muss bereits $X = \Ell(x,y)$.
Somit ist $\hyperbolicline = \hyperbolic \cap \Ell(x,y)$ die eindeutige Gerade in $\hyperbolic$, die $x$ und $y$ enthält.










\section{}





\subsection{}


Wegen der Bilinearität von $\beta$ ist $\beta(x, -) \colon V \to \Rbb$, $v \mapsto \beta(x,v)$ eine lineare Abbildung.
Es gilt $\beta(x, -) \neq 0$, da $\beta(x,-)(x) = \beta(x,x) = -1$.
Folglich ist $\im \beta(x, -)$ ein Untervektorraum von $\Rbb$, der nicht der Nullvektorraum ist; es muss $\im \beta(x, -) = \Rbb$ gelten.
Damit ergibt sich, dass $T_x = \ker \beta(x, -)$ ein Untervektorraum von $V$ ist, für den nach der Dimensionsformel
\[
    \dim T_x
  = \dim \ker \beta(x,-)
  = \dim V - \dim \im \beta(x, -)
  = 3 - 1
  = 2.
\]

Um zu zeigen, dass $\beta|_{T_x \times T_x}$ ein Skalarprodukt ist, zeigen wir für $v \in T_x$ mit $v \neq 0$, dass $\beta(v,v) > 0$.
Hierfür nehmen wir an, dass $\beta(v, v) \leq 0$, dass $v$ also licht- oder zeitartig ist.

Wir bemerken, dass $x$ und $v$ linear unabhängig sind; ansonsten wäre nämlich $v = \lambda x$ für ein $\lambda \in \Rbb$ mit $\lambda \neq 0$, weshalb $\beta(v,x) = \beta(\lambda x, x) = \lambda \beta(x,x) = - \lambda \neq 0$ wäre.
(Hier haben wir die Bedingung $v \neq 0$ gebraucht.)

Nach Aufgabe 1 ist nun
\[
     0
  = \beta(x,v)^2
  > \beta(x,x) \beta(v,v)
  = -\beta(v,v).
\]
Dabei haben wir in der ersten Gleichung genutzt, dass $v \in T_x$ und somit $\beta(x,v) = 0$, und für die Striktheit der Ungleichung nutzen wir, dass $x$ und $v$ linear unabhängig sind.
Nach der obigen Gleichskette ist nun $\beta(v,v) > 0$, im Widerspruch zur Annahme $\beta(v,v) \leq 0$.
Also muss bereits $\beta(v,v) > 0$ gelten.


\begin{remark}
  Eine weiter Möglichkeit besteht darin, die Gerade $U \coloneqq \Ell(x)$ zu betrachten.
  Dies ist ein $1$-dimensionaler Untervektorraum von $V$, und da $\beta(x,x) = -1$ ist die Einschränkung $\beta|_{U \times U}$ negativ definit und insbesondere nicht-entartet.
  Da $\beta|_{U \times U}$ nicht-entartet ist (!) gilt $V = U \oplus U^\perp$, wobei $U^\perp = x^\perp = T_x$.
  
  Die Einschränkung $\beta|_{U \times U}$ ist vom Typ $(0,1)$.
  Ist $\beta|_{T_x \times T_x}$ vom Typ $(n,m)$, so folgt aus der Orthogonalität der Summe $V = U \oplus T_x$, dass $\beta$ vom Typ $(0,1) + (n,m) = (n,m+1)$ ist.
  Da $\beta$ von Typ $(2,1)$ ist, ergibt sich hieraus, dass $(n,m) = (2,1) - (0,1) = (2,0)$.
  Also ist $\beta|_{T_x \times T_x}$ vom Typ $(2,0)$.
\end{remark}











\section{}

Man bemerke zunächst, dass $A$, $B$ und $C$ linear unabhängig sind.
Ansonsten gebe es nämlich einen zweidimensionalen Untervektorraum $X \subseteq V$ mit $A, B, C \in X$, weshalb $X \cap \hyperbolic$ eine Gerade in $\hyperbolic$ wäre, die alle drei Punkte enthält.


Wir zeigen zunächst, dass das angegebene Element
\[
  \tangent
  \coloneqq
  \frac{B - \cosh(c) A}{\sinh(c)}
\]
die gewünschten Eigenschaften hat.
Das zeigt inbesondere die Existenz des gewünschten Tangentialvektors.
Anschließend zeigen wir für einen beliebigen entsprechenden Tangentialvektor $\tangent_{AB}$, dass $\tangent_{AB} = \tangent$.

Es ist klar, dass $\tangent \in \Ell(A,B)$, und dass $1/\sinh(c) > 0$.
Man bemerke, dass
\[
  \cosh(c) = \cosh(\arccosh(-\beta(A,B))) = -\beta(A,B).
\]
Deshalb ist
\begin{align*}
      \beta(\tangent, A)
   =  \beta\left( \frac{B - \cosh(c) A}{\sinh(c)}, A \right)
  &=  \frac{\beta(B, A) - \cosh(c) \beta(A,A)}{\sinh(c)}      \\
  &=  \frac{\beta(A,B) - \beta(A,B)}{\sinh(c)}
   =  0,
\end{align*}
also $\tangent \in T_A$.
Da $\sinh(c)^2 = \cosh(c)^2 - 1$ ist außerdem
\begin{align*}
      \beta(\tangent, \tangent)
  &=  \beta\left( \frac{B - \cosh(c) A}{\sinh(c)}, \frac{B - \cosh(c) A}{\sinh(c)} \right)    \\
  &=  \frac{ \beta(B, B) - 2 \cosh(c) \beta(A,B) + \cosh(c)^2 \beta(A,A) }{ \cosh(c)^2 - 1 }  \\
  &=  \frac{ -1 - 2 (- \beta(A, B)) \beta(A,B) + \beta(A,B)^2 (-1) }{ \beta(A,B)^2 - 1 }
   =  \frac{ \beta(A, B)^2 - 1 }{ \beta(A, B)^2 - 1 }
   =  1.
\end{align*}
Also ist $\tangent$ normiert.
Somit erfüllt $\tangent$ alle geforderten Eigenschaften.

Es sei nun $\tangent_{AB}$ ein weiterer Vektor, der alle angegebenen Eigenschaften erfüllt.
Da $A$ und $B$ linear unabhängig sind, ist der Untervektorraum $\Ell(A,B) \subseteq V$ zweidimensional.

Der Untervektorraum $U \coloneqq \{t \in \Ell(A,B) \mid \beta(t,A) = 0\} = \Ell(A,B) \cap T_A$ ist eindimensional: Da $\tangent \in U$ ist $U \neq 0$ und somit $\dim U \geq 1$.
Da $A \notin U$ ist $U \subsetneq \Ell(A,B)$ und somit $\dim U < \dim \Ell(A,B) = 2$.

Da $\tangent$ und $\tangent_{AB}$ zwei bezüglich $\beta|_{U \times U}$ normierte Vektoren des eindimensionalen Vektorraums $U$ sind, muss $\tangent_{AB} = \pm \tangent$.
Wäre $\tangent_{AB} = -\tangent$, so wäre $\tangent_{AB} = (-B + \cosh(c) A)/\sinh(c)$.
Wegen der linearen Unabhängigkeit von $A$ und $B$ wäre diese Linearkombination von $A$ und $B$ eindeutig; der Koeffizient $-1/\sinh(c)$ von $B$ ist aber negativ, im Widerspruch zur Definition von $\tangent_{AB}$.
Also muss bereits $\tangent_{AB} = \tangent$ gelten.













\end{document}