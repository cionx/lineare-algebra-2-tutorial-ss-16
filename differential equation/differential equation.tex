%\documentclass[a4paper, 10pt]{article}
\documentclass[a4paper, 10pt]{scrartcl}

\usepackage{../generalstyle}

\setromanfont[Mapping=tex-text]{Linux Libertine O}
% \setsansfont[Mapping=tex-text]{DejaVu Sans}
% \setmonofont[Mapping=tex-text]{DejaVu Sans Mono}

\title{Lösung zu Zettel 5, Aufgabe 2}
\author{Jendrik Stelzner}
\date{\today}

\begin{document}
\maketitle


Dass $f,g,h \in C^\infty(\Rbb, \Rbb)$ Lösungen der angegebenen Differenzialgleichung sind, ist äquivalent dazu, dass
\[
    \vect{f \\ g \\ h}'
  = \vect{f' \\ g' \\ h'}
  =
  \underbrace{
  \begin{pmatrix}
    4 &  1  &  -2 \\
    3 & 14  & -18 \\
    2 &  6  &  -7
  \end{pmatrix}
  }_{\eqqcolon A}
  \vect{f \\ g \\ h}.
\]
Die Lösungen der Gleichung $y' = A y$ mit $y \in C^\infty(\Rbb, \Rbb^3)$ sind durch
\[
  y(t) = \exp(At) C
  \quad
  \text{für alle $t \in \Rbb$}
\]
gegeben, wobei $C \in \Rbb^3$ nicht von $t$ abhängt.
Der Lösungsraum wird also von den Spalten der Matrix $\exp(At)$ aufgespannt.
(Da $\exp(At)$ für alle $t \in \Rbb$ invertierbar ist, sind diese Spalten zu jedem Zeitpunkt $t \in \Rbb$ linear unabhängig, weshalb auch die entsprechenden Funktionen linear unabhängig sind.
Also sind die Spalten von $\exp(At)$ bereits eine Basis des Lösungsraums, und für eine fest vorgegebene Lösung $y$ der entsprechende Konstantenvektor $C \in \Rbb^3$ eindeutig.)
Zum Lösen der Differentialgleichung müssen wir also $\exp(At)$ berechen.

Hierfür und bestimmen wir die Jordan-Normalform von $A$, inklusive entsprechender Basiswechselmatrizen:
Für das charakteristische Polynom ergibt sich
\[
    \chi_A(t)
  = t^3 - 11t + 39t - 45
  = (t-3)^2 (t-5).
\]
Da alle Eigenwerte reell sind hat $A$ bereits über $\Rbb$ eine Jordan-Normalform.
Für den Eigenraum $E(\lambda)$ von $A$ zum Eigenwert $\lambda$ ergibt sich, dass
\[
  E(3) = \Rbb \vect{1 \\ 3 \\ 2}
  \quad\text{und}\quad
  E(5) = \Rbb \vect{0 \\ 2 \\ 1}.
\]
Inbesondere ist $A$ nicht diagonalisierbar, da der Eigenraum zum Eigenwert $3$ nicht zweidimensional ist.
Für den entsprechenden Hauptraum $H(3)$ erhalten wir
\[
    H(3)
  = \Rbb \vect{1 \\ 3 \\ 2} + \Rbb \vect{1 \\ 0 \\ 0},
\]
wobei die Basisvektoren bereits so gewählt sind, dass
\[
    A \vect{1 \\ 0 \\ 0}
  = \vect{4 \\ 3 \\ 2}
  = \vect{1 \\ 3 \\ 2}+ 3 \vect{1 \\ 0 \\ 0}.
\]
Es handelt es sich also bereits um eine Jordanbasis des Hauptraums.
Für die Matrix
\[
  S
  \coloneqq
  \begin{pmatrix}
    1 & 1 & 0 \\
    3 & 0 & 2 \\
    2 & 0 & 1
  \end{pmatrix}
\]
ergibt sich damit, dass
\[
  S^{-1} A S
  =
  \begin{pmatrix}
    3 & 1 & 0 \\
    0 & 3 & 0 \\
    0 & 0 & 5
  \end{pmatrix}
  \eqqcolon J,
\]
wobei
\[
  S^{-1}
  =
  \begin{pmatrix*}[r]
    0 & -1  &  2  \\
    1 &  1  & -2  \\
    0 &  2  & -3
  \end{pmatrix*}.
\]
Für alle $t \in \Rbb$ ist deshalb
\[
  At = S (Jt) S^{-1}.
\]
Da nun
\[
    \exp(At)
  = \exp(S Jt S^{-1})
  = S \exp(Jt) S^{-1}
\]
genügt es, $\exp(Jt)$ zu berechnen.
Hierfür nutzen wir, dass für für $J$ die Jordanzerlegung
\[
  J
  =
  \begin{pmatrix}
    3 & 1 & 0 \\
    0 & 3 & 0 \\
    0 & 0 & 5
  \end{pmatrix}
  =
  \underbrace{
  \begin{pmatrix}
    3 & 0 & 0 \\
    0 & 3 & 0 \\
    0 & 0 & 5
  \end{pmatrix}
  }_{\eqqcolon D}
  +
  \underbrace{
  \begin{pmatrix}
    0 & 1 & 0 \\
    0 & 0 & 0 \\
    0 & 0 & 0
  \end{pmatrix}
  }_{\eqqcolon N}
\]
in eine Diagonalmatrix $D$ und eine nilpotente Matrix $N$ haben, wobei $D$ und $N$ kommutieren.
Für alle $t \in \Rbb$ haben wir damit eine Zerlegung
\[
  Jt
  = (D + N)t
  = Dt + Nt,
\]
in eine Diagonalmatrix $Dt$ und nilpotente Matrix $Nt$, die miteinander kommutieren.
Deshalb ist
\[
    \exp(Jt)
  = \exp(Dt + Nt)
  = \exp(Dt) \exp(Nt),
\]
wobei wir beide Faktoren leicht bestimmen können.
Zum einen ist
\[
  \exp(Dt)
  =
  \exp
  \begin{pmatrix}
    3t  & 0   & 0 \\
    0   & 3t  & 0 \\
    0   & 0   & 5t
  \end{pmatrix}
  =
  \begin{pmatrix}
    e^{3t}  & 0       & 0       \\
    0       & e^{3t}  & 0       \\
    0       & 0       & e^{5t}
  \end{pmatrix}.
\]
Zum anderen haben wir wegen $N^2 = 0$ auch $(Nt)^2 = 0$ für alle $t \in \Rbb$, und somit
\[
  \exp(Nt)
  =
  I + Nt
  =
  \begin{pmatrix}
    1 & t & 0 \\
    0 & 1 & 0 \\
    0 & 0 & 1
  \end{pmatrix}.
\]
Also ist
\begin{align*}
      \exp(Jt)
   =  \exp(Dt) \exp(Nt)
  &=
  \begin{pmatrix}
    e^{3t}  & 0       & 0       \\
    0       & e^{3t}  & 0       \\
    0       & 0       & e^{5t}
  \end{pmatrix}
  \begin{pmatrix}
    1 & t & 0 \\
    0 & 1 & 0 \\
    0 & 0 & 1
  \end{pmatrix} \\
  &=
  \begin{pmatrix}
    e^{3t}  & e^{3t} t  & 0       \\
    0       & e^{3t}    & 0       \\
    0       & 0         & e^{5t}
  \end{pmatrix}
\end{align*}
und damit
\[
    \exp(At)
  = S \exp(Jt) S^{-1}
  =
  e^{3t}
  \begin{pmatrix}
    1 + t &              t      &           - 2 t     \\
       3t & 4 e^{2t} + 3 t - 3  &    e^{2t} +   t - 1 \\
       2t &   e^{2t} +   t - 1  & -3 e^{2t} - 4 t + 4
  \end{pmatrix}.
\]

Der Lösungsraum der gegebenen Differentialgleichung hat also eine Basis $(b_1, b_2, b_3)$ mit den drei Funktionen $b_1, b_2, b_3 \colon \Rbb \to \Rbb^3$, die durch
\begin{gather*}
  b_1(t) = e^{3t} \vect{1+t \\ 3t \\ 2t},
  \quad
  b_2(t) = e^{3t} \vect{t \\ 4 e^{2t} + 3 t - 3 \\ e^{2t} + t - 1},     \\
  b_3(t) = e^{3t} \vect{-2t \\ e^{2t} + t - 1 \\ -3 e^{2t} - 4 t + 4}.
\end{gather*}
gegeben sind.

















\end{document}
