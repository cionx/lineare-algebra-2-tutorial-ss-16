\documentclass[a4paper,10pt]{article}
%\documentclass[a4paper,10pt]{scrartcl}

\usepackage{../generalstyle}

\title{Simultane Diagonalisierbarkeit}
\author{Jendrik Stelzner}
\date{\today}

\begin{document}
\maketitle










\section{Endlich viele Endomorphismen}


Zunächst wollen wir Notation einführen.


\begin{definition}
  Es sei $V$ ein $K$-Vektorraum.
  \begin{enumerate}
    \item
      Für jeden Endomorphismus $f \colon V \to V$ und Skalar $\lambda \in K$ sei
      \[
        V(f,\lambda) \coloneqq \{ v \in V \mid f(v) = v \}
      \]
      der \emph{Eigenraum} von $f$ zum Eigenwert $\lambda$.
      Ein Element $v \in V(f, \lambda)$ mit $v \neq 0$ heißt \emph{Eigenvektor} von $f$ zum Eigenwert $\lambda$.
    \item
      Für Endomorphismen $f_1, \dotsc, f_n \colon V \to V$ und paarweise verschiedene Skalare $\lambda_1, \dotsc, \lambda_n \in K$ sei
      \[
                  V(f_1, \lambda_1; \dotsc; f_n, \lambda_n)
        \coloneqq \{ v \in V \mid \text{$f_i(v) = \lambda_i v$ für alle $i = 1, \dotsc, n$} \}
      \]
      der \emph{gemeinsame Eigenraum} der Endomorphismen $f_1, \dotsc, f_n$ zu den (paarweise verschiedenen) Eigenwerten $\lambda_1, \dotsc, \lambda_n$.
      Ein Element $v \in V(f_1, \lambda_1; \dotsc; f_n, \lambda_n)$ mit $v \neq 0$ heißt \emph{gemeinsamer Eigenvektor} der Endomorphismen $f_1, \dotsc, f_n$ zu den Eigenwerten $\lambda_1, \dotsc, \lambda_n$.
  \end{enumerate}
\end{definition}


\begin{proposition}\label{prop: basis of common eigenvectors equivalent to sum of common eigenspaces}
  Es sei $V$ ein endlichdimensionaler $K$-Vektorraum und $f_1, \dotsc, f_n \colon V \to V$ Endomorphismen.
  Dann sind die folgenden beiden Aussagen äquivalent:
  \begin{enumerate}
    \item
      Es gibt eine Basis $\mc{B}$ von $V$, so dass jeder Endomorphismus $f_i$ bezüglich $\mc{B}$ durch eine Diagonalmatrix dargestellt wird.
    \item
      Es gibt eine Basis $\mc{B}$ von $V$ aus gemeinsamen Eigenvektoren der Endomorphismen $f_1, \dotsc, f_n$.
    \item
      Es ist
      \[
        V = \bigoplus_{\lambda_1, \dotsc, \lambda_n \in K} V(f_1, \lambda_1; \dotsc; f_n, \lambda_n).
      \]
  \end{enumerate}
\end{proposition}


\begin{remark}
  Die Endlichdimensionalität in Proposition~\ref{prop: basis of common eigenvectors equivalent to sum of common eigenspaces} wird zur Existenz von Basen benötigt.
  Nutzt man, dass nach dem Auswahlaxiom jeder Vektorraum eine Basis besitzt, so gilt die Aussage auch für unendlichdimensionale Vektorräume.
\end{remark}


\begin{lemma}
  Es sei $V$ ein $K$-Vektorraum.
  \begin{enumerate}
    \item
     Ist $(f_i)_{i \in I}$ eine Familie von Endomorphismen $f_i \colon V \to V$ und $g \colon V \to V$ ein Endomorphismus, der mit jedem der $f_i$ kommutiert (also $f_i g = g f_i$ für alle $i \in I$ erfüllt), so ist für jede Familie $(\lambda_i)_{i \in I}$ von Skalaren $\lambda_i \in K$ der gemeinsame Eigenraum $V((f_i)_{i \in I}, (\lambda_i)_{i \in I})$ invariant unter $g$.
  \end{enumerate}
\end{lemma}


\begin{definition}
  Es sei $V$ ein $K$-Vektorraum.
  \begin{enumerate}[leftmargin=*]
    \item
      Ein Endomorphismus $f \colon V \to V$ heißt \emph{diagonalisierbar} falls $V = \bigoplus_{\lambda \in K} V(f, \lambda)$.
    \item
      Mehrere Endomorphismen $f_1, \dotsc, f_n \colon V \to V$ heißen \emph{simultan diagonalisierbar}, falls $V = \bigoplus_{\lambda_1, \dotsc, \lambda_n} V(f_1, \lambda_1; \dotsc; f_n, \lambda_n)$.
  \end{enumerate}
\end{definition}


\begin{theorem}
  Es sei $V$ ein $K$-Vektorraum und es seien $f_1, \dotsc, f_n \colon V \to V$ Endomorphismen.
  Dann sind die folgenden beiden Aussagen äquivalent:
  \begin{enumerate}
    \item
      Die Endomorphismen $f_1, \dotsc, f_n$ sind simultan diagonalisierbar.
    \item
      Jeder Endomorphismus $f_i$ ist (einzeln) diagonalisierbar, und die Endomorphismen $f_1, \dotsc, f_n$ kommutieren paarweise miteinander (d.h.\ es ist $f_i f_j = f_j f_i$ für alle $1 \leq i,j \leq n$).
  \end{enumerate}
\end{theorem}


\begin{corollary}
  Für einen endlichdimensionalen $K$-Vektorraum $V$ und Endomorphismen $f_1, \dotsc, f_n \colon V \to V$ sind die folgenden Bedingungen äquivalent:
  \begin{enumerate}
    \item
      Es gibt eine Basis $\mc{B}$ von $V$, so dass jeder Endomorphismus $f_i$ bezüglich $\mc{B}$ durch eine Diagonalmatrix dargestellt wird.
    \item
      Es gibt eine Basis $\mc{B}$ aus gemeinsamen Eigenvektoren von $f_1, \dotsc, f_n$.
    \item
      Es ist $V = \bigoplus_{\lambda_1, \dotsc, \lambda_n \in K} V(f_1, \lambda_1; \dotsc; f_n, \lambda_n)$.
    \item
      Jeder der Endomorphismen $f_i$ ist (einzeln) diagonalisierbar, und die Endomorphismen $f_1, \dotsc, f_n$ kommutieren paarweise miteinander.
  \end{enumerate}
\end{corollary}










\section{Unendlich viele Endomorphismen}


\begin{definition}
  Es sei $V$ ein $K$-Vektorraum und $(f_i)_{i \in I}$ eine Familie von Endomorphismen $f_i \colon V \to V$.
  Für eine Familie $(\lambda_i)_{i \in I}$ von Skalaren $\lambda_i \in K$ ist
  \[
      V((f_i)_{i \in I}, (\lambda_i)_{i \in I})
    = \{ v \in V \mid \text{$f_i(v) = \lambda_i v$ für alle $i \in I$} \}
  \]
  der \emph{gemeinsame Eigenraum} der Endomorphismen $(f_i)_{i \in I}$ zu den Eigenwerten $(\lambda_i)_{i \in I}$.
  Ein Element $v \in V((f_i)_{i \in I}, (\lambda_i)_{i \in I})$ mit $v \neq 0$ heißt \emph{gemeinsamer Eigenvektor} der Endomorphismen $(f_i)_{i \in I}$ zu den Eigenwerten $(\lambda_i)_{i \in I}$.
\end{definition}


\begin{remark}
  Für endlich viele Endomorphismen $f_1, \dotsc, f_n \colon V \to V$ und Skalare $\lambda_1, \dotsc, \lambda_n \in K$ ist
  \[
      V((f_1, \dotsc, f_n), (\lambda_1, \dotsc, \lambda_n))
    = V(f_1, \lambda_1; \dotsc; f_n, \lambda_n).
  \]
\end{remark}




\begin{definition}
  Es sei $V$ ein $K$-Vektorraum und $H \subseteq \End_K(V)$ ein Untervektorraum.
  Für jedes $\lambda \in H^*$ sei
  \[
              V_\lambda
    \coloneqq \{v \in V \mid \text{$f(v) = \lambda(f) v$ für alle $f \in H$}\}
  \]
  der \emph{weight space} von $V$ bezüglich $\lambda$.
\end{definition}





\end{document}
