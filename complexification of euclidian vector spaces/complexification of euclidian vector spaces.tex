\documentclass[a4paper,10pt]{article}
%\documentclass[a4paper,10pt]{scrartcl}

\usepackage{../generalstyle}

\title{Komplexifizierung euklidischer Vektorräume}
\author{Jendrik Stelzner}
\date{\today}

\begin{document}
\maketitle


\begin{abstract}
  Wir geben einen kurzen Überblick über die Komplexifizierung euklidischer Vektorräume, und zeigen einige Kompatiblitäten von Orthogonalitäten und Adjungierten mit dieser Komplexifizierung an.
  Anschließend folgern wir den Fall $\Kbb = \Rbb$ von für Aufgabe 5 von Zettel 6 aus dem Fall $\Kbb = \Cbb$.
\end{abstract}


\section{Vorbereitung}


Bevor wir uns dem eigentlichen Thema zuwenden, zeigen wir noch einige nützliche Aussagen über die Komplexifizierung reeller Vektorräume.
Im Folgenden seien $U$, $V$ und $W$ drei $\Rbb$-Vektorräume.


\begin{proposition}\label{prop: functoriality of the complexification}
  \begin{enumerate}[leftmargin=*]
    \item
      Sind $f \colon U \to V$ und $g \colon V \to W$ zwei $\Rbb$-lineare Abbildungen, so ist
      \[
        (g \circ f)_\Cbb = g_\Cbb \circ f_\Cbb.
      \]
    \item
      Es gilt $(\id_V)_\Cbb = \id_{V_\Cbb}$.
  \end{enumerate}
\end{proposition}


\begin{proof}
  \begin{enumerate}[leftmargin=*]
    \item
      Per Definition von $f_\Cbb$ und $g_\Cbb$ kommutiert das folgende Diagram:
      \[
      \begin{tikzcd}
          U       \arrow[r, "f"]  \arrow[d, "\iota_U"]
        & V       \arrow[r, "g"]  \arrow[d, "\iota_V"]
        & W                       \arrow[d, "\iota_W"]
        \\
          U_\Cbb  \arrow[r, "f_\Cbb"]
        & V_\Cbb  \arrow[r, "g_\Cbb"]
        & W_\Cbb
      \end{tikzcd}
      \]
      Dabei bezeichnen die vertikalen Pfeile die jeweiligen kanonische Inklusionen.
      Indem wir den mittleren Teils des Diagrams weglassen, erhalten wir das folgende kommutative Diagram:
      \[
      \begin{tikzcd}[column sep=large]
          U       \arrow[r, "g \circ f"]            \arrow[d, "\iota_U"]
        & W                                         \arrow[d, "\iota_W"]
        \\
          U_\Cbb  \arrow[r, "g_\Cbb \circ f_\Cbb"]
        & W_\Cbb
      \end{tikzcd}
      \]
      Also erfüllt $f_\Cbb \circ g_\Cbb$ die definierene Eigenschaft von $(f \circ g)_\Cbb$.
    \item
      Die Abbildung $\id_{V_\Cbb}$ erfüllt die definierende Eigenschaft von $(\id_V)_\Cbb$, da das folgende Diagram kommutiert:
      \[
      \begin{tikzcd}
          V       \arrow[r, "\id_V"]        \arrow[d, "\iota_V"]
        & V                                 \arrow[d, "\iota_V"]
        \\
          V_\Cbb  \arrow[r, "\id_{V_\Cbb}"]
        & V_\Cbb
      \end{tikzcd}
      \]
  \end{enumerate}
\end{proof}


\begin{remark}
  Wegen den Eigenschaften aus Proposition~\ref{prop: functoriality of the complexification} bezeichnet man die Komplexifizierung als \emph{(kovariant) funktoriell}.
  Man vergleiche dies etwa mit Aufgabe 1 vom 12.\ Zettel aus Lineare Algebra I.
\end{remark}



\begin{lemma}\label{lemma: commuting in complexification}
  Sind $f, g \colon V \to W$ zwei $\Rbb$-lineare Abbildungen, so ist genau dann $f = g$, wenn $f_\Cbb = g_\Cbb$.
\end{lemma}


\begin{proof}
  Ist $f = g$, so ist auch $f_\Cbb = g_\Cbb$.
  Ist andererseits $f_\Cbb = g_\Cbb$, so ist
  \[
    f(v) + i \cdot 0
    = f_\Cbb(v + i \cdot 0)
    = g_\Cbb(v + i \cdot 0)
    = g(v) + i \cdot 0
    \quad
    \text{für alle $v \in V$},
  \]
  und somit $f(v) = g(v)$ für alle $v \in V$.
\end{proof}


\begin{corollary}
  Sind $f, g \colon V \to V$ zwei Endomorphismen, so kommutieren $f$ und $g$ genau dann, wenn $f_\Cbb$ und $g_\Cbb$ kommutieren.
\end{corollary}


\begin{proof}
  Da $(f \circ g)_\Cbb = f_\Cbb \circ g_\Cbb$ und $(g \circ f)_\Cbb = g_\Cbb \circ f_\Cbb$ ist nach Lemma~\ref{lemma: commuting in complexification} genau dann $f \circ g = g \circ f$, wenn $f_\Cbb \circ g_\Cbb = g_\Cbb \circ f_\Cbb$.
\end{proof}


\begin{lemma}\label{lem: invariance of induced subspaces}
  Es sei $f \colon V \to V$ ein Endomorphismus und $U \subseteq V$ ein Untervektorraum.
  Dann ist $U$ genau dann invariant unter $f$, wenn $U_\Cbb$ invariant unter $f_\Cbb$ ist.
\end{lemma}


\begin{proof}
  Es gilt
  \begin{align*}
        &\, \text{$U_\Cbb$ ist $f_\Cbb$-invariant}                            \\
    \iff&\, f_\Cbb(U_\Cbb) \subseteq U_\Cbb                                   \\
    \iff&\, \text{$f_\Cbb(u_1 + i u_2) \in U_\Cbb$ für alle $u_1, u_2 \in U$} \\
    \iff&\, \text{$f(u_1) + i f(u_2) \in U_\Cbb$ für alle $u_1, u_2 \in U$}   \\
    \iff&\, \text{$f(u_1), f(u_2) \in U$ für alle $u_1, u_2 \in U$}           \\
    \iff&\, \text{$f(u) \in U$ für alle $u \in U$}                            \\
    \iff&\, f(U) \subseteq U                                                  \\
    \iff&\, \text{$U$ ist $f$-invariant}.
    \qedhere
  \end{align*}
\end{proof}










\section{Hauptteil}


Im Folgenden seien $V$ und $W$ zwei euklidische Vektorräume.


\begin{proposition}
  Es gibt ein eindeutiges (komplexes) Skalarprodukt $\bil{\cdot, \cdot}_\Cbb$ auf $V_\Cbb$, so dass
  \[
      \bil{v_1 + i \cdot 0, v_2 + i \cdot 0}_\Cbb
    = \bil{v_1, v_2}
    \quad
    \text{für alle $v_1, v_2 \in V$},
  \]
  d.h.\ es gilt
  \[
    \bil{\iota(v_1), \iota(v_2)}_\Cbb = \bil{v_1, v_2}
    \quad\text{für alle $v_1, v_2 \in V$},
  \]
  wobei $\iota \colon V \to V_\Cbb$, $v \mapsto v + i \cdot 0$ die kanonische Inklusion bezeichnet.
\end{proposition}


\begin{proof}
  Wenn sich $\bil{\cdot, \cdot}$ zu einem solchen komplexen Skalarprodukt $\bil{\cdot, \cdot}_\Cbb$ fortsetzen lässt, so ist $\bil{\cdot, \cdot}_\Cbb$ insbesondere sesquilinear (also linear im ersten Argument und antilinear im zweiten Argument).
  Für alle $v_1, v_2, w_1, w_2 \in V$ ist dann
  \[
      \bil{v_1 + i v_2, w_1 + i w_2}
    = ( \bil{v_1, w_1} + \bil{v_2, w_2} ) + i( \bil{v_2, w_1} - \bil{v_1, w_2} ).
  \]
  Somit ist $\bil{\cdot, \cdot}_\Cbb$ durch $\bil{\cdot, \cdot}$ eindeutig bestimmt.
  
  Zum Beweis der Existenz definieren wir
  \[
              \bil{v_1 + iv_2, w_1 + iw_2}
    \coloneqq ( \bil{v_1, w_1} + \bil{v_2, w_2} ) + i(\bil{v_2, w_1} - \bil{v_1, w_2} )
  \]
  für alle $v_1, v_2, w_1, w_2 \in V$, und zeigen, dass dies ein komplexes Skalarprodukt auf $V_\Cbb$ ist.
  
  Dass $\bil{\cdot, \cdot}_\Cbb$ hermitsch ist, ergibt sich daraus, dass
  \begin{align*}
      \overline{ \bil{w_1 + i w_2, v_1 + i v_2}_\Cbb }
    &= \overline{ ( \bil{w_1, v_1} + \bil{w_2, v_2} ) + i( \bil{w_2, v_1} - \bil{w_1, v_2} ) } \\
    &= ( \bil{w_1, v_1} + \bil{w_2, v_2} ) - i( \bil{w_2, v_1} - \bil{w_1, v_2} ) \\
    &= ( \bil{w_1, v_1} + \bil{w_2, v_2} ) + i( \bil{w_1, v_2} - \bil{w_2, v_1} ) \\
    &= ( \bil{v_1, w_1} + \bil{v_2, w_2} ) + i( \bil{v_2, w_1} - \bil{v_1, w_2} ) \\
    &= \bil{v_1 + i v_2, w_1 + i w_2}_\Cbb
  \end{align*}
  für alle $v_1, v_2, w_1, w_2 \in V$.
  Da $\bil{\cdot, \cdot}$ $\Rbb$-bilinear ist, ergibt sich, dass $\bil{\cdot, \cdot}_\Cbb$ im ersten Argument $\Rbb$-linear ist (und auch im zweiten).
  Im ersten Argument gilt außerdem
  \begin{align*}
      \bil{i \cdot (v_1 + i v_2), w_1 + i w_2}_\Cbb
    &= \bil{- v_2 + i v_1, w_1 + i w_2}_\Cbb  \\
    &= (\bil{- v_2, w_1} + \bil{v_1, w_2}) + i(\bil{v_1, w_1} - \bil{-v_2, w_2}) \\
    &= (\bil{v_1, w_2} - \bil{v_2, w_1}) + i(\bil{v_1, w_1} + \bil{v_2, w_2}) \\
    &= i \cdot ( (\bil{v_1, w_1} + \bil{v_2, w_2}) + i (\bil{v_2, w_1} - \bil{v_1, w_2}) )  \\
    &= i \cdot \bil{v_1 + i v_2, w_1 + i w_2}_\Cbb.
  \end{align*}
  Zusammen mit der $\Rbb$-Linearität im ersten Argument ergibt sich damit, dass $\bil{\cdot, \cdot}_\Cbb$ im ersten Argument $\Cbb$-linear ist.
  Damit ist $\bil{\cdot, \cdot}$ im zweiten Argument $\Cbb$-antiliner, denn $\bil{\cdot, \cdot}$ ist hermitsch.
  
  Für alle $v_1, v_2 \in V$ ist
  \begin{gather*}
          \bil{v_1 + i v_2, v_1 + i v_2}_\Cbb
    =     \bil{v_1, v_1} + \bil{v_2, v_2}
    =     \|v_1\|^2 + \|v_2\|^2
    \geq  0,
  \intertext{und insbesondere}
          \bil{v_1 + i v_2, v_1 + i v_2}_\Cbb = 0
    \iff  v_1 = v_2 = 0
    \iff  v_1 + i v_2 = 0.
    \qedhere
  \end{gather*}
  Also ist $\bil{\cdot, \cdot}$ positiv definit.
\end{proof}


Im Nachweis der positiven Definitheit von $\bil{\cdot, \cdot}_\Cbb$ haben wir gesehen, wie sich die Norm auf $V_\Cbb$ aus der Norm auf $V$ ergibt:


\begin{lemma}
  Ist $\|\cdot\|_\Cbb$ die Norm auf $V_\Cbb$, die durch $\bil{\cdot, \cdot}_\Cbb$ induziert wird, so gilt
  \[
    \|v_1 + i v_2\| = \sqrt{\|v_1\|^2 + \|v_2\|^2}
    \quad
    \text{für alle $v_1, v_2 \in V$}.
  \]
\end{lemma}


\begin{proof}
  Für alle $v_1, v_2 \in V$ ist
  \begin{align*}
        \|v_1 + i v_2\|_\Cbb^2
    &=  \bil{ v_1 + i v_2, v_1 + i v_2}_\Cbb \\
    &=  (\bil{v_1, v_1} + \bil{v_2, v_2}) + i(\bil{v_2, v_1} - \bil{v_1, v_2}) \\
    &=  \bil{v_1, v_1} + \bil{v_2, v_2}
     =  \|v_1\|^2 + \|v_2\|^2.
    \qedhere
  \end{align*}
\end{proof}


Wir wollen nun noch ein paar Resultate darüber angeben, wie sich Eigenschaften, die im reellen mithilfe des Skalarproduktes definiert sind, auf die Komplexifizierung übertragen.


\begin{proposition}
  Für jeden Untervektorraum $U \subseteq V$ gilt
  \[
    (U^\perp)_\Cbb = (U_\Cbb)^\perp.
  \]
\end{proposition}


\begin{proof}
  Sind $u_1, u_2 \in U$ und $v_1, v_2 \in U^\perp$, so ist
  \[
      \bil{u_1 + i u_2, v_1 + i v_2}_\Cbb
    = (\bil{u_1, v_1} + \bil{u_2, v_2}) + i(\bil{u_2, v_1} - \bil{u_1, v_2})
    = 0,
  \]
  da $\bil{u_i, v_j} = 0$ für alle $i,j \in \{1,2\}$.
  Damit ist $(U^\perp)_\Cbb \subseteq (U_\Cbb)^\perp$.
  
  Andererseits seien $v_1, v_2 \in V$ mit $v_1 + i v_2 \in (U_\Cbb)^\perp$.
  Für alle $u \in U$ ist dann $u + i \cdot 0 \in U_\Cbb$ und somit
  \[
      0
    = \bil{u + i \cdot 0, v_1 + i v_2}
    = \bil{u, v_1} - i \bil{u, v_2}.
  \]
  Also ist $\bil{u, v_1} = \bil{u, v_2} = 0$ für alle $u \in U$ und somit $v_1, v_2 \in U^\perp$.
  Damit ist auch $(U_\Cbb)^\perp \subseteq (U^\perp)_\Cbb$.a
\end{proof}


\begin{proposition}\label{prop: complexification has adjoints}
  \begin{enumerate}[leftmargin=*]
    \item
      Zwei lineare Abbildungen $f \colon V \to W$ und $g \colon W \to V$ sind genau dann adjungiert zueinander, wenn $f_\Cbb$ und $g_\Cbb$ adjungiert zueinander sind.
    \item
      Wenn $f$ ein Adjungiertes $f^*$ besitzt, so hat auch $f_\Cbb$ ein Adjungiertes, und es gilt
      \[
        (f_\Cbb)^* = (f^*)_\Cbb.
      \]
  \end{enumerate}
\end{proposition}


\begin{proof}
  \begin{enumerate}[leftmargin=*]
    \item
      Wir nehmen zunächst an, dass $f$ und $g$ adjungiert zueinander sind.
      Für alle $v_1, v_2, w_1, w_2 \in V$ ist dann
      \begin{align*}
        &\,  \bil{ f_\Cbb( v_1 + i v_2 ), w_1 + i w_2 }_\Cbb
        =     \bil{ f(v_1) + i f(v_2), w_1 + i w_2}_\Cbb \\
        =&\,  (\bil{f(v_1), w_1} + \bil{f(v_2), w_2}) + i(\bil{f(v_2), w_1} - \bil{f(v_1), w_2}) \\
        =&\,  (\bil{v_1, g(w_1)} + \bil{v_2, g(w_2)}) + i(\bil{v_2, g(w_1)} - \bil{v_1, g(w_2)}) \\
        =&\,  \bil{ v_1 + i v_2, g(w_1) + i g(w_2) }_\Cbb
        =     \bil{ v_1 + i v_2, g_\Cbb( w_1 + i w_2 ) }_\Cbb.
      \end{align*}
      Also sind $f_\Cbb$ und $g_\Cbb$ adjungiert zueinander.
      
      Wir nehmen nun an, dass $f_\Cbb$ und $g_\Cbb$ adjungiert zunander sind.
      Für alle $v \in V$, $w \in W$ ist dann
      \begin{align*}
            \bil{f(v), w}
        &=  \bil{f(v) + i \cdot 0, w + i \cdot 0}_\Cbb
         =  \bil{f_\Cbb(v + i \cdot 0), w + i \cdot 0}_\Cbb \\
        &=  \bil{v + i \cdot 0, g_\Cbb(w + i \cdot 0)}_\Cbb
         =  \bil{v + i \cdot 0, g(w) + i \cdot 0}_\Cbb
         =  \bil{v, g(w)}.
      \end{align*}
      Also sind $f$ und $g$ adjungiert zueinander.
      
    \item
      Besitzt $f$ ein Adjungiertes $f^*$, so sind $f_\Cbb$ und $(f^*)_\Cbb$ nach dem vorherigen Teil adjungiert zueinander.
      Also besitzt $f_\Cbb$ ein Adjungiertes $(f_\Cbb)^*$, nämlich $(f^*)_\Cbb$.
  \end{enumerate}
\end{proof}


\begin{remark}
  Besitzt $f$ ein Adjungiertes, so schreiben wir im Folgenden meist nur $f^*_\Cbb$ statt $(f^*)_\Cbb$ oder $(f_\Cbb)^*$.
\end{remark}


\begin{corollary}\label{cor: complexification preserves normally stuff}
  Es sei $f \colon V \to V$ ein Endomorphismus, der ein Adjungiertes besitzt.
  \begin{enumerate}[leftmargin=*]
    \item
      Ist $f$ normal, so ist $f_\Cbb$ normal.
    \item
      Ist $f$ selbstadjungiert, so ist $f_\Cbb$ selbstadjungiert.
    \item
      Ist $f$ antiselbstadjungiert, so ist $f_\Cbb$ antiselbstadjungiert.
    \item
      Ist $f$ orthogonal, so ist $f_\Cbb$ unitär.
  \end{enumerate}
\end{corollary}


\begin{proof}
  Nach Proposition~\ref{prop: complexification has adjoints} ist $f^*_\Cbb = (f^*)_\Cbb$ adjungiert zu $f_\Cbb$.
  \begin{enumerate}[leftmargin=*]
    \item
      Da $f$ normal ist, gilt $f^* f = f^* f$.
      Nach Lemma~\ref{lemma: commuting in complexification} ist deshalb $f^*_\Cbb f_\Cbb = f^*_\Cbb f$.
      Also ist auch $f_\Cbb$ normal.
    \item
      Wegen der Selbstadjungiertheit von $f$ ist $f = f^*$.
      Also ist $f_\Cbb = f^*_\Cbb$, und $f_\Cbb$ somit selbstadjungiert.
    \item
      Es gilt $f^* = -f$ und somit $f^*_\Cbb = (-f)_\Cbb = - f_\Cbb$.
    \item
      Die Orthogonalität von $f$ bedeutet, dass $f f^* = \id_V = f^* f$.
      Durch Komplexifizieren ergibt sich, dass
      \[
        f_\Cbb f^*_\Cbb = \id_{V_\Cbb} = f^*_\Cbb f_\Cbb.
      \]
      Also ist $f_\Cbb$ orthogonal.
  \end{enumerate}
\end{proof}


\begin{proof}
  Es ist
  \[
      f_\Cbb (f_\Cbb)^*
    = f_\Cbb (f^*)_\Cbb
    = (f f^*)_\Cbb
    = (f^* f)_\Cbb
    = (f^*)_\Cbb f_\Cbb
    = (f_\Cbb)^* f_\Cbb.
    \qedhere
  \]
\end{proof}










\section{Zettel 6, Aufgabe 5}


Wir erinnern an die folgende Aussage:


\begin{theorem}\label{thrm: invariance for unitary}
  Es sei $V$ ein endlichdimensionaler unitärer Vektorraum, $f \colon V \to V$ ein normaler Endomorphismus und $U \subseteq V$ ein $f$-invarianter Untervektorraum.
  Dann ist auch $U^\perp$ $f$-invariant.
\end{theorem}


\begin{proof}
  Dies wurde auf dem sechsten Übungszettel, bzw.\ im Tutorium gezeigt
\end{proof}


Mithilfe der Komplexifizierung lässt sich die analoge Aussage für endlichdimensionale euklidische Vektorräume nun direkt folgern:


\begin{corollary}
  Es sei $V$ ein endlichdimensionaler euklidischer Vektorraum, $f \colon V \to V$ ein normaler Endomorphismus und $U \subseteq V$ ein $f$-invarianter Untervektorraum.
  Dann ist $U^\perp$ ebenfalls $f$-invariant.
\end{corollary}


\begin{proof}
  Wegen der Normalität von $f$ ist nach Korollar~\ref{cor: complexification preserves normally stuff} auch $f_\Cbb \colon V_\Cbb \to V_\Cbb$ normal, und wegen der Endlichdimensionalität von $V$ ist auch $V_\Cbb$ endlichdimensional.
  Wegen der $f$-Invarianz von $U$ ist nach Lemma~\ref{lem: invariance of induced subspaces} der induzierte Unterraum $U_\Cbb$ invariant unter $f_\Cbb$.
  Nach Theorem~\ref{thrm: invariance for unitary} ist $(U_\Cbb)^\perp = (U^\perp)_\Cbb$ invariant unter $f_\Cbb$.
  Nach Lemma~\ref{lem: invariance of induced subspaces} ist somit $U^\perp$ invariant unter $f$.
\end{proof}
















\end{document}
