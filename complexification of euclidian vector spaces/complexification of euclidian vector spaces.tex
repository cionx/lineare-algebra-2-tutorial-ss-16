\documentclass[a4paper,10pt]{article}
%\documentclass[a4paper,10pt]{scrartcl}

\usepackage{../generalstyle}

\setromanfont[Mapping=tex-text]{Linux Libertine O}
% \setsansfont[Mapping=tex-text]{DejaVu Sans}
% \setmonofont[Mapping=tex-text]{DejaVu Sans Mono}

\title{Komplexifizierung euklidischer Vektorräume}
\author{Jendrik Stelzner}
\date{\today}

\begin{document}
\maketitle


Es sei $V$ ein euklidischer Vektorraum mit Skalarprodukt $\bil{\cdot, \cdot}$.
Wir zeigen zunächst, dass sich dieses Skalarprodukt eindeutig auf $V_\Cbb$ fortsetzen lässt:


\section{Vorbereitung}


Bevor wir uns dem eigentlichen Thema zuwenden, zeigen wir noch einige nützliche Aussagen über die Komplexifizierung reeller Vektorräume.


\begin{lemma}\label{lemma: commuting in complexification}
  Es seien $V$ und $W$ zwei $\Rbb$-Vektorräume, und $f, g \colon V \to w$ $\Rbb$-lineare Abbildungen.
  Dann ist genau dann $f = g$, wenn $f_\Cbb = g_\Cbb$.
\end{lemma}


\begin{proof}
  Ist $f = g$, so ist auch $f_\Cbb = g_\Cbb$.
  Ist andererseits $f_\Cbb = g_\Cbb$, so ist
  \[
    f(v) + i \cdot 0
    = f_\Cbb(v + i \cdot 0)
    = g_\Cbb(v + i \cdot 0)
    = g(v) + i \cdot 0
    \quad
    \text{für alle $v \in V$},
  \]
  und somit $f(v) = g(v)$ für alle $v \in V$.
\end{proof}


\begin{corollary}
  Sind $f, g \colon V \to V$ Endomorphismen eines $\Rbb$-Vektorraums $V$, so kommutieren $f$ und $g$ genau dann, wenn $f_\Cbb$ und $g_\Cbb$ kommutieren.
\end{corollary}


\begin{proof}
  Da $(f \circ g)_\Cbb = g_\Cbb \circ f_\Cbb$ und $(g \circ f)_\Cbb = g_\Cbb \circ f_\Cbb$ ist nach Lemma~\ref{lemma: commuting in complexification} genau dann $f \circ g = g \circ f$, wenn $f_\Cbb \circ g_\Cbb = g_\Cbb \circ f_\Cbb$.
\end{proof}


\begin{lemma}\label{lem: invariance of induced subspaces}
  Es sei $V$ ein $\Rbb$-Vektorraum, $f \colon V \to V$ ein Endomorphismus und $U \subseteq V$ ein Untervektorraum.
  Dann ist $U$ genau dann $f$-invariant, wenn $U_\Cbb$ invariant unter $f_\Cbb$ ist.
\end{lemma}


\begin{proof}
  Es gilt
  \begin{align*}
        &\, \text{$U_\Cbb$ ist $f_\Cbb$-invariant}                            \\
    \iff&\, f_\Cbb(U_\Cbb) \subseteq U_\Cbb                                   \\
    \iff&\, \text{$f_\Cbb(u_1 + i u_2) \in U_\Cbb$ für alle $u_1, u_2 \in U$} \\
    \iff&\, \text{$f(u_1) + i f(u_2) \in U_\Cbb$ für alle $u_1, u_2 \in U$}   \\
    \iff&\, \text{$f(u_1), f(u_2) \in U$ für alle $u_1, u_2 \in U$}           \\
    \iff&\, \text{$f(u) \in U$ für alle $u \in U$}                            \\
    \iff&\, f(U) \subseteq U                                                  \\
    \iff&\, \text{$U$ ist $f$-invariant}.
    \qedhere
  \end{align*}
\end{proof}










\section{Hauptteil}


\begin{proposition}
  Es gibt ein eindeutiges (komplexes) Skalarprodukt $\bil{\cdot, \cdot}_\Cbb$ auf $V_\Cbb$, so dass
  \[
      \bil{v_1 + i \cdot 0, v_2 + i \cdot 0}_\Cbb
    = \bil{v_1, v_2}
    \quad
    \text{für alle $v_1, v_2 \in V$},
  \]
  d.h.\ es gilt
  \[
    \bil{\iota(v_1), \iota(v_2)}_\Cbb = \bil{v_1, v_2}
    \quad\text{für alle $v_1, v_2 \in V$},
  \]
  wobei $\iota \colon V \to V_\Cbb$, $v \mapsto v + i \cdot 0$ die kanonische Inklusion bezeichnet.
\end{proposition}


\begin{proof}
  Wenn sich $\bil{\cdot, \cdot}$ zu einem solchen komplexen Skalarprodukt $\bil{\cdot, \cdot}_\Cbb$ fortsetzen lässt, so ist $\bil{\cdot, \cdot}_\Cbb$ insbesondere sesquilinear (also linear im ersten Argument und antilinear im zweiten Argument).
  Für alle $v_1, v_2, w_1, w_2 \in V$ ist dann
  \[
      \bil{v_1 + i v_2, w_1 + i w_2}
    = ( \bil{v_1, w_1} + \bil{v_2, w_2} ) + i( \bil{v_2, w_1} - \bil{v_1, w_2} ).
  \]
  Somit ist $\bil{\cdot, \cdot}_\Cbb$ eindeutig.
  
  Zum Beweis der Existenz definieren wir
  \[
              \bil{v_1 + iv_2, w_1 + iw_2}
    \coloneqq ( \bil{v_1, w_1} + \bil{v_2, w_2} ) + i(\bil{v_2, w_1} - \bil{v_1, w_2} )
  \]
  für alle $v_1, v_2, w_1, w_2 \in V$, und zeigen, dass dies ein komplexes Skalarprodukt auf $V_\Cbb$ ist.
  
  $\bil{\cdot, \cdot}_\Cbb$ is hermitsch, denn für alle $v_1, v_2, w_1, w_2 \in V$ ist
  \begin{align*}
      \overline{ \bil{w_1 + i w_2, v_1 + i v_2}_\Cbb }
    &= \overline{ ( \bil{w_1, v_1} + \bil{w_2, v_2} ) + i( \bil{w_2, v_1} - \bil{w_1, v_2} ) } \\
    &= ( \bil{w_1, v_1} + \bil{w_2, v_2} ) - i( \bil{w_2, v_1} - \bil{w_1, v_2} ) \\
    &= ( \bil{w_1, v_1} + \bil{w_2, v_2} ) + i( \bil{w_1, v_2} - \bil{w_2, v_1} ) \\
    &= ( \bil{v_1, w_1} + \bil{v_2, w_2} ) + i( \bil{v_2, w_1} - \bil{v_1, w_2} ) \\
    &= \bil{v_1 + i v_2, w_1 + i w_2}_\Cbb.
  \end{align*}
  Da $\bil{\cdot, \cdot}$ $\Rbb$-bilinear ist, ergibt sich, dass $\bil{\cdot, \cdot}_\Cbb$ im ersten Argument $\Rbb$-linear ist (und auch im zweiten).
  Im ersten Argument gilt außerdem
  \begin{align*}
      \bil{i \cdot (v_1 + i v_2), w_1 + i w_2}_\Cbb
    &= \bil{- v_2 + i v_1, w_1 + i w_2}_\Cbb  \\
    &= (\bil{- v_2, w_1} + \bil{v_1, w_2}) + i(\bil{v_1, w_1} - \bil{-v_2, w_2}) \\
    &= (\bil{v_1, w_2} - \bil{v_2, w_1}) + i(\bil{v_1, w_1} + \bil{v_2, w_2}) \\
    &= i \cdot ( (\bil{v_1, w_1} + \bil{v_2, w_2}) + i (\bil{v_2, w_1} - \bil{v_1, w_2}) )  \\
    &= i \cdot \bil{v_1 + i v_2, w_1 + i w_2}_\Cbb.
  \end{align*}
  Zusammen mit der $\Rbb$-Linearität im ersten Argument ergibt sich damit, dass $\bil{\cdot, \cdot}_\Cbb$ im ersten Argument $\Cbb$-linear ist.
  Da $\bil{\cdot, \cdot}_\Cbb$ hermitsch und im ersten Argument $\Cbb$-linear ist, ist $\bil{\cdot, \cdot}_\Cbb$ im zweiten Argument $\Cbb$-antilinear.
  
  Für alle $v_1, v_2 \in V$ ist
  \begin{gather*}
      \bil{v_1 + i v_2, v_1 + i v_2}_\Cbb
    = \bil{v_1, v_1} + \bil{v_2, v_2}
    = \|v_1\| + \|v_2\|,
  \shortintertext{und somit}
          \bil{v_1 + i v_2, v_1 + i v_2} = 0
    \iff  v_1 = v_2 = 0
    \iff  v_1 + i v_2 = 0.
    \qedhere
  \end{gather*}
\end{proof}


\begin{lemma}
  Es sei $V$ ein euklidischer Vektorraum, und $\|\cdot\|_\Cbb$ die Norm die durch $\bil{\cdot, \cdot}$ induziert wird.
  Dann gilt
  \[
    \|v_1 + i v_2\| = \sqrt{\|v_1\|^2 + \|v_2\|^2}
    \quad
    \text{für alle $v_1, v_2 \in V$}.
  \]
\end{lemma}


\begin{proof}
  Für alle $v_1, v_2 \in V$ ist
  \begin{align*}
        \|v_1 + i v_2\|_\Cbb^2
    &=  \bil{ v_1 + i v_2, v_1 + i v_2}_\Cbb \\
    &=  (\bil{v_1, v_1} + \bil{v_2, v_2}) + i(\bil{v_2, v_1} - \bil{v_1, v_2}) \\
    &=  \bil{v_1, v_1} + \bil{v_2, v_2}
     =  \|v_1\|^2 + \|v_2\|^2.
    \qedhere
  \end{align*}
\end{proof}



\begin{proposition}
  Es sei $V$ ein euklidischer Vektorraum und $U \subseteq V$ ein Untervektorraum.
  Dann ist
  \[
    (U^\perp)_\Cbb = (U_\Cbb)^\perp.
  \]
\end{proposition}


\begin{proof}
  Sind $v_1, v_2 \in U$ und $w_1, w_2 \in U^\perp$, so ist
  \[
      \bil{v_1 + i v_2, w_1 + i w_2}_\Cbb
    = (\bil{v_1, w_1} + \bil{v_2, w_2}) + i(\bil{v_2, w_1} - \bil{v_1, w_2})
    = 0,
  \]
  da $\bil{v_i, w_j} = 0$ für alle $i,j \in \{1,2\}$.
  Damit ist $(U^\perp)_\Cbb \subseteq (U_\Cbb)^\perp$.
  
  Andererseits seien $w_1, w_2 \in V$ mit $w_1 + i w_2 \in (U_\Cbb)^\perp$.
  Für alle $v \in U$ ist dann $v + i \cdot 0 \in U_\Cbb$ und somit
  \[
      0
    = \bil{v + i \cdot 0, w_1 + i w_2}
    = \bil{v, w_1} - i \bil{v, w_2}.
  \]
  Also ist $\bil{v, w_1} = \bil{v, w_2} = 0$ für alle $v \in U$ und somit $w_1, w_2 \in U^\perp$.
  Damit ist auch $(U_\Cbb)^\perp \subseteq (U^\perp)_\Cbb$.
\end{proof}


\begin{proposition}
  Es seien $V$ und $W$ euklidische Vektorräume und es sei $f \colon V \to W$ eine lineare Abbildung.
  Dann ist
  \[
    (f^*)_{\Cbb} = (f_\Cbb)^*.
  \]
\end{proposition}


\begin{proof}
  Dies lässt sich etwa in Koordinaten durchrechnen:
  Für alle $v_1, v_2, w_1, w_2$ ist
  \begin{align*}
     &\,  \bil{ f_\Cbb( v_1 + i v_2 ), w_1 + i w_2 }_\Cbb
    =     \bil{ f(v_1) + i f(v_2), w_1 + i w_2}_\Cbb \\
    =&\,  (\bil{f(v_1), w_1} + \bil{f(v_2), w_2}) + i(\bil{f(v_2), w_1} - \bil{f(v_1), w_2}) \\
    =&\,  (\bil{v_1, f^*(w_1)} + \bil{v_2, f^*(w_2)}) + i(\bil{v_2, f^*(w_1)} - \bil{v_1, f^*(w_2)}) \\
    =&\,  \bil{ v_1 + i v_2, f^*(w_1) + i f^*(w_2) }_\Cbb
    =     \bil{ v_1 + i v_2, (f^*)_\Cbb( w_1 + i w_2 ) }_\Cbb,
  \end{align*}
  also hat $(f^*)_\Cbb$ die definierende Eigenschaft von $(f_\Cbb)^*$.
\end{proof}


\begin{corollary}\label{cor: complexification preserves normal}
  Ist $V$ ein euklidischer Vektorraum und $f \colon V \to V$ ein normaler Endomorphismus, so ist $f_\Cbb$ ein normaler Endomorphismus von $V_\Cbb$.
\end{corollary}


\begin{proof}
  Es ist
  \[
      f_\Cbb (f_\Cbb)^*
    = f_\Cbb (f^*)_\Cbb
    = (f f^*)_\Cbb
    = (f^* f)_\Cbb
    = (f^*)_\Cbb f_\Cbb
    = (f_\Cbb)^* f_\Cbb.
    \qedhere
  \]
\end{proof}


\begin{theorem}\label{thrm: invariance for unitary}
  Es sei $V$ ein endlichdimensionaler unitärer Vektorraum, $f \colon V \to V$ ein normaler Endomorphismus und $U \subseteq V$ ein $f$-invarianter Untervektorraum.
  Dann ist auch $U^\perp$ $f$-invariant.
\end{theorem}


\begin{proof}
  Dies wurde auf dem sechsten Übungszettel gezeigt.
\end{proof}


\begin{corollary}
  Es sei $V$ ein endlichdimensionaler euklidischer Vektorraum, $f \colon V \to V$ ein normaler Endomorphismus und $U \subseteq V$ ein $f$-invarianter Untervektorraum.
  Dann ist $U^\perp$ ebenfalls $f$-invariant.
\end{corollary}


\begin{proof}
  Wegen der Normalität von $f$ ist nach Korollar~\ref{cor: complexification preserves normal} auch $f_\Cbb \colon V_\Cbb \to V_\Cbb$ normal, und wegen der Endlichdimensionalität von $V$ ist auch $V_\Cbb$ endlichdimensional.
  Wegen der $f$-Invarianz von $U$ ist nach Lemma~\ref{lem: invariance of induced subspaces} der induzierte Unterraum $U_\Cbb$ invariant unter $f_\Cbb$.
  Nach Theorem~\ref{thrm: invariance for unitary} ist $(U_\Cbb)^\perp = (U^\perp)_\Cbb$ invariant unter $f_\Cbb$.
  Nach Lemma~\ref{lem: invariance of induced subspaces} ist somit $U^\perp$ invariant unter $f$.
\end{proof}
















\end{document}
