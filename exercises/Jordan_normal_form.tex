\section{Jordannormalform, Haupträume und Trigonalisierbarkeit}





%%% PRIORITY 1


\begin{question}[subtitle = Spur und Determinante durch Eigenwerte]{1}
  \begin{enumerate}[leftmargin=*]
    \item
      Es sei $f \colon V \to V$ ein Endomorphismus eines endlichdimensionalen $\Cbb$-Vektorraums $V$.
      Drücken Sie $\tr f$ und $\det f$ durch die (nicht notwendigerweise verschiedenen) Eigenwerte $\lambda_1, \dotsc, \lambda_n \in \Cbb$ von $f$ aus.
    \item
      Drücken Sie für den Fall $n = 2$ die Determinante $\det f$ durch $\tr f$ und $\tr f^2$ aus.
  \end{enumerate}
\end{question}


\begin{question}[subtitle = Berechnung von Jordannormalformen]{1}
  Bestimmen Sie für die folgenden komplexen Matrizen jeweils eine Jordannormalform, inklusive entsprechender Basiswechselmatrizen:
  \begin{gather*}
    \begin{pmatrix*}[r]
      2 & 2 &  -5 \\
      3 & 7 & -15 \\
      1 & 2 &  -4
    \end{pmatrix*},
    \quad
    \begin{pmatrix*}[r]
      0 & 0 &  1 \\
      1 & 0 & -3 \\
      0 & 1 &  3
    \end{pmatrix*},
    \quad
    \begin{pmatrix*}[r]
      1 & 0 & 1 & -1  \\
      0 & 1 & 1 &  0  \\
      0 & 0 & 1 &  1  \\
      0 & 0 & 0 &  1
    \end{pmatrix*},
    \quad
    \begin{pmatrix*}[r]
      3 & -4  &  0  &  2  \\
      4 & -5  & -2  &  4  \\
      0 &  0  &  3  & -2  \\
      0 &  0  &  2  & -1
    \end{pmatrix*},
  \\
    \begin{pmatrix*}[r]
       3  &  3  &  1  &  5  \\
       0  & -2  &  2  & -8  \\
      -1  & -2  &  0  & -3  \\
       0  &  2  & -1  &  6
    \end{pmatrix*},
    \quad
    \begin{pmatrix*}[r]
       7  & 1 &  2  &  2  \\
       1  & 4 & -1  & -1  \\
      -2  & 1 &  5  & -1  \\
       1  & 1 &  2  &  8
    \end{pmatrix*},
    \quad
    \begin{pmatrix*}[r]
       5  &  4  &  2  &  1  \\
       0  &  1  & -1  & -1  \\
      -1  & -1  &  3  &  0  \\
       1  &  1  & -1  &  2
    \end{pmatrix*}
  \end{gather*}
\end{question}
\begin{solution}
  \begin{enumerate}[leftmargin=*]
    \item
      Eine mögliche Jordannormalform ist
      $
        \begin{pmatrix}
          1 &   &   \\
            & 1 &   \\
            &   & 3
        \end{pmatrix},
      $
      mit Basiswechselmatrix
      $
        \begin{pmatrix*}[r]
          5 & -2  & 1 \\
          0 &  1  & 3 \\
          1 &  0  & 1
        \end{pmatrix*}.
      $
    \item
      Die Jordannormalform ist
      $
        \begin{pmatrix*}[r]
          1 & 1 &   \\
            & 1 & 1 \\
            &   & 1
        \end{pmatrix*},
      $
      mit Basiswechselmatrix
      $
        \begin{pmatrix*}[r]
           1  & -1  & 1 \\
          -2  &  1  & 0 \\
           1  &  0  & 0
        \end{pmatrix*}.
      $
    \item
      Eine mögliche Jordannormalform ist
      $
        \begin{psmallmatrix*}[r]
          1 &   &   &   \\
            & 1 & 1 &   \\
            &   & 1 & 1 \\
            &   &   & 1
        \end{psmallmatrix*},
      $
      mit Basiswechselmatrix
      $
        \begin{psmallmatrix*}[r]
          1 & 1 & -1 & 0  \\
          0 & 1 &  0 & 0  \\
          0 & 0 &  1 & 0  \\
          0 & 0 &  0 & 1
        \end{psmallmatrix*}.
      $
    \item
      Eine mögliche Jordannormalform ist
      $
        \begin{psmallmatrix*}[r]
          -1  &  1  &   &   \\
              & -1  &   &   \\
              &     & 1 & 1 \\
              &     &   & 1
        \end{psmallmatrix*},
      $
      mit Basiswechselmatrix
      $
        \begin{pmatrix*}[r]
          1 & \frac{1}{4} & 1 & \frac{1}{2} \\
          1 & 0           & 1 & 0           \\
          0 & 0           & 1 & \frac{1}{2} \\
          0 & 0           & 1 & 0
        \end{pmatrix*}.
      $
    \item
      Eine mögliche Jordannormalform ist
      $
        \begin{psmallmatrix}
          1 &   &   &   \\
            & 2 & 1 &   \\
            &   & 2 & 1 \\
            &   &   & 2
        \end{psmallmatrix},
      $
      mit Basiswechselmatrix
      $
        \begin{psmallmatrix*}[r]
           0  &  1  &  2  &  7  \\
          -2  & -2  &  0  & -1  \\
           1  &  0  & -1  & -2  \\
           1  &  1  &  0  &  0
        \end{psmallmatrix*}.
      $
    \item
      Eine mögliche Jordannormalform ist
      $
        \begin{psmallmatrix}
          6 &   &   &   \\
            & 6 & 1 &   \\
            &   & 6 & 1 \\
            &   &   & 6
        \end{psmallmatrix},
      $
      mit Basiswechselmatrix
      $
        \renewcommand*{\arraystretch}{1.3}
        \begin{pmatrix*}[r]
           \frac{2}{3}  &  3  &  2  & 0 \\
           \frac{2}{3}  &  3  & -1  & 0 \\
          -\frac{1}{3}  & -6  & -1  & 0 \\
          -\frac{1}{3}  &  3  &  2  & 0
        \end{pmatrix*}.
      $
    \item
      Eine mögliche Jordannormalform ist
      $
        \begin{psmallmatrix}
          1 &   &   &   \\
            & 2 &   &   \\
            &   & 4 & 1 \\
            &   &   & 4
        \end{psmallmatrix},
      $
      mit Basiswechselmatrix
      $
        \begin{psmallmatrix*}[r]
          -1  &  1  &  1  & 1 \\
           1  & -1  &  0  & 0 \\
           0  &  0  & -1  & 0 \\
           0  &  1  &  1  & 0
        \end{psmallmatrix*}.
      $
  \end{enumerate}
\end{solution}


\begin{question}{1}
  \begin{enumerate}[leftmargin=*]
    \item
      Bestimmen Sie alle möglichen Jordannormalformen einer nicht-diagonalisierbare Matrix $A \in \Mat_2(\Cbb)$ mit $\tr = 0$.
    \item
      Bestimmen Sie alle möglichen Jordannormalformen für $A \in \Mat_3(\Cbb)$ mit $\det A = 0$ und $\tr A = 0$.
  \end{enumerate}
\end{question}


\begin{question}[subtitle = Lösen linearer homogener Differentialgleichungen]{1}
  Bestimmen Sie die Lösungsräume der folgenden homogenen linearen Differentialgleichungen.
  Dabei seien $f, g, h \in C^\infty(\Rbb)$.
  \[
    \left\{
      \begin{array}{ccrr}
        f'  & = & -f  & - 6g, \\
        g'  & = & 2f  & + 6g;
      \end{array}
    \right.
    \quad
    \left\{
      \begin{array}{ccrr}
        f'  & = & -f  & - g,  \\
        g'  & = & 2f  & + g;
      \end{array}
    \right.
    \quad
    \left\{
      \begin{array}{ccrrr}
        f'  & = & 2f  & + 2g  & + 3h, \\
        g'  & = &  f  & + 3g  & + 3h, \\
        h'  & = & -f  & - 2f  & - 2h.
      \end{array}
    \right.
  \]
\end{question}





%%% PRIORITY 2


\begin{question}{2}
  Es sei $A \in \Mat_2(\Cbb)$ mit $\tr A = 0$ und $\tr A^2 = -2$.
  Bestimmen Sie $\det A$.
\end{question}


\begin{question}[subtitle = Zu Haupträumen]{2}
  Es sei $V$ ein endlichdimensionaler $K$-Vektorraum und $f \colon V \to V$ ein Endomorphismus.
  \begin{enumerate}[leftmargin=*]
    \item
      Es sei $n \colon V \to V$ ein nilpotenter Endomorphismus.
      Zeigen Sie, dass ${\id_V} - n$ invertierbar ist.
      
      (\emph{Hinweis}:
       Es ist $\id_V = \id_V - n^{p+1}$ für $p \in \Nbb$ groß genug.)
    \item
      Zeigen, bzw.\ folgern Sie allgemeiner, dass $\lambda {\id_V} + n$ für alle $\lambda \in K^\times$ invertierbar ist.
    \item
      Es sei $f \colon V \to V$ ein beliebiger Endomorphismus.
      Zeigen Sie für alle $\lambda, \mu \in K$ mit $\lambda \neq \mu$, dass $V^\sim_\lambda(f)$ invariant unter $f - \mu {\id_V}$ ist, und dass die Einschränkung $(f - \mu {\id_V})|_{V^\sim_\lambda(f)}$ invertierbar ist.
    \item
      Folgern Sie:
      Ist $(f - \lambda_1)^{n_1} \dotsm (f - \lambda_k)^{n_k} = 0$ mit $\lambda_1, \dotsc, \lambda_k \in K$ paarweise verschieden, so sind $\lambda_1, \dotsc, \lambda_k$ die möglichen Eigenwerte von $f$, und für alle $1 \leq i \leq k$ ist der Nilpotenzindex von $(f - \lambda_i)|_{V^\sim_\lambda(f)}$ durch $\leq n_k$ abschätzbar.
    \item
      Folgern Sie:
      Ist $K$ algebraisch abgeschlossen und $(f - \lambda_1) \dotsm (f - \lambda_n) = 0$, so ist $f$ diagonalisierbar mit möglichen Eigenwerten $\lambda_1, \dotsc, \lambda_n \in K$.
  \end{enumerate}
\end{question}


\begin{question}[subtitle = Ein Gegenbeispiel]{2}
  Es sei $f \colon V \to V$ ein Endomorphismus eines $\Cbb$-Vektorraums $V$ und $\lambda \in \Cbb$.
  Zeigen Sie, dass die Einschränkung $(f - \lambda {\id_V})|_{V^\sim_\lambda(f)}$ nicht notwendigerweise nilpotent ist.
\end{question}


\begin{question}[subtitle = Bestimmung möglicher Jordannormalformen]{2}
  Bestimmen Sie in den Folgenden alle Möglichkeiten der Jordannormalform von $A \in \Mat_n(\Cbb)$.
  \begin{enumerate}[leftmargin=*]
    \item
     Es ist $\chi_A(T) = (T-3)^4 (T-5)^4$ und $(A - 3I)^2 (A - 5I)^2 = 0$.
    \item
      Es ist $A^3 = 0$ und alle nicht-trivialen Eigenräume von $A$ sind eindimensional.
    \item
      Es ist $\chi_A(T) = (T-2)(T+2)^3$ und $(A - 2I)(A + 2I) = 0$.
    \item
      Es ist $\chi_A(T) = T^3 - T$.
    \item
      Es ist $\chi_A(T) = (T^2 -5T + 6)^2$ und alle Eigenräume von $A$ sind entweder null-\ oder eindimensional.
    \item
      Es ist $A^2 = A$ und alle nicht-trivialen Eigenräume von $A$ sind zweidimensional.
    \item
      Es ist $\chi_A(T) = T^5$ und alle Eigenräume von $A$ sind entweder null-\ oder eindimensional.
    \item
      Es ist $\chi_A(T) = (T+3)^3 T^2$ und $A$ hat keine zweidimensionalen Eigenräume.
    \item
      Es ist $\chi_A(T) = T^5 - 2 T^4$.
  \end{enumerate}
\end{question}


\begin{question}{2}
  Bestimmen Sie die Potenz $A^{10}$ der Matrix
  \[
    A
    \coloneqq
    \begin{pmatrix*}[r]
       3  & 4 &  3      \\
      -1  & 0 & -1      \\
       1  & 2 &  3
    \end{pmatrix*}
    \in \Mat_3(\Cbb).
  \]
\end{question}





%%% PRIORITY 3


\begin{question}[subtitle = Shiften von Haupträumen]{3}
  Es sei $V$ ein endlichdimensionaler $\Cbb$-Vektorraum, und es seien $K, E \colon V \to V$ zwei Endomorphismen mit
  \[
    \text{$K$ ist invertierbar}
    \quad\text{und}\quad
    KE = 2EK.
  \]
  \begin{enumerate}[leftmargin=*]
    \item
      Zeigen Sie, dass
      \[
        (K - 2 \lambda \id_V)^n E = 2^n E (K - \lambda \id_V)^n
      \]
      für alle $n \in \Nbb$.
    \item
      Folgern Sie, dass $E( V^\sim_\lambda(K) ) \subseteq V^\sim_{2\lambda}(K)$ für alle $\lambda \in \Cbb$.
    \item
      Folgern Sie, dass $E$ nilpotent ist.
  \end{enumerate}
\end{question}


\begin{question}[subtitle = Determinante durch Tracepowers ausdrücken]{3}
  Es sei $K$ ein algebraisch abgeschlossener Körper mit $\ringchar K \notin \{2,3\}$.
  Zeigen Sie, dass
  \[
    \det A = \frac{1}{6} (\tr A)^3 - \frac{1}{2} (\tr A^2)(\tr A) + \frac{1}{3} (\tr A^3)
    \quad
    \text{für jedes $A \in \Mat_3(K)$}.
  \]
  
  (\emph{Hinweis}:
   Wenn die Rechnungen zu kompliziert werden, dann macht man es falsch.)
\end{question}






%%% PRIORITY 4


\begin{question}[subtitle = Die multiplikative Jordanzerlegung]{4}
  Es sei $V$ ein $\Cbb$-Vektorraum.
  \begin{enumerate}[leftmargin=*]
    \item
      Es sei $n \colon V \to V$ ein nilpotenter Endomorphismus.
      Zeigen Sie, dass der Endomorphismus ${\id_V} + n$ invertierbar ist.
      
      (\emph{Hinweis}: Es ist $\id_V = \id_V \pm n^{p+1}$ für $p \in \Nbb$ groß genug.)
  \end{enumerate}
  Ein Endomorphismus $u \colon V \to V$ heißt \emph{unipotent}, falls $u - \id_V$ nilpotent ist.
  \begin{enumerate}[leftmargin=*, resume]
    \item
      Folgern Sie, dass jeder unipotente Endomorphismus von $V$ invertierbar ist.
  \end{enumerate}
  Von nun an sei $V$ endlichdimensional.
  Auf dem fünften Übungszettel wurde gezeigt, dass es für jeden Endomorphismus $f \colon V \to V$ eindeutige Endomorphismen $d,n \colon V \to V$ gibt, so dass
  \begin{itemize}
    \item 
      $f = d + n$,
    \item
      $d$ ist diagonalisierbar und $n$ ist nilpotent, und
    \item
      $d$ und $n$ kommutieren.
  \end{itemize}
  Dies ist die \emph{additive Jordanzerlegung} auf $\End(V)$.
  \begin{enumerate}[leftmargin=*, resume]
    \item
      Zeigen Sie, dass $f$ genau dann invertierbar ist, wenn der diagonalisierbare Anteil $d$ der additiven Jordanzerlegung von $f$ invertierbar ist.
  \end{enumerate}
  Folgern Sie damit aus der obigen additiven Jordanzerlegung von $\End(V)$ die folgende \emph{multiplikative Jordanzerlegung} von $\GL(V)$:
  \begin{enumerate}[leftmargin=*, resume]
    \item
      Zeigen Sie, dass es für jedes $s \in \GL(V)$ eindeutige $d, u \in \GL(V)$ gibt, so dass
      \begin{itemize}
        \item
          $s = d \cdot u$,
        \item
          $d$ ist diagonalisierbar und $u$ ist unipotent, und
        \item
          $d$ und $u$ kommutieren.
      \end{itemize}
  \end{enumerate}
\end{question}


\begin{question}[subtitle = Dichtheit der diagonalisierbaren Matrizen]{4}
  Es sei $\|\cdot\|$ eine Norm auf $\Mat_n(\Cbb)$.
  Für alle $\lambda_1, \dotsc, \lambda_n \in \Cbb$ sei
  \[
    \diag(\lambda_1, \dotsc, \lambda_n)
    \coloneqq
    \begin{pmatrix}
      \lambda_1 &         &           \\
                & \ddots  &           \\
                &         & \lambda_n
    \end{pmatrix}
    \in \Mat_n(\Cbb).
  \]
  Es sei
  \[
    \Diagonal_n(\Cbb)
    \coloneqq
    \left\{
      S \diag(\lambda_1, \dotsc, \lambda_n) S^{-1}
    \,\middle|\,
      S \in \GL_n(\Cbb),
      \lambda_1, \dotsc, \lambda \in \Cbb
    \right\}
    \subseteq \Mat_n(\Cbb)
  \]
  die Menge der diagonalisierbaren komplexen $n \times n$-Matrizen.
  Wir zeigen, dass $D_n(\Cbb) \subseteq \Mat_n(\Cbb)$ dicht ist, d.h.\ dass es für jede Matrix $A \in \Mat_n(\Cbb)$ und jedes $\varepsilon > 0$ eine diagonalisierbare Matrix $D \in \Diagonal_n(\Cbb)$ mit $\|A-D\| < \varepsilon$ gibt.
  
  Es sei $S \in \GL_n(\Cbb)$, so dass $S A S^{-1}$ eine obere Dreiecksmatrix
  \[
    S A S^{-1}
    =
    \begin{pmatrix}
      \lambda_1 & *       & \cdots  & *         \\
                & \ddots  & \ddots  & \vdots    \\
                &         & \ddots  & *         \\
                &         &         & \lambda_n
    \end{pmatrix}
  \]
  ist.
  Es seien $z_1, \dotsc, z_n \in \Cbb$ paarweise verschieden und
  \[
    B(t)
    \coloneqq
    A + t S \diag(z_1, \dotsc, z_n) S^{-1}
    \quad
    \text{für alle $t \in \Rbb$}.
  \]
  \begin{enumerate}[leftmargin=*]
    \item
      Zeigen Sie, dass zum Zeitpunkt $t \in \Rbb$ die Zahlen $\mu_1(t), \dotsc, \mu_n(t) \in \Cbb$ mit
      \[
        \mu_i(t) \coloneqq \lambda_i + t z_i
        \quad
        \text{für jedes $i = 1, \dotsc, n$}
      \]
      die Eigenwerte von $B(t)$ ist.
    \item
      Zeigen Sie, dass die Zahlen $\mu_1(t), \dotsc, \mu_n(t)$ für fast alle $t \in \Rbb$ paarweise verschieden sind.
    \item
      Folgern Sie, dass $B(t)$ für fast alle $t \in \Rbb$ diagonalisierbar ist.
    \item
      Folgern Sie, dass es für alle $\varepsilon > 0$ ein $D \in \Diagonal_n(\Cbb)$ mit $\| A - D \| < \varepsilon$ gibt.
  \end{enumerate}
  Wir wollen die Dichtheit von $\Diagonal_n(\Cbb) \subseteq \Mat_n(\Cbb)$ nutzen, um den Satz von Cayley-Hamilton für komplexe Matrizen zu zeigen:
  \begin{enumerate}[leftmargin=*, resume]
    \item
      Zeigen Sie, dass die Abbildung
      \[
        F \colon \Mat_n(\Cbb) \to \Mat_n(\Cbb),
        \quad
        A \mapsto \chi_A(A)
      \]
      stetig ist, wobei $\chi_A(T) \in \Cbb[T]$ das charakteristische Polynom von $A$ ist.
      
      (\emph{Hinweis}: Die Matrixpotenzen $A \mapsto A^k$ sind stetig, und die Koeffizienten des charakteristischen Polynoms $\chi_A$ sind Polynome in den Einträgen von $A$.)
      
    \item
      Zeigen Sie, dass $F(D) = 0$ für jede Diagonalmatrix $D \in \Mat_n(\Cbb)$.
    \item
      Zeigen Sie, dass $P(SAS^{-1}) = S P(A) S^{-1}$ für alle $P \in \Cbb[T]$, $A \in \Mat_n(\Cbb)$ und $S \in \GL_n(\Cbb)$.
      Folgern Sie, dass $F(D) = 0$ für jede Matrix $D \in \Diagonal_n(\Cbb)$.
    \item
      Folgern Sie, dass $F(A) = 0$ für alle $A \in \Mat_n(\Cbb)$.
  \end{enumerate}
\end{question}

