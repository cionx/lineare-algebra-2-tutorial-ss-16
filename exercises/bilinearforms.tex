\section{Bilinearformen}





%%% PRIORITY 1


\begin{question}[subtitle = Berechnung der Signatur quadratischer Formen]{1}
  Bestimmen Sie die Signatur $(n_0, n_+, n_-)$ der folgenden quadratischen Formen auf $\Rbb^n$:
  \begin{enumerate}[leftmargin=*]
    \item
      $
          q(x_1, x_2)
        = 2 x_1^2 - 3 x_2^2 + 2 x_1 x_2
      $
    \item
      $
          q(x_1, x_2)
        = - x_1^2 + x_2^2 + a x_1 x_2
      $
      mit $a \in \Rbb$
    \item
      $
          q(x_1, x_2)
        = x_1^2 + 15 x_2^2 + 6 x_1 x_2
      $
    \item
      $
          q(x_1, x_2)
        = 2 x_1 x_2
      $
    \item
      $
          q(x_1, x_2, x_3)
        = x_1^2 + 2 x_1 x_2 - 2 x_1 x_3 + x_2^2 - 2 x_2 x_3 - x_3^2
      $
    \item
      $
          q(x_1, x_2, x_3, x_4)
        = x_1^2 - 7 x_2^2 - x_3^2 - x_4^2 + 2 x_1 x_2 - 6 x_2 x_3 + 6 x_2 x_4 + 2 x_3 x_4.
      $
  \end{enumerate}
\end{question}
\begin{solution}
  \begin{enumerate}[leftmargin=*]
    \item
      Die Signatur ist $(0,1,1)$.
    \item
      Die Signatur ist $(0,1,1)$.
    \item
      Die Signatur ist $(0,2,0)$.
    \item
      Die Signatur ist $(0,1,1)$
    \item
      Die Signatur ist $(1,1,1)$.
    \item
      Die Signatur ist $(1,2,1)$.
  \end{enumerate}
\end{solution}


\begin{question}[subtitle = You should be able to solve this]{1}
  \begin{enumerate}[leftmargin=*]
    \item
      Bestimmen Sie für die Matrix
      \[
        A \coloneqq
        \begin{pmatrix*}[r]
           2  & -1  & 1 \\
          -1  &  2  & 1 \\
           1  &  1  & 2
        \end{pmatrix*}
        \in \Mat_n(\Rbb)
      \]
      eine orthogonale Matrix $S \in \Orthogonal(3)$, so dass $S^T A S$ eine Diagonalmatrix ist.
    \item
      Bestimmen Sie für die symmetrische Bilinearform $\beta \colon \Rbb^2 \times \Rbb^2 \to \Rbb$ mit
      \[
        \beta\left( \vect{x_1 \\ x_2}, \vect{y_1 \\ y_2} \right)
        \coloneqq
        x_1 y_2 + x_2 y_1
        \quad
        \text{für alle $\vect{x_1 \\ x_2}, \vect{y_1 \\ y_2} \in \Rbb^2$}
      \]
      eine Basis $\mc{B}$ von $\Rbb^2$, so dass $\beta$ bezüglich $\mc{B}$ durch eine Diagonalmatrix mit möglichen Diagonaleinträgen $0, 1, -1$ dargestellt wird.
  \end{enumerate}
\end{question}





%%% PRIORITY 2


\begin{question}[subtitle ={Eine Beschreibung vom Typ $(1,1)$}]{2}
  Es sei $V$ ein zweidimensionaler $\Rbb$-Vektorraum und $\beta \colon V \times V \to \Rbb$ eine symmetrische Bilinearform.
  Zeigen Sie, dass $\beta$ genau dann vom Typ $(1,1)$ ist, wenn es $v_+, v_- \in V$ mit $\beta(v_+, v_+) > 0$ und $\beta(v_-, v_-) < 0$ gibt.
\end{question}


\begin{question}[subtitle = Einschränkung nicht-entarteter Bilinearformen]{2}
  Es sei $V$ ein $K$-Vektorraum und $\beta \colon V \times V \to K$ eine nicht-entartete Bilinearform, d.h.\ für jedes $v \in V$ mit $v \neq 0$ gibt es $w \in V$ mit $\beta(v,w) \neq 0$.
  Entscheiden Sie, ob für jeden Untervektorraum $U \subseteq V$ die Einschränkung $\beta|_{U \times U}$ notwendigerweise ebenfalls nicht-entartet ist; geben Sie gegebenenfalls ein Gegenbeispiel an.
\end{question}


\begin{question}[subtitle = Die Polarisationsformel(n)]{2}
  Es sei $V$ ein $K$-Vektorraum, $\beta \colon V \times V \to K$ eine symmetrische Bilinearform und $q \colon V \to K$, $v \mapsto \beta(v,v)$ die zugehörige quadratische Form.
  \begin{enumerate}[leftmargin=*]
    \item
      Zeigen Sie für den Fall $\ringchar K \neq 2$ mithilfe einer Polarisationsformel, dass $\beta$ durch $q$ bereits eindeutig bestimmt ist.
    \item
      Folgern Sie:
      Ist $\ringchar K \neq 2$, $V \neq 0$ und $\beta$ nicht-entartet, d.h.\ für jedes $v \in V$ mit $v \neq 0$ gibt es ein $w \in V$ mit $\beta(v, w) \neq 0$, so gibt es ein $v \in V$ mit $\beta(v,v) \neq 0$.
    \item
      Zeigen Sie für den Fall $\ringchar K = 2$, dass es unterschiedliche symmetrische Bilinearformen mit gleicher quadratischer Form geben kann, indem Sie ein explizites Beispiel angeben.
  \end{enumerate}
\end{question}


\begin{question}[subtitle = Multiple Choice zu reellen symmetrischen Bilinearformen]{2}
  Entscheiden Sie, welche der folgenden Aussagen für jeden reellen Vektorraum $V$ und jede symmetrische Bilinearform $\bil{\cdot, \cdot} \colon V \times V \to \Rbb$ mit $\beta \neq 0$ gilt.
  Geben Sie gegebenenfalls ein Gegenbeispiel.
  \begin{enumerate}[leftmargin=*]
    \item
      Ist $\bil{v, v} \geq 0$ für alle $v \in V$, so ist $\bil{\cdot, \cdot}$ ein Skalarprodukt.
    \item
      Ist $\mc{B}$ eine Basis von $V$ mit $\bil{b_1, b_2} > 0$ für alle $b_1, b_2 \in \mc{B}$, so ist $\bil{\cdot, \cdot}$ ein Skalarprodukt.
    \item
      Die Teilmenge $\rad \beta \coloneqq \{ v \in V \mid \text{$\bil{v, w} = 0$ für alle $w \in V$} \}$ ist ein Untervektorraum von $V$.
    \item
      Die Teilmengen
      \[
        U_+ \coloneqq \{ v \in V \mid \bil{v, v} \geq 0 \}
        \quad\text{und}\quad
        U_- \coloneqq \{ v \in V \mid \bil{v, v} \leq 0 \}
      \]
      sind Untervektorräume von $V$.
    \item
      Die Teilmenge $U_0 \coloneqq \{ v \in V \mid \bil{v, v} = 0 \}$ ist ein Untervektorraum von $V$.
    \item
      Ist $V$ endlichdimensional, so gilt für jeden Untervektorraum $U \subseteq V$ die Gleichung $\dim V = \dim U + \dim U^\perp$
    \item
      Ist $\beta \neq 0$ und $U \subseteq V$ ein Untervektorraum mit $(U^\perp)^\perp = V$, so ist $U = V$.
    \item
      Für alle Untervektorräume $U_1, U_2 \subseteq V$ gilt $(U_1 + U_2)^\perp = U_1^\perp \cap U_2^\perp$.
    \item
      Für $\rad \beta = \{ v \in V \mid \text{$\bil{v, w} = 0$ für alle $w \in V$} \}$ und jeden Untervektorraum $U \subseteq V$ gilt
      \[
        (U^\perp)^\perp = U + \rad \beta.
      \]
  \end{enumerate}
\end{question}
\begin{solution}
  \begin{enumerate}[leftmargin=*]
    \item
      Falsch, siehe
      $
       \begin{pmatrix}
        0 & 0 \\
        0 & 1
       \end{pmatrix}
      $
      und die Standardbasis.
    \item
      Falsch, siehe
      $
        \begin{pmatrix}
          1 & 2 \\
          2 & 1
        \end{pmatrix}
      $
      und die Standardbasis.
    \item
      Wahr.
    \item
      Falsch, für
      $
        \begin{pmatrix}
          1 & 2 \\
          2 & 1
        \end{pmatrix}
      $
      sind $e_1$ und $e_2$ in $U_+$, aber $e_1 - e_2$ nicht.
    \item
      Nein,für
      $
        \begin{pmatrix}
          0 & 1 \\
          1 & 0
        \end{pmatrix}.
      $
      ist $e_1, e_2 \in U_0$, aber $e_1 + e_2 \notin U_0$.
    \item
      Nein, für
      $
        \begin{pmatrix}
          0 & 0 \\
          0 & 1
        \end{pmatrix}
      $
      und $U = \Ell(e_1)$ ist $U^\perp = \Rbb^2$.
    \item
      Falsch:
      Für
      $
        \begin{pmatrix}
          1 & 0 \\
          0 & 0
        \end{pmatrix}
      $
      und $U \coloneqq \Ell(e_1)$ ist $U^\perp = \Ell(e_2)$ und $(U^\perp)^\perp = \Rbb^2$.
    \item
      Die Aussage gilt.
    \item
      Falsch:
      Nehme ein Skalarprodukt auf einem unendlichdimensionalen Raum, und einen dichten, echten Unterraum.
      Man siehe etwa Übung~\ref{qst: counterexample with integral scalar product} und Übung~\ref{qst: counterexample sequences}.
  \end{enumerate}
\end{solution}


\begin{question}[subtitle = Matrizen als Bilinearformen und lineare Abbildungen]{2}
  Es sei $V$ ein endlichdimensionaler $K$-Vektorraum, $b \colon V \times V \to K$ eine Bilinearform, $\mc{B} = (b_1, \dotsc, b_n)$ eine Basis von $V$, und $\mc{B}^* = (b_1^*, \dotsc, b_n^*)$ die entsprechende duale Basis von $V^*$.
  \begin{enumerate}[leftmargin=*]
    \item
      Zeigen Sie, dass die Abbildung $B \colon V \to V^*$, $v \mapsto b(-, v)$ linear ist.
    \item
      Zeigen Sie die Gleichheit $\Mat_\mc{B}(b) = \Mat_{\mc{B}, \mc{B}^*}(B)$.
      (Beachten Sie, dass auf der linken Seite die darstellende Matrix einer Bilinearform steht, und auf der rechten Seite die darstellende Matrix einer linearen Abbildung.)
  \end{enumerate}
\end{question}





%%% PRIORITY 3


\begin{question}[subtitle = Die duale Abbildung als Adjungiertes]{3}
  Es seien $V$ und $W$ zwei $K$-Vektorräume und $f \colon V \to W$ sei eine lineare Abbildung.
  \begin{enumerate}[leftmargin=*]
    \item
      Geben Sie die Definition der dualen Abbildung $f^* \colon W^* \to V^*$ an, und zeigen Sie ihre Linearität.
    \item
      Zeigen Sie für jeden $K$-Vektorraum $U$, dass die Abbildung
      \[
        \bil{\cdot, \cdot} \colon U \times U^* \to K
        \quad\text{mit}\quad
        \bil{v, \varphi} = \varphi(v)
        \quad\text{für alle $v \in V$, $\varphi \in V^*$}
      \]
      eine Bilinearform ist.
    \item
      Zeigen Sie, dass $\bil{f(v), \psi} = \bil{v, f^*(\psi)}$ für alle $v \in V$ und $\psi \in W^*$.
  \end{enumerate}
\end{question}


\begin{question}[subtitle = Induzierte nicht-entartete Bilinearformen auf Quotienten]{3}
  Es sei $V$ ein $K$-Vektorraum und $\beta \colon V \times V \to K$ eine symmetrische Bilinearform.
  \begin{enumerate}[leftmargin=*]
    \item
      Zeigen Sie, dass
      \[
        \rad(\beta) \coloneqq \{ v \in V \mid \text{$\beta(v, w) = 0$ für alle $w \in V$} \}
      \]
      ein Untervektorraum von $V$ ist.
      (Man bezeichnet $\rad(\beta)$ als das \emph{Radikal} von $\beta$.)
    \item
      Zeigen Sie: $\beta$ induziert eine symmetrische Bilinearform
      \[
        \bar{\beta} \colon (V / \rad(\beta)) \times (V /\rad(\beta)) \to K
        \quad\text{mit}\quad
        \bar{\beta}(\overline{v}, \overline{w})
        \coloneqq
        \beta(v,w)
        \quad
        \text{für alle $v, w \in V$}.
      \]
    \item
      Zeigen Sie, dass $\bar{\beta}$ nicht-entartet ist, d.h.\ dass für das Radikal
      \[
                  \rad(\bar{\beta})
        \coloneqq \{ x \in V / \rad(\beta) \mid \text{$\bar{\beta}(x,y) = 0$ für alle $y \in V / \rad(\beta)$} \}
      \]
      bereits $\rad(\bar{\beta}) = 0$ gilt.
    \item
      Inwiefern gelten die obigen Aussagen, wenn man $\rad(\beta)$ durch $W \coloneqq \{v \in V \mid \beta(v,v) = 0\}$ ersetzt?
  \end{enumerate}
\end{question}


\begin{question}[subtitle = Zerlegung einer Bilinearform in symmetrischen und alternierenden Teil]{3}
   Es sei $V$ ein $K$-Vektorraum und $b \colon V \times V \to K$ eine Bilinearform.
  \begin{enumerate}[leftmargin=*]
    \item
      Zeigen Sie für $\ringchar K \neq 2$, dass es eindeutige Bilinearformen $b_s, b_a \colon V \times V \to K$ gibt, so dass
      \begin{itemize}
        \item
          $b = b_s + b_a$ und
        \item
          $b_s$ ist symmetrisch und $b_a$ ist alternierend.
      \end{itemize}
    \item
      Zeigen Sie durch Angabe eines Gegenbeispiels, dass die Aussage für $\ringchar K = 2$ nicht notwendigerweise gilt.
  \end{enumerate}
  Es sei nun $V$ der reelle Vektorraum der Polynomfunktionen $\Rbb \to \Rbb$.
  \begin{enumerate}[leftmargin=*, resume]
    \item
      Zeigen Sie, dass die Abbildung $b \colon V \times V \to \Rbb$ mit $b(p, q) \coloneqq \int_{-1}^1 p(t) q'(t) \dd{t}$ eine Bilinearform ist.
    \item
      Geben Sie den symmetrischen Anteil $b_s$ in einer Form an, in der kein Integral vorkommt.
%     \item
%       Geben Sie eine Basis von $V_n$ an, bezüglich der $b_s$ durch eine Diagonalmatrix mit Einträgen $0, 1, -1$ beschrieben wird, und bestimmen Sie die Signatur von $b_s$ auf $V_n$.
  \end{enumerate}
\end{question}


\begin{question}[subtitle = Die Traceform]{3}
  \begin{enumerate}[leftmargin=*]
    \item
      Zeigen Sie, dass die Abbildung $\sigma \colon \Mat_n(K) \times \Mat_n(K) \to K$ mit $\sigma(A, B) \coloneqq \tr(AB)$ eine symmetrische Bilinearform ist.
      Man bezeichnet diese als die \emph{Traceform}.
    \item
      Zeigen Sie, dass $\sigma$ in dem Sinne assoziativ ist, dass $\sigma(AB, C) = \sigma(A, BC)$ für alle $A, B, C \in \Mat_n(K)$.
    \item
      Zeigen Sie, dass $\sigma$ auch in dem Sinne assoziativ ist, dass $\sigma([A,B],C) = \sigma(A,[B,C])$ für alle $A, B, C \in \Mat_n(K)$.
    \item
      Zeigen sie, dass $\sigma$ nicht-entartet ist, d.h.\ dass es für jedes $A \in \Mat_n(K)$ mit $A \neq 0$ ein $B \in \Mat_n(K)$ mit $\sigma(A, B) \neq 0$ gibt.
      
      (\emph{Hinweis}:
       Betrachten Sie die Matrizen $E_{ij} \in \Mat_n(K)$.)
  \end{enumerate}
  Es sei nun
  \[
    S_+ \coloneqq \{ A \in \Mat_n(K) \mid A^T = A \}
  \]
  der Untervektorraum der symmetrischen Matrizen und
  \[
    S_- \coloneqq \{ A \in \Mat_n(K) \mid A^T = -A \}
  \]
  der Untervektorraum der schiefsymmetrischen Matrizen.
  Zeigen Sie:
  \begin{enumerate}[leftmargin=*, resume]
    \item
      Ist $\ringchar K \neq 2$, so sind $S_+$ und $S_-$ bezüglich $\sigma$ orthogonal zueinander.
      
      (\emph{Hinweis}:
       Überlegen Sie sich, dass $\sigma(A^T, B^T) = \sigma(A, B)$ für alle $A, B \in \Mat_n(K)$.)
    \item
      Ist $K = \Rbb$, so ist die Einschränkung von $\sigma$ auf $S_+$ positiv definit, und die Einschränkung auf $S_-$ negativ definit.
  \end{enumerate}
\end{question}


\begin{question}[subtitle = Dualität von Basen]{3}
  Ist $\beta \colon V \times W \to K$ eine Bilinearform, so heißen eine Basis $\mc{B} = (v_i)_{i \in I}$ von $V$ und eine Basis $\mc{C} = (w_i)_{i \in I}$ von $W$ \emph{dual bezüglich $\beta$}, falls
  \[
    \beta(v_i, w_j) = \delta_{ij}
    \quad
    \text{für alle $i, j \in I$}.
  \]
  Es sei zunächst $V$ ein $K$-Vektorraum.
  \begin{enumerate}[leftmargin=*]
    \item
      Zeigen Sie, dass die \emph{Evaluation} $e \colon V \times V^* \to K$ mit $e(v, \varphi) = \varphi(e)$ eine $K$-bilineare Abbildung ist.
    \item
      Zeigen Sie: Ist $V$ endlichdimensional, so gibt es zu jeder Basis $\mc{B} = (b_1, \dotsc, b_n)$ von $V$ genau eine Basis $\mc{C}$ von $V^*$, die bezüglich $e$ dual zu $\mc{B}$ ist.
      Woher kennen Sie diese Basis?
  \end{enumerate}
  Von nun an sei $V$ ein endlichdimensionaler euklidischer Vektorraum mit Skalarprodukt $\bil{\cdot, \cdot}$.
  \begin{enumerate}[leftmargin=*, resume]
    \item
      Zeigen Sie, dass die Abbildung $\Phi \colon V \to V^*$, $v \mapsto \bil{-, v}$ ein Isomorphismus ist.
    \item
      Folgern Sie, dass es für jede Basis $\mc{B} = (b_1, \dotsc, b_n)$ von $V$ genau eine Basis $\mc{B}^\circ = (b_1^\circ, \dotsc, b_n^\circ)$ von $V$ gibt, die bezüglich $\bil{\cdot, \cdot}$ dual zu $\mc{B}$ ist.
      
      (\emph{Hinweis}:
       Formulieren Sie die Aussage, dass $\mc{B}^\circ$ dual zu $\mc{B}$ ist, mithilfe von $\Phi$ um.)
    \item
      Zeigen Sie, dass die Abbildung
      \[
            \left\{ \text{geordnete Basen von $V$} \right\}
        \to \left\{ \text{geordnete Basen von $V$} \right\},
        \quad
        \mc{B} \mapsto \mc{B}^\circ
      \]
      eine Involution ist.
    \item
      Unter welchen Namen kennen Sie Basen von $V$, die bezüglich $(-)^\circ$ selbstdual sind, die also $\mc{B}^\circ = \mc{B}$ erfüllen?
  \end{enumerate}
\end{question}


\begin{question}[subtitle = Existenz raumartiger Vektoren]{3}
  Es sei $n \geq 1$ und $V$ ein $(n+1)$-dimensionaler reeller Vektorraum und $\beta \colon V \times V \to \Rbb$ eine symmetrische Bilinearform vom Typ $(n,1)$.
  \begin{enumerate}[leftmargin=*]
    \item
      Es seien $u, v \in V$ zwei linear unabhängige Vektoren mit $\beta(u,u), \beta(v,v) \leq 0$.
      Zeigen Sie, dass es $w \in \Ell(u,v)$ mit $\beta(w, w) > 0$ gibt.
    \item
      Folgern Sie, dass es für jeden zweidimensionalen Untervektorraum $U \subseteq V$ einen Vektor $w \in U$ mit $\beta(w,w) > 0$ gibt.
  \end{enumerate}
\end{question}





%%% PRIORITY 4


\begin{question}[subtitle = Isometriegruppe und Lie-Algebra einer Bilinearform in Koordinaten]{4}
  \label{qst: isometry group and lie algebra of bilinear form coordinate ver}
  Es sei $B \in \Mat_n(\Kbb)$.
  Es seien
  \[
              \Orthogonal(B)
    \coloneqq \{ S \in \GL_n(\Kbb) \mid S^T B S = B \}
  \]
  und
  \[          \gLie(B)
    \coloneqq \{ A \in \Mat_n(\Kbb) \mid A^T B = - B A \}
  \]
  \begin{enumerate}[leftmargin=*]
    \item
      Zeigen Sie, dass $\Orthogonal(B)$ eine Untergruppe von $\GL_n(\Kbb)$ ist.
    \item
      Zeigen Sie, dass $\gLie(B)$ eine Lie-Unteralgebra von $\gl_n(\Kbb)$ ist, dass also $[A_1, A_2] \in \gLie(B)$ für alle $A_1, A_2 \in \gLie(B)$.
    \item
      Zeigen Sie, dass $\exp(A) \in \Orthogonal(B)$ für alle $A \in \gLie(B)$.
      
      (\emph{Hinweis}:
       Zeigen Sie zunächst, dass $\exp(A)^T B = B \exp(-A)$.)
    \item
      Geben Sie für $\Kbb = \Rbb$ eine Matrix $B \in \Mat_n(\Rbb)$ an, so dass $\Orthogonal(B) = \Orthogonal(n)$.
      Was sind dann die Elemente von $\gLie(B)$?
  \end{enumerate}
\end{question}


\begin{question}[subtitle = Isometriegruppe und Lie-Algebra einer Bilinearform ohne Koordinaten]{4}
  Dies ist eine koordinatenfreie Version von Übung~\ref{qst: isometry group and lie algebra of bilinear form coordinate ver}.
  Für einen endlichdimensionalen $\Kbb$-Vektorraum $V$ und eine Bilinearform $\beta \colon V \times V \to \Kbb$ sei
  \[
              \Orthogonal(\beta)
    \coloneqq \{ \phi \in \GL(V) \mid \text{$\beta(\phi(x), \phi(y)) = \beta(x,y)$ für alle $x,y \in V$} \}
  \]
  die Isometriegruppe von $\beta$, und
  \[
              \gLie(\beta)
    \coloneqq \{ f \in \End(V) \mid \text{$\beta(f(x),y) = -\beta(x, f(y))$ für alle $x, y \in V$} \}
  \]
  die assoziierte Lie-Algebra.
  \begin{enumerate}[leftmargin=*]
    \item
      Zeigen Sie, dass $\Orthogonal(\beta)$ eine Untergruppen von $\GL(V)$ ist.
    \item
      Zeigen Sie, dass $\gLie(\beta)$ eine Lie-Unteralgebra von $\gl(V)$ ist, d.h.\ dass $[f,g] \in \gLie(\beta)$ für alle $f, g \in \gLie(\beta)$.
    \item
      Zeigen Sie, dass $\exp(f) \in \Orthogonal(\beta)$ für alle $f \in \gLie(\beta)$.
      
      (\emph{Hinweis}:
       Zeigen Sie zunächst, dass $\beta(\exp(f)(x) ,y) = \beta(x, \exp(-f)(y))$ für alle $x, y \in V$.
       Nutzen Sie hierfür, dass die Bilinearform $\beta$ in beiden Argumenten stetig ist.)
    \item
      Es sei $\Kbb = \Rbb$ und $\bil{\cdot, \cdot}$ ein Skalarprodukt auf $V$.
      Unter welchen Begriffen sind die Elemente aus $G(\bil{\cdot, \cdot})$ und $\gLie(\bil{\cdot, \cdot})$ bekannt?
  \end{enumerate}
\end{question}


\begin{question}[subtitle = Symbolmanipulation und kanonische Abbildungen]{4}
  Für je zwei $K$-Vektorräume $V$ und $W$ sei
  \[
              \Bil(V, W)
    \coloneqq \{b \colon V \times  W \to K \mid \text{$b$ ist bilinear}\}
  \]
  der $K$-Vektorraum der Bilinearformen $V \times W \to K$.
  \begin{enumerate}[leftmargin=*]
    \item
      Zeigen Sie, dass die Flipabbildung $F_{V,W} \colon \Bil(V, W) \to \Bil(W, V)$ mit
      \[
        F_{V,W}(b)(w,v) = b(v,w)
        \quad
        \text{für alle $v \in V$, $w \in W$}
      \]
      ein Isomorphismus von $K$-Vektorräumen ist.
    \item
      Es sei $b \in \Bil(V, W)$ eine Bilinearform.
      Zeigen Sie, dass $b$ ein lineare Abbildung
      \[
        \Phi_{V,W}(b) \colon V \to W^*,
        \quad
        v \mapsto b(v, -)
      \]
      induziert.
      Dabei ist $b(v, -) \colon W \to K$, $w \mapsto b(v,w)$.
  \item
      Zeigen Sie, dass die Abbildung
      \[
        \Phi_{V,W} \colon \Bil(V, W) \to \Hom(V, W^*),
        \quad
        b \mapsto \Phi_{V,W}(b)
      \]
      ein Isomorphismus von $K$-Vektorräumen ist.
    \item
      Konstruieren Sie mithilfe der vorherigen Aufgabenteile einen Isomorphismus von $K$-Vektorräumen $T_{V,W} \colon \Hom(V, W^*) \to \Hom(W, V^*)$, so dass das folgende Diagramm kommutiert:
      \[
        \begin{tikzcd}[row sep = large, column sep = large, ampersand replacement = \&]
                \Bil(V, W)    \arrow{r}{F_{V,W}}
                              \arrow[swap]{d}{\Phi_{V,W}}
            \&  \Bil(W, V)    \arrow{d}{\Phi_{W,V}}
          \\
                \Hom(V, W^*)  \arrow{r}{T_{V,W}}
            \&  \Hom(W, V^*)
        \end{tikzcd}
      \]
  \end{enumerate}
  Wir betrachten nun den Fall $W = V^*$.
  \begin{enumerate}[resume, leftmargin=*]
    \item
      Zeigen Sie, dass die Evaluation $e \colon V \times V^* \to K$, $(v, \varphi) \mapsto \varphi(v)$ eine Bilinearform ist.
   \item
      Nach den vorherigen Aufgabenteilen entspricht die Bilinearform $e \in \Bil(V, V^*)$ einer linearen Abbildung $V \to V^{**}$, sowie einer linearen Abbildung $V^* \to V^*$.
      Bestimmen Sie diese Abbildungen.
    \item
      Woher kennen Sie diese Abbildung?
  \end{enumerate}
\end{question}