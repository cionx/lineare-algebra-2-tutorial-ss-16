\documentclass[a4paper,10pt]{article}
%\documentclass[a4paper,10pt]{scrartcl}

\usepackage{../generalstyle}
\usepackage{specificstyle}

\setromanfont[Mapping=tex-text]{Linux Libertine O}
% \setsansfont[Mapping=tex-text]{DejaVu Sans}
% \setmonofont[Mapping=tex-text]{DejaVu Sans Mono}

\title{Übungen zu Lineare Algebra II}
\author{Jendrik Stelzner}
\date{\today}

\begin{document}
\maketitle


% COMPLEXIFICATION









\begin{question}
  Es sei $V$ ein reeller Vektorraum und $(U_i)_{i \in I}$ eine Familie von Untervektorräumen $U_i \subseteq V$.
  Zeigen Sie, dass genau dann $V = \bigoplus_{i \in I} U_i$, wenn $V_\Cbb = \bigoplus_{i \in I} (U_i)_{\Cbb}$.
\end{question}


\begin{question}
  Es seien $V$ und $W$ zwei reelle Vektorräume, und $f \colon V \to W$ sei $\Rbb$-linear.
  \begin{enumerate}
    \item
      Zeigen Sie, dass $\ker (f_\Cbb) = (\ker f)_\Cbb$.
    \item
      Folgern Sie, dass $f_\Cbb$ genau dann injektiv ist, wenn $f$ injektiv ist.
    \item
      Folgern Sie ferner, dass $(V_\Cbb)_\lambda(f_\Cbb) = V_\lambda(f)_\Cbb$ für jedes $\lambda \in \Rbb$.
    \item
      Zeigen Sie, dass $\im (f_\Cbb) = (\im f)_\Cbb$.
    \item
      Folgern Sie, dass $f_\Cbb$ genau dann surjektiv ist, wenn $f$ surjektiv ist.
  \end{enumerate}
\end{question}


\begin{question}
  Es sei $V$ ein reeller Vektorraum und $f \colon V \to V$ ein Endomorphismus.
  Zeigen Sie, dass $f$ genau dann diagonalisierbar ist, wenn $f_\Cbb$ diagonalisierbar mit reellen Eigenwerten ist.
\end{question}










% SCALAR PRODUCTS AND BILINEAR FORMS


\begin{question}
  Es seien $V$ und $W$ zwei endlichdimensionale euklidische Vektorräume.
  Ferner sei $f \colon V \to W$ eine $\Rbb$-lineare Abbildung.
  \begin{enumerate}[leftmargin=*]
    \item
      Zeigen Sie, dass die Abbildung
      \[
        \Phi_V \colon V \to V^*,
        \quad
        v \mapsto \bil{-, v}
      \]
      ein $\Rbb$-linearer Isomorphismus ist.
    \item
      Geben Sie die Definition der dualen Abbildung $f^* \colon W^* \to V^*$ an.
      Zeigen Sie, dass $f^*$ $\Rbb$-linear ist.
    \item
      Zeigen Sie, dass die Abbildung $g \coloneqq \Phi_V^{-1} \circ f^* \circ \Phi_W$ $\Rbb$-linear ist, und dass
      \[
        \bil{f(v), w} = \bil{v, g(w)}
        \quad
        \text{für alle $v \in V$, $w \in W$}.
      \]
    \item
      Inwiefern ändern sich die obigen Resultate, wenn $V$ und $W$ unitäre Vektorräume sind?
  \end{enumerate}
\end{question}


\begin{question}
  Für je zwei $K$-Vektorräume $V$ und $W$ sei
  \[
              \Bil(V, W)
    \coloneqq \{b \colon V \times  W \to K \mid \text{$b$ ist bilinear}\}
  \]
  der Raum der Bilinearformen $V \times W \to K$.
  \begin{enumerate}
    \item
      Zeigen Sie, dass die Flipabbildung
      \[
        F \colon \Bil(V, W) \to \Bil(W, V),
        \quad
        b \mapsto F(b)
        \quad\text{mit}\quad
        F(b)(w,v) = b(v,w)
      \]
      ein Isomorphismus von $K$-Vektorräumen ist.
    \item
      Es sei $b \in \Bil(V, W)$ eine Bilinearform.
      Zeigen Sie, dass $b$ ein lineare Abbildung
      \[
        \Phi_{V,W}(b) \colon V \to W^*,
        \quad
        v \mapsto b(v, -)
      \]
      induziert.
      Dabei ist
      \[
        b(v, -) \colon W \to K,
        \quad
        w \mapsto b(v,w).
      \]
    \item
      Zeigen Sie, dass die Abbildung
      \[
        \Phi_{V,W} \colon \Bil(V, W) \to \Hom(V, W^*),
        \quad
        b \mapsto \Phi_{V,W}(b)
      \]
      ein Isomorphismus von $K$-Vektorräumen ist.
    \item
      Geben Sie mithilfe der vorherigen Aufgabenteile explizit einen Isomorphismus $\Hom(V, W^*) \to \Hom(W, V^*)$ an.
  \end{enumerate}
  Wir betrachten nun den Fall $W = V^*$.
  \begin{enumerate}[resume]
    \item
      Zeigen Sie, dass die Evaluation
      \[
        e \colon V \times V^* \to K,
        \quad
        (v, \varphi) \mapsto \varphi(v)
      \]
      eine Bilinearform ist.
   \item
      Nach den vorherigen Aufgabenteilen entspricht die Bilinearform $e$ einer linearen Abbildung $V \to V^{**}$, sowie einer linearen Abbildung $V^* \to V^*$.
      Bestimmen Sie diese Abbildungen.
    \item
      Woher kennen Sie diese Abbildung?
  \end{enumerate}
\end{question}


\begin{question}
  Es seien $V$ und $W$ zwei $K$-Vektorräume und $f \colon V \to W$ eine lineare Abbildung.
  \begin{enumerate}
    \item
      Geben Sie die Definition der dualen Abbildung $f^* \colon W^* \to V^*$ an, und zeigen sie ihre Linearität.
    \item
      Zeigen Sie für jeden $K$-Vektorraum $U$, dass die Abbildung
      \[
        \bil{\cdot, \cdot} \colon U \times U^* \to K,
        \quad\text{mit}\quad
        \bil{v, \varphi} = \varphi(v)
      \]
      eine Bilinearform ist.
    \item
      Zeigen Sie, dass
      \[
        \bil{f(v), \psi} = \bil{v, f^*(\psi)}
        \quad
        \text{für alle $v \in V$, $\psi \in W^*$}.
      \]
  \end{enumerate}
\end{question}


\begin{question}
  \begin{enumerate}[leftmargin=*]
    \item
      Zeigen Sie, dass die Abbildung
      \[
        \sigma \colon \Mat_n(K) \times \Mat_n(K) \to K
        \quad\text{mit}\quad
        \sigma(A, B) \coloneqq \tr(AB)
      \]
      eine symmetrische Bilinearform ist.
    \item
      Zeigen Sie, dass $\sigma$ in dem Sinne assoziativ ist, dass
      \[
        \sigma(AB, C) = \sigma(A, BC)
        \quad
        \text{für alle $A, B, C \in \Mat_n(K)$}.
      \]
  \end{enumerate}
\end{question}







\end{document}
