%\documentclass[a4paper,10pt]{article}
\documentclass[a4paper,10pt]{scrartcl}

\usepackage{../generalstyle}
\usepackage{specificstyle}

\setromanfont[Mapping=tex-text]{Linux Libertine O}
% \setsansfont[Mapping=tex-text]{DejaVu Sans}
% \setmonofont[Mapping=tex-text]{DejaVu Sans Mono}

\title{Übungen zu Lineare Algebra II}
\author{Jendrik Stelzner}
\date{\today}

\begin{document}
\maketitle



% GENERAL STUFF


\begin{question}
  Ein Endomorphismus $f \colon V \to V$ eines $K$-Vektorraums $V$ heißt \emph{lokal nilpotent}, falls es für jedes $v \in V$ ein $n \in \Nbb$ mit $f^n(v) = 0$ gibt.
  \begin{enumerate}[leftmargin=*]
    \item
      Zeigen Sie, dass jeder nilpotente Endomorphismus auch lokal nilpotent ist.
    \item
      Zeige Sie, dass $0$ der einzige mögliche Eigenwert eines lokal nilpotenten Endomorphismus ist.
    \item
      Geben Sie ein Beispiel für einen Vektorraum $V$ und einen Endomorphismus $f \colon V \to V$ an, so dass $f$ zwar lokal nilpotent, nicht aber nilpotent ist.
    \item
      Zeigen Sie, dass jeder lokal nilpotente Endomorphismus eines endlichdimensionalen Vektorraums bereits nilpotent ist.
  \end{enumerate}
\end{question}


\begin{question}
  Es sei $V$ ein $K$-Vektorraum und $f \colon V \to V$ ein Endomorphismus.
  Zeigen Sie:
  \begin{enumerate}[leftmargin=*]
    \item
      Ist $f^2 = f$, so ist $V = \im f \oplus \ker f$, und es gilt $\im f = V_1(f)$ und $\ker f = V_0(f)$.
    \item
      Ist $f^2 = \id_V$ und $\ringchar K \neq 2$, so ist $f$ diagonalisierbar mit (möglichen) Eigenwerten $1$ und $-1$.
    \item
      Sind $\lambda, \mu \in K$ mit $\lambda \neq \mu$ und $(f-\lambda)(f-\mu) = 0$, so ist $f$ diagonalisierbar mit (möglichen) Eigenwerten $\lambda$ und $\mu$.
      Inwiefern sind die vorherigen beiden Aufgabenteile Sonderfälle hiervon?
  \end{enumerate}
\end{question}


\begin{question}
  Zeigen Sie im folgenden jeweils, dass der Vektorraum $V$ die direkte Summe der Untervektorräume $U_1$ und $U_2$ ist, indem sie einen idempotenten Endomorphisus $e \colon V \to V$ mit $U_1 = \im e$ und $U_2 = \ker e$ angeben.
  \begin{enumerate}[leftmargin=*]
    \item
      Es sei $\ringchar K \neq 2$, $V \coloneqq \Mat_n(K)$ der Vektorraum der ($n \times n$)-Matrizen über $K$,
      \[
        U_1 \coloneqq \{ A \in \Mat_n(K) \mid A^T = A \}
      \]
      der Untervektorraum der symmetrischen Matrizen, und
      \[
        U_2 \coloneqq \{ A \in \Mat_n(K) \mid A^T = -A \}
      \]
      der Untervektorraum der schiefsymmetrischen Matrizen.
    \item
      Es sei $V \coloneqq \{ f \mid f \colon \Rbb \to \Rbb \}$ der Vektorraum der reellwertigen Folgen auf $\Rbb$, sowie
      \[
        U_1 \coloneqq \{ f \in V \mid \text{$f(x) = f(-x)$ für alle $x \in \Rbb$} \}
      \]
      der Untervektorraum der geraden Funktionen und
      \[
        U_2 \coloneqq \{ f \in V \mid \text{$f(x) = f(-x)$ für alle $x \in \Rbb$} \}
      \]
      der Untervektorraum der ungeraden Funktionen.
    \item
      Die Ebene $V = \Rbb^2$ und als Untervektorräume die beiden Geraden
      \[
        U_1 \coloneqq \Rbb \vect{1 \\  1}
        \quad\text{und}\quad
        U_2 \coloneqq \Rbb \vect{1 \\ -1}.
      \]
    \item
      Für $\ringchar K \neq 2$ und einen Vektorraum $W$ sei
      \[
                  V
        \coloneqq \{b \colon W \times W \to K \mid \text{$b$ ist bilinear}\}
      \]
      der Vektorraum der Bilinearformen auf $W$.
      Es sei
      \[
                  U_1
        \coloneqq \{ s \in V \mid \text{$s$ ist symmetrisch} \}
      \]
      der Untervektorraum der symmetrischen Bilinearformen, und
      \[
                  U_2
        \coloneqq \{ a \in V \mid \text{$a$ ist alternierend} \}
      \]
      der Untervektorraum der alternierenden Bilinearformen.
    \item
      Der Vektorraum $V \coloneqq \Cbb(I, \Rbb)$ der stetigen reellwertigen Funktionen auf dem Einheitsintervall $I = [0,1]$ mit den Untervektorräumen
      \[
        U_1 \coloneqq \{f \in V \mid f(0) = 0\}
        U_2 \coloneqq \{f \in V \mid \text{$f$ ist konstant}\}.
      \]
    \item
      Es sei erneut $V \coloneqq \Cbb(I, \Rbb)$ der Vektorraum der stetigen reellwertigen Funktionen auf dem Einheitsintervall $I = [0,1]$.
     Es sei nun
      \[
        U_1 \coloneqq \{ f \in V \mid f(0) = f(1) = 0 \}
      \]
      der Untervektorraum der Funktion mit Nullrandwerten, und
      \[
        U_2 \coloneqq \{ h_{x,y} \mid x, y \in \Rbb \}
      \]
      der Untervektorraum der affin-linearen Funktionen, wobei
      \[
        h_{x,y} \colon I \to \Rbb,
        \quad
        t \mapsto (1-t)x + ty = x + t(y-x)
      \]
      die affin lineare Funktion mit den Randwerten $x$ und $y$ ist.
      (\emph{Hinweis}: Es hilft, sich diese Zerlegung anschaulich vorzustellen.)
    \item
      Für einen Körper mit $\ringchar K \nmid n$ die Zerlegung von $V \coloneqq \Mat_n(K)$ in die Untervektorräume
      \[
        U_1 \coloneqq \slLie(K) = \{ A \in \Mat_n(K) \mid \tr A  = 0 \}
        \quad\text{und}\quad
        U_2 \coloneqq K I = \{ \lambda I \mid \lambda \in K \}
      \]
      der spurlosen Matrizen und der Skalarmatrizen.
    \item
      Es sei $V$ ein diagonalisierbarer Endomorphismus mit zwei verschiedenen Eigenwerten $\lambda, \mu \in K$ und $U_1 \coloneqq V_\lambda(f)$ und $U_2 \coloneqq V_\mu(f)$.
  \end{enumerate}
\end{question}


\begin{question}
  Es seien $V$ ein $K$-Vektorraum.
  \begin{enumerate}[leftmargin=*]
    \item
      Zeigen Sie, dass sich durch jeden idempotenten Endomorphismus $e \colon V \to V$ (d.h.\ $e^2 = e$) eine Zerlegung
      \[
        V = \im e \oplus \ker e
      \]
      ergibt, wobei
      \[
        e(v + w) = v
        \quad
        \text{für alle $v \in \im e$ und $w \in \ker e$}.
      \]
    \item
      Es sei $(U_1, U_2)$ ein Paar von Untervektorräumen $U_1, U_2 \subseteq V$ mit $V = U_1 \oplus U_2$.
      Zeigen Sie, dass es einen eindeutigen Endomorphismus $p_{U_1, U_2} \colon V \to V$ gibt, so dass
      \[
          p_{U_1, U_2}(u_1 + u_2)
        = u_1
        \quad
        \text{für alle $u_1 \in U_1$ und $u_2 \in U_2$}.
      \]
    \item
      Zeigen Sie, dass die obigen Konstruktionen eine Bijektion
      \begin{align*}
        \left\{
          (U_1, U_2)a
          \,\middle|\,
          \begin{tabular}{c}
            $U_1, U_2 \subseteq V$ sind \\
            Untervektorräume            \\
            mit $V = U_1 \oplus U_2$
          \end{tabular}
        \right\}
        &\longleftrightarrow
        \{ e \in \End(V) \mid \text{$e$ ist idempotent} \},
      \\
        (U_1, U_2) &\longmapsto p_{U_1, U_2}
      \\
        (\im e, \ker e) &\longmapsfrom e
      \end{align*}
      ergeben.
    \item
      Auf der linken Seite der obigen Bijektion gibt es eine Involution $(U_1, U_2) \mapsto (U_2, U_1)$.
      Geben Sie an, wie die entsprechende Involution auf der rechten Seite aussieht.
  \end{enumerate}
\end{question}


% TODO: 1 + (lokal nilpotent) ist invertierbar

% TODO: Bijektion zwischen Idempotenten und Involutiven mit gleicher Zerlegung

% TODO: Matrixexponentialszeugs, allgemein Lie-Algebra -> Exponential

% TODO: X ist diagonalisierbar ==> ad(X) ist diagonalisierbar


\begin{question}
  Es sei $V$ ein $K$-Vektorraum und $f \colon V \to V$ ein Endomorphismus.
  \begin{enumerate}[leftmargin=*]
    \item
      Zeigen Sie für $\ringchar K \neq 2$, dass $f$ diagonalisierbar mit möglichen Eigenwerten $1$ und $-1$ ist.
    \item
      Zeigen Sie, dass die Aussage für $\ringchar K = 2$ nicht mehr gelten muss.
  \end{enumerate}
\end{question}


\begin{question}
  Es seien $V$ und $W$ zwei $K$-Vektorräume, und $f \colon V \to W$ eine lineare Abbildung, die ein Rechtsinverses $g \colon W \to V$ besitzt.
  Zeigen Sie, dass
  \[
    V = \ker f \oplus \im g
  \]
  auf die folgenden beiden Weisen:
  \begin{enumerate}[leftmargin=*]
    \item
      Durch explizites Nachrechnen, dass $V = \ker f + \im g$ und $\ker f \cap \im g = 0$.
    \item
      Mithilfe des Endomorphismus $gf \colon V \to V$.
  \end{enumerate}
\end{question}


\begin{question}
  Es sei $V$ ein $K$-Vektorraum.
  Zeigen Sie, dass die folgenden Aussagen allgemein gelten, oder geben Sie jeweils ein Gegenbeispiel an.
  \begin{enumerate}[leftmargin=*]
    \item
      Ist $V = V_1 \oplus V_2$ für Untervektorräume $V_1, V_2 \subseteq V$, so gilt für jeden Untervektorraum $U \subseteq V$ die Zerlegung
      \[
        U = (U \cap V_1) \oplus (U \cap V_2).
      \]
    \item
      Ist $V = U_1 \oplus W_1 = U_2 \oplus W_2$ mit $W_1 \supseteq W_2$, so ist
      \[
        W_1 = (U_2 \cap W_1) \oplus W_2.
      \]
    \item
      Ist $f \colon V \to V$ ein Endomorphismus und $U \subseteq V$ ein $f$-invarinter Untervektorraum, so gibt es einen $f$-invarianten Untervektorraum $W \subseteq V$ mit $V = U \oplus W$.
    \item
      Für alle Untervektorräume $W, U_1, U_2 \subseteq V$ mit $U_1 \subseteq U_2$ gilt
      \[
        (U_1 + W) \cap U_2 =  U_1 + (W \cap U_2).
      \]
    \item
      Ist $\mc{E} \subseteq V$ ein Erzeugendensystem und $U \subseteq V$ ein Untervektorraum, so ist die Einschränkung $\mc{E}' \coloneqq \mc{E} \cap U$ ein Erzeugendensystem von $U$.
    \item
      Ist $(U_i)_{i \in I}$ eine Famlie von Untervektorräumen $U_i \subseteq V$ mit $V = \sum_{i \in I} U_i$ und $U_i \cap U_j = 0$ für $i \neq j$, so ist $V = \bigoplus_{i \in I} U_i$.
  \end{enumerate}
\end{question}


\begin{question}
  Ein Endomorphismus $f \colon V \to V$ eines $K$-Vektorraums $V$ heißt \emph{algebraisch (über $K$)}, falls es ein Polynom $P \in K[T]$ mit $P \neq 0$ gibt, so dass $P(f) = 0$ gilt.
  \begin{enumerate}[leftmargin=*]
    \item
      Zeigen Sie, dass jeder Endomorphismus eines endlichdimensionalen Vektorraums algebraisch ist.
    \item
      Geben Sie ein Beispiel für einen $K$-Vektorraum $V$ und einen Endomorphismus $f \colon V \to V$ an, der nicht algebraisch ist.
  \end{enumerate}
\end{question}










% QUOTIENTS


\begin{question}
  Es sei $V$ ein Vektorraum und $f \colon V \to V$ ein Endomorphismus.
  Es sei $(U_i)_{i \in I}$ eine Familie von $f$-invarianten Untervektorräumen, und $U \subseteq V$ ein $f$-invarianter Untervektorraum.
  Zeigen Sie:
  \begin{enumerate}[leftmargin=*]
    \item
      Auch der Schnitt $\bigcap_{i \in I} U_i$ ist $f$-invariant.
    \item
      Auch die Summe $\sum_{i \in i} U_i$ ist $f$-invariant.
    \item
      $f$ induziert eine lineare Abbildung
      \[
        \bar{f} \colon V/U \to V/U,
        \quad
        [x] \mapsto [f(x)].
      \]
  \end{enumerate}
\end{question}


\begin{question}
  Es sei $V$ ein $K$-Vektorraum und $U \subseteq V$ ein $K$-Untervektorraum.
  Konstruieren Sie für den Annihilator
  \[
      U^\circ
    = \{ \varphi \in V^* \mid \varphi|_U = 0 \}
  \]
  einen Isomorphismus $F \colon U^\circ \to (V/U)^*$.
\end{question}


\begin{question}
  Es sei $V$ ein $K$-Vektorraum mit zwei Untervektorräumen $U_1, U_2 \subseteq V$.
  Zeigen Sie die folgenden beiden Isomorphiesätze:
  \begin{enumerate}[leftmargin=*]
    \item
      Die Inklusion $U_1 \to U_1 + U_2$, $x \mapsto x$ induziert einen isomorphismus
      \[
        U_1 / (U_1 \cap U_2) \to (U_1 + U_2) / U_2,
        \quad
        [x] \mapsto [x]
        \quad
        \text{für alle $x \in V$}.
      \]
    \item
      Ist $U_1 \subseteq U_2$, so ist $U_2 / U_1$ ein Untervektorraum von $V / U_1$, und die Abbildung
      \[
        (V / U_1) / (U_2 / U_1) \to V / U_2,
        \quad
        [[x]] \mapsto [x]
        \quad
        \text{für alle $x \in V$}.
      \]
      ist ein wohldefinierter Isomorphismus.
  \end{enumerate}
\end{question}



\begin{question}
  Es sei $K$ ein algebraisch abgeschlossener Körper und $f \colon V \to V$ ein Endomorphismus eines endlichdimensionalen $K$-Vektorraums $V$.
  Zeigen Sie, dass die folgenden beiden Aussagen äquivalent sind:
  \begin{enumerate}
    \item
      $f$ ist diagonalisierbar.
    \item
      Für jeden $f$-invarianten Untervektorraum $U \subseteq V$ gibt es einen $f$-invarianten Untervektorraum $W \subseteq V$ mit $V = U \oplus W$.
  \end{enumerate}
\end{question}


\begin{question}
  Es sei $V$ ein $K$-Vektorraum und $f \colon V \to V$ ein Endomorphismus.
  Für alle $k \in \Nbb$ sei
  \[
    R_k \coloneqq \im f^k
    \quad\text{und}\quad
    N_k \coloneqq \ker f^k.
  \]
  \begin{enumerate}
    \item
      Zeigen Sie, dass $R_0 = V$, und dass $R_i \supseteq R_{i+1}$ für alle $i \in \Nbb$.
      Es gibt also eine absteigende Kette
      \[
        V = R_0 \supseteq R_1 \supseteq R_2 \supseteq R_3 \supseteq R_4 \supseteq \dotsb
      \]
      von Untervektorräumen.
    \item
      Zeigen Sie für $i \in \Nbb$ mit $R_i = R_{i+1}$, dass auch $R_{i+1} = R_{i+2}$.
    \item
      Folgern Sie:
      Gilt in der obigen absteigenden Kette einmal Gleichheit, also $R_i = R_{i+1}$, so stabilisiert die Kette bereits, d.h.\ es gilt $R_j = R_i$ für alle $j \geq i$.
    \item
      Zeigen Sie, dass $N_0 = 0$, und dass $N_i \subseteq N_{i+1}$ für alle $i \in \Nbb$.
      Es gibt also eine aufsteigende Kette
      \[
        0 = N_0 \subseteq N_1 \subseteq N_2 \subseteq N_3 \subseteq N_4 \subseteq \dotsb
      \]
      von Untervektorräumen.
    \item
      Zeigen Sie, dass für $i \in \Nbb$ mit $N_i = N_{i+1}$ auch $N_{i+1} = N_{i+2}$.
    \item
      Folgern Sie:
      Gilt in der obigen aufsteigende Kette einmal Gleichheit, also $N_i = N_{i+1}$, so stabilisiert die Kette bereits, d.h.\ es gilt $N_j = N_i$ für alle $j \geq i$.
  \end{enumerate}

\end{question}











% QUOTIENTS


\begin{question}
  Es sei $V$ ein $K$-Vektorraum und $U \subseteq V$ ein Untervektorraum.
  Es sei $\pi \colon V \to V/U$, $v \mapsto [v]$ die kanonische Projektion.
  \begin{enumerate}[leftmargin=*]
    \item
      Es sei $(b_i)_{i \in I}$ eine Basis von $V$, und für eine Teilmenge $J \subseteq I$ sei $(b_j)_{j \in J}$ eine Basis von $U$.
      Zeigen Sie, dass $([b_i])_{i \in I \smallsetminus J}$ eine Basis von $V/U$ ist.
    \item
      Es sei $(b_i)_{i \in I}$ eine Basis von $U$ und $(c_j)_{j \in J}$ eine Basis von $V/U$, wobei $I \cap J = \emptyset$.
      Für $j \in J$ sei $b_j \in V$ mit $\pi(b_j) = c_j$.
      Zeigen Sie, dass $(b_l)_{l \in L}$ für $L \coloneqq I \cap J$ ist eine Basis von $V$ ist.
  \end{enumerate}
\end{question}


\begin{question}
  Es seien $V$ und $W$ zwei $K$-Vektorräume und $f \colon V \to W$ eine lineare Abbildung.
  \begin{enumerate}
    \item
      Es sei $U \subseteq V$ ein Untervektorraum mit $f|_U = 0$.
      Zeigen Sie, dass $V$ eine lineare Abbildung
      \[
        \bar{f} \colon V/U \to W, \quad [v] \mapsto f(u)
      \]
      induziert.
    \item
      Zeigen Sie, dass $\im \bar{f} = \im f$.
      Folgern Sie, dass $\bar{f}$ genau dann surjektiv ist, wenn $f$ surjektiv ist.
    \item
      Zeigen Sie, dass $U \subseteq \ker f$, und dass $\ker \bar{f} = (\ker f)/U$.
      Folgern Sie, dass $\bar{f}$ genau dann injektiv ist, wenn bereits $U = \ker f$ gilt.
    \item
      Folgern Sie, dass $f$ einen Isomorphismus
      \[
        V/(\ker f) \to \im f,
        \quad
        [v] \mapsto f(v)
      \]
      induziert.
  \end{enumerate}
\end{question}


\begin{question}
  Es sei $V$ ein $K$-Vektorraum mit Erzeugendensystem $E \subseteq V$.
  Es sei $W$ ein $K$-Vektorraum mit Basis $\{b_e\}_{e \in E} \subseteq W$.
  Konstruieren Sie einen Isomorphismus $V/U \to W$ für einen passenden Untervektorraum $U \subseteq W$.
\end{question}


\begin{question}
  Es sei $V$ ein $K$-Vektorraum und $U \subseteq V$ ein Untervektorraum.
  Es seien $f \colon V \to V$ ein Endomorphismus, so dass $U$ invariant unter $f$ ist (d.h.\ es ist $f(U) \subseteq U$).
  \begin{enumerate}
    \item
      Zeigen Sie, dass $f$ einen Endomorphismus
      \[
        \bar{f} \colon V/U \to V/U,
        \quad
        [v] \mapsto [f(v)]
      \]
    induziert.
  \end{enumerate}
  Es sei nun $g \colon V \to V$ ein weiterer Endomorphismus, so dass $U$ invariant unter $g$ ist, und es sei $\bar{g} \colon V/U \to V/U$ der induzierte Endomorphismus.
  \begin{enumerate}[resume]
    \item
      Es seien $f|_U = g|_U$ und $\bar{f} = \bar{g}$.
      Beweisen oder widerlegen Sie, dass bereits $f = g$ gelten muss.
  \end{enumerate}
\end{question}













% COMPLEXIFICATION


\begin{question}
  Es sei $V$ ein $\Rbb$-Vektorraum und $W$ ein $\Cbb$-Vektorraum.
  Zeigen Sie:
  \begin{enumerate}[leftmargin=*]
    \item
      Für jede $\Rbb$-lineare Abbildung $f \colon V \to W$ gibt genau eine $\Cbb$-lineare Abbildung $f_\Cbb \colon V_\Cbb \to W_\Cbb$, die das folgende Diagram kommutieren lässt:
      \[
        \begin{tikzcd}[row sep = large, column sep = large, ampersand replacement=\&]
                V       \arrow{d}[swap]{\iota}
                        \arrow{rd}{f}
            \&  {}
          \\
                V_\Cbb  \arrow{r}[swap]{f_\Cbb}
            \&  W_\Cbb
        \end{tikzcd}
      \]
    \item
      Für je zwei $\Cbb$-lineare Abbildungen $g_1, g_2 \colon V_\Cbb \to W$ die Äquivalenz
      \[
        g_1 = g_2
        \iff
        g_1 \circ \iota = g_2 \circ \iota
      \]
      gilt.
    \item
      Für jeden $\Cbb$-Vektorraum $W'$ gilt für jede $\Rbb$-lineare Abbildung $f \colon V \to W$ und jede $\Cbb$-lineare Abbildung $g \colon W \to W'$ die Gleichheit
      \[
        (g \circ f)_\Cbb = g \circ f_\Cbb.
      \]
  \end{enumerate}
\end{question}


\begin{question}
  \begin{enumerate}[leftmargin=*]
    \item
      Zeigen Sie, dass für jedes $\Rbb$-Vektorraum $V$ und $\Cbb$-Vektorraum $W$ die Abbildung
      \[
        \Phi_{V,W} \colon \Hom_\Rbb(V, W) \to \Hom_\Cbb(V_\Cbb, W),
        \quad
        f \mapsto f_\Cbb
      \]
      ein Isomorphismus von $\Rbb$-Vektorräumen ist.
      Geben Sie auch $\Phi_{V,W}^{-1}$ an.
    \item
      Es seien vier $K$-Vektorräume $V, V', W, W'$ und zwei $K$-lineare Abbildungen $f \colon V \to V'$ und $g \colon W \to W'$ gegeben.
      Zeigen Sie, dass die beidseitige Komposition
      \[
        g \circ - \circ f
        \colon
        \Hom_K(V, W) \to \Hom_K(V', W'),
        \quad
        h \mapsto g \circ h \circ f
      \]
      eine $K$-lineare Abbildung ist.
    \item
      Zeigen Sie, dass die Isomorphismen $\Phi_{V,W}$ in dem folgenden Sinne \emph{natürlich} sind:
      Es seien $V$ und $V'$ zwei $\Rbb$-Vektorräume und es sei $f \colon V \to V'$ eine $\Rbb$-lineare Abbildung.
      Es seien $W$ und $W'$ zwei $\Cbb$-Vektorräume und es sei $g \colon W \to W'$ eine $\Cbb$-lineare Abbildung.
      Dann kommutiert das folgende Diagram von $\Rbb$-Vektorräumen und $\Rbb$-linearen Abbildungen:
      \[
        \begin{tikzcd}[row sep = large, column sep = large, ampersand replacement=\&]
                \Hom_\Rbb(V, W)         \arrow[swap]{d}{g \circ - \circ f}
                                        \arrow{r}{\Phi_{V,W}}
            \&  \Hom_\Cbb(V_\Cbb, W)    \arrow{d}{g \circ - \circ f_\Cbb}
          \\
                \Hom_\Rbb(V', W')       \arrow{r}{\Phi_{V',W'}}
            \&  \Hom_\Cbb(V'_\Cbb, W')
        \end{tikzcd}
      \]

  \end{enumerate}
\end{question}


\begin{question}
  Zeigen Sie, dass die $\Rbb$-lineare Inklusion $\Rbb \to \Cbb$, $x \mapsto x$ einen Isomorphismus $\Rbb_\Cbb \to \Cbb$ von $\Cbb$-Vektorräumen induziert.
\end{question}


\begin{question}
  Es seien $V$ und $W$ zwei $\Rbb$-Vektorräume.
  Zeigen Sie, dass die $\Rbb$-lineare Abbildung
  \[
    \varphi \colon \Hom_\Rbb(V, W) \to \Hom_\Cbb(V_\Cbb, W_\Cbb),
    \quad
    f \mapsto f_\Cbb
  \]
  einen Isomorphismus von $\Cbb$-Vektorräumen
  \[
    \Phi \colon \Hom_\Rbb(V, W)_\Cbb \to \Hom_\Cbb(V_\Cbb, W_\Cbb)
  \]
  induziert.
  (\emph{Hinweis}: Beachten Sie, dass $V$ und $W$ nicht notwendigerweise endlichdimensional sind.)
\end{question}


\begin{question}
  Es sei $V$ ein $\Rbb$-Vektorraum.
  Konstruieren Sie einen Isomorphismus $(V^*)_\Cbb \to (V_\Cbb)^*$.
  (\emph{Hinweis}: Beachten Sie, dass $V$ ist nicht notwendigerweise endlichdimensional ist.)
\end{question}


\begin{question}
  Es sei $V$ ein reeller Vektorraum und $(U_i)_{i \in I}$ eine Familie von Untervektorräumen $U_i \subseteq V$.
  Zeigen Sie:
  \begin{enumerate}[leftmargin=*]
    \item
      Es gilt
      \[
          \left( \bigcap_{i \in I} U_i \right)_\Cbb
        = \bigcap_{i \in I} (U_i)_\Cbb
      \]
    \item
      Es gilt
      \[
          \left( \sum_{i \in I} U_i \right)_\Cbb
        = \sum_{i \in I} (U_i)_\Cbb.
      \]
    \item
      Folgern Sie, dass genau dann $V = \bigoplus_{i \in I} U_i$, wenn $V_\Cbb = \bigoplus_{i \in I} (U_i)_{\Cbb}$.
  \end{enumerate}
\end{question}


\begin{question}
  Es seien $V$ und $W$ zwei reelle Vektorräume, und $f \colon V \to W$ sei $\Rbb$-linear.
  \begin{enumerate}[leftmargin=*]
    \item
      Zeigen Sie, dass $\ker (f_\Cbb) = (\ker f)_\Cbb$.
    \item
      Folgern Sie, dass $f_\Cbb$ genau dann injektiv ist, wenn $f$ injektiv ist.
    \item
      Folgern Sie ferner, dass $(V_\Cbb)_\lambda(f_\Cbb) = V_\lambda(f)_\Cbb$ für jedes $\lambda \in \Rbb$.
    \item
      Zeigen Sie, dass $\im (f_\Cbb) = (\im f)_\Cbb$.
    \item
      Folgern Sie, dass $f_\Cbb$ genau dann surjektiv ist, wenn $f$ surjektiv ist.
  \end{enumerate}
\end{question}


\begin{question}
  Es sei $V$ ein reeller Vektorraum und $f \colon V \to V$ ein Endomorphismus.
  Zeigen Sie, dass $f$ genau dann diagonalisierbar ist, wenn $f_\Cbb$ diagonalisierbar mit reellen Eigenwerten ist.
\end{question}


\begin{question}
  Zeigen Sie, dass die kanonische Inklusion $\iota \colon \Rbb[X] \to \Cbb[X]$, $x \mapsto x$ $\Rbb$-linear ist, und einen Isomorphismus $\Rbb[X]_\Cbb \to \Cbb[X]$ von $\Cbb$-Vektorräumen induziert.
\end{question}


\begin{question}
  Es sei $\mc{B} = (b_1, \dotsc, b_n)$ eine Basis eines $\Rbb$-Vektorraums $V$ und $\mc{C} = (c_1, \dotsc, c_m)$ eine Basis eines $\Rbb$-Vektorraums $W$.
  Es seien
  \[
    \mc{B}_\Cbb \coloneqq (b_1 + i \cdot 0, \dotsc, b_n + i \cdot 0)
    \quad\text{und}\quad
    \mc{C}_\Cbb \coloneqq (c_1 + i \cdot 0, \dotsc, c_m + i \cdot 0)
  \]
  die entsprechenden $\Cbb$-Basen der Komplexifizierungen $V_\Cbb$ und $W_\Cbb$.
  Es seiena
  \begin{gather*}
    \Phi^\Rbb \colon \Hom_\Rbb(V,W) \to \Mat(m \times n, \Rbb),
    \quad
    f \mapsto \Mat_{\mc{B}, \mc{C}}(f)
  \shortintertext{und}
    \Phi^\Cbb \colon \Hom_\Cbb(V_\Cbb, W_\Cbb) \to \Mat(m \times n, \Cbb),
    \quad
    g \mapsto \Mat_{\mc{B}, \mc{C}}(g).
  \end{gather*}
  Es seien
  \begin{align*}
    \iota_1 \colon \Hom_\Rbb(V, W) \to \Hom_\Rbb(V,W)_\Cbb,
    &\quad
    f \mapsto f + i \cdot 0,
    \\
    \iota_2 \colon \Hom_\Rbb(V,W) \to \Hom_\Cbb(V_\Cbb, W_\Cbb),
    &\quad
    f \mapsto f_\Cbb
    \\
    \iota_3 \colon \Mat(m \times n, \Rbb) \to \Mat(m \times n, \Rbb)_\Cbb,
    &\quad
    A \mapsto A + i \cdot 0,
    \\
    \iota_4 \colon \Mat(m \times n, \Rbb) \to \Mat(m \times n, \Cbb),
    &\quad
    A \mapsto A,
  \end{align*}
  die jeweiligen kanonischen Inklusionen.
  \begin{enumerate}[leftmargin=*]
    \item
      Zeigen Sie, dass das folgende Diagram kommutiert:
      \[
        \begin{tikzcd}[row sep = large, column sep = large, ampersand replacement = \&]
                \Hom_\Rbb(V, W)           \arrow{r}{\iota_2}
                                          \arrow[swap]{d}{\Phi^\Rbb}
            \&  \Hom_\Cbb(V_\Cbb, W_\Cbb) \arrow{d}{\Phi^\Cbb}
          \\
                \Mat(m \times n, \Rbb)    \arrow{r}{\iota_4}
            \&  \Mat(m \times n, \Cbb)
        \end{tikzcd}
      \]
      Folgern Sie, dass $\iota_4$ tatsächlich injektiv ist, wie der oben verwendete Begriff \emph{Inklusion} vermuten lässt.
    \item
      Zeigen Sie, dass das folgende Diagram kommutiert:
      \[
        \begin{tikzcd}[row sep = large, column sep = large, ampersand replacement = \&]
                \Hom_\Rbb(V, W)             \arrow{r}{\iota_1}
                                            \arrow[swap]{d}{\Phi^\Rbb}
            \&  \Hom_\Rbb(V, W)_\Cbb        \arrow{d}{(\Phi^\Rbb)_\Cbb}
          \\
                \Mat(m \times n, \Rbb)      \arrow{r}{\iota_3}
            \&  \Mat(m \times n, \Rbb)_\Cbb
        \end{tikzcd}
      \]
    \item
      Zeigen Sie, dass die Inklusion $\iota_1$ eine eindeutige $\Cbb$-lineare Abbildung
      \[
        \Psi_1 \colon \Hom_\Rbb(V,W)_\Cbb \to \Hom_\Cbb(V_\Cbb, W_\Cbb)
      \]
      induziert, die das folgende Diagram zum kommutieren bringt:
      \[
        \begin{tikzcd}[row sep = large, column sep = large, ampersand replacement = \&]
                  {}
              \&  \Hom_\Rbb(V,W)            \arrow[swap]{ld}{\iota_2}
                                            \arrow{rd}{\iota_1}
              \&  {}
          \\
                  \Hom_\Rbb(V,W)_\Cbb       \arrow{rr}{\Phi_1}
              \&  {}
              \&  \Hom_\Cbb(V_\Cbb, W_\Cbb).
        \end{tikzcd}
      \]
    \item
      Zeigen Sie auf analoge Weise, dass die Inklusion $\iota_2$ eine eindeutige $\Cbb$-lineare Abbildung
      \[
        \Psi_2 \colon \Mat(m \times n, \Rbb)_\Cbb \to \Mat(m \times n, \Cbb)
      \]
      induziert, die das folgende Diagram zum kommutieren bringt:
      \[
        \begin{tikzcd}[row sep = large, column sep = large, ampersand replacement = \&]
                  {}
              \&  \Mat(m \times n, \Rbb)      \arrow[swap]{ld}{\iota_4}
                                              \arrow{rd}{\iota_3}
              \&  {}
          \\
                  \Mat(m \times n, \Rbb)_\Cbb \arrow{rr}{\Phi_2}
              \&  {}
              \&  \Mat(m \times n, \Cbb)
        \end{tikzcd}
      \]
    \item
      Wir haben nun das folgende Diagram:
      \[
        \begin{tikzcd}[row sep = large, column sep = large, ampersand replacement = \&]
                  {}
              \&  \Hom_\Rbb(V, W)               \arrow[swap]{ld}{\iota_2}
                                                \arrow{rd}{\iota_1}
                                                \arrow[near end, swap]{dd}{\Phi^\Rbb}
              \&  {}
          \\
                  \Hom_\Rbb(V, W)_\Cbb          \arrow[crossing over, near start]{rr}{\Psi_1}
                                                \arrow[swap]{dd}{(\Phi^\Rbb)_\Cbb}
              \&  {}
              \&  \Hom_\Cbb(V_\Cbb, W_\Cbb)     \arrow{dd}{\Phi^\Cbb}
          \\
                  {}
              \&  \Mat(m \times n, \Rbb)        \arrow[swap]{ld}{\iota_4}
                                                \arrow{rd}{\iota_3}
              \&  {}
          \\
                  \Mat(m \times n, \Rbb)_\Cbb   \arrow{rr}{\Psi_2}
              \&  {}
              \&  \Mat(m \times n, \Cbb)
        \end{tikzcd}
      \]
      Von diesem Diagram wissen wir bereits, dass Deckel, Boden und beide Rückseiten kommutieren.
      Zeigen Sie damit, dass auch die Vorderseite kommutiert.
      (\emph{Hinweis}: Nutzen Sie, dass zwei $\Cbb$-lineare Abbildung $f, g \colon \Hom_\Rbb(V, W)_\Cbb \to \Mat(m \times n, \Cbb)$ genau dann übereinstimmen, wenn die Kompositionen $f \circ \iota_2$ und $g \circ \iota_2$ übereinstimmen.)
    \item
      Zeigen Sie, dass $\Psi_2$ ein Isomorphismus von $\Cbb$-Vektorräumen ist.
    \item
      Folgen Sie, dass auch $\Psi_1$ ein Isomorphismus von $\Cbb$-Vektorräumen ist.
  \end{enumerate}
\end{question}










% EIGENSTUFF


\begin{question}
  Es sei $V$ ein $K$-Vektorraum, wobei $V \neq 0$ und $K$ algebraisch abgeschlossen ist.
  Es seien $f_1, \dotsc, f_n \colon V \to V$ paarweise kommutierende Endomorphismen.
  Zeigen Sie, dass die Endomorphismen $f_1, \dotsc, f_n$ einen gemeinsamen Eigenvektor besitzen, d.h.\ dass es ein $v \in V$ gibt, das für jedes $f_i$ eine Eigenvektor ist.
\end{question}


\begin{question}
  Es sei $V$ ein $K$-Vektorraum.
  Für alle Endomorphismen $f_1, \dotsc, f_n \in \End(V)$ und Skalare $\lambda_1, \dotsc, \lambda_n \in K$ sei
  \[
              V(f_1, \lambda_1; \dotsc; f_n, \lambda_n)
    \coloneqq \{ v \in V \mid \text{$f_i(v) = \lambda_i v$ für alle $i = 1, \dotsc, n$} \}
  \]
  der \emph{gemeinsame Eigenraum} der Endomorphismen $f_1, \dotsc, f_n$ zu den Eigenwerten $\lambda_1, \dotsc, \lambda_n$.
  \begin{enumerate}[leftmargin=*]
    \item
      Zeigen Sie, dass
      \[
          V(f_1, \lambda_1; \dotsc; f_n, \lambda_n)
        = \bigcap_{i=1}^n V(f_i, \lambda_i)
      \]
      für alle Endomorphismen $f_1, \dotsc, f_n \in \End(V)$ und Eigenwerte $\lambda_1, \dotsc, \lambda_n \in K$.
    \item
      Es seien $f_1, \dotsc, f_n, g \in \End(V)$ Endomorphismen, so dass $g$ mit jedem $f_i$ kommutiert.
      Zeigen sie, dass der gemeinsame Eigenraum $V(f_1, \lambda_1; \dots; f_n, \lambda_n)$ für alle $\lambda_1, \dotsc, \lambda_n \in K$ invariant unter $g$ ist.
    \item
      Zeigen Sie: Sind die Endomorphismen $f_1, \dotsc, f_n \in \End(V)$ diagonalisierbar (d.h.\ es ist $V = \bigoplus_{\lambda \in K} V(f_i, \lambda)$ für alle $i = 1, \dotsc, n$) und paarweise kommutierend, so sind die Endomorphismen \emph{simultan diagonalisierbar}, d.h.\ es ist
      \[
          V
        = \bigoplus_{\lambda_1, \dotsc, \lambda_n \in K}  V(f_1, \lambda_1; \dotsc; f_n, \lambda_n).
      \]
    \item
      Es sei nun $V$ endlichdimensional und $H \subseteq \End(V)$ ein Untervektorraum aus diagonalisierbaren und paarweise kommutierenden Endomorphismen.
      Zeigen Sie, dass es eine Basis $\mc{B}$ von $V$ gibt, so dass $\Mat_\mc{B}(f)$ für jedes $f \in H$ eine Diagonalmatrix ist.
  \end{enumerate}
\end{question}



\begin{question}
  Es sei $A \in \Mat_2(\Rbb)$ mit $\tr A = 0$ und $\tr A^2 = -2$.
  Bestimmen Sie $\det A$.
\end{question}


\begin{question}
  Zeigen Sie, dass es für $A \in \GL_n(K)$ ein Polynom $P \in K[T]$ mit $\deg P \leq n-1$ gibt, so dass $A^{-1} = P(A)$.
\end{question}


\begin{question}
  Es sei $K$ ein algebraisch abgeschlossener Körper mit $\ringchar K \notin \{2,3\}$.
  Zeigen Sie, dass
  \[
    \det A = \frac{1}{6} (\tr A)^3 - \frac{1}{2} (\tr A^2)(\tr A) + \frac{1}{3} (\tr A^3)
    \quad
    \text{für jedes $A \in \Mat_3(K)$}.
  \]
  (\emph{Hinweis}: Wenn die Rechnungen zu kompliziert werden, dann macht man es falsch.)
\end{question}



% JORDANSTUFF


\begin{question}
  Es sei $V$ ein endlichdimensionaler $\Cbb$-Vektorraum.
  \begin{enumerate}
    \item
      Es sei $n \colon V \to V$ ein nilpotenter Endomorphismus.
      Zeigen Sie, dass der Endomorphismus ${\id_V} + n$ invertierbar ist.
  \end{enumerate}
  Ein Endomorphismus $u \colon V \to V$ heißt \emph{unipotent}, falls $u - \id_V$ nilpotent it.
  \begin{enumerate}[resume]
    \item
      Folgern Sie, dass jeder unipotente Endomorphismus von $V$ invertierbar ist.
  \end{enumerate}
  Auf dem fünften Übungszettel wurde gezeigt, dass es für jeden Endomorphismus $f \colon V \to V$ eindeutige Endomorphismen $d,n \colon V \to V$ gibt, so dass
  \begin{itemize}
    \item 
      $f = d + n$,
    \item
      $d$ ist diagonalisierbar und $n$ ist nilpotent, und
    \item
      $d$ und $n$ kommutieren.
  \end{itemize}
  Folgern Sie aus dieser \emph{additiven Jordanzerlegung} von $\End(V)$ die folgende \emph{multiplikative Jordanzerlegung} von $\GL(V)$.
  \begin{enumerate}[resume]
    \item
      Zeigen Sie, dass es für jedes $s \in \GL(V)$ eindeutige $d, u \in \GL(V)$ gibt, so dass
      \begin{itemize}
        \item
          $s = d \cdot u$,
        \item
          $d$ ist diagonalisierbar und $u$ ist unipotent, und
        \item
          $d$ und $u$ kommutieren.
      \end{itemize}
  \end{enumerate}
\end{question}



\begin{question}
  Für alle $\lambda_1, \dotsc, \lambda_n \in \Cbb$ sei
  \[
    \diag(\lambda_1, \dotsc, \lambda_n)
    \coloneqq
    \begin{pmatrix}
      \lambda_1 &         &           \\
                & \ddots  &           \\
                &         & \lambda_n
    \end{pmatrix}
    \in \Mat_n(\Cbb).
  \]
  Es sei
  \[
    \Diagonal_n(\Cbb)
    \coloneqq
    \left\{
      S \diag(\lambda_1, \dotsc, \lambda_n) S^{-1}
    \,\middle|\,
      S \in \GL_n(\Cbb),
      \lambda_1, \dotsc, \lambda \in \Cbb
    \right\}
    \subseteq \Mat_n(\Cbb)
  \]
  die Menge der diagonalisierbaren komplexen $n \times n$-Matrizen.
  Wir zeigen, dass $D_n(\Cbb) \subseteq \Mat_n(\Cbb)$ dicht ist, d.h.\ dass es für jede Matrix $A \in \Mat_n(\Cbb)$ und jedes $\varepsilon > 0$ eine diagonalisierbare Matrix $D \in \Diagonal_n(\Cbb)$ mit $\|A-D\| < \varepsilon$ gibt.
  
  Es sei $S \in \GL_n(\Cbb)$, so dass $S A S^{-1}$ eine obere Dreiecksmatrix mit Diagonaleinträgen $\lambda_1, \dotsc, \lambda_n$ ist, also
  \[
    S A S^{-1}
    =
    \begin{pmatrix}
      \lambda_1 & *       & \cdots  & *         \\
                & \ddots  & \ddots  & \vdots    \\
                &         & \ddots  & *         \\
                &         &         & \lambda_n
    \end{pmatrix}.
  \]
  Es seien $z_1, \dotsc, z_n \in \Cbb$ paarweise verschieden und
  \[
    B(t)
    \coloneqq
    A + t S \diag(z_1, \dotsc, z_n) S^{-1}
    \quad
    \text{für alle $t \in \Rbb$}.
  \]
  \begin{enumerate}
    \item
      Zeigen Sie, dass $\mu_1(t), \dotsc, \mu_n(t) \in \Cbb$ mit
      \[
        \mu_i(t) \coloneqq \lambda_i + t z_i
        \quad
        \text{für $i = 1, \dotsc, n$}
      \]
      die Eigenwerte von $B(t)$ ist.
    \item
      Zeigen Sie, dass die Zahlen $\mu_1(t), \dotsc, \mu_n(t)$ für fast alle $t \in \Rbb$ paarweise verschieden sind.
    \item
      Folgern Sie, dass $B(t)$ für fast alle $t \in \Rbb$ diagonalisierbar ist.
    \item
      Folgern Sie, dass es für alle $\varepsilon > 0$ ein $D \in \Diagonal_n(\Cbb)$ mit $\| A - D \| < \varepsilon$ gibt.
  \end{enumerate}
  Wir wollen die Dichtheit von $\Diagonal_n(\Cbb) \subseteq \Mat_n(\Cbb)$ nutzen, um den Satz von Cayley-Hamilton zu zeigen:
  \begin{enumerate}[resume]
    \item
      Zeigen Sie, dass die Abbildung
      \[
        F \colon \Mat_n(\Cbb) \to \Mat_n(\Cbb),
        \quad
        A \mapsto \chi_A(A)
      \]
      stetig ist, wobei $\chi_A(T) \in \Cbb[T]$ das charakteristische Polynom von $A$ ist.
    \item
      Zeigen Sie, dass $F(D) = 0$ für jede Diagonalmatrix $D \in \Mat_n(\Cbb)$.
    \item
      Zeigen Sie, dass $P(SAS^{-1}) = S P(A) S^{-1}$ für alle $P \in \Cbb[T]$, $A \in \Mat_n(\Cbb)$ und $S \in \GL_n(\Cbb)$.
      Folgern Sie, dass $F(D) = 0$ für jede Matrix $D \in \Diagonal_n(\Cbb)$.
    \item
      Folgern Sie, dass $F(A) = 0$ für alle $A \in \Mat_n(\Cbb)$.
  \end{enumerate}
\end{question}



















% SCALAR PRODUCTS


\begin{question}
  Es seien $V$ und $W$ zwei $\Kbb$-Skalarprodukträume und $f \colon V \to W$ eine lineare Abbildung.
  Es sei $\mc{B} = (b_1, \dotsc, b_n)$ eine Orthonormalbasis von $V$ und $\mc{C} = (c_1, \dotsc, c_m)$ eine Orthonormalbasis von $W$.
  Zeigen Sie die Gleichheit
  \[
      \Mat_{\mc{B}, \mc{C}}(f^*)
    = \Mat_{\mc{C}, \mc{B}}(f)^*.
  \]
\end{question}


\begin{question}
  Es sei $V$ ein endlichdimensionaler unitärer Vektorraum und $f \colon V \to V$ ein normaler Endomorphismus.
  Zeigen Sie:
  \begin{enumerate}
    \item
      $f$ ist genau dann unitär, wenn alle Eigenwerte von $f$ Betrag $1$ haben.
    \item
      $f$ ist genau dann selbstadjungiert, wenn alle Eigenwerte von $f$ reell sind.
    \item
      $f$ ist genau dann antiselbstadjungiert, wenn alle Eigenweret von $f$ rein imaginär sind.
    \item
      $f$ ist genau dann eine Orthogonalprojektion, wenn $0$ und $1$ die einzigen Eigenwerte von $f$ sind.
  \end{enumerate}
\end{question}


\begin{question}
  Es seien $V$ und $W$ zwei endlichdimensionale euklidische Vektorräume.
  Ferner sei $f \colon V \to W$ eine $\Rbb$-lineare Abbildung.
  \begin{enumerate}[leftmargin=*]
    \item
      Zeigen Sie, dass die Abbildung
      \[
        \Phi_V \colon V \to V^*,
        \quad
        v \mapsto \bil{-, v}
      \]
      ein $\Rbb$-linearer Isomorphismus ist.
    \item
      Geben Sie die Definition der dualen Abbildung $f^* \colon W^* \to V^*$ an.
      Zeigen Sie, dass $f^*$ $\Rbb$-linear ist.
    \item
      Zeigen Sie, dass die Abbildung $g \coloneqq \Phi_V^{-1} \circ f^* \circ \Phi_W$ $\Rbb$-linear ist, und dass
      \[
        \bil{f(v), w} = \bil{v, g(w)}
        \quad
        \text{für alle $v \in V$, $w \in W$}.
      \]
    \item
      Inwiefern ändern sich die obigen Resultate für denn Fall $\Kbb = \Cbb$, wenn also $V$ und $W$ endlichdimensionale unitäre Vektorräume sind?
  \end{enumerate}
\end{question}


\begin{question}
  Es sei $V$ ein endlichdimensionale $\Kbb$-Vektorraum und $f \colon V \to V$ ein Endomorphismus.
  \begin{enumerate}[leftmargin=*]
    \item
      Zeigen Sie für denn Fall $\Kbb = \Rbb$, dass $f$ genau dann diagonalisierbar ist, wenn es ein Skalarprodukt auf $V$ gibt, bezüglich dessen $f$ selbstadjungiert ist.
    \item
      Zeigen oder widerlegen Sie die analoge Aussage für $\Kbb = \Cbb$.
  \end{enumerate}
\end{question}



\begin{question}
  Es sei $V \coloneqq \mathcal{C}([0,1], \Rbb)$ der Raum der stetigen Funktionen $[0,1] \to \Rbb$, und es sei
  \[
    U \coloneqq \{ f \in V \mid f(0) = 0 \}.
  \]
  \begin{enumerate}[leftmargin=*]
    \item
      Zeigen Sie, dass $U$ ein Untervektorraum von $V$ ist.
    \item
      Zeigen Sie, dass
      \[
        \bil{f, g} \coloneqq \int_0^1 f(t) g(t) \dd{t}
        \quad
        \text{für alle $f, g \in V$}
      \]
      ein Skalarprodukt auf $V$ definiert.
    \item
      Zeigen Sie, dass $U^\perp = 0$.
      Folgern Sie, dass $V \neq U \oplus U^\perp$.
      (\emph{Hinweis}: Betrachten Sie für $g \in U^\perp$ die Funktion $h \colon [0,1] \to \Rbb$ mit $h(t) = t^2 g(t)$.)
    \item
      Zeigen Sie ferner, dass $V/U$ eindimensional ist.
  \end{enumerate}
\end{question}


\begin{question}
  Es sei
  \[
    W = \{(a_n)_{n \in \Zbb} \mid \text{$a_n \in \Rbb$ für alle $n \in \Zbb$}\}
  \]
  der Vektorraum der beidseitigen reellwertigen Folgen.
  Wir betrachten den Untervektorraum
  \[
    V \coloneqq
    \left\{
      (a_n)_{n \in \Zbb} \in W
    \,\middle|\,
      \sum_{n \in \Zbb} |a_n|^2 < \infty
   \right\}
  \]
  der quadratsummierbaren Folgen.
  \begin{enumerate}[leftmargin=*]
    \item
      Zeigen Sie, dass $V$ ein Untervektorraum von $W$ ist.
    \item
      Zeigen Sie für alle $(a_n)_{n \in \Zbb}, (b_n)_{n \in \Zbb} \in V$, dass
      \[
        \sum_{n \in \Zbb} a_n b_n < \infty.
      \]
      (\emph{Hinweis:} Zeigen sie zunächst, dass $ab \leq (a^2 + b^2)/2$ für alle $a, b \in \Rbb$.)
    \item
      Zeigen sie, dass
      \[
                  \bil{ (a_n)_{n \in \Zbb}, (b_n)_{n \in \Zbb} }
        \coloneqq \sum_{n \in \Zbb} a_n b_n
        \quad
        \text{für alle $(a_n)_{n \in \Zbb}, (b_n)_{n \in \Zbb} \in V$}
      \]
      ein Skalarprodukt auf $V$ definiert.
    \item
      Es sei
      \[
        R \colon V \to V,
        \quad
        (a_n)_{n \in \Zbb} \mapsto (a_{n-1})_{n \in \Zbb}
      \]
      der Rechtsshift-Operator.
      Zeigen Sie, dass $R$ ein Adjungiertes besitzt, und entscheiden Sie, ob $R$ selbstadjungiert, orthogonal, bzw.\ normal ist.
    \item
      Zeigen Sie, dass $R$ keine Eigenwerte besitzt.
    \item
      Es sei
      \[
        S \colon V \to V,
        \quad
        (a_n)_{n \in \Nbb} \mapsto (a_{-n})_{n \in \Nbb}.
      \]
      Zeigen Sie, dass $S$ ein Adjungiertes besitzt, und entscheiden Sie, ob $R$ selbstadjungiert, orthogonal, bzw.\ normal ist.
    \item
      Zeigen Sie, dass $S$ diagonalisierbar ist.
    \item
      Es sei
      \[
        U \coloneqq \{(a_n)_{n \in \Zbb} \in V \mid \text{$a_n = 0$ für fast alle $n \in \Zbb$}\}.
      \]
      Bestimmen Sie $U^\perp$ und entscheiden Sie, ob $V = U \oplus U^\perp$.
    \item
      Bestimmen Sie eine Orthonormalbasis von $U$.
  \end{enumerate}
\end{question}


\begin{question}
  \begin{enumerate}[leftmargin=*]
    \item
      Zeigen Sie, dass durch
      \[
        \sigma(A, B) \coloneqq \tr\left( A^T B \right)
        \quad
        \text{für alle $A, B \in \Mat_n(\Rbb)$}
      \]
      ein Skalarprodukt auf $\Mat_n(\Rbb)$ definiert wird.
    \item
      Zeigen Sie, dass die Standardbasis $(E_{ij})_{i,j=1,\dotsc,n}$ von $\Mat_n(\Rbb)$ mit
      \[
        (E_{ij})_{kl} \coloneqq \delta_{ik} \delta_{jl}
        \quad
        \text{für alle $1 \leq i,j,k,l \leq n$}
      \]
      eine Orthonormalbasis von $\Mat_n(\Rbb)$ bezüglich $\sigma$ bilden.
    \item
      Es sei
      \[
        S_+ \coloneqq \{A \in \Mat_n(\Rbb) \mid A^T = A\}
      \]
      der Untervektorraum der symmetrischen Matrizen, und
      \[
        S_- \coloneqq \{A \in \Mat_n(\Rbb) \mid A^T  = -A\}
      \]
      der Untervektorraum der schiefsymmetrischen Matrizen.
      Zeigen Sie, dass
      \[
        \Mat_n(\Rbb) = S_+ \oplus S_-,
      \]
      und dass die Summe orthogonal ist.
  \end{enumerate}
\end{question}


\begin{question}
  Es sei $V$ ein Skalarproduktraum und
  \[
    O(V) \coloneqq \{ f \in \End(V) \mid f f^* = \id \}.
  \]
  Zeigen Sie, dass $O(V)$ eine Untergruppe von $\GL(V)$ bildet.
\end{question}


\begin{question}
Zeigen sie, dass für eine Matrix $A \in \Mat_n(\Kbb)$ die folgenden Bedingungen äquivalent sind:
  \begin{enumerate}
    \item
      $A$ ist invertierbar mit $A^{-1} = A^*$.
    \item
      $A A^* = I$.
    \item
      $A^* A = I$.
    \item
      Die Spalten von $A$ bilden eine Orthonormalbasis des $\Kbb^n$.
    \item
      Die Zeilen von $A$ bilden eine Orthonormalbasis des $\Kbb^n$.
  \end{enumerate}
\end{question}


\begin{question}
  Es sei $A \in \Mat_n(\Cbb)$.
  \begin{enumerate}[leftmargin=*]
    \item
      Zeigen Sie, dass es eindeutige hermitsche Matrizen $B, C \in \Mat_n(\Cbb)$ mit $A = B + i C$ gibt.
    \item
      Zeigen Sie, dass $A$ genau dann normal ist, wenn $B$ und $C$ kommutieren.
  \end{enumerate}
\end{question}


\begin{question}
  Es sei $V$ ein endlichdimensionaler euklidischer Vektorraum mit Skalarprodukt $\bil{\cdot, \cdot}$, und es sei $G \subseteq \GL(V)$ eine endliche Untergruppe.
  \begin{enumerate}[leftmargin=*]
    \item
      Zeigen Sie, dass
      \[
        \bil{x,y}_G \coloneqq \frac{1}{|G|} \sum_{\phi \in G} \bil{\phi(x), \phi(y)}
        \quad
        \text{für alle $x, y \in V$}
      \]
      ein Skalarprodukt auf $G$ definiert.
    \item
      Zeigen Sie, dass $\bil{\cdot, \cdot}_G$ in dem Sinne $G$-invariant ist, dass
      \[
        \bil{\phi(x), \phi(y)} = \bil{x,y}
        \quad
        \text{für alle $x, y \in V$ und $\phi \in G$}.
      \]
    \item
      Folgern Sie, dass es eine Basis $\mc{B}$ von $V$ gibt, so dass $M_\mc{B}(\phi)$ für alle $\phi \in G$ eine orthogonale Matrix ist.
    \item
      Folgern Sie damit, dass es für $n = \dim V$ einen injektiven Gruppenhomomorphismus $\Phi \colon G \to O_n(\Rbb)$ gibt, $G$ also isomorph zu der Untergruppe $\im \Phi$ von $O_n(\Rbb)$ ist.
  \end{enumerate}
\end{question}


\begin{question}
  Es sei $V$ ein euklidischer Vektorraum.
  Für jedes $\alpha \in V$ mit $\alpha \neq 0$ sei
  \[
    s_\alpha \colon V \to V,
    \quad\text{mit}\quad
              s_\alpha(x)
    \coloneqq x - 2 \frac{\bil{x, \alpha}}{\|\alpha\|^2} \alpha.
  \]
  Ferner seien
  \[
              L_\alpha
    \coloneqq \Rbb \alpha
    \quad\text{und}\quad
              H_\alpha
    \coloneqq L_\alpha^\perp
    =         \alpha^\perp
    =         \{ v \in V \mid \bil{v, \alpha} = 0 \}.
  \]
  \begin{enumerate}[leftmargin=*]
    \item
      Zeigen Sie, dass $s_\alpha^2 = \id_V$, und dass $s_{\lambda \alpha} = s_\alpha$ für alle $\lambda \in \Rbb^\times$.
    \item
      Zeigen Sie, dass $s_\alpha$ diagonalisierbar ist, und dass
      \[
        V_{-1}(s_\alpha) = L_\alpha
        \quad\text{und}\quad
        V_1(s_\alpha) = H_\alpha.
      \]
    \item
      Interpretieren Sie $V$ geometrisch anschaulich.
    \item
      Es sei $s' \colon V \to V$ ein Endomorphismus mit $s'(\alpha) = -\alpha$ und $s'(x) = x$ für alle $x \in H_\alpha$.
      Zeigen Sie, dass bereits $s' = s_\alpha$ gilt.
    \item
      Zeigen Sie, dass für jeden orthogonalen Isomorphismus $t \colon V \to V$ die Identität
      \[
        t s_\alpha t^{-1} = s_{t(\alpha)}
      \]
      gilt.
  \end{enumerate}
\end{question}


\begin{question}
  Es sei $V$ ein endlichdimensionaler unitärer Vektorraum.
  Zeigen Sie, dass für eine lineare Abbildung $S \colon V \to V$ die folgenden Bedingungen äquivalent sind:
  \begin{enumerate}
    \item
      $S$ ist normal.
    \item
      $V$ hat eine Orthonormalbasis aus Eigenvektoren von $S$.
    \item
      Für jeden $S$-invarianten Untervektorraum $U \subseteq V$ ist auch das orthogonale Komplement $U^\perp$ invariant unter $S$.
  \end{enumerate}
\end{question}


\begin{question}
  Es sei $V$ ein endlichdimensionaler Skalarproduktraum über $\Kbb$.
  Zeigen Sie, dass
  \[
    \langle f, g \rangle \coloneqq \tr(f \circ g^*)
  \]
  ein Skalarprodukt auf $\End_\Kbb(V)$ definiert.
\end{question}


\begin{question}
  Es sei $\det \colon \Mat_n(\Cbb) \to \Cbb^\times$ die Determinantenabbildung, wobei $\Cbb^\times$ die multiplikative Gruppe des Körpers bezeichnet.
  \begin{enumerate}
    \item
      Zeigen Sie, dass $\det$ ein surjektiver Gruppenhomomorphismus ist.
    \item
      Bestimmen Sie den Kern von $\det$.
    \item
      Bestimmen Sie Bild und Kern der Einschränkung $\det|_{\GL_n(\Rbb)}$.
    \item
      Bestimmen Sie Bild und Kern der Einschränkung $\det|_{U_n}$.
    \item
      Bestimmen Sie Bild und Kern der Einschränkung $\det|_{O_n}$.
  \end{enumerate}
\end{question}


\begin{question}
  Es sei
  \[
    \Phi \colon SU_2 \to S^3,
    \quad
    \begin{pmatrix}
      a & b \\
      c & 
    \end{pmatrix}
    \mapsto
    \vect{a \\ b}
  \]
  die Abbildung auf die erste Spalte, wobei
  \[
              S^3
    \coloneqq \left\{ \vect{z_1 \\ z_2} \in \Cbb^2 \,\middle|\, |z_1|^2 + |z_2|^2 = 1 \right\}.
  \]
  \begin{enumerate}
    \item
      Zeigen Sie, dass $\Phi$ wohldefiniert ist.
    \item
      Zeigen Sie, dass $\Phi$ bijektiv ist.
  \end{enumerate}
\end{question}


\begin{question}
  Zeigen Sie, dass die drei Gruppen $SO_2$, $S^1$ und $U_1$ isomorph sind.
\end{question}












% BILINEAR FORMS


\begin{question}
  Ist $\beta \colon V \times W \to K$ eine Bilinearform, so heißen eine Basis $\mc{B} = (v_i)_{i \in I}$ von $V$ und eine Basis $\mc{C} = (w_i)_{i \in I}$ von $W$ \emph{dual bezüglich $\beta$}, falls
  \[
    \beta(v_i, w_j) = \delta_{ij}
    \quad
    \text{für alle $i, j \in I$}.
  \]
  Es sei zunächst $V$ ein $K$-Vektorraum.
  \begin{enumerate}
    \item
      Zeigen Sie, dass die \emph{Evaluation}
      \[
        e \colon V \times V^* \to K
        \quad\text{mit}\quad
        e(v, \varphi) = \varphi(e)
      \]
      eine $K$-bilineare Abbildung ist.
    \item
      Zeigen Sie im Falle der Endlichdimensionalität von $V$, dass es zu jeder Basis $\mc{B} = (b_1, \dotsc, b_n)$ von $V$ genau eine Basis $\mc{C}$ von $V^*$ gibt, die bezüglich $e$ dual zu $\mc{B}$ ist.
      Woher kennen Sie diese Basis?
  \end{enumerate}
  Von nun an sei $V$ ein endlichdimensionaler euklidischer Vektorraum mit Skalarprodukt $\bil{\cdot, \cdot}$.
  \begin{enumerate}[resume]
    \item
      Zeigen Sie, dass die Abbildung
      \[
        \Phi \colon V \to V^*,
        \quad
        v \mapsto \bil{-, v}
      \]
      ein Isomorphismus ist.
    \item
      Folgern Sie, dass es für jede Basis $\mc{B} = (b_1, \dotsc, b_n)$ von $V$ genau eine Basis $\mc{B}^\circ = (b_1^\circ, \dotsc, b_n^\circ)$ von $V$ gibt, die bezüglich $\bil{\cdot, \cdot}$ dual zu $\mc{B}$ ist.
      (\emph{Hinweis}: Formulieren Sie die Aussage, dass $\mc{C}$ dual zu $\mc{B}$ ist, mithilfe von $\Phi$ um.)
    \item
      Zeigen Sie, dass für jede Basis $\mc{B}$ von $V$ die Gleichheit $(\mc{B}^\circ)^\circ = \mc{B}$ gilt.
      Folgern Sie, dass die Abbildung
      \[
            \left\{ \text{geordnete Basen von $V$} \right\}
        \to \left\{ \text{geordenet Basen von $V$} \right\},
        \quad
        \mc{B} \mapsto \mc{B}^\circ
      \]
      bijektiv ist.
    \item
      Unter welchen Namen kennen Sie Basen von $V$, die bezüglich $(-)^\circ$ selbstdual sind, die also $\mc{B}^\circ = \mc{B}$ erfüllen?
  \end{enumerate}
\end{question}


\begin{question}
  Es sei $V$ ein $K$-Vektorraum und $\beta \colon V \times V \to K$ eine symmetrische Bilinearform.
  \begin{enumerate}[leftmargin=*]
    \item
      Zeigen Sie, dass
      \[
        \rad(\beta) \coloneqq \{ v \in V \mid \text{$\beta(v, w) = 0$ für alle $w \in V$} \}
      \]
      ein Untervektorraum von $V$ ist.
      (Man bezeichnet $\rad(\beta)$ als das \emph{Radikal} von $\beta$.)
    \item
      Zeigen Sie, dass $\beta$ eine symmetrische Bilinearform $\bar{\beta} \colon (V/U) \times (V/U) \to K$ mit
      \[
        \bar{\beta}([v], [w])
        \coloneqq
        \beta(v,w)
        \quad
        \text{für alle $v, w \in V$}
      \]
      induziert.
    \item
      Zeigen Sie, dass $\bar{\beta}$ nicht entartet ist, d.h.\ dass für das Radikal
      \[
                  \rad(\bar{\beta})
        \coloneqq \{ x \in V/U \mid \text{$\bar{\beta}(x,y) = 0$ für alle $y \in V/U$} \}
      \]
      bereits $\rad(\bar{\beta}) = 0$ gilt.
    \item
      Inwiefern gelten die obigen Aussagen noch, wenn man $U$ durch
      \[
        W \coloneqq \{v \in V \mid \beta(v,v) = 0\}
      \]
      ersetzt?
  \end{enumerate}
\end{question}


\begin{question}
  Für je zwei $K$-Vektorräume $V$ und $W$ sei
  \[
              \Bil(V, W)
    \coloneqq \{b \colon V \times  W \to K \mid \text{$b$ ist bilinear}\}
  \]
  der Raum der Bilinearformen $V \times W \to K$.
  \begin{enumerate}[leftmargin=*]
    \item
      Zeigen Sie, dass die Flipabbildung
      \[
        F \colon \Bil(V, W) \to \Bil(W, V),
        \quad
        b \mapsto F(b)
        \quad\text{mit}\quad
        F(b)(w,v) = b(v,w)
      \]
      ein Isomorphismus von $K$-Vektorräumen ist.
    \item
      Es sei $b \in \Bil(V, W)$ eine Bilinearform.
      Zeigen Sie, dass $b$ ein lineare Abbildung
      \[
        \Phi_{V,W}(b) \colon V \to W^*,
        \quad
        v \mapsto b(v, -)
      \]
      induziert.
      Dabei ist
      \[
        b(v, -) \colon W \to K,
        \quad
        w \mapsto b(v,w).
      \]
    \item
      Zeigen Sie, dass die Abbildung
      \[
        \Phi_{V,W} \colon \Bil(V, W) \to \Hom(V, W^*),
        \quad
        b \mapsto \Phi_{V,W}(b)
      \]
      ein Isomorphismus von $K$-Vektorräumen ist.
    \item
      Geben Sie mithilfe der vorherigen Aufgabenteile explizit einen Isomorphismus
      \[
        \Hom(V, W^*) \to \Hom(W, V^*)
      \]
      an.
  \end{enumerate}
  Wir betrachten nun den Fall $W = V^*$.
  \begin{enumerate}[resume, leftmargin=*]
    \item
      Zeigen Sie, dass die Evaluation
      \[
        e \colon V \times V^* \to K,
        \quad
        (v, \varphi) \mapsto \varphi(v)
      \]
      eine Bilinearform ist.
   \item
      Nach den vorherigen Aufgabenteilen entspricht die Bilinearform $e$ einer linearen Abbildung $V \to V^{**}$, sowie einer linearen Abbildung $V^* \to V^*$.
      Bestimmen Sie diese Abbildungen.
    \item
      Woher kennen Sie diese Abbildung?
  \end{enumerate}
\end{question}


\begin{question}
  Es seien $V$ und $W$ zwei $K$-Vektorräume und $f \colon V \to W$ eine lineare Abbildung.
  \begin{enumerate}[leftmargin=*]
    \item
      Geben Sie die Definition der dualen Abbildung $f^* \colon W^* \to V^*$ an, und zeigen Sie ihre Linearität.
    \item
      Zeigen Sie für jeden $K$-Vektorraum $U$, dass die Abbildung
      \[
        \bil{\cdot, \cdot} \colon U \times U^* \to K
        \quad\text{mit}\quad
        \bil{v, \varphi} = \varphi(v)
        \quad\text{für alle $v \in V$, $\varphi \in V^*$}
      \]
      eine Bilinearform ist.
    \item
      Zeigen Sie, dass
      \[
        \bil{f(v), \psi} = \bil{v, f^*(\psi)}
        \quad
        \text{für alle $v \in V$, $\psi \in W^*$}.
      \]
  \end{enumerate}
\end{question}


\begin{question}
  \begin{enumerate}[leftmargin=*]
    \item
      Zeigen Sie, dass die Abbildung
      \[
        \sigma \colon \Mat_n(K) \times \Mat_n(K) \to K
        \quad\text{mit}\quad
        \sigma(A, B) \coloneqq \tr(AB)
      \]
      eine symmetrische Bilinearform ist.
      Man bezeichnet diese als die \emph{Traceform}.
    \item
      Zeigen Sie, dass $\sigma$ in dem Sinne assoziativ ist, dass
      \[
        \sigma(AB, C) = \sigma(A, BC)
        \quad
        \text{für alle $A, B, C \in \Mat_n(K)$}.
      \]
    \item
      Zeigen sie, dass $\sigma$ nicht entartet ist, d.h.\ dass es für jedes $A \in \Mat_n(K)$ mit $A \neq 0$ ein $B \in \Mat_n(K)$ mit $\sigma(A, B) \neq 0$ gibt.
  \end{enumerate}
\end{question}


\begin{question}
  Es sei $V$ ein endlichdimensionaler $K$-Vektorraum, $b \colon V \times V \to K$ eine Bilinearform, $\mc{B} = (b_1, \dotsc, b_n)$ eine Basis von $V$, und $\mc{B}^* = (b_1^*, \dotsc, b_n^*)$ die entsprechende duale Basis von $V^*$.
  \begin{enumerate}[leftmargin=*]
    \item
      Zeigen Sie, dass die Abbildung
      \[
        B \colon V \to V^*,
        \quad
        v \mapsto b(-,v)
      \]
      $K$-linear ist.
    \item
      Zeigen Sie die Gleihheit
      \[
        \Mat_\mc{B}(b) = \Mat_{\mc{B}, \mc{B}^*}(B).
      \]
      (Beachten Sie, dass auf der linken Seite die darstellende Matrix einer Bilinearform steht, und auf der rechten Seite die darstellende Matrix einer linearen Abbildung.)
  \end{enumerate}
\end{question}









% LIE STUFF AND COMMUTATORS



\begin{question}
  Das \emph{Zentrum} eines Rings $R$ ist definiert als
  \[
    Z(R) \coloneqq \{r \in R \mid \text{$rs = sr$ für alle $s \in R$}.
  \]
  Man bemerke, dass $R$ genau dann kommutativ ist, wenn $Z(R) = R$.
  Wir werden $Z(\Mat_n(K))$ bestimmen.
  Hierfür sei
  \[
    D_n(K) \coloneqq K I = \{\lambda I \mid \lambda \in K\}
  \]
  der Untervektorraum der Skalarmatrizen.
  \begin{enumerate}[leftmargin=*]
    \item
      Zeigen Sie, dass $D_n(K) \subseteq Z(\Mat_n(K))$.
    \item
      Zeigen Sie für $A \in Z(\Mat_n(K))$, dass $A$ eine Diagonalmatrix ist.
      (\emph{Hinweis}: Betrachten Sie die Matrizen $E_{ii}$ für $1 \leq i \leq n$.)
    \item
      Zeigen Sie ferner, dass alle Diagonaleinträge von $A$ bereits gleich sind.
      (\emph{Hinweis}: Betrachten Sie die Matrizen $E_{ij}$ mit $1 \leq i,j \leq n$.)
    \item
      Folgern Sie, dass $Z(\Mat_n(K)) = D_n(K)$.
  \end{enumerate}
\end{question}


\begin{question}
  Es sei $V$ ein endlichdimensionaler $\Cbb$-Vektorraum, und es seien $K, E \colon V \to V$ zwei Endomorphismen mit
  \[
    \text{$K$ ist invertierbar}
    \quad\text{und}\quad
    KE = 2EK.
  \]
  \begin{enumerate}[leftmargin=*]
    \item
      Zeigen Sie, dass
      \[
        (K - 2 \lambda \id_V)^n E = 2^n E (K - \lambda \id_V)^n
      \]
      für alle $n \in \Nbb$.
    \item
      Folgern Sie, dass $E( V^\sim_\lambda(K) ) \subseteq V^\sim_{2\lambda}(K)$ für alle $\lambda \in \Cbb$.
    \item
      Folgern Sie, dass $E$ nilpotent ist.
  \end{enumerate}
\end{question}



\begin{question}
  Es sei $V$ ein $K$-Vektorraum und $m \colon V \times V \to V$ eine bilineare Abbildung.
  Eine lineare Abbildung $D \colon V \to V$ heißt \emph{$m$-Derivation}, falls
  \[
    D(m(x,y))
    = m(D(x), y) + m(x, D(y))
    \quad
    \text{für alle $x, y \in V$}.
  \]
  Es sei
  \[
              \Der(m)
    \coloneqq \{ D \colon V \to V \mid \text{$D$ ist eine $m$-Derivation} \}.
  \]
  \begin{enumerate}[leftmargin=*]
    \item
      Zeigen Sie für den Fall $V = K[X]$ und die Multiplikation
      \[
        m(p,q) \coloneqq p \cdot q
        \quad
        \text{für alle $p, q \in K[X]$},
      \]
      dass die Ableitung
      \[
        D \colon K[X] \to K[X],
        \quad
        \sum_{d=0}^n a_d X^d  \mapsto \sum_{d=1}^n a_d d X^{d-1} 
      \]
      eine $m$-Derivation ist.
      Unter welchem Namen ist dieser Umstand für gewöhnlich bekannt?
    \item
      Zeigen Sie, dass $\Der(m)$ ein Untervektorraum von $\End(V)$ ist.
    \item
      Zeigen Sie, dass $\Der(m)$ eine Lie-Unteralgebra von $\End(V)$ ist, d.h.\ dass für alle $D_1, D_2 \in \Der(m)$ auch $[D_1, D_2] \in \Der(m)$.
  \end{enumerate}
\end{question}


\begin{question}
  Es sei $V$ ein $K$-Vektorrraum und $[-,-] \colon V \times V \to V$ eine alternierend bilineare Abbildung.
  Für jedes $x \in V$ sei
  \[
    \ad_x \coloneqq [x,-] \colon V \to V, \quad y \mapsto [x,y].
  \]
  Zeigen Sie, dass die folgenden beiden Aussagen äquivalent sind:
  \begin{enumerate}
    \item
      $[-,-]$ erfüllt die Jacobi-Identität, d.h.\ es ist
      \[
        [[x,y],z] + [[y,z],x] + [[z,y],x] = 0
        \quad
        \text{für alle $x, y, z \in V$}.
      \]
    \item
      Es gilt
      \[
          \ad_x([y,z])
        = [\ad_x(y), z] + [y, \ad_x(z)]
        \quad
        \text{für alle $x, y, z \in V$}.
      \]
      (Für jedes $x \in V$ ist also $\ad_x$ eine Derivation bezüglich $[-,-]$.)
  \end{enumerate}
\end{question}



\begin{question}
  Es seien $E$ und $H$ zwei Endomorphismen eines $\Cbb$-Vektorraums $V$, so dass $[H,E] = 2E$.
  \begin{enumerate}[leftmargin=*]
    \item
      Zeigen Sie, dass $E(V_\lambda(H)) \subseteq V_{\lambda + 2}(H)$ für alle $\lambda \in K$.
    \item
      Folgern Sie: Ist $V$ endlichdimensional und $H$ diagonalisierbar, so ist $E$ nilpotent.
  \end{enumerate}
\end{question}


\begin{question}
  Für einen endlichdimensionalen $\Kbb$-Vektorraum $V$ und eine Bilinearform $\beta \colon V \times V \to \Kbb$ sei
  \[
              O(\beta)
    \coloneqq \{ \phi \in \GL(V) \mid \text{$\beta(\phi(x), \phi(y)) = \beta(x,y)$ für alle $x,y \in V$} \}
  \]
  die Isometriegruppe von $\beta$, und
  \[
              \gLie(\beta)
    \coloneqq \{ f \in \End(V) \mid \text{$\beta(f(x),y) = -\beta(x, f(y))$ für alle $x, y \in V$} \}
  \]
  die assoziierte Lie-Algebra.
  \begin{enumerate}[leftmargin=*]
    \item
      Zeigen Sie, dass $O(\beta)$ eine Untergruppen von $\GL(V)$ ist.
    \item
      Zeigen Sie, dass $\gLie(\beta)$ eine Lie-Unteralgebra von $\gl(V)$ ist, d.h.\ dass für alle $f, g \in \gLie(\beta)$ auch $[f,g] \in \gLie(\beta)$.
    \item
      Zeigen Sie, dass $\exp(f) \in O(\beta)$ für alle $f \in \gLie(\beta)$.
      (\emph{Hinweis}: Die bilineare Abbildung $\beta$ ist in beiden Argumenten stetig.)
    \item
      Es sei $\Kbb = \Rbb$ und $\bil{\cdot, \cdot}$ ein Skalarprodukt auf $V$.
      Unter welchen Begriffen sind die Elemente aus $G(\bil{\cdot, \cdot})$ und $\gLie(\bil{\cdot, \cdot})$ bekannt?
  \end{enumerate}
\end{question}







\end{document}
