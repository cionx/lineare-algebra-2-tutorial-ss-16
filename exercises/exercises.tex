%\documentclass[a4paper,10pt]{article}
\documentclass[a4paper,10pt]{scrartcl}

\usepackage{../generalstyle}
\usepackage{specificstyle}


\title{Übungen zu Lineare Algebra II}
\author{Jendrik Stelzner}
\date{\today}

\begin{document}
\maketitle




% TODO: 1 + (lokal nilpotent) ist invertierbar

% TODO: Endomorphism corresponding to the matrix transpose

% TODO: Quotients and direct summands









% GENERAL STUFF


\begin{question}
  Es sei $V$ ein $K$-Vektorraum und $f, g \colon V \to V$ seien zwei Endomorphismen.
  \begin{enumerate}[leftmargin=*]
    \item
      Es sei $f \circ g = \id_V$ und $V$ sei endlichdimensional.
      Zeigen Sie, dass auch $g \circ f = \id_V$.
    \item
      Zeigen Sie, dass die Aussage nicht mehr notwendigerweise gilt, wenn $V$ unendlichdimensional ist.
  \end{enumerate}
\end{question}


\begin{question}
  Es sei $K$ ein endlicher Körper.
  \begin{enumerate}[leftmargin=*]
    \item
      Geben Sie ein Polynom $p \in K[X]$ an, so dass $p \neq 0$ aber $p(\lambda) = 0$ für alle $\lambda \in K$.
    \item
      Geben Sie ein Polynom $p \in K[X]$ an, so dass $\deg p \geq 1$, aber $p(\lambda) = 1$ für alle $\lambda \in K$.
    \item
      Folgern Sie, dass jeder algebraisch abgeschlossene Körper unendlich ist.
  \end{enumerate}
\end{question}


\begin{question}
  Es seien $V$ und $W$ zwei $K$-Vektorräume, so dass $V$ endlichdimensional ist, und $f \colon V \to W$ eine lineare Abbildung.
  Zeigen Sie die Dimensionsformel
  \[
    \dim V = \dim \ker V + \dim \im V.
  \]
\end{question}


\begin{question}
  Ein Endomorphismus $f \colon V \to V$ eines $K$-Vektorraums $V$ heißt \emph{lokal nilpotent}, falls es für jedes $v \in V$ ein $n \in \Nbb$ mit $f^n(v) = 0$ gibt.
  \begin{enumerate}[leftmargin=*]
    \item
      Zeigen Sie, dass jeder nilpotente Endomorphismus auch lokal nilpotent ist.
    \item
      Zeige Sie, dass $0$ der einzige mögliche Eigenwert eines lokal nilpotenten Endomorphismus ist.
    \item
      Geben Sie ein Beispiel für einen Vektorraum $V$ und einen Endomorphismus $f \colon V \to V$ an, so dass $f$ zwar lokal nilpotent, nicht aber nilpotent ist.
    \item
      Zeigen Sie, dass jeder lokal nilpotente Endomorphismus eines endlichdimensionalen Vektorraums bereits nilpotent ist.
  \end{enumerate}
\end{question}


\begin{question}
  Es sei $K$ ein Körper.
  \begin{enumerate}[leftmargin=*]
    \item
      Zeigen Sie für alle $A, B \in \Mat_n(K)$ die Gleichheit $\tr(AB) = \tr(BA)$.
    \item
      Folgern Sie, dass die Spur invariant unter Konjugation ist, d.h.\ dass
      \[
        \tr(S A S^{-1}) = \tr(A)
        \quad
        \text{für alle $A \in \Mat_n(K)$ und $S \in \GL_n(K)$}.
      \]
  \end{enumerate}
\end{question}


\begin{question}
  Es sei $V$ ein $K$-Vektorraum und $f \colon V \to V$ ein Endomorphismus.
  Für alle $k \in \Nbb$ sei
  \[
    R_k \coloneqq \im f^k
    \quad\text{und}\quad
    N_k \coloneqq \ker f^k.
  \]
  \begin{enumerate}[leftmargin=*]
    \item
      Zeigen Sie, dass $R_0 = V$, und dass $R_i \supseteq R_{i+1}$ für alle $i \in \Nbb$.
      Es gibt also eine absteigende Kette
      \[
        V = R_0 \supseteq R_1 \supseteq R_2 \supseteq R_3 \supseteq R_4 \supseteq \dotsb
      \]
      von Untervektorräumen.
    \item
      Zeigen Sie, dass für $i \in \Nbb$ mit $R_i = R_{i+1}$ auch $R_{i+1} = R_{i+2}$ gilt.
    \item
      Folgern Sie:
      Gilt in der obigen absteigenden Kette einmal Gleichheit, also $R_i = R_{i+1}$ für ein $i \in \Nbb$, so stabilisiert die Kette bereits, d.h.\ es gilt $R_j = R_i$ für alle $j \geq i$.
    \item
      Zeigen Sie, dass $N_0 = 0$, und dass $N_i \subseteq N_{i+1}$ für alle $i \in \Nbb$.
      Es gibt also eine aufsteigende Kette
      \[
        0 = N_0 \subseteq N_1 \subseteq N_2 \subseteq N_3 \subseteq N_4 \subseteq \dotsb
      \]
      von Untervektorräumen.
    \item
      Zeigen Sie, dass für $i \in \Nbb$ mit $N_i = N_{i+1}$ auch $N_{i+1} = N_{i+2}$ gilt.
    \item
      Folgern Sie:
      Gilt in der obigen aufsteigende Kette einmal Gleichheit, also $N_i = N_{i+1}$ für ein $i \in \Nbb$, so stabilisiert die Kette bereits, d.h.\ es gilt $N_j = N_i$ für alle $j \geq i$.
    \item
      Folgern Sie:
      Ist $V$ endlichdimensional, so stabilisieren beiden Ketten.
  \end{enumerate}
\end{question}


\begin{question}
  Es seien $f, g \colon V \to V$ zwei Endomorphismen eines $K$-Vektorraums $V$.
  Beweisen Sie entweder, dass die folgenden Aussagen im Allgemeinen gelten, oder geben Sie ein Gegenbeispiel an.
  \begin{enumerate}[leftmargin=*]
    \item
      Sind $f$ und $g$ diagonalisierbar, so ist auch $f \circ g$ diagonalisierbar.
    \item
      Kommutieren $f$ und $g$ und ist $f \circ g$ diagonalisierbar, so ist $f$ oder $g$ diagonalisierbar.
    \item
      Sind $f$ und $g$ diagonalisierbar, so ist auch $f + g$ diagonalisierbar.
    \item
      Falls $f$ und $g$ kommutieren und diagonalisierbar sind, so ist auch $f + g$ diagonalisierbar.
    \item
      Falls $f$ und $g$ kommutieren und diagonalisierbar sind, so ist $f \circ g$ invertierbar.
    \item
      Ist $f$ diagonalisierbar, so ist für jedes $p \in K[X]$ auch $p(f)$ diagonalisierbar.
  \end{enumerate}
\end{question}

\begin{solution}
  \begin{enumerate}
    \item
      Nein, braucht etwa simultan diagonalisierbar.
    \item
      Nein.
    \item
      Nein, braucht etwa simultan diagonalisierbar.
    \item
      Ja, da simultan diagonalisierbar.
    \item
      Ja, da simultan diagonalsierbar.
    \item
      Ja.
  \end{enumerate}
\end{solution}


\begin{question}
  \begin{enumerate}[leftmargin=*]
    \item
      Formulieren Sie den Satz von Cayley-Hamilton.
    \item
      Zeigen Sie den Satz für ($2 \times 2$)-Matrizen durch explizites Nachrechnen.
    \item
      Zeigen Sie den Satz für Diagonalmatrizen.
    \item
      Folgern Sie den Satz für diagonalisierbare Matrizen.
  \end{enumerate}
\end{question}


\begin{question}
  Ein Endomorphismus $f \colon V \to V$ eines $K$-Vektorraums $V$ heißt \emph{algebraisch (über $K$)}, falls es ein Polynom $P \in K[T]$ mit $P \neq 0$ gibt, so dass $P(f) = 0$ gilt.
  \begin{enumerate}[leftmargin=*]
    \item
      Zeigen Sie, dass jeder Endomorphismus eines endlichdimensionalen Vektorraums algebraisch ist.
    \item
      Geben Sie ein Beispiel für einen $K$-Vektorraum $V$ und einen Endomorphismus $f \colon V \to V$ an, der nicht algebraisch ist.
    \item
      Entscheiden Sie, ob die lineare Abbildung $K[X] \to K[X]$, $p \mapsto X \cdot p$ algebraisch ist.
    \item
      Zeigen Sie, dass ein diagonalisierbarer Endomorphismus genau dann algebraisch ist, wenn er nur endlich viele Eigenwerte hat.
  \end{enumerate}
\end{question}


\begin{question}
  Es sei $V$ ein Vektorraum und $f \colon V \to V$ ein Endomorphismus.
  Es sei $(U_i)_{i \in I}$ eine Familie von $f$-invarianten Untervektorräumen, und $U \subseteq V$ ein $f$-invarianter Untervektorraum.
  Zeigen Sie:
  \begin{enumerate}[leftmargin=*]
    \item
      Auch der Schnitt $\bigcap_{i \in I} U_i$ ist $f$-invariant.
    \item
      Auch die Summe $\sum_{i \in i} U_i$ ist $f$-invariant.
  \end{enumerate}
\end{question}


\begin{question}
  Es sei $V$ ein $K$-Vektorraum, $f \colon V \to V$ ein Automorphismus und $U \subseteq V$ ein $f$-invarianter Untervektorraum.
  \begin{enumerate}[leftmargin=*]
    \item
      Zeigen Sie:
      Ist $U$ endlichdimensional, so ist $U$ auch invariant unter $f^{-1}$.
    \item
      Zeigen Sie, dass die Aussage nicht gelten muss, falls $U$ unendlichdimensional ist.
  \end{enumerate}
\end{question}


\section{Direkte Summen}





%%% PRIORITY 1


\begin{question}[subtitle = Definition der direkten Summe]{1}
  Es sei $V$ ein Vektorraum und $(U_i)_{i \in I}$ eine Familie von Untervektorräumen $U_i \subseteq V$.
  Definieren Sie, wann $V = \bigoplus_{i \in I} U_i$.
\end{question}





%%% PRIORITY 2


\begin{question}[subtitle = Multiple Choice für Direkte Summen]{2}
  Es sei $V$ ein $K$-Vektorraum.
  Entscheiden Sie, welche der folgenden Aussagen allgemein gültig sind.
  Geben Sie gegebenfalls ein Gegenbeispiel an.
  \begin{enumerate}[leftmargin=*]
    \item
      Ist $V = U \oplus W_1 = U \oplus W_2$ für Untervektorräume $U, W_1, W_2 \subseteq V$, so ist $W_1 = W_2$.
    \item
      Ist $V = V_1 \oplus V_2$ für Untervektorräume $V_1, V_2 \subseteq V$, so gilt für jeden Untervektorraum $U \subseteq V$ die Zerlegung
      \[
        U = (U \cap V_1) \oplus (U \cap V_2).
      \]
    \item
      Ist $f \colon V \to V$ ein Endomorphismus und $U \subseteq V$ ein $f$-invarianter Untervektorraum, so gibt es einen $f$-invarianten Untervektorraum $W \subseteq V$ mit $V = U \oplus W$.
    \item
      Für alle Untervektorräume $W, U_1, U_2 \subseteq V$ mit $U_1 \subseteq U_2$ gilt
      \[
        (U_1 + W) \cap U_2 =  U_1 + (W \cap U_2).
      \]
    \item
      Sind $U_1, U_2, W \subseteq V$ Untervektorräume mit $U_1 \subseteq U_2$ und $V = U_1 \oplus W$, so ist
      \[
        U_2 = U_1 \oplus (W \cap U_2).
      \]
    \item
      Ist $\mc{E} \subseteq V$ ein Erzeugendensystem von $V$ und $U \subseteq V$ ein Untervektorraum, so ist der Schnitt $\mc{E} \cap U$ ein Erzeugendensystem von $U$.
    \item
      Ist $(U_i)_{i \in I}$ eine Famlie von Untervektorräumen $U_i \subseteq V$ mit $V = \sum_{i \in I} U_i$ und $U_i \cap U_j = 0$ für alle $1 \leq i \neq j \leq n$, so ist $V = \bigoplus_{i \in I} U_i$.
  \end{enumerate}
\end{question}





%%% PRIORITY 3


\begin{question}[subtitle = Äquivalenz von Idempotenten und direkten Summen]{3}
  Es seien $V$ ein $K$-Vektorraum.
  \begin{enumerate}[leftmargin=*]
    \item
      Zeigen Sie, dass sich durch jeden idempotenten Endomorphismus $e \colon V \to V$ (d.h.\ $e^2 = e$) eine Zerlegung
      \[
        V = \im e \oplus \ker e
      \]
      ergibt, und dass
      \[
        e(v + w) = v
        \quad
        \text{für alle $v \in \im e$ und $w \in \ker e$}.
      \]
    \item
      Zeigen, Sie, dass für jeden idempotenten Endomorphismus $e \colon V \to V$ auch $\id_V - e$ idempotent ist, und dass $\im (\id_V - e) = \ker e$ und $\ker (\id_V - e) = \im e$.
    \item
      Es sei $(U_1, U_2)$ ein Paar von Untervektorräumen $U_1, U_2 \subseteq V$ mit $V = U_1 \oplus U_2$.
      Zeigen Sie, dass es einen eindeutigen Endomorphismus $p_{U_1, U_2} \colon V \to V$ gibt, so dass
      \[
          p_{U_1, U_2}(u_1 + u_2)
        = u_1
        \quad
        \text{für alle $u_1 \in U_1$ und $u_2 \in U_2$}.
      \]
    \item
      Zeigen Sie, dass die obigen Konstruktionen wie folgt eine Bijektion ergeben.
      \begin{align*}
        \left\{
          (U_1, U_2)
          \,\middle|\,
          \begin{tabular}{c}
            $U_1, U_2 \subseteq V$ sind \\
            Untervektorräume            \\
            mit $V = U_1 \oplus U_2$
          \end{tabular}
        \right\}
        &\longleftrightarrow
        \{ e \in \End(V) \mid \text{$e$ ist idempotent} \},
      \\
        (U_1, U_2) &\longmapsto p_{U_1, U_2},
      \\
        (\im e, \ker e) &\longmapsfrom e.
      \end{align*}
    \item
      Auf der linken Seite der obigen Bijektion gibt es eine Involution $(U_1, U_2) \mapsto (U_2, U_1)$.
      Zeigen Sie, dass dies unter der gegebenen Bijektion der Involutions $e \mapsto \id_V - e$ auf der rechten Seite entspricht.
  \end{enumerate}
\end{question}


\begin{question}[subtitle = Direkte Summen durch Splits]{3}
  Es seien $V$ und $W$ zwei $K$-Vektorräume, und $f \colon V \to W$ sei eine lineare Abbildung, die ein lineares Rechtsinverses $g \colon W \to V$ besitzt.
  Zeigen Sie auf die folgenden beiden Weisen, dass
  \[
    V = \ker f \oplus \im g.
  \]
  \begin{enumerate}[leftmargin=*]
    \item
      Durch explizites Nachrechnen, dass $V = \ker f + \im g$ und $\ker f \cap \im g = 0$.
    \item
      Durch geschickte Betrachtung des Endomorphismus $gf \colon V \to V$.
  \end{enumerate}
\end{question}


\begin{question}[subtitle = Diagonalisierbarkeit involutiver Endomorphismen]{3}
  Es sei $V$ ein $K$-Vektorraum und $f \colon V \to V$ ein Endomorphismus mit $f^2 = 1$.
  \begin{enumerate}[leftmargin=*]
    \item
      Zeigen Sie für $\ringchar K \neq 2$, dass $V = V_1(f) \oplus V_{-1}(f)$, dass also $f$ diagonalisierbar mit möglichen Eigenwerten $1$ und $-1$ ist.
    \item
      Zeigen Sie, dass die Aussage für $\ringchar K = 2$ nicht mehr gelten muss.
  \end{enumerate}
\end{question}


\begin{question}[subtitle = Idempotente konstruieren]{3}
  Zeigen Sie im Folgenden jeweils, dass der Vektorraum $V$ die direkte Summe der Untervektorräume $U_1$ und $U_2$ ist, indem Sie einen idempotenten Endomorphisus $e \colon V \to V$ mit $U_1 = \im e$ und $U_2 = \ker e$ angeben.
  \begin{enumerate}[leftmargin=*]
    \item
      Es sei $\ringchar K \neq 2$, $V \coloneqq \Mat_n(K)$ der $K$-Vektorraum der ($n \times n$)-Matrizen über $K$,
      \[
        U_1 \coloneqq \{ A \in \Mat_n(K) \mid A^T = A \}
      \]
      der Untervektorraum der symmetrischen Matrizen, und
      \[
        U_2 \coloneqq \{ A \in \Mat_n(K) \mid A^T = -A \}
      \]
      der Untervektorraum der schiefsymmetrischen Matrizen.
    \item
      Es sei $V \coloneqq \{ f \mid f \colon \Rbb \to \Rbb \}$ der $\Rbb$-Vektorraum der reellwertigen Funktionen auf $\Rbb$, sowie
      \[
        U_1 \coloneqq \{ f \in V \mid \text{$f(-x) = f(x)$ für alle $x \in \Rbb$} \}
      \]
      der Untervektorraum der geraden Funktionen und
      \[
        U_2 \coloneqq \{ f \in V \mid \text{$f(-x) = -f(x)$ für alle $x \in \Rbb$} \}
      \]
      der Untervektorraum der ungeraden Funktionen.
    \item
      Als $\Rbb$-Vektorraum die Ebene $V = \Rbb^2$ und als Untervektorräume die beiden Geraden
      \[
        U_1 \coloneqq \Rbb \vect{1 \\  1}
        \quad\text{und}\quad
        U_2 \coloneqq \Rbb \vect{\phantom{-}1 \\ -1}.
      \]
%     \item
%       Für $\ringchar K \neq 2$ und einen Vektorraum $W$ sei
%       \[
%                   V
%         \coloneqq \{b \colon W \times W \to K \mid \text{$b$ ist bilinear}\}
%       \]
%       der Vektorraum der Bilinearformen auf $W$.
%       Es sei
%       \[
%                   U_1
%         \coloneqq \{ s \in V \mid \text{$s$ ist symmetrisch} \}
%       \]
%       der Untervektorraum der symmetrischen Bilinearformen, und
%       \[
%                   U_2
%         \coloneqq \{ a \in V \mid \text{$a$ ist antisymmetrisch} \}
%       \]
%       der Untervektorraum der alternierenden Bilinearformen.
    \item
      Der $\Rbb$-Vektorraum $V \coloneqq \mathcal{C}(I, \Rbb)$ der stetigen reellwertigen Funktionen auf dem Einheitsintervall $I = [0,1]$ mit den Untervektorräumen
      \[
        U_1 \coloneqq \{f \in V \mid f(0) = 0\}
        \quad\text{und}\quad
        U_2 \coloneqq \{f \in V \mid \text{$f$ ist konstant}\}.
      \]
%     \item
%       Es sei erneut $V \coloneqq \Cbb(I, \Rbb)$ der Vektorraum der stetigen reellwertigen Funktionen auf dem Einheitsintervall $I = [0,1]$.
%      Es sei nun
%       \[
%         U_1 \coloneqq \{ f \in V \mid f(0) = f(1) = 0 \}
%       \]
%       der Untervektorraum der Funktion mit Nullrandwerten, und
%       \[
%         U_2 \coloneqq \{ h_{x,y} \mid x, y \in \Rbb \}
%       \]
%       der Untervektorraum der affin-linearen Funktionen, wobei
%       \[
%         h_{x,y} \colon I \to \Rbb,
%         \quad
%         t \mapsto (1-t)x + ty = x + t(y-x)
%       \]
%       die affin lineare Funktion mit den Randwerten $x$ und $y$ ist.
%       
%       (\emph{Hinweis}:
%        Es hilft, sich diese Zerlegung anschaulich vorzustellen.)
    \item
      Für einen Körper $K$ mit $\ringchar K \nmid n$ der $K$-Vektorraum $V \coloneqq \Mat_n(K)$ der $(n \times n)$-Matrizen über $K$, und die Untervektorräume der spurlosen Matrizen und der Skalarmatrizen, d.h.\
      \[
        U_1 \coloneqq \slLie_n(K) = \{ A \in \Mat_n(K) \mid \tr A  = 0 \}
        \quad\text{und}\quad
        U_2 \coloneqq K I = \{ \lambda I \mid \lambda \in K \}
      \]
    \item
      Es sei $V$ ein $K$-Vektorraum und $f \colon V \to V$ ein Endomorphismus, so dass es $\lambda, \mu \in K$ mit $\lambda \neq \mu$ und $(f-\lambda)(f-\mu) = 0$ gibt.
      Es seien $U_1 = V_\lambda(f)$ und $U_2 = V_\mu(f)$.
      
      (\emph{Hinweis}:
       Die Behauptung ist also, dass $f$ diagonalisierbar mit Eigenwerten $\lambda$ und $\mu$ ist.)
  \end{enumerate}
\end{question}


\begin{question}[subtitle = Äquivalenz von complete sets of orthogonal idempotents und endlichen direkten Summen]{3}
  Es sei $V$ ein $K$-Vektorraum.
  Eine Kollektion $e_1, \dotsc, e_n \in \End(V)$ von Endomorphismen heißt \emph{complete set of orthogonal idempotents} falls die folgenden Bedingungen erfüllt sind:
  \begin{itemize}
    \item
      Für alle $i = 1, \dotsc, n$ ist $e_i$ idempotent, also $e_i^2 = e_i$ (\emph{idempotents}).
    \item
      Für alle $1 \leq i \neq j \leq n$ ist $e_i e_j = 0$ (\emph{orthogonal}).
    \item
      Es gilt $\id_V = e_1 + \dotsb + e_n$ (\emph{complete}).
  \end{itemize}
  \begin{enumerate}[leftmargin=*]
    \item
      Es sei $e_1, \dotsc, e_n \colon V \to V$ ein \emph{complete set of orthogonal idempotents}.
      Zeigen Sie, dass
      \[
        V = \im e_1 \oplus \dotsb \oplus \im e_n.
      \]
    \item
      Es seien $U_1, \dotsc, U_n \subseteq V$ Untervektorräume mit $V = U_1 \oplus \dotsb \oplus U_n$.
      Zeigen Sie, dass es für alle $i = 1, \dotsc, n$ einen eindeutigen Endomorphismus $p^{(i)}_{U_1, \dotsc, U_n} \colon V \to V$ mit
      \[
          p^{(i)}_{U_1, \dotsc, U_n}(u_1 + \dotsb + u_n)
        = u_i
        \quad
        \text{für alle $u_1 \in U_1, \dotsc, u_n \in U_n$},
      \]
      gibt.
      Zeigen Sie ferner, dass $p^{(1)}_{U_1, \dots, U_n}, \dotsc, p^{(n)}_{U_1, \dotsc, U_n}$ ein \emph{complete set of orthogonal idempotents} ist.
    \item
      Zeigen Sie, dass die obigen Konstruktionen wie folgt eine Bijektion ergeben:
      \begin{align*}
        \left\{
          (U_1, \dotsc, U_n)
          \,\middle|\,
          \begin{tabular}{c}
            $U_1, \dotsc, U_n \subseteq V$      \\
            sind Untervek-                      \\
            torräume mit                        \\
            $U = U_1 \oplus \dotsb \oplus U_n$
          \end{tabular}
        \right\}
        &\longleftrightarrow
        \left\{
          (e_1, \dotsc, e_n)
          \,\middle|\,
          \begin{tabular}{c}
            $e_1, \dotsc, e_n \in \End(V)$  \\
            ist ein \emph{complete set}     \\
            \emph{of orthogonal}            \\
            \emph{idempotents}
          \end{tabular}
        \right\}
        \\
        (U_1, \dotsc, U_n)
        &\longmapsto
        \left( p^{(1)}_{U_1, \dotsc, U_n}, \dotsc, p^{(n)}_{U_1, \dotsc, U_n} \right)
        \\
        (\im e_1, \dotsc, \im e_n)
        &\longmapsfrom
        (e_1, \dotsc, e_n)
      \end{align*}
    \item
      Es sei $f \colon V \to V$ ein diagonalisierbarer Endomorphismus mit Eigenwerten $\lambda_1, \dotsc, \lambda_n \in K$.
      Es sei $e_1, \dotsc, e_n \in K$ das \emph{complete set of orthogonal idempotents}, dass der Zerlegung
      \[
        V = V_{\lambda_1}(f) \oplus \dotsb \oplus V_{\lambda_n}(f)
      \]
      entspricht, d.h.\ für alle $i = 1, \dotsc, n$ sei $e_i = p^{(i)}_{V_{\lambda_1}(f), \dots, V_{\lambda_n}(f)}$.
      Geben Sie eine Formel an, durch die sich $e_i$ aus $f$ ergibt.
  \end{enumerate}
\end{question}


\begin{question}[subtitle = Complete sets of orthogonal idempotents]{3}
  Es sei $V$ ein $K$-Vektorraum und $e_1, \dotsc, e_n \in \End(V)$ sei eine Kollektion von Endomorphismen mit den folgenden Eigenschaften:
  \begin{itemize}
    \item
      Für alle $i = 1, \dotsc, n$ ist $e_i$ idempotent, also $e_i^2 = e_i$.
    \item
      Die idempotenten Endomorphismen $e_1, \dotsc, e_n$ sind paarweise orthogonal, d.h.\ es ist $e_i e_j = 0$ für alle $1 \leq i \neq j \leq n$.
    \item
      Es gilt $\id_V = e_1 + \dotsb + e_n$.
  \end{itemize}
  Man sagt, dass $e_1, \dotsc, e_n$ ein \emph{complete set of orthogonal idempotents} ist.
  \begin{enumerate}[leftmargin=*]
    \item
      Zeigen Sie, dass $V = \im e_1 \oplus \dotsb \oplus \im e_n$ gilt.
    \item
      Zeigen Sie für alle $i = 1, \dotsc, n$, dass $\im e_i = V_1(e_i)$ und $\bigoplus_{j \neq i} \im e_j = \ker e_i$ gelten.
    \item
      Folgern Sie, dass es für jeden idempotenten Endomorphismus $e \colon V \to V$ eine Zerlegung
      \[
        V = \im e \oplus \ker e
      \]
      mit $\im e = V_1(e)$ gibt.
      
      (\emph{Hinweis}:
       Erweitern Sie $e$ zu einem complete set of idempotents, dass die Zerlegung liefert.)
    \item
      Für alle $i = 1, \dotsc, n$ sei $E_{ii} \in \Mat_n(K)$ die Matrix mit $1$ als $i$-ten Diagonaleintrag, und alle anderen Einträge sind $0$.
      Zeigen Sie, dass die Endomorphismen $e_1, \dotsc, e_n$ mit
      \[
        e_i \colon \Mat_n(K) \to \Mat_n(K),
        \quad
        A \mapsto A E_{ii}
      \]
      ein \emph{complete set of orthogonal idempotents} bildet, und bestimmen Sie die Zerlegung
      \[
        \Mat_n(K) = \im e_1 \oplus \dotsb \oplus \im e_n.
      \]
  \end{enumerate}
\end{question}





%%% PRIORITY 4


\begin{question}[subtitle = Eine Charakterisierung von Diagonalisierbarkeit über direkte Komplemente]{4}
  Es sei $K$ ein algebraisch abgeschlossener Körper und $f \colon V \to V$ ein Endomorphismus eines endlichdimensionalen $K$-Vektorraums $V$.
  Zeigen Sie, dass die folgenden beiden Aussagen äquivalent sind:
  \begin{enumerate}
    \item
      $f$ ist diagonalisierbar.
    \item
      Für jeden $f$-invarianten Untervektorraum $U \subseteq V$ gibt es einen $f$-invarianten Untervektorraum $W \subseteq V$ mit $V = U \oplus W$.
  \end{enumerate}
\end{question}


\begin{question}[subtitle = Ein Kriterium für Diagonalisierbarkeit mithilfe von complete sets of orthogonal idempotents]{4}
  Es sei $V$ ein $K$-Vektorraum.
  \begin{enumerate}[leftmargin=*]
    \item
      Es seien $e_1, \dotsc, e_n \in \End(V)$ Endomorphismen mit den folgenden Eigenschaften:
      \begin{itemize}
        \item
          Für alle $i = 1, \dotsc, n$ ist $e_i$ idempotent, also $e_i^2 = e_i$.
        \item
          Die idempotenten Endomorphismen $e_1, \dotsc, e_n$ sind paarweise orthogonal, d.h.\ es ist $e_i e_j = 0$ für alle $1 \leq i \neq j \leq n$.
        \item
          Es gilt $\id_V = e_1 + \dotsb + e_n$.
      \end{itemize}
      Man nennt $e_1, \dotsc, e_n$ ein \emph{complete set of orthogonal idempotents}.
      Zeigen Sie, dass
      \[
        V = \im e_1 \oplus \dotsb \oplus \im e_n.
      \]
  \end{enumerate}
  Es sei nun $f \colon V \to V$ ein Endomorphismus.
  Wir nehmen zunächst an, dass $f$ diagonalisierbar mit paarweise verschiedenen Eigenwerten $\lambda_1, \dotsc, \lambda_n \in K$ ist.
  \begin{enumerate}[resume]
    \item
      Zeigen Sie, dass $(f - \lambda_1) \dotsm (f - \lambda_n) = 0$.
    \item
      Folgern Sie aus der Eigenraumzerlegung $V = V_{\lambda_1}(f) \oplus \dotsb \oplus V_{\lambda_n}(f)$, dass es für alle $i = 1, \dotsc, n$ eine eindeutige lineare Abbildung $e_i \colon V \to V$ gibt, so dass
      \[
          e_i(v_1 + \dotsb + v_n)
        = v_i
        \quad
        \text{für alle $v_1 \in V_{\lambda_1}(f), \dotsc, v_n \in V_{\lambda_n}(f)$}.
      \]
      (Die Abbildungen $e_1, \dotsc, e_n$ sind also die Projektionen auf die einzelnen Eigenräume bezüglich der Eigenraumzerlegung.)
    \item
      Zeigen Sie, dass die Endomorphismen $e_1, \dotsc, e_n$ ein \emph{complete set of orthogonal idempotents} bilden.
    \item
      Zeigen Sie, dass $\im e_i = V_{\lambda_i}(f)$ für alle $i = 1, \dotsc, n$.
      Die Zerlegung $V = \im e_1 \oplus \dotsb \oplus e_n$ stimmt also mit der Eigenraumzerlegung von $V$ bezüglich $f$ überein.
    \item
      Zeigen Sie, dass
      \[
          e_i
        = \prod_{j \neq i} \frac{f - \lambda_j}{\lambda_i - \lambda_j}
        = \frac{\prod_{j \neq i} (f-\lambda_j)}{\prod_{j \neq i} (\lambda_i - \lambda_j)}
        \quad
        \text{für alle $i = 1 \dotsc, n$}.
      \]
      
      (\emph{Hinweis}:
       Wenden Sie den rechten Ausdruck auf die Eigenräume von $f$ an.)
  \end{enumerate}
  Wir nehmen nun umgekehrt an, dass $(f - \lambda_1) \dotsm (f - \lambda_n) = 0$ für paarweise verschiedene Skalare $\lambda_1, \dotsc, \lambda_n \in K$.
  Für alle $i = 1, \dotsc, n$ sei
  \[
              e_i
    \coloneqq \prod_{j \neq i} \frac{f - \lambda_j}{\lambda_i - \lambda_j}
    =         \frac{\prod_{j \neq i} (f-\lambda_j)}{\prod_{j \neq i} (\lambda_i - \lambda_j)}.
  \]
  \begin{enumerate}[resume]
    \item
      Zeigen Sie, dass die Endomorphismen $e_1, \dotsc, e_n$ idempotent sind, indem Sie zeigen, dass
      \[
        e_i^2 - e_i = 0
        \quad
        \text{für alle $i = 1, \dotsc, n$}.
      \]
    \item
      Zeigen Sie, dass die idempotenten Endomorphismen $e_1, \dotsc, e_n$ orthogonal sind.
    \item
      Zeigen Sie, dass $\id_V = e_1 + \dotsb + e_n$.
      Gehen Sie hierfür wie folgt vor:
      
      Für alle $i = 1, \dotsc, n$ sei
      \[
                  P_i(T)
        \coloneqq \prod_{j \neq i} \frac{T - \lambda_j}{\lambda_i - \lambda_j}
        =         \frac{\prod_{j \neq i} (T-\lambda_j)}{\prod_{j \neq i} (\lambda_i - \lambda_j)}
        \in K[T].
      \]
      Zeigen Sie für alle $i = 1, \dotsc, n$, dass $P_i$ ein Polynom vom Grad $n-1$ ist, so dass $e_i = P_i(f)$.
      Zeigen Sie auch, dass $P_i(\lambda_i) = 1$ und $P_i(\lambda_j) = 0$ für alle $1 \leq i \neq j \leq n$.
      
      Folgern Sie für das Polynom $P(T) \coloneqq 1 - \sum_{i=1}^n P_i(T)$, dass $\deg P \leq n-1$, und dass $P(\lambda_i) = 0$ für alle $i = 1, \dotsc, n$.
      Folgern Sie, dass $P = 0$, und somit $1 = \sum_{i=1}^n P_i(T)$.
      
      Folgern Sie durch Einsetzen von $f$, dass $\id_V = \sum_{i=1}^n e_i$.
  \end{enumerate}
  Also ist $e_1, \dotsc, e_n$ ein \emph{complete set of orthogonal idempotents}, und somit $V = \im e_1 \oplus \dotsb \oplus \im e_n$.
  \begin{enumerate}[resume]
    \item
      Zeigen Sie, dass $\im e_i \subseteq V_{\lambda_i}(f)$ für alle $i = 1, \dotsc, n$.
      
      (\emph{Hinweis}:
       Überlegen sie sich, dass $(f - \lambda_i) e_i = 0$.)
    \item
      Folgern Sie mithilfe der Zerlegung $V = \im e_1 \oplus \dotsb \oplus \im e_n$, dass $V$ diagonalisierbar ist, und dass $\im e_i = V_{\lambda_i}(f)$ für alle $i = 1, \dotsc, n$.
  \end{enumerate}
  Ingesamt zeigt dies, dass genau dann $(f - \lambda_1) \dotsm (f - \lambda_n) = 0$ für paarweise verschieden $\lambda_1, \dotsc, \lambda_n \in K$, wenn $f$ diagonalisierbar ist und $\lambda_1, \dotsc, \lambda_n \in K$ die einzigen möglichen Eigenwerte von $f$ sind.
  \begin{enumerate}[resume]
    \item 
      Es sei nun $K = \Cbb$.
      Folgern Sie, dass $f$ in den folgenden Fällen diagonalisierbar ist, und bestimmen Sie jeweils die möglichen Eigenwerte:
      \begin{itemize}
        \item
          Es gilt $f^2 = f$,
        \item
          es gilt $f^3 = f$,
        \item
          es gilt $f^3 = -f$,
        \item
          es gilt $f^n = \id_V$ für ein $n \geq 1$.
      \end{itemize}
  \end{enumerate}
\end{question}


\section{Quotientenvektorräume}


\begin{question}
  Es seien $V$ und $W$ zwei $K$-Vektorräume und $f \colon V \to W$ sei eine lineare Abbildung.
  \begin{enumerate}[leftmargin=*]
    \item
      Es sei $U \subseteq V$ ein Untervektorraum mit $f|_U = 0$.
      Zeigen Sie, dass $f$ eine lineare Abbildung
      \[
        \bar{f} \colon V\!/U \to W,
        \quad
        [v] \mapsto f(v)
      \]
      induziert.
    \item
      Zeigen Sie, dass $\im \bar{f} = \im f$.
      Folgern Sie, dass $\bar{f}$ genau dann surjektiv ist, wenn $f$ surjektiv ist.
    \item
      Zeigen Sie, dass $U \subseteq \ker f$, und dass $\ker \bar{f} = (\ker f)/U$.
      Folgern Sie, dass $\bar{f}$ genau dann injektiv ist, wenn bereits die Gleichheit $U = \ker f$ gilt.
    \item
      Folgern Sie, dass $f$ einen Isomorphismus $V/(\ker f) \to \im f$, $[v] \mapsto f(v)$ induziert.
  \end{enumerate}
\end{question}


\begin{question}
  Zeigen Sie, dass eine Teilmenge $U \subseteq V$ eines $K$-Vektorraums $V$ genau dann ein Untervektorraum ist, wenn es einen $K$-Vektorraum $W$ und eine lineare Abbildung $f \colon V \to W$ gibt, so dass $U = \ker f$.
\end{question}


\begin{question}
  Es sei $V$ ein $K$-Vektorraum mit Erzeugendensystem $E \subseteq V$.
  Es sei $W$ ein $K$-Vektorraum mit Basis $(b_e)_{e \in E}$.
  Konstruieren Sie einen Isomorphismus $W\!/U \to V$ für einen passenden Untervektorraum $U \subseteq W$.
\end{question}


\begin{question}
  Es sei $V$ ein $K$-Vektorraum mit zwei Untervektorräumen $U_1, U_2 \subseteq V$.
  Zeigen Sie die folgenden beiden Isomorphiesätze:
  \begin{enumerate}[leftmargin=*]
    \item
      Die Inklusion $U_1 \to U_1 + U_2$, $x \mapsto x$ induziert einen isomorphismus
      \[
        U_1 / (U_1 \cap U_2) \to (U_1 + U_2) / U_2,
        \quad
        [x] \mapsto [x]
        \quad
        \text{für alle $x \in V$}.
      \]
    \item
      Ist $U_1 \subseteq U_2$, so ist $U_2 / U_1$ ein Untervektorraum von $V / U_1$, und die Abbildung
      \[
        (V \! / U_1) / (U_2 / U_1) \to V \! / U_2,
        \quad
        [[x]] \mapsto [x]
        \quad
        \text{für alle $x \in V$}.
      \]
      ist ein wohldefinierter Isomorphismus.
  \end{enumerate}
\end{question}


\begin{question}
  Es sei $V$ ein $K$-Vektorraum und $U \subseteq V$ ein Untervektorraum.
  Konstruieren Sie für den Annihilator
  \[
      U^\circ
    = \{ \varphi \in V^* \mid \text{$\varphi(u) = 0$ für alle $u \in U$} \}
  \]
  einen Isomorphismus $F \colon U^\circ \to (V\!/U)^*$.
\end{question}


\begin{question}
  Es sei $V$ ein $K$-Vektorraum und $\sim$ eine Äquivalenzrelation auf $V$, so dass auf $V/{\sim}$ die Addition
  \[
      \overline{v} + \overline{w}
    = \overline{v + w}
    \quad
    \text{für alle $v, w \in V$}
  \]
  und die Skalarmultiplikation
  \[
      \lambda \cdot \overline{v}
    = \overline{\lambda \cdot v}
    \quad
    \text{für alle $\lambda \in K$, $v \in V$}
  \]
  wohldefiniert sind.
  \begin{enumerate}[leftmargin=*]
    \item
      Zeigen Sie, dass $V\!/{\sim}$ mit den obigen Operationen ein $K$-Vektorraum ist, und dass die Äquivalenzklasse $\overline{0}$ das Nullelement von $V/{\sim}$ ist.
    \item
      Zeigen Sie, dass die kanonische Abbildung $\rho \colon V \to V/{\sim}$ mit $v \mapsto \overline{v}$ ein Epimorphismus ist.
    \item
      Zeigen Sie für $U \coloneqq \ker \rho$, dass
      \[
        v \sim w
        \iff
        v - w \in U
        \quad
        \text{für alle $v, w \in V$}.
      \]
    \item
      Folgern Sie, dass $V/{\sim} = V/U$, und dass $\rho$ die kanonische Projektion des Quotientenvektorraums ist.
  \end{enumerate}
\end{question}


\begin{question}
  Es sei $V$ ein $\Kbb$-Vektorraum.
  Eine Abbildung $[\,\cdot\,] \colon V \to V$ heißt \emph{Seminorm}, falls
  \begin{itemize}
    \item
      $[\lambda x] = |\lambda| [x]$ für alle $\lambda \in \Kbb$ und $x \in V$ (Homogenität), und
    \item
      $[x + y] \leq [x] + [y]$ für alle $x, y \in V$ (Dreiecksungleichung).
  \end{itemize}
  Zeigen Sie:
  \begin{enumerate}[leftmargin=*]
    \item
      Die Teilmenge $N \coloneqq \{x \in V \mid [x] = 0\}$ ist ein Untervektorraum von $V$.
    \item
      Die Seminorm $[\,\cdot\,]$ induziert auf $V\!/U$ eine Norm $\|\cdot\|$ durch
      \[
        \| \overline{x} \| \coloneqq [x]
        \quad
        \text{für alle $x \in V$}.
      \]
  \end{enumerate}
\end{question}


\begin{question}
  Es seien $V$ und $W$ zwei $K$-Vektorräume und $f \colon V \to W$ eine lineare Abbildung.
  Es sei
  \[
    i \colon \ker f \to V,
    \quad
    v \mapsto v
  \]
  die kanonische Inklusion und
  \[
    p \colon W \to \coker f,
    \quad
    w \mapsto [w]
  \]
  die kanonische Projektion.
  \begin{enumerate}[leftmargin=*]
    \item
      Zeigen Sie, dass $f \circ i = 0$ und $p \circ f = 0$.
    \item
      Zeigen Sie, dass es für jeden $K$-Vektorraum $U$ und jede lineare Abbildung $h \colon U \to V$ mit $f \circ h = 0$ eine eindeutige lineare Abbildung $\bar{h} \colon U \to \ker f$ gibt, so dass das folgende Diagram kommutiert:
      \[
        \begin{tikzcd}[row sep = large, column sep = large, ampersand replacement = \&]
                \ker f  \arrow{r}{i}
            \&  V       \arrow{r}{f}
            \&  W
          \\
                U       \arrow[swap]{ru}{h}
                        \arrow[dashed]{u}{\bar{h}}
            \&  {}
            \&  {}
        \end{tikzcd}
      \]
    \item
      Zeigen Sie, dass es für jeden $K$-Vektorraum $U$ und jede lineare Abbildung $g \colon W \to U$ mit $g \circ f = 0$ eine eindeutige lineare Abbildnug $\bar{g} \colon \coker f \to U$ gibt, so dass das folgende Diagram kommutiert:
      \[
        \begin{tikzcd}[row sep = large, column sep = large, ampersand replacement = \&]
                {}
            \&  {}
            \&  U
          \\
                V         \arrow{r}{f}
            \&  W         \arrow{r}{p}
                          \arrow{ru}{g}
            \&  \coker f  \arrow[swap, dashed]{u}{\bar{g}}
        \end{tikzcd}
      \]
  \end{enumerate}
\end{question}


\begin{question}
  Es sei $V$ ein $K$-Vektorraum und $f, g \colon V \to V$ seien Endomorphismus.
  Außdem sei $U \subseteq V$ ein Untervektorraum, der invariant unter $f$ und $g$ ist.
  \begin{enumerate}[leftmargin=*]
    \item
      Zeigen Sie: Der Endomorphismus $f$ induziert einen Endomorphismus
      \[
        \bar{f} \colon V\!/U \to V\!/U,
        \quad
        [v] \mapsto [f(v)].
      \]
      Analog induziert dann auch $g$ einen Endomorphismus $\bar{g} \colon V \to V$, $[v] \mapsto [g(v)]$.
    \item
      Es seien $f|_U = g|_U$ und $\bar{f} = \bar{g}$.
      Beweisen oder widerlegen Sie, dass bereits $f = g$ gelten muss.
  \end{enumerate}
\end{question}
\section{Komplexifizierung}


\begin{question}
  Es sei $V$ ein $\Rbb$-Vektorraum und $W$ ein $\Cbb$-Vektorraum.
  Es sei $\iota \colon V \to V$, $v \mapsto v + i \cdot 0$ die kanonische Inklusion.
  Zeigen Sie:
  \begin{enumerate}[leftmargin=*]
    \item
      Für jede $\Rbb$-lineare Abbildung $f \colon V \to W$ gibt genau eine $\Cbb$-lineare Abbildung $f^\Cbb \colon V_\Cbb \to W$, die das folgende Diagram kommutieren lässt:
      \[
        \begin{tikzcd}[row sep = large, column sep = large, ampersand replacement=\&]
                V       \arrow{d}[swap]{\iota}
                        \arrow{rd}{f}
            \&  {}
          \\
                V_\Cbb  \arrow{r}[swap]{f^\Cbb}
            \&  W
        \end{tikzcd}
      \]
    \item
      Für je zwei $\Cbb$-lineare Abbildungen $g_1, g_2 \colon V_\Cbb \to W$ gilt die Äquivalenz
      \[
        g_1 = g_2
        \iff
        g_1 \circ \iota = g_2 \circ \iota.
      \]
    \item
      Für jeden $\Cbb$-Vektorraum $W'$ gilt für jede $\Rbb$-lineare Abbildung $f \colon V \to W$ und jede $\Cbb$-lineare Abbildung $g \colon W \to W'$ die Gleichheit
      \[
        (g \circ f)^\Cbb = g \circ f^\Cbb.
      \]
  \end{enumerate}
\end{question}


\begin{question}
  Es sei $V$ ein $\Rbb$-Vektorraum.
  \begin{enumerate}
    \item
      Definieren Sie, wann ein $\Cbb$-Untervektorraum $W \subseteq V_\Cbb$ induziert ist.
    \item
      Zeigen Sie, dass für einen induzierten $\Cbb$-Untervektorraum $W \subseteq V_\Cbb$ die Menge
      \[
        U \coloneqq \{ v \in V \mid v + i \cdot 0 \in W \}
      \]
      ein $\Rbb$-Untervektorraum von $V$ ist, durch den $W$ induziert wird.
    \item
      Ist $U$ eindeutig mit dieser Eigenschaft?
    \item
      Folgern Sie, dass eine $\Cbb$-Vektorraum $W \subseteq V_\Cbb$ genau dann induziert ist, wenn $\overline{W} = W$.
    \item
      Folgern Sie, dass die Abbildung
      \[
        \left\{
          U \subseteq V
        \,\middle|\,
          \text{$U$ ist ein $\Rbb$-UVR}
        \right\}
        \to
        \left\{
          W \subseteq V_\Cbb
         \,\middle|\,
          \text{$W$ ist ein $\Cbb$-UVR}
        \right\},
        \quad
        U \mapsto U_\Cbb
      \]
      injektiv ist, und geben Sie ein Linksinverses an.
  \end{enumerate}
\end{question}


\begin{question}
  Für jeden $\Rbb$-Vektorraum $V$ sei $\iota_V \colon V \to V$, $v \mapsto v + i \cdot 0$ die kanonische Inklusion und für jeden $\Cbb$-Vektorraum $W$ und jede $\Rbb$-lineare Abbildung $f \colon V \to W$ sei $f^\Cbb \colon V_\Cbb \to W$ die eindeutige $\Cbb$-lineare Abbildung mit $f^\Cbb \circ \iota_V = f$.
  \begin{enumerate}[leftmargin=*]
    \item
      Zeigen Sie, dass für jedes $\Rbb$-Vektorraum $V$ und $\Cbb$-Vektorraum $W$ die Abbildung
      \[
        \Phi_{V,W} \colon \Hom_\Rbb(V, W) \to \Hom_\Cbb(V_\Cbb, W),
        \quad
        f \mapsto f^\Cbb
      \]
      ein Isomorphismus von $\Rbb$-Vektorräumen ist.
      Geben Sie auch $\Phi_{V,W}^{-1}$ an.
    \item
      Es seien  $V, V', W, W'$ vier $K$-Vektorräume und $g_1 \colon V' \to V$ und $g_2 \colon W \to W'$ zwei $K$-lineare Abbildungen.
      Zeigen Sie, dass die beidseitige Komposition
      \[
        g_2 \circ - \circ g_1
        \colon
        \Hom_K(V, W) \to \Hom_K(V', W'),
        \quad
        h \mapsto g_2 \circ f \circ g_1
      \]
      eine $K$-lineare Abbildung ist.
    \item
      Zeigen Sie, dass die Isomorphismen $\Phi_{V,W}$ in dem folgenden Sinne \emph{natürlich} sind:
      Es seien $V$ und $V'$ zwei $\Rbb$-Vektorräume und es sei $h \colon V' \to V$ eine $\Rbb$-lineare Abbildung.
      Es seien $W$ und $W'$ zwei $\Cbb$-Vektorräume und es sei $g \colon W \to W'$ eine $\Cbb$-lineare Abbildung.
      Dann kommutiert das folgende Diagram von $\Rbb$-Vektorräumen und $\Rbb$-linearen Abbildungen:
      \[
        \begin{tikzcd}[row sep = large, column sep = large, ampersand replacement=\&]
                \Hom_\Rbb(V, W)         \arrow[swap]{d}{g \circ - \circ h}
                                        \arrow{r}{\Phi_{V,W}}
            \&  \Hom_\Cbb(V_\Cbb, W)    \arrow{d}{g \circ - \circ h^\Cbb}
          \\
                \Hom_\Rbb(V', W')       \arrow{r}{\Phi_{V',W'}}
            \&  \Hom_\Cbb(V'_\Cbb, W')
        \end{tikzcd}
      \]

  \end{enumerate}
\end{question}


\begin{question}
  Es sei $V$ ein $\Rbb$-Vektorraum mit $\Rbb$-Basis $\mc{B} = (v_j)_{j \in J}$.
  Zeigen Sie, dass dann $\mc{B}_\Cbb = (v_j + i \cdot 0)_{j \in J}$ eine $\Cbb$-Basis von $V_\Cbb$ ist.
\end{question}


\begin{question}
  Zeigen Sie, dass die $\Rbb$-lineare Inklusion $\Rbb \to \Cbb$, $x \mapsto x$ einen Isomorphismus $\Rbb_\Cbb \to \Cbb$ von $\Cbb$-Vektorräumen induziert.
\end{question}


\begin{question}
  Es seien $V$ und $W$ zwei $\Rbb$-Vektorräume.
  Zeigen Sie, dass die $\Rbb$-lineare Abbildung
  \[
    \varphi \colon \Hom_\Rbb(V, W) \to \Hom_\Cbb(V_\Cbb, W_\Cbb),
    \quad
    f \mapsto f_\Cbb
  \]
  einen Isomorphismus von $\Cbb$-Vektorräumen
  \[
    \Phi \colon \Hom_\Rbb(V, W)_\Cbb \to \Hom_\Cbb(V_\Cbb, W_\Cbb)
  \]
  induziert.
  \newline
  (\emph{Hinweis}:
   Beachten Sie, dass $V$ und $W$ nicht notwendigerweise endlichdimensional sind.)
\end{question}


\begin{question}
  Es sei $V$ ein $\Rbb$-Vektorraum.
  Konstruieren Sie einen Isomorphismus $(V^*)_\Cbb \to (V_\Cbb)^*$.
  
  (\emph{Hinweis}:
   Beachten Sie, dass $V$ ist nicht notwendigerweise endlichdimensional ist.)
\end{question}


\begin{question}\label{qst: compatibility of sums and intersections with complexification}
  Es sei $V$ ein reeller Vektorraum und $(U_i)_{i \in I}$ eine Familie von Untervektorräumen $U_i \subseteq V$.
  Zeigen Sie:
  \begin{enumerate}[leftmargin=*]
    \item
      Es gilt
      \[
          \left( \bigcap_{i \in I} U_i \right)_\Cbb
        = \bigcap_{i \in I} (U_i)_\Cbb
      \]
    \item
      Es gilt
      \[
          \left( \sum_{i \in I} U_i \right)_\Cbb
        = \sum_{i \in I} (U_i)_\Cbb.
      \]
    \item
      Folgern Sie, dass genau dann $V = \bigoplus_{i \in I} U_i$, wenn $V_\Cbb = \bigoplus_{i \in I} (U_i)_{\Cbb}$.
  \end{enumerate}
\end{question}


\begin{question}
  Es seien $V$ und $W$ zwei reelle Vektorräume, und $f \colon V \to W$ sei eine $\Rbb$-lineare Abbildung.
  \begin{enumerate}[leftmargin=*]
    \item
      Zeigen Sie, dass $\ker (f_\Cbb) = (\ker f)_\Cbb$.
    \item
      Folgern Sie, dass $f_\Cbb$ genau dann injektiv ist, wenn $f$ injektiv ist.
    \item
      Folgern Sie ferner, dass $(V_\Cbb)_\lambda(f_\Cbb) = V_\lambda(f)_\Cbb$ für jedes $\lambda \in \Rbb$.
    \item
      Zeigen Sie, dass $\im (f_\Cbb) = (\im f)_\Cbb$.
    \item
      Folgern Sie, dass $f_\Cbb$ genau dann surjektiv ist, wenn $f$ surjektiv ist.
  \end{enumerate}
\end{question}


\begin{question}
  Es sei $V$ ein reeller Vektorraum und $f \colon V \to V$ ein Endomorphismus.
  Zeigen Sie, dass $f$ genau dann diagonalisierbar ist, wenn $f_\Cbb$ diagonalisierbar mit reellen Eigenwerten ist.
  \newline
  (\emph{Hinweis}:
   Man betrachte etwa Übung~\ref{qst: compatibility of sums and intersections with complexification}.
   Beachten Sie aber auf jeden Fall, dass $V$ nicht notwendigerweise endlichdimensional ist.)
\end{question}


\begin{question}
  Zeigen Sie, dass die kanonische Inklusion $\iota \colon \Rbb[X] \to \Cbb[X]$, $x \mapsto x$ $\Rbb$-linear ist, und einen Isomorphismus $\Rbb[X]_\Cbb \to \Cbb[X]$ von $\Cbb$-Vektorräumen induziert.
\end{question}


\begin{question}
  Es sei $\mc{B} = (b_1, \dotsc, b_n)$ eine Basis eines $\Rbb$-Vektorraums $V$ und $\mc{C} = (c_1, \dotsc, c_m)$ eine Basis eines $\Rbb$-Vektorraums $W$.
  Es seien
  \[
    \mc{B}_\Cbb \coloneqq (b_1 + i \cdot 0, \dotsc, b_n + i \cdot 0)
    \quad\text{und}\quad
    \mc{C}_\Cbb \coloneqq (c_1 + i \cdot 0, \dotsc, c_m + i \cdot 0)
  \]
  die entsprechenden $\Cbb$-Basen der Komplexifizierungen $V_\Cbb$ und $W_\Cbb$.
  Es seien
  \begin{gather*}
    \Phi^\Rbb \colon \Hom_\Rbb(V,W) \to \Mat(m \times n, \Rbb),
    \quad
    f \mapsto \Mat_{\mc{B}, \mc{C}}(f)
  \shortintertext{und}
    \Phi^\Cbb \colon \Hom_\Cbb(V_\Cbb, W_\Cbb) \to \Mat(m \times n, \Cbb),
    \quad
    g \mapsto \Mat_{\mc{B}_\Cbb, \mc{C}_\Cbb}(g).
  \end{gather*}
  Es seien
  \[
  \begin{array}{ll}
      \iota_1 \colon \Hom_\Rbb(V, W) \to \Hom_\Rbb(V,W)_\Cbb,
    & f \mapsto f + i \cdot 0,
    \\
      \iota_2 \colon \Hom_\Rbb(V,W) \to \Hom_\Cbb(V_\Cbb, W_\Cbb),
    & f \mapsto f_\Cbb
    \\
      \iota_3 \colon \Mat(m \times n, \Rbb) \to \Mat(m \times n, \Rbb)_\Cbb,
    & A \mapsto A + i \cdot 0,
    \\
      \iota_4 \colon \Mat(m \times n, \Rbb) \to \Mat(m \times n, \Cbb),
    & A \mapsto A,
  \end{array}
  \]
  die jeweiligen kanonischen Inklusionen.
  \begin{enumerate}[leftmargin=*]
    \item
      Zeigen Sie, dass das folgende Diagram kommutiert:
      \[
        \begin{tikzcd}[row sep = large, column sep = large, ampersand replacement = \&]
                \Hom_\Rbb(V, W)           \arrow{r}{\iota_2}
                                          \arrow[swap]{d}{\Phi^\Rbb}
            \&  \Hom_\Cbb(V_\Cbb, W_\Cbb) \arrow{d}{\Phi^\Cbb}
          \\
                \Mat(m \times n, \Rbb)    \arrow{r}{\iota_4}
            \&  \Mat(m \times n, \Cbb)
        \end{tikzcd}
      \]
      Folgern Sie, dass $\iota_4$ tatsächlich injektiv ist, wie der oben verwendete Begriff \emph{Inklusion} vermuten lässt.
    \item
      Zeigen Sie, dass das folgende Diagram kommutiert:
      \[
        \begin{tikzcd}[row sep = large, column sep = large, ampersand replacement = \&]
                \Hom_\Rbb(V, W)             \arrow{r}{\iota_1}
                                            \arrow[swap]{d}{\Phi^\Rbb}
            \&  \Hom_\Rbb(V, W)_\Cbb        \arrow{d}{(\Phi^\Rbb)_\Cbb}
          \\
                \Mat(m \times n, \Rbb)      \arrow{r}{\iota_3}
            \&  \Mat(m \times n, \Rbb)_\Cbb
        \end{tikzcd}
      \]
    \item
      Zeigen Sie, dass die Inklusion $\iota_2$ eine eindeutige $\Cbb$-lineare Abbildung
      \[
        \Psi_1 \colon \Hom_\Rbb(V,W)_\Cbb \to \Hom_\Cbb(V_\Cbb, W_\Cbb)
      \]
      induziert, die das folgende Diagram zum kommutieren bringt:
      \[
        \begin{tikzcd}[row sep = large, column sep = large, ampersand replacement = \&]
                  {}
              \&  \Hom_\Rbb(V,W)            \arrow[swap]{ld}{\iota_1}
                                            \arrow{rd}{\iota_2}
              \&  {}
          \\
                  \Hom_\Rbb(V,W)_\Cbb       \arrow{rr}{\Psi_1}
              \&  {}
              \&  \Hom_\Cbb(V_\Cbb, W_\Cbb).
        \end{tikzcd}
      \]
    \item
      Zeigen Sie auf analoge Weise, dass die Inklusion $\iota_4$ eine eindeutige $\Cbb$-lineare Abbildung
      \[
        \Psi_2 \colon \Mat(m \times n, \Rbb)_\Cbb \to \Mat(m \times n, \Cbb)
      \]
      induziert, die das folgende Diagram zum kommutieren bringt:
      \[
        \begin{tikzcd}[row sep = large, column sep = large, ampersand replacement = \&]
                  {}
              \&  \Mat(m \times n, \Rbb)      \arrow[swap]{ld}{\iota_3}
                                              \arrow{rd}{\iota_4}
              \&  {}
          \\
                  \Mat(m \times n, \Rbb)_\Cbb \arrow{rr}{\Psi_2}
              \&  {}
              \&  \Mat(m \times n, \Cbb)
        \end{tikzcd}
      \]
    \item
      Wir haben nun das folgende Diagram:
      \[
        \begin{tikzcd}[row sep = large, column sep = large, ampersand replacement = \&]
                  {}
              \&  \Hom_\Rbb(V, W)               \arrow[swap]{ld}{\iota_1}
                                                \arrow{rd}{\iota_2}
                                                \arrow[near end, swap]{dd}{\Phi^\Rbb}
              \&  {}
          \\
                  \Hom_\Rbb(V, W)_\Cbb          \arrow[crossing over, near start]{rr}{\Psi_1}
                                                \arrow[swap]{dd}{(\Phi^\Rbb)_\Cbb}
              \&  {}
              \&  \Hom_\Cbb(V_\Cbb, W_\Cbb)     \arrow{dd}{\Phi^\Cbb}
          \\
                  {}
              \&  \Mat(m \times n, \Rbb)        \arrow[swap]{ld}{\iota_3}
                                                \arrow{rd}{\iota_4}
              \&  {}
          \\
                  \Mat(m \times n, \Rbb)_\Cbb   \arrow{rr}{\Psi_2}
              \&  {}
              \&  \Mat(m \times n, \Cbb)
        \end{tikzcd}
      \]
      Von diesem Diagram wissen wir bereits, dass Deckel, Boden und beide Rückseiten kommutieren.
      Folgern Sie daraus, dass auch die Vorderseite kommutiert.
      
      (\emph{Hinweis}:
       Nutzen Sie, dass zwei $\Cbb$-lineare Abbildung $f, g \colon \Hom_\Rbb(V, W)_\Cbb \to \Mat(m \times n, \Cbb)$ genau dann übereinstimmen, wenn die Kompositionen $f \circ \iota_1$ und $g \circ \iota_1$ übereinstimmen.)
    \item
      Zeigen Sie, dass $\Psi_2$ ein Isomorphismus von $\Cbb$-Vektorräumen ist.
    \item
      Folgen Sie, dass auch $\Psi_1$ ein Isomorphismus von $\Cbb$-Vektorräumen ist.
  \end{enumerate}
\end{question}



% EIGENSTUFF


\begin{question}
  Es sei $V \neq 0$ ein $K$-Vektorraum, wobei $K$ algebraisch abgeschlossen ist.
  Es seien $f_1, \dotsc, f_n \colon V \to V$ paarweise kommutierende Endomorphismen.
  \begin{enumerate}
    \item
      Zeigen Sie, dass für alle $I \subseteq \{1, \dotsc, n\}$ und Skalare $\lambda_i \in K$ mit $i \in I$ der gemeinsame Eigenraum
      \[
                   V( (f_i, \lambda_i)_{i \in I} )
        \coloneqq  \{ v \in V \mid \text{$f_i(v) = \lambda_i v$ für alle $i \in I$} \}.
      \]
      invariant unter $f_1, \dotsc, f_n$ ist.
     \item
      Folgern Sie, dass die Endomorphismen $f_1, \dotsc, f_n$ einen gemeinsamen Eigenvektor besitzen, d.h.\ dass es einen Vektor $v \in V$ gibt, so dass $v$ für jedes $f_i$ eine Eigenvektor ist.
  \end{enumerate}
\end{question}


\begin{question}
  Es sei $V$ ein $K$-Vektorraum.
  Für alle Endomorphismen $f_1, \dotsc, f_n \colon V  \to V$ und Skalare (Eigenwerte) \mbox{$\lambda_1, \dotsc, \lambda_n \in K$} sei
  \[
              V(f_1, \lambda_1; \dotsc; f_n, \lambda_n)
    \coloneqq \{ v \in V \mid \text{$f_i(v) = \lambda_i v$ für alle $i = 1, \dotsc, n$} \}
  \]
  der \emph{gemeinsame Eigenraum} der Endomorphismen $f_1, \dotsc, f_n$ zu den Eigenwerten $\lambda_1, \dotsc, \lambda_n$.
  \begin{enumerate}[leftmargin=*]
    \item
      Zeigen Sie, dass
      \[
          V(f_1, \lambda_1; \dotsc; f_n, \lambda_n)
        = \bigcap_{i=1}^n V(f_i, \lambda_i)
      \]
      für alle Endomorphismen $f_1, \dotsc, f_n \in \End(V)$ und Eigenwerte $\lambda_1, \dotsc, \lambda_n \in K$.
    \item
      Es seien $f_1, \dotsc, f_n, g \in \End(V)$ Endomorphismen, so dass $g$ mit jedem $f_i$ kommutiert.
      Zeigen sie, dass der gemeinsame Eigenraum $V(f_1, \lambda_1; \dots; f_n, \lambda_n)$ für alle $\lambda_1, \dotsc, \lambda_n \in K$ invariant unter $g$ ist.
    \item
      Zeigen Sie: Sind die Endomorphismen $f_1, \dotsc, f_n \colon V \to V$ diagonalisierbar (d.h.\ für alle $i = 1, \dotsc, n$ ist $V = \bigoplus_{\lambda \in K} V(f_i, \lambda)$ ) und paarweise kommutierend, so sind die Endomorphismen \emph{simultan diagonalisierbar}, d.h.\ es ist
      \[
          V
        = \bigoplus_{\lambda_1, \dotsc, \lambda_n \in K}  V(f_1, \lambda_1; \dotsc; f_n, \lambda_n).
      \]
ö    ö\item
      Es sei nun $V$ endlichdimensional und $H \subseteq \End(V)$ ein Untervektorraum aus diagonalisierbaren und paarweise kommutierenden Endomorphismen.
      Zeigen Sie, dass es eine Basis $\mc{B}$ von $V$ gibt, so dass $\Mat_\mc{B}(f)$ für jedes $f \in H$ eine Diagonalmatrix ist.
      
      (\emph{Hinweis}:
       Nutzen Sie, dass $\End(V)$ endlichdimensioal ist.)
  \end{enumerate}
\end{question}


\begin{question}
  Es sei $f \colon V \to V$ ein Endomorphismus eines endlichdimensionalen $\Cbb$-Vektorraums $V$.
  Drücken Sie $\tr f$ und $\det f$ durch die  mit (nicht notwendigerweise verschiedenen) Eigenwerten $\lambda_1, \dotsc, \lambda_n \in \Cbb$ von $f$ aus.
\end{question}


\begin{question}
  Es sei $A \in \Mat_2(\Rbb)$ mit $\tr A = 0$ und $\tr A^2 = -2$.
  Bestimmen Sie $\det A$.
\end{question}


\begin{question}
  Es sei $A \in \GL_n(K)$ und $\chi_A(T)$ das charakteristische Polynom von $A$.
  \begin{enumerate}[leftmargin=*]
    \item
      Zeigen Sie, dass der konstante Term von $\chi_A(T)$ nicht verschwindet.
    \item
      Zeigen Sie, dass es ein Polynom $P \in K[T]$ gibt, so dass $A^{-1} = P(A)$.
  \end{enumerate}
\end{question}


\begin{question}
  Es sei $K$ ein algebraisch abgeschlossener Körper mit $\ringchar K \notin \{2,3\}$.
  Zeigen Sie, dass
  \[
    \det A = \frac{1}{6} (\tr A)^3 - \frac{1}{2} (\tr A^2)(\tr A) + \frac{1}{3} (\tr A^3)
    \quad
    \text{für jedes $A \in \Mat_3(K)$}.
  \]
  
  (\emph{Hinweis}:
   Wenn die Rechnungen zu kompliziert werden, dann macht man es falsch.)
\end{question}


\begin{question}
  Es sei $f \colon V \to V$ ein Endomorphismus eines $n$-dimensionalen $K$-Vektorraums $V$ und $\{ v_1, \dotsc, v_{n+1} \} \subseteq V$ eine Teilmenge aus Eigenvektoren von $f$, so dass jede $n$-elementige Teilmenge linear unabhängig ist.
  Zeigen Sie, dass $f$ bereits ein skalares Vielfaches der Identität ist.
\end{question}











% JORDANSTUFF AND GENERALIZED EIGENSPACES


\begin{question}
  Bestimmen Sie alle nicht-diagonalierbaren $A \in \Mat_2(\Cbb)$ mit $\tr = 0$.
\end{question}


\begin{question}
  Bestimmen Sie für die folgenden komplexen Matrizen jeweils eine Jordannormalform, inklusiver entsprechender Basiswechselmatrizen:
  \begin{gather*}
    \begin{pmatrix*}[r]
      2 & 2 &  -5 \\
      3 & 7 & -15 \\
      1 & 2 &  -4
    \end{pmatrix*},
    \quad
    \begin{pmatrix*}[r]
      0 & 0 &  1 \\
      1 & 0 & -3 \\
      0 & 1 &  3
    \end{pmatrix*},
    \quad
    \begin{pmatrix*}[r]
      1 & 0 & 1 & -1  \\
      0 & 1 & 1 &  0  \\
      0 & 0 & 1 &  1  \\
      0 & 0 & 0 &  1
    \end{pmatrix*},
    \quad
    \begin{pmatrix*}[r]
      3 & -4  &  0  &  2  \\
      4 & -5  & -2  &  4  \\
      0 &  0  &  3  & -2  \\
      0 &  0  &  2  & -1
    \end{pmatrix*},
  \\
    \begin{pmatrix*}[r]
       3  &  3  &  1  &  5  \\
       0  & -2  &  2  & -8  \\
      -1  & -2  &  0  & -3  \\
       0  &  2  & -1  &  6
    \end{pmatrix*}
  \end{gather*}
\end{question}
\begin{solution}
  \begin{enumerate}leftmargin=*]
    \item
      \[
        \begin{pmatrix}
          1 &   &   \\
            & 1 &   \\
            &   & 3
        \end{pmatrix}
      \]
      mit Basiswechselmatrix
      \[
        \begin{pmatrix*}[r]
          5 & -2  & 1 \\
          0 &  1  & 3 \\
          1 &  0  & 1
        \end{pmatrix*}
      \]
    \item
      \[
        \begin{pmatrix*}[r]
          1 &   &   \\
          1 & 1 &   \\
            & 1 & 1
        \end{pmatrix*}
      \]
      mit Basiswechselmatrix
      \[
        \begin{pmatrix*}[r]
           1  & -1  & 1 \\
          -2  &  1  & 0 \\
           1  &  0  & 0
        \end{pmatrix*}
      \]
    \item
      \[
        \begin{pmatrix*}[r]
          1 & 0 & 0 & 0 \\
            & 1 & 1 & 0 \\
            &   & 1 & 1 \\
            &   & 0 & 1
        \end{pmatrix*}
      \]
      mit Basiswechselmatrix
      \[
        \begin{pmatrix*}[r]
          1 & 1 & -1 & 0  \\
          0 & 1 &  0 & 0  \\
          0 & 0 &  1 & 0  \\
          0 & 0 &  0 & 1
        \end{pmatrix*}
      \]
    \item
      \[
        \begin{pmatrix*}[r]
          -1  &  1  &   &   \\
              & -1  &   &   \\
              &     & 1 & 1 \\
              &     &   & 1
        \end{pmatrix*}
      \]
      mit Basiswechselmatrix
      \[
        \begin{pmatrix*}[r]
          1 & \frac{1}{4} & 1 & \frac{1}{2} \\
          1 & 0           & 1 & 0           \\
          0 & 0           & 1 & \frac{1}{2} \\
          0 & 0           & 1 & 0
        \end{pmatrix*}
      \]
    \item
      \[
        \begin{pmatrix}
          1 &   &   &   \\
            & 2 & 1 &   \\
            &   & 2 & 1 \\
            &   &   & 2
        \end{pmatrix}
      \]
      mit Basiswechselmatrix
      \[
        \begin{pmatrix}
           0  &  1  &  2  &  7  \\
          -2  & -2  &  0  & -1  \\
           1  &  0  & -1  & -2  \\
           1  &  1  &  0  &  0
        \end{pmatrix}
      \]
  \end{enumerate}
\end{solution}


\begin{question}
  Es sei $V$ ein endlichdimensionaler $K$-Vektorraum und $f \colon V \to V$ ein Endomorphismus.
  \begin{enumerate}[leftmargin=*]
    \item
      Es sei $n \colon V \to V$ ein nilpotenter Endomorphismus.
      Zeigen Sie, dass ${\id_V} + n$ invertierbar ist.
    \item
      Zeigen, bzw.\ folgern Sie allgemeiner, dass $\lambda {\id_V} + n$ für alle $\lambda \in K^\times$ invertierbar ist.
    \item
      Es sei $f \colon V \to V$ ein beliebiger Endomorphismus.
      Zeigen Sie für alle $\lambda, \mu \in K$ mit $\lambda \neq \mu$, dass $V^\sim_\lambda(f)$ invariant unter $f - \mu {\id_V}$ ist, und dass die Einschränkung $(f - \mu {\id_V})|_{V^\sim_\lambda(f)}$ invertierbar ist.
    \item
      Folgern Sie:
      Ist $(f - \lambda_1)^{n_1} \dotsm (f - \lambda_k)^{n_k} = 0$ mit $\lambda_1, \dotsc, \lambda_k \in K$ paarweise verschieden, so sind $\lambda_1, \dotsc, \lambda_k$ die möglichen Eigenwerte von $f$, und für alle $1 \leq i \leq k$ ist $\dim V^\sim_{\lambda_i}(f) \leq n_i$.
    \item
      Folgern Sie:
      Ist $K$ algebraisch abgeschlossen und $(f - \lambda_1) \dotsm (f - \lambda_n) = 0$, so ist $f$ diagonalisierbar mit möglichen Eigenwerten $\lambda_1, \dotsc, \lambda_n \in K$.
  \end{enumerate}
\end{question}


\begin{question}
  Es sei $f \colon V \to V$ Endomorphismus eines $\Cbb$-Vektorraums und $\lambda \in \Cbb$.
  Zeigen Sie, dass die Einschränkung $(f - \lambda {\id_V})|_{V^\sim_\lambda(f)}$ nicht notwendigerweise nilpotent ist.
\end{question}


\begin{question}
  Bestimmen Sie in den Folgenden alle Möglichkeiten der Jordannormalform von $A \in \Mat_n(\Cbb)$.
  \begin{enumerate}[leftmargin=*]
    \item
     Es ist $\chi_A(T) = (T-3)^4 (T-5)^4$ und $(A - 3I)^2 (A - 5I)^2 = 0$.
    \item
      Es ist $A^3 = 0$ und alle Eigenräume von $A$ sind eindimensional.
    \item
      Es ist $\chi_A(T) = (T-2)(T+2)^3$ und $(A - 2I)(A + 2I) = 0$.
    \item
      Es ist $\chi_A(T) = T(T-1)(T+1)$.
    \item
      Es ist $\chi_A(T) = (T-2)^2(T-3)^2$ und alle Eigenräume von $A$ sind entweder null-\ oder eindimensional.
    \item
      Es ist $A^2 = A$ und alle Eigenräume von $A$ sind zweidimensional.
    \item
      Es ist $\chi_A(T) = T^5$ und alle Eigenräume von $A$ sind entweder null-\ oder eindimensional.
    \item
      Es ist $\chi_A(T) = (T+3)^3 T^2$ und $A$ hat keine zweidimensionalen Eigenräume.
  \end{enumerate}
\end{question}


\begin{question}
  Es sei $V$ ein endlichdimensionaler $\Cbb$-Vektorraum.
  \begin{enumerate}
    \item
      Es sei $n \colon V \to V$ ein nilpotenter Endomorphismus.
      Zeigen Sie, dass der Endomorphismus ${\id_V} + n$ invertierbar ist.
  \end{enumerate}
  Ein Endomorphismus $u \colon V \to V$ heißt \emph{unipotent}, falls $u - \id_V$ nilpotent it.
  \begin{enumerate}[resume]
    \item
      Folgern Sie, dass jeder unipotente Endomorphismus von $V$ invertierbar ist.
  \end{enumerate}
  Auf dem fünften Übungszettel wurde gezeigt, dass es für jeden Endomorphismus $f \colon V \to V$ eindeutige Endomorphismen $d,n \colon V \to V$ gibt, so dass
  \begin{itemize}
    \item 
      $f = d + n$,
    \item
      $d$ ist diagonalisierbar und $n$ ist nilpotent, und
    \item
      $d$ und $n$ kommutieren.
  \end{itemize}
  Folgern Sie aus dieser \emph{additiven Jordanzerlegung} von $\End(V)$ die folgende \emph{multiplikative Jordanzerlegung} von $\GL(V)$.
  \begin{enumerate}[resume]
    \item
      Zeigen Sie, dass es für jedes $s \in \GL(V)$ eindeutige $d, u \in \GL(V)$ gibt, so dass
      \begin{itemize}
        \item
          $s = d \cdot u$,
        \item
          $d$ ist diagonalisierbar und $u$ ist unipotent, und
        \item
          $d$ und $u$ kommutieren.
      \end{itemize}
  \end{enumerate}
\end{question}


\begin{question}
  Bestimmen Sie die Potenz $A^{10}$ der Matrix
  \[
    A
    \coloneqq
    \begin{pmatrix*}[r]
       3  & 4 &  3      \\
      -1  & 0 & -1      \\
       1  & 2 &  3
    \end{pmatrix*}
    \in \Mat_n(\Cbb).
  \]
\end{question}


\begin{question}
  Bestimmen Sie die Lösungsräume der folgenden homogenen linearen Gleichungsssysteme mit $f, g, h \in C^\infty(\Rbb)$:
  \[
    \left\{
      \begin{array}{ccrr}
        f'  & = & -f  & - 6g, \\
        g'  & = & 2f  & + 6g,
      \end{array}
    \right.
    \quad
    \left\{
      \begin{array}{ccrr}
        f'  & = & -f  & - g,  \\
        g'  & = & 2f  & + g,
      \end{array}
    \right.
    \quad
    \left\{
      \begin{array}{ccrrr}
        f'  & = & 2f  & + 2g  & + 3h, \\
        g'  & = &  f  & + 3g  & + 3h, \\
        h'  & = & -f  & - 2f  & - 2h.
      \end{array}
    \right.
  \]
\end{question}


\begin{question}
  Es sei $\|\cdot\|$ eine Norm auf $\Mat_n(\Cbb)$.
  Für alle $\lambda_1, \dotsc, \lambda_n \in \Cbb$ sei
  \[
    \diag(\lambda_1, \dotsc, \lambda_n)
    \coloneqq
    \begin{pmatrix}
      \lambda_1 &         &           \\
                & \ddots  &           \\
                &         & \lambda_n
    \end{pmatrix}
    \in \Mat_n(\Cbb).
  \]
  Es sei
  \[
    \Diagonal_n(\Cbb)
    \coloneqq
    \left\{
      S \diag(\lambda_1, \dotsc, \lambda_n) S^{-1}
    \,\middle|\,
      S \in \GL_n(\Cbb),
      \lambda_1, \dotsc, \lambda \in \Cbb
    \right\}
    \subseteq \Mat_n(\Cbb)
  \]
  die Menge der diagonalisierbaren komplexen $n \times n$-Matrizen.
  Wir zeigen, dass $D_n(\Cbb) \subseteq \Mat_n(\Cbb)$ dicht ist, d.h.\ dass es für jede Matrix $A \in \Mat_n(\Cbb)$ und jedes $\varepsilon > 0$ eine diagonalisierbare Matrix $D \in \Diagonal_n(\Cbb)$ mit $\|A-D\| < \varepsilon$ gibt.
  
  Es sei $S \in \GL_n(\Cbb)$, so dass $S A S^{-1}$ eine obere Dreiecksmatrix mit Diagonaleinträgen $\lambda_1, \dotsc, \lambda_n$ ist, also
  \[
    S A S^{-1}
    =
    \begin{pmatrix}
      \lambda_1 & *       & \cdots  & *         \\
                & \ddots  & \ddots  & \vdots    \\
                &         & \ddots  & *         \\
                &         &         & \lambda_n
    \end{pmatrix}.
  \]
  Es seien $z_1, \dotsc, z_n \in \Cbb$ paarweise verschieden und
  \[
    B(t)
    \coloneqq
    A + t S \diag(z_1, \dotsc, z_n) S^{-1}
    \quad
    \text{für alle $t \in \Rbb$}.
  \]
  \begin{enumerate}
    \item
      Zeigen Sie, dass $\mu_1(t), \dotsc, \mu_n(t) \in \Cbb$ mit
      \[
        \mu_i(t) \coloneqq \lambda_i + t z_i
        \quad
        \text{für $i = 1, \dotsc, n$}
      \]
      die Eigenwerte von $B(t)$ ist.
    \item
      Zeigen Sie, dass die Zahlen $\mu_1(t), \dotsc, \mu_n(t)$ für fast alle $t \in \Rbb$ paarweise verschieden sind.
    \item
      Folgern Sie, dass $B(t)$ für fast alle $t \in \Rbb$ diagonalisierbar ist.
    \item
      Folgern Sie, dass es für alle $\varepsilon > 0$ ein $D \in \Diagonal_n(\Cbb)$ mit $\| A - D \| < \varepsilon$ gibt.
  \end{enumerate}
  Wir wollen die Dichtheit von $\Diagonal_n(\Cbb) \subseteq \Mat_n(\Cbb)$ nutzen, um den Satz von Cayley-Hamilton zu zeigen:
  \begin{enumerate}[resume]
    \item
      Zeigen Sie, dass die Abbildung
      \[
        F \colon \Mat_n(\Cbb) \to \Mat_n(\Cbb),
        \quad
        A \mapsto \chi_A(A)
      \]
      stetig ist, wobei $\chi_A(T) \in \Cbb[T]$ das charakteristische Polynom von $A$ ist.
    \item
      Zeigen Sie, dass $F(D) = 0$ für jede Diagonalmatrix $D \in \Mat_n(\Cbb)$.
    \item
      Zeigen Sie, dass $P(SAS^{-1}) = S P(A) S^{-1}$ für alle $P \in \Cbb[T]$, $A \in \Mat_n(\Cbb)$ und $S \in \GL_n(\Cbb)$.
      Folgern Sie, dass $F(D) = 0$ für jede Matrix $D \in \Diagonal_n(\Cbb)$.
    \item
      Folgern Sie, dass $F(A) = 0$ für alle $A \in \Mat_n(\Cbb)$.
  \end{enumerate}
\end{question}










% SCALAR PRODUCTS


\begin{question}
  Definieren Sie die Begriffe eines reellen, bzw.\ komplexen Skalarprodukts, sowie eines reellen, bzw.\ komplexen Hilbertraums.
\end{question}


\begin{question}
  Formulieren und Beweisen Sie die Cauchy-Schwarz-Ungleichung.
\end{question}


\begin{question}
  Es sei $V$ ein Skalarproduktraum.
  Zeigen Sie für Endomorphismen $f, g_1, g_2 \colon V \to V$ Endomorphismen die folgende Kürzungsregel:
  Falls $f^*$ existiert und $f^* f g_1 = f^* f g_2$, dann ist bereits $f g_1 = f g_2$.
\end{question}


\begin{question}
  Zeigen Sie, dass jeder endlichdimensionale Skalarproduktraum eine Orthonormalbasis besitzt.
\end{question}


\begin{question}
  Es sei $V$ ein $\Kbb$-Vektorraum mit abzählbarer Orthonormalbasis $(e_i)_{i \in \Nbb}$.
  Es sei $T \colon V \to V$ die eindeutige lineare Abbildung mit $T(e_i) = e_1$ für alle $i \in \Nbb$.
  Zeigen Sie, dass $T$ kein Adjungiertes besitzt.
\end{question}



\begin{question}
  Es sei $V$ ein Skalarproduktraum und $v_1, \dotsc, v_n \in V$ seien paarweise orthogonal zueinander.
  Zeigen Sie: Ist $v_1, \dotsc, v_n \neq 0$, so ist die Familie $(v_1, \dotsc, v_n)$ linear unabhängig.
\end{question}


\begin{question}
  Es sei $V$ ein Skalarproduktraum und $f \colon V \to V$ ein normaler Endomorphismus.
  Zeigen Sie, dass die Eigenräume $V_\lambda(f)$ und $V_\mu(f)$ für alle $\lambda \neq \mu$ orthogonal sind.
\end{question}


\begin{question}
  Es seien $V$ und $W$ zwei $\Kbb$-Skalarprodukträume und $f \colon V \to W$ eine lineare Abbildung.
  Es sei $\mc{B} = (b_1, \dotsc, b_n)$ eine Orthonormalbasis von $V$ und $\mc{C} = (c_1, \dotsc, c_m)$ eine Orthonormalbasis von $W$.
  Zeigen Sie die Gleichheit
  \[
      \Mat_{\mc{B}, \mc{C}}(f^*)
    = \Mat_{\mc{C}, \mc{B}}(f)^*.
  \]
\end{question}


\begin{question}
  Es sei $V$ ein endlichdimensionaler Skalarproduktraum und $f \colon V \to V$ ein normaler Endomorphismus.
  \begin{enumerate}[leftmargin=*]
    \item
      Zeigen Sie, dass $\|f(v)\| = \|f^*(v)\|$ für alle $v \in V$.
    \item
      Zeigen Sie, dass $V_\lambda(f) = V_{\overline{\lambda}}(f^*)$ und $V^\sim_\lambda(f) = V^\sim_{\overline{\lambda}}(f^*)$.
  \end{enumerate}
\end{question}


\begin{question}
  Es sei $V$ ein endlichdimensionaler Skalarproduktraum und $v_1, \dotsc, v_n \in V$ seien Einheitsvektoren.
  Zeigen Sie, dass die folgenden beiden Aussage äquivalent sind:
  \begin{enumerate}
    \item
      $(v_1, \dotsc, v_n)$ ist eine Orthonormalbasis von $V$.
    \item
      Für alle $v \in V$ ist $\| v \|^2 = \sum_{i=1}^n |\bil{v, v_i}|^2$.
  \end{enumerate}
\end{question}


\begin{question}
  Es sei $V$ ein Skalarproduktraum und $f \colon V \to V$ ein selbstadjungierter, nilpotenter Endomorphismus.
  Zeigen Sie, dass $f = 0$.
\end{question}


\begin{question}
  Es sei $\pi \in S_n$ eine Permutation und $P_\pi \colon \Rbb^n \to \Rbb^n$ die eindeutige lineare Abbildung mit
  \[
    P_\pi(e_i) = e_{\pi(i)}
    \quad
    \text{für alle $i = 1, \dotsc, n$},
  \]
  wobei $(e_1, \dotsc, e_n)$ die Standardbasis von $\Rbb^n$ ist.
  \begin{enumerate}[leftmargin=*]
    \item
      Zeigen Sie, dass $P_\pi$ orthogonal ist.
    \item
      Bestimmen Sie die möglichen Eigenwerte von $P_\pi$.
    \item
      Geben Sie ein Beispiel an, bei dem alle möglichen Eigenwerte auftreten.
  \end{enumerate}
\end{question}


\begin{question}
  Es sei $V$ ein endlichdimensionaler Skalarproduktraum und $U \subseteq V$ ein Untervektorraum mit Orthonormalbasis $(u_1, \dotsc, u_n)$.
  Zeigen Sie, dass die lineare Abbildung
  \[
    P \colon V \to V,
    \quad
    v \mapsto \sum_{i=1}^n \bil{v, u_i} u_i
  \]
  die orthogonale Projektion auf $U$ ist.
\end{question}


\begin{question}
  Bestimmen Sie die Signatur $(n_0, n_+, n_-)$ der folgenden quadratischen Formen auf $\Rbb^n$:
  \begin{enumerate}[leftmargin=*]
    \item
      $
          q(x_1, x_2)
        = 2 x_1^2 - 3 x_2^2 + 2 x_1 x_2
      $
    \item
      $
          q(x_1, x_2)
        = - x^1 + x_2 + a x_1 x_2
      $
      mit $a \in \Rbb$
    \item
      $
          q(x_1, x_2)
        = x_1^2 + 15 x_2^2 + 6 x_1 x_2
      $
    \item
      $
          q(x_1, x_2)
        = 2 x_1 x_2
      $
    \item
      $
          q(x_1, x_2, x_3)
        = x_1^2 + 2 x_1 x_2 - 2 x_1 x_3 + x_2^2 - 2 x_2 x_3 - x_3^2
      $
    \item
      $
          q(x_1, x_2, x_3, x_4)
        = x_1^2 - 7 x_2^2 - x_3^2 - x_4^2 + 2 x_1 x_2 - 6 x_2 x_3 + 6 x_2 x_4 + 2 x_3 x_4.
      $
  \end{enumerate}
\end{question}
\begin{solution}
  \begin{enumerate}[leftmargin=*]
    \item
      Die Signatur ist $(0,1,-1)$.
    \item
      Die Signatur ist $(0,1,-1)$.
    \item
      Die Signatur ist $(0,2,0)$.
    \item
      Die Signatur ist $(0,1,-1)$
    \item
      Die Signatur ist $(1,1,1)$.
    \item
      Die Signatur ist $(1,2,1)$.
  \end{enumerate}
\end{solution}


\begin{question}
  Es sei $V \neq 0$ ein $\Kbb$-Vektorraum mit Skalarprodukt $\bil{\cdot, \cdot}$.
  Für alle $\lambda \in \Kbb$ sei
  \[
    \bil{x, y}_\lambda \coloneqq \lambda \bil{x, y}
    \quad
    \text{für alle $x, y \in V$}.
  \]
  Bestimmen Sie alle $\lambda \in \Kbb$, für die $\bil{\cdot, \cdot}_\lambda$ ein Skalarprodukt auf $V$ ist.
\end{question}


\begin{question}
  Es sei $V$ ein endlichdimensionaler unitärer Vektorraum und $f \colon V \to V$ ein normaler Endomorphismus.
  Zeigen Sie:
  \begin{enumerate}[leftmargin=*]
    \item
      $f$ ist genau dann unitär, wenn alle Eigenwerte von $f$ Betrag $1$ haben.
    \item
      $f$ ist genau dann selbstadjungiert, wenn alle Eigenwerte von $f$ reell sind.
    \item
      $f$ ist genau dann antiselbstadjungiert, wenn alle Eigenwerte von $f$ rein imaginär sind.
    \item
      $f$ ist genau dann eine Orthogonalprojektion, wenn $0$ und $1$ die einzigen Eigenwerte von $f$ sind.
  \end{enumerate}
\end{question}


\begin{question}
  Es sei $V$ ein endlichdimensionaler Skalarproduktsraum und $f \colon V \to V$ ein Endomorphismus.
  \begin{enumerate}[leftmargin=*]
    \item
      Zeigen Sie, dass $\ker f^* \subseteq (\im f)^\perp$.
    \item
      Folgern Sie daraus, dass $\im f^* \subseteq (\ker f)^\perp$.
    \item
      Folgern Sie aus den beiden Inklusionen $\ker f^* \subseteq (\im f)^\perp$ und $\im f^* \subseteq (\ker f)^\perp$ mithilfe der Endlichdimensionalität von $V$, dass bereits Gleichheiten gelten, dass also
      \[
        \ker f^* = (\im f)^\perp
        \quad\text{und}\quad
        \im f^* = (\ker f)^\perp.
      \]
  \end{enumerate}
  Von nun an sei $f$ normal.
  \begin{enumerate}[leftmargin=*, resume]
    \item
      Zeigen Sie, dass $\|f(x)\| = \|f^*(x)\|$ für alle $x \in V$.
    \item
      Folgern Sie, dass $\ker f = \ker f^*$.
    \item
      Folgern Sie damit aus den obigen Gleichheiten, dass $V = \im f \oplus \ker f$ gilt, und dass die Summe orthogonal ist.
      
      (\emph{Hinweis}:
       Zeigen Sie zuerst, dass $\im f$ und $\ker f$ orthogonal sind, und nutzen Sie dann die Endlichdimensionalität von $V$.)
  \end{enumerate}
\end{question}


\begin{question}
  \begin{enumerate}[leftmargin=*]
    \item
      Bestimmen Sie für die Matrix
      \[
        A \coloneqq
        \begin{pmatrix*}[r]
          2  & -1  & 1 \\
          -1  &  2  & 1 \\
          1  &  1  & 2
        \end{pmatrix*}
        \in \Mat_n(\Rbb)
      \]
      eine orthogonale Matrix $S \in \Orthogonal(3)$, so dass $S^T A S$ eine Diagonalmatrix ist.
    \item
      Bestimmen Sie für die symmetrische Bilinearform $\beta \colon \Rbb^2 \times \Rbb^2 \to \Rbb$ mit
      \[
        \beta\left( \vect{x_1 \\ x_2}, \vect{y_1 \\ y_2} \right)
        \coloneqq
        x_1 y_2 + x_2 y_1
        \quad
        \text{für alle $\vect{x_1 \\ x_2}, \vect{y_1 \\ y_2} \in \Rbb^2$}
      \]
      eine Basis $\mc{B}$ von $\Rbb^2$, so dass $\beta$ bezüglich $\mc{B}$ durch eine Diagonalmatrix mit möglichen Diagonaleinträgen $0, 1, -1$ dargestellt wird.
  \end{enumerate}
\end{question}


\begin{question}
  Es seien $V$ und $W$ zwei endlichdimensionale euklidische Vektorräume.
  Ferner sei $f \colon V \to W$ eine $\Rbb$-lineare Abbildung.
  \begin{enumerate}[leftmargin=*]
    \item
      Zeigen Sie, dass die Abbildung
      \[
        \Phi_V \colon V \to V^*,
        \quad
        v \mapsto \bil{-, v}
      \]
      ein $\Rbb$-linearer Isomorphismus ist.
    \item
      Geben Sie die Definition der dualen Abbildung $f^* \colon W^* \to V^*$ an.
      Zeigen Sie, dass $f^*$ $\Rbb$-linear ist.
    \item
      Zeigen Sie, dass die Abbildung $g \coloneqq \Phi_V^{-1} \circ f^* \circ \Phi_W$ $\Rbb$-linear ist, und dass
      \[
        \bil{f(v), w} = \bil{v, g(w)}
        \quad
        \text{für alle $v \in V$, $w \in W$}.
      \]
    \item
      Zeigen Sie:
      Eine Basis $\mc{B} = (v_1, \dotsc, v_n)$ von $V$ ist genau dann eine Orthonormalbasis, wenn die Basis $\Phi_V(\mc{B}) = (\Phi_V(v_1), \dotsc, \Phi_V(v_n))$ von $V^*$ die duale Basis $\mc{B}^*$ ist.
    \item
      Inwiefern ändern sich die obigen Resultate für denn Fall $\Kbb = \Cbb$, wenn also $V$ und $W$ endlichdimensionale unitäre Vektorräume sind?
  \end{enumerate}
\end{question}


\begin{question}
  Es sei $V$ ein endlichdimensionale $\Kbb$-Vektorraum und $f \colon V \to V$ ein Endomorphismus.
  \begin{enumerate}[leftmargin=*]
    \item
      Zeigen Sie für denn Fall $\Kbb = \Rbb$, dass $f$ genau dann diagonalisierbar ist, wenn es ein Skalarprodukt auf $V$ gibt, bezüglich dessen $f$ selbstadjungiert ist.
    \item
      Zeigen oder widerlegen Sie die analoge Aussage für $\Kbb = \Cbb$.
  \end{enumerate}
\end{question}


\begin{question}
  Es sei $x \in \Rbb^n$ ein normierter Spaltenvektor und
  \[
    A \coloneqq x x^T \in \Mat(n \times n, \Rbb).
  \]
  Zeigen Sie, dass die Abbildung
  \[
    P \colon \Rbb^n \to \Rbb^n,
    \quad
    y \mapsto Ay
  \]
  die orthogonale Projektion auf die Gerade $\Rbb x$ ist.
\end{question}


\begin{question}
  Es seien $V$ und $W$ zwei endlichdimensionale euklidische Vektorräume.
  Für jeden Untervektorraum $U \subseteq V$ sei
  \[
              U^\perp
    \coloneqq \{ v \in V \mid \text{$\bil{u, v} = 0$ für alle $u \in U$} \}
  \]
  das orthogonale Komplement von $U$, und
  \[
              U^\circ
    \coloneqq \{ \varphi \in V^* \mid \text{$\varphi(u) = 0$ für alle $u \in U$}
  \]
  der Annihilator von $U$.
  Für jeden Endomorphismus $f \colon V \to W$ sei $f^* \colon W \to V$ die Adjungierte von $f$, und
  \[
    f^T \colon W^* \to V^*,
    \quad
    \varphi \mapsto \varphi \circ f
  \]
  die zu $f$ duale Abbildung.
  \begin{enumerate}
    \item
      Zeigen Sie, dass die Abbildung
      \[
        \Phi_V \colon V \to V^*,
        \quad
        v \mapsto \bil{-, v}
      \]
      ein Isomorphismus ist.
    \item
      Zeigen Sie, dass für jeden Untervektorraum $U \subseteq V$ die Gleichheit $\Phi_V(U^\perp) = U^\circ$ gilt.
    \item
      Zeigen Sie, dass $f^T \circ \Phi_W = \Phi_V \circ f^*$, dass alsa das folgende Diagram kommutiert:
      \[
        \begin{tikzcd}[row sep = large, column sep = large, ampersand replacement = \&]
                V   \arrow[swap]{d}{\Phi_V}
            \&  W   \arrow[swap]{l}{f^*}
                    \arrow{d}{\Phi_W}
          \\
                V^* 
            \&  W^* \arrow{l}{f^T}
        \end{tikzcd}
      \]
      Folgern Sie, dass $f^* = \Phi_V^{-1} \circ f^T \circ \Phi_W$.
  \end{enumerate}
  In Linear Algebra I wurde gezeigt, dass
  \[
      \ker f^T
    = (\im f)^\circ
    \quad\text{und}\quad
      \im f^T
    = (\ker f )^\circ,
  \]
  und dass für je zwei Untervektorräume $U_1, U_2 \subseteq V$ die Gleichheiten
  \[
      (U_1 + U_2)^\circ
    = U_1^\circ \cap U_2^\circ
    \quad\text{und}\quad
      (U_1 \cap U_2)^\circ
    = U_1^\circ + U_2^\circ
  \]
  gelten.
  \begin{enumerate}[resume]
    \item
      Folgen Sie aus den vorherigen Aufgabenteilen und den Aussagen aus Lineare Algebra I, dass für alle Untervektorräume $U_1, U_2 \subseteq V$ die Gleichheiten
      \[
          (U_1 + U_2)^\perp
        = U_1^\perp \cap U_2^\perp
        \quad\text{und}\quad
          (U_1 \cap U_2)^\perp
        = U_1^\perp + U_2^\perp
      \]
      gelten.
      
      (\emph{Hinweis}:
       Nutzen Sie, dass $\Phi_V$ ein Isomorphismus ist.)
    \item
      Folgen Sie aus den vorherigen Aufgabenteilen und den Aussagen aus Lineare Algebra I, dass
      \[
          \ker f^*
        = (\im f)^\perp
        \quad\text{und}\quad
          \im f^*
        = (\ker f)^\perp.
      \]
      
      (\emph{Hinweis}:
       Nutzen Sie, dass $\Phi_V$ und $\Phi_W$ Isomorphismen sind.)
  \end{enumerate}
\end{question}


\begin{question}
  Es sei $V \coloneqq \mathcal{C}([0,1], \Rbb)$ der Raum der stetigen Funktionen $[0,1] \to \Rbb$, und es sei
  \[
    U \coloneqq \{ f \in V \mid f(0) = 0 \}.
  \]
  \begin{enumerate}[leftmargin=*]
    \item
      Zeigen Sie, dass $U$ ein Untervektorraum von $V$ ist.
    \item
      Zeigen Sie, dass
      \[
        \bil{f, g} \coloneqq \int_0^1 f(t) g(t) \dd{t}
        \quad
        \text{für alle $f, g \in V$}
      \]
      ein Skalarprodukt auf $V$ definiert.
    \item
      Zeigen Sie, dass $U^\perp = 0$.
      Folgern Sie, dass $V \neq U \oplus U^\perp$.
      
      (\emph{Hinweis}:
       Betrachten Sie für $g \in U^\perp$ die Funktion $h \colon [0,1] \to \Rbb$ mit $h(t) = t^2 g(t)$.)
    \item
      Zeigen Sie ferner, dass $V\!/U$ eindimensional ist.
  \end{enumerate}
\end{question}


\begin{question}
  Es sei $V$ ein endlichdimensionaler euklidischer Vektorraum und $f \colon V \to V$ ein selbstadjungierter, orthogonaler Endomorphismus mit nur positiven Eigenwerten.
  Zeigen Sie, dass bereits $f = \id_V$ gilt.
\end{question}


\begin{question}
  Es sei $V$ ein endlichdimensionaler euklidischer Vektorraum.
  \begin{enumerate}[leftmargin=*]
    \item
      Zeigen Sie, dass die Abbildung
      \[
        \Phi \colon V \to V^*,
        \quad
        v \mapsto \bil{-,v}
      \]
      ein Isomorphismus ist.
    \item
      Zeigen Sie:
      Eine Basis $\mc{B} = (v_1, \dotsc, v_n)$ von $V$ ist genau dann eine Orthonormalbasis, wenn
      \[
          \varphi
        = \sum_{i=1}^n \varphi(v_i) \bil{-,v_i}
        \quad
        \text{für alle $\varphi \in V^*$}.
      \]
  \end{enumerate}
  Es sei nun $V$ der Vektorraum der Polynomsfunktionen $\Rbb \to \Rbb$ vom Grad $\leq 2$.
  Für $a \in \Rbb$ sei $\varphi_a \in V^*$ durch $\varphi_a(f) = f(a)$ definiert.
  \begin{enumerate}[leftmargin=*, resume]
    \item
      Zeigen Sie, dass
      \[
        \bil{f, g} \coloneqq \int_{-1}^1 f(t)g(t) \dd{t}
        \quad
        \text{für alle $f, g \in V$}
      \]
      ein Skalarprodukt auf $V$ definiert.
    \item
      Bestimmen Sie eine Orthonormalbasis von $V$.
    \item
      Zeigen Sie, dass es für alle $a \in \Rbb$ ein eindeutiges $g_a \in V$ gibt, so dass
      \[
          f(a)
        = \int_{-1}^1 f(t) g_a(t) \dd{t}
        \quad
        \text{für alle $f \in V$}.
      \]
    \item
      Bestimmen Sie $g_a$ für beliebiges $a \in \Rbb$.
  \end{enumerate}
\end{question}


\begin{question}
  Es sei $V$ ein endlichdimensionaler euklidischer Vektorraum und $f \colon V \to V$ ein Endomorphismus.
  Entscheiden Sie, welche der folgenden Aussagen sich implizieren.
  \begin{enumerate}
    \item
      $f$ ist selbstadjungiert mit positiven Eigenwerten.
    \item
      $f$ ist orthogonal, und alle Eigenwerte von $f$ sind positiv.
    \item
      $f$ ist normal mit $\det f > 0$.
    \item
      Es gibt einen selbstadjungierten Endomorphismus $g \colon V \to V$ mit $f = \exp(g)$.
    \item
      $f$ ist selbstadjungiert und orthogonal.
  \end{enumerate}
\end{question}


\begin{question}
  Es sei $V$ der reelle Vektorraum der Polynomfunktionen $\Rbb \to \Rbb$, und für alle $n \in \Nbb$ sei $V_n \subseteq V$ der Untervektorraum der Polynomfunktionen von Grad $\leq n$.
  \begin{enumerate}[leftmargin=*]
    \item
      Zeigen Sie, dass
      \[
                  \bil{f, g}
        \coloneqq \int_{-1}^1 f(t) g(t) \dd{t}
        \quad
        \text{für alle $f, g \in V$}
      \]
      ein Skalarprodukt auf $V$ definiert.
    \item
      Zeigen Sie, dass die lineare Abbildung $\psi \colon V \to V$ mit
      \[
        \psi(f)(t) \coloneqq (t^2 - 1) f''(t) + 2t f'(t)
        \quad
        \text{für alle $f \in V$ und $t \in \Rbb$}
      \]
      selbstadjungiert bezüglich $\bil{\cdot, \cdot}$ ist.
  \end{enumerate}
  Es sei $\mc{G} \coloneqq (p_n)_{n \geq 0}$ die Orthonormalbasis von $V$, die durch Anwenden des Gram-Schmidt-Verfahrens auf die Standardbasis $\mc{B} \coloneqq (x^n)_{n \geq 0}$ ensteht.
  \begin{enumerate}[resume, leftmargin=*]
    \item
      Zeigen Sie für alle $n \geq 0$, dass $V_n$ invariant unter $\psi$ ist.
    \item
      Zeigen Sie für alle $n \geq 0$, dass $\mc{G}_n \coloneqq (p_0, \dotsc, p_n)$ eine Basis von $V_n$ ist.
    \item
      Zeigen Sie für alle $n \geq 0$, dass $\Mat_{\mc{G}_n}(\psi|_{V_n})$ eine obere Dreiecksmatrix ist.
      Betrachten Sie hierfür die Filtration
      \[
        0 \subseteq V_0 \subseteq V_1 \subseteq V_2 \subseteq V_3 \subseteq \dotsb \subseteq V_n,
      \]
      und nutzen Sie, dass $V_i = \Ell(\mc{G}_i)$ invariant unter $\psi$ ist.
    \item
      Folgen Sie mithilfe der Selbstadjungiertheit von $\psi$, dass $\Mat_{\mc{G}_n}(\psi|_{V_n})$ für alle $n \geq 0$ bereits eine Diagonalmatrix ist.
      Folgern Sie, dass $\mc{G}$ eine Basis aus Eigenvektoren von $\psi$ ist.
    \item
      Bestimmen Sie für alle $n \geq 0$ die Eigenwerte der Einschränkung $\psi|_{V_n}$, indem Sie die darstellende Matrix bezüglich der Basis $\mc{B}_n = (1, x, \dotsc, x^n)$ von $V_n$ bestimmen.
    \item
      Geben Sie den zu $p_n$ gehörigen Eigenwert von $\psi$ an.
    \item
      Berechnen Sie $\mc{G}_4$.
  \end{enumerate}
\end{question}


\begin{question}
  Es sei $V$ ein endlichdimensionaler Skalarproduktraum und $f \colon V \to V$ ein Endomorphismus.
  Zeigen Sie:
  \begin{enumerate}[leftmargin=*]
    \item
      Es gilt $\exp(f)^* = \exp(f^*)$.
    \item
      Ist $f$ normal, so ist auch $f^*$ normal.
    \item
      Ist $f$ selbstadjungiert, so ist auch $\exp(f)$ selbstadjungiert.
    \item
      Ist $f$ antiselbstadjungiert, so ist $\exp(f)$ orthogonal ($\Kbb = \Rbb$), bzw.\ unitär ($\Kbb = \Cbb$).
  \end{enumerate}
\end{question}


\begin{question}
  Es sei $V$ ein endlichdimensionaler Skalarproduktraum und $f \colon V \to V$ ein Endomorphismus.
  \newline
  Zeigen Sie die folgenden Äquivalenzen für den Fall $\Kbb = \Cbb$:
  \begin{enumerate}[leftmargin=*]
    \item
      Es gibt genau dann einen normalen Endomorphismus $g \colon V \to V$ mit $f = \exp(g)$, wenn $f$ normal und invertierbar ist.
    \item
      Es gibt genau dann einen antiselbstadjungierten Endomorphismus $g \colon V \to V$ mit $f = \exp(g)$, wenn $f$ unitär ist.
    \item
      Es gibt genau dann einen selbstadjungierten Endomorphismus $g \colon V \to V$ mit $f = \exp(g)$, wenn $f$ selbstadjungiert mit positiven Eigenwerten ist.
  \end{enumerate}
  Zeigen Sie die folgenden Aussagen für den Fall $\Kbb = \Rbb$:
  \begin{enumerate}[leftmargin=*, resume]
    \item
      Es gibt genau dann einen normalen Endomorphismus $g \colon V \to V$ mit $f = \exp(g)$, wenn $f$ normal und invertierbar ist, und alle (reellen) Eigenwerte von $g$ gerade Vielfachheit haben.
    \item
      Es gibt genau dann einen antiselbstadjungierten Endomorphismus $g \colon V \to V$ mit $f = \exp(g)$, wenn $f$ orthogonal ist und alle negativen (reellen) Eigenwerte von $f$ gerade Vielfachheit haben.
    \item
      Es gibt genau dann einen selbstadjungierten Endomorphismus $g \colon V \to V$ mit $f = \exp(g)$, wenn $f$ selbstadjungiert mit positiven (reellen) Eigenwerten ist.
  \end{enumerate}
\end{question}


\begin{question}
  Es sei
  \[
    W = \{(a_n)_{n \in \Zbb} \mid \text{$a_n \in \Rbb$ für alle $n \in \Zbb$}\}
  \]
  der Vektorraum der beidseitigen reellwertigen Folgen.
  Wir betrachten den Untervektorraum
  \[
    V \coloneqq
    \left\{
      (a_n)_{n \in \Zbb} \in W
    \,\middle|\,
      \sum_{n \in \Zbb} |a_n|^2 < \infty
   \right\}
  \]
  der quadratsummierbaren Folgen.
  \begin{enumerate}[leftmargin=*]
    \item
      Zeigen Sie, dass $V$ ein Untervektorraum von $W$ ist.
    \item
      Zeigen Sie für alle $(a_n)_{n \in \Zbb}, (b_n)_{n \in \Zbb} \in V$, dass
      \[
        \sum_{n \in \Zbb} a_n b_n < \infty.
      \]
      
      (\emph{Hinweis}:
       Zeigen sie zunächst, dass $ab \leq (a^2 + b^2)/2$ für alle $a, b \in \Rbb$.)
    \item
      Zeigen sie, dass
      \[
                  \bil{ (a_n)_{n \in \Zbb}, (b_n)_{n \in \Zbb} }
        \coloneqq \sum_{n \in \Zbb} a_n b_n
        \quad
        \text{für alle $(a_n)_{n \in \Zbb}, (b_n)_{n \in \Zbb} \in V$}
      \]
      ein Skalarprodukt auf $V$ definiert.
    \item
      Es sei
      \[
        R \colon V \to V,
        \quad
        (a_n)_{n \in \Zbb} \mapsto (a_{n-1})_{n \in \Zbb}
      \]
      der Rechtsshift-Operator.
      Zeigen Sie, dass $R$ ein Adjungiertes besitzt, und entscheiden Sie, ob $R$ selbstadjungiert, orthogonal, bzw.\ normal ist.
    \item
      Zeigen Sie, dass $R$ keine Eigenwerte besitzt.
    \item
      Es sei
      \[
        S \colon V \to V,
        \quad
        (a_n)_{n \in \Nbb} \mapsto (a_{-n})_{n \in \Nbb}.
      \]
      Zeigen Sie, dass $S$ ein Adjungiertes besitzt, und entscheiden Sie, ob $R$ selbstadjungiert, orthogonal, bzw.\ normal ist.
    \item
      Zeigen Sie, dass $S$ diagonalisierbar ist.
    \item
      Es sei
      \[
        U \coloneqq \{(a_n)_{n \in \Zbb} \in V \mid \text{$a_n = 0$ für fast alle $n \in \Zbb$}\}.
      \]
      Bestimmen Sie $U^\perp$ und entscheiden Sie, ob $V = U \oplus U^\perp$.
    \item
      Bestimmen Sie eine Orthonormalbasis von $U$.
  \end{enumerate}
\end{question}


\begin{question}
  \begin{enumerate}[leftmargin=*]
    \item
      Zeigen Sie, dass durch
      \[
        \sigma(A, B) \coloneqq \tr\left( A^T B \right)
        \quad
        \text{für alle $A, B \in \Mat_n(\Rbb)$}
      \]
      ein Skalarprodukt auf $\Mat_n(\Rbb)$ definiert wird.
    \item
      Zeigen Sie, dass die Standardbasis $(E_{ij})_{i,j=1,\dotsc,n}$ von $\Mat_n(\Rbb)$ mit
      \[
        (E_{ij})_{kl} \coloneqq \delta_{ik} \delta_{jl}
        \quad
        \text{für alle $1 \leq i,j,k,l \leq n$}
      \]
      eine Orthonormalbasis von $\Mat_n(\Rbb)$ bezüglich $\sigma$ bilden.
    \item
      Es sei
      \[
        S_+ \coloneqq \{A \in \Mat_n(\Rbb) \mid A^T = A\}
      \]
      der Untervektorraum der symmetrischen Matrizen, und
      \[
        S_- \coloneqq \{A \in \Mat_n(\Rbb) \mid A^T  = -A\}
      \]
      der Untervektorraum der schiefsymmetrischen Matrizen.
      Zeigen Sie, dass
      \[
        \Mat_n(\Rbb) = S_+ \oplus S_-,
      \]
      und dass die Summe orthogonal ist.
  \end{enumerate}
\end{question}


\begin{question}
  Es sei $V$ ein Skalarproduktraum und
  \[
    \Orthogonal(V) \coloneqq \{ f \in \End(V) \mid f f^* = \id \}.
  \]
  Zeigen Sie, dass $\Orthogonal(V)$ eine Untergruppe von $\GL(V)$ bildet.
\end{question}


\begin{question}
Zeigen sie, dass für eine Matrix $A \in \Mat_n(\Kbb)$ die folgenden Bedingungen äquivalent sind:
  \begin{enumerate}
    \item
      $A$ ist invertierbar mit $A^{-1} = A^*$.
    \item
      $A A^* = I$.
    \item
      $A^* A = I$.
    \item
      Die Spalten von $A$ bilden eine Orthonormalbasis des $\Kbb^n$.
    \item
      Die Zeilen von $A$ bilden eine Orthonormalbasis des $\Kbb^n$.
  \end{enumerate}
\end{question}


\begin{question}
  Es sei $A \in \Mat_n(\Cbb)$.
  \begin{enumerate}[leftmargin=*]
    \item
      Zeigen Sie, dass es eindeutige hermitsche Matrizen $B, C \in \Mat_n(\Cbb)$ mit $A = B + i C$ gibt.
    \item
      Zeigen Sie, dass $A$ genau dann normal ist, wenn $B$ und $C$ kommutieren.
  \end{enumerate}
\end{question}


\begin{question}
  Es sei $V$ ein endlichdimensionaler euklidischer Vektorraum mit Skalarprodukt $\bil{\cdot, \cdot}$, und es sei $G \subseteq \GL(V)$ eine endliche Untergruppe.
  \begin{enumerate}[leftmargin=*]
    \item
      Zeigen Sie, dass
      \[
        \bil{x,y}_G \coloneqq \frac{1}{|G|} \sum_{\phi \in G} \bil{\phi(x), \phi(y)}
        \quad
        \text{für alle $x, y \in V$}
      \]
      ein Skalarprodukt auf $G$ definiert.
    \item
      Zeigen Sie, dass $\bil{\cdot, \cdot}_G$ in dem Sinne $G$-invariant ist, dass
      \[
        \bil{\phi(x), \phi(y)} = \bil{x,y}
        \quad
        \text{für alle $x, y \in V$ und $\phi \in G$}.
      \]
    \item
      Folgern Sie, dass es eine Basis $\mc{B}$ von $V$ gibt, so dass $M_\mc{B}(\phi)$ für alle $\phi \in G$ eine orthogonale Matrix ist.
    \item
      Folgern Sie damit, dass es für $n = \dim V$ einen injektiven Gruppenhomomorphismus $\Phi \colon G \to \Orthogonal(n)$ gibt, $G$ also isomorph zu der Untergruppe $\im \Phi$ von $\Orthogonal(n)$ ist.
  \end{enumerate}
\end{question}


\begin{question}
  Es seien $F$ und $G$ zwei selbstadjungierte Endomorphismen eines Skalarproduktraums $V$.
  Zeigen Sie, dass $F \circ G$ genau dann selbstadjungierti ist, wenn $F$ und $G$ kommutieren.
\end{question}


\begin{question}
  Es sei $V$ ein euklidischer Vektorraum.
  Für jedes $\alpha \in V$ mit $\alpha \neq 0$ sei
  \[
    s_\alpha \colon V \to V,
    \quad\text{mit}\quad
              s_\alpha(x)
    \coloneqq x - 2 \frac{\bil{x, \alpha}}{\|\alpha\|^2} \alpha.
  \]
  Ferner seien
  \[
              L_\alpha
    \coloneqq \Rbb \alpha
    \quad\text{und}\quad
              H_\alpha
    \coloneqq L_\alpha^\perp
    =         \alpha^\perp
    =         \{ v \in V \mid \bil{v, \alpha} = 0 \}.
  \]
  \begin{enumerate}[leftmargin=*]
    \item
      Zeigen Sie, dass $s_\alpha^2 = \id_V$, und dass $s_{\lambda \alpha} = s_\alpha$ für alle $\lambda \in \Rbb^\times$.
    \item
      Zeigen Sie, dass $s_\alpha$ diagonalisierbar ist, und dass
      \[
        V_{-1}(s_\alpha) = L_\alpha
        \quad\text{und}\quad
        V_1(s_\alpha) = H_\alpha.
      \]
    \item
      Interpretieren Sie $V$ geometrisch anschaulich.
    \item
      Es sei $s' \colon V \to V$ ein Endomorphismus mit $s'(\alpha) = -\alpha$ und $s'(x) = x$ für alle $x \in H_\alpha$.
      Zeigen Sie, dass bereits $s' = s_\alpha$ gilt.
    \item
      Es sei $t \colon V \to V$ ein orthogonaler Isomorphismus. Zeigen Sie die Gleichheit
      \[
        t s_\alpha t^{-1} = s_{t(\alpha)}.
      \]
  \end{enumerate}
\end{question}


\begin{question}
  Es seien $V$ und $W$ euklidische Vektorräume, und $f \colon V \to V$ eine surjektive Funktion (!) mit
  \[
    \bil{f(v_1), f(v_2)} = \bil{v_1, v_2}
    \quad
    \text{für alle $v_1, v_2 \in V$}.
  \]
  Zeigen Sie, dass $f$ ein Isomorphismus ist.
\end{question}


\begin{question}
  Es sei $V$ ein endlichdimensionaler unitärer Vektorraum.
  Zeigen Sie, dass für eine lineare Abbildung $S \colon V \to V$ die folgenden Bedingungen äquivalent sind:
  \begin{enumerate}
    \item
      $S$ ist normal.
    \item
      $V$ hat eine Orthonormalbasis aus Eigenvektoren von $S$.
    \item
      Für jeden $S$-invarianten Untervektorraum $U \subseteq V$ ist auch das orthogonale Komplement $U^\perp$ invariant unter $S$.
  \end{enumerate}
\end{question}


\begin{question}
  Es sei $V$ ein endlichdimensionaler Skalarproduktraum über $\Kbb$.
  Es sei
  \[
    S \coloneqq \{ f \in \End_\Kbb(V) \mid f^* = f \}
  \]
  der Untervektorraum der selbstadjungierten Endomorphismen und
  \[
    A \coloneqq \{ f \in \End_\Kbb(V) \mid f^* = -f \}
  \]
  der Untervektorraum der antiselbstadjungierten Endomorphismen.
  \begin{enumerate}
    \item
      Zeigen Sie, dass
      \[
        \langle f, g \rangle \coloneqq \tr(f \circ g^*)
      \]
      ein Skalarprodukt auf $\End_\Kbb(V)$ definiert.
    \item
      Folgern Sie, dass
      \[
        |\tr(f g^*)|^2 \leq \tr(f f^*) \tr(g g^*)
        \quad
        \text{für alle $f, g \in \End_\Kbb(V)$}.
      \]
    \item
      Zeigen Sie, dass $\End_\Kbb(V) = S \oplus A$, und dass die Summe orthogonal ist.
  \end{enumerate}
\end{question}


\begin{question}
  Es sei $\det \colon \Mat_n(\Cbb) \to \Cbb^\times$ die Determinantenabbildung, wobei $\Cbb^\times$ die multiplikative Gruppe des Körpers bezeichnet.
  \begin{enumerate}[leftmargin=*]
    \item
      Zeigen Sie, dass $\det$ ein surjektiver Gruppenhomomorphismus ist.
    \item
      Bestimmen Sie den Kern von $\det$.
    \item
      Bestimmen Sie Bild und Kern der Einschränkung $\det|_{\GL_n(\Rbb)}$.
    \item
      Bestimmen Sie Bild und Kern der Einschränkung $\det|_{\Unitary_n}$.
    \item
      Bestimmen Sie Bild und Kern der Einschränkung $\det|_{\Orthogonal_n}$.
  \end{enumerate}
\end{question}


\begin{question}
  Es sei
  \[
    \Phi \colon \SUnitary(2) \to S^3,
    \quad
    \begin{pmatrix}
      a & b \\
      c & d
    \end{pmatrix}
    \mapsto
    \vect{a \\ c}
  \]
  die Abbildung auf die erste Spalte, wobei
  \[
              S^3
    \coloneqq \left\{ \vect{z_1 \\ z_2} \in \Cbb^2 \,\middle|\, |z_1|^2 + |z_2|^2 = 1 \right\}.
  \]
  \begin{enumerate}[leftmargin=*]
    \item
      Zeigen Sie, dass $\Phi$ wohldefiniert ist.
    \item
      Zeigen Sie, dass $\Phi$ bijektiv ist.
  \end{enumerate}
\end{question}


\begin{question}
  Zeigen Sie, dass die drei Gruppen $\SOrthogonal(2)$, $S^1$ und $\Unitary(1)$ isomorph sind.
\end{question}


\begin{question}
  Es sei $V$ ein euklidischer Vektorraum, und die Abbildung $P \colon V \to V$ sei orthogonal und eine Orthogonalprojektion.
  Bestimmen Sie $P$.
\end{question}


\begin{question}
  Es sei $A \in \Unitary(n)$.
  Zeigen Sie, dass $|\tr A| \leq n$.
  Wann gilt Gleichheit?
\end{question}


\begin{question}
  Es sei $V$ ein endlichdimensionaler Skalarproduktraum und $\mc{B} = (b_1, \dotsc, b_n)$ und $\mc{C} = (c_1, \dotsc, c_n)$ seien zwei geordnete Basen von $V$.
  \begin{enumerate}
    \item
      Die Basis $\mc{C}$ sei genau die Orthonormalbasis von $V$, die aus $\mc{B}$ durch Anwendung des Gram-Schmidt-Verfahrens entstehen.
      Zeigen Sie, dass die Basiswechselmatrix $T_{\mc{C} \to \mc{B}}$ (von $\mc{C}$ nach $\mc{B}$) eine obere Dreiecksmatrix mit positiven reellen Diagonaleinträgen ist.
    \item
      Zeigen oder widerlegen Sie die umgekehrte Aussage:
      Ist die Basiswechselmatrix $T_{\mc{C} \to \mc{B}}$ eine obere Dreiecksmatrix mit positiven reellen Diagonaleinträgen, so ist $\mc{C}$ notwendigerweise die Orthonormalbasis von $V$, die durch das Gram-Schmidt-Verfahren aus $\mc{B}$ entsteht.
  \end{enumerate}
\end{question}


\begin{question}
  \begin{enumerate}[leftmargin=*]
    \item
      Es seien $z_1, \dotsc, z_n \in \Cbb$ paarweise verschieden Punkte.
      Zeigen Sie, dass es für beliebige Werte $w_1, \dotsc, w_n \in \Cbb$ ein Polynom $P \in \Cbb[T]$ mit $P(z_j) = w_j$ für alle $j = 1, \dotsc, n$ gibt.
    \item
      Es sei $f \colon V \to V$ ein normaler Endomorphismus eines endlichdimensionalen unitären Vektorraums $V$.
      Zeigen Sie, dass es ein Polynom $P \in \Cbb[T]$ mit $f^* = P(f)$ gibt.
      
      (\emph{Hinweis}:
       $f$ ist diagonalisierbar.)
  \end{enumerate}
\end{question}


\begin{question}
  Es sei $V$ ein endlichdimensionaler reeller Vektorraum mit $n \coloneqq \dim V$.
  \begin{enumerate}
    \item
      Zeigen Sie, dass es für jede Basis $\mc{B} = \{b_1, \dotsc, b_n\}$ von $V$ genau ein Skalarprodukt $\bil{\cdot, \cdot}_\mc{B}$ auf $V$ gibt, so dass $\mc{B}$ eine Orthonormalbasis von $V$ bezüglich $\bil{\cdot, \cdot}_\mc{B}$ ist.
    \item
      Untersuchen Sie die Abbildung
      \[
        \{\text{Basen von $V$}\} \to \{\text{Skalarprodukte auf $V$}\},
        \quad
        \mc{B} \mapsto \bil{\cdot, \cdot}_\mc{B}
      \]
      auf Injektivität und Surjektivität.
  \end{enumerate}
\end{question}












% BILINEAR FORMS


\begin{question}
  Es sei $V$ ein $K$-Vektorraum, $\beta \colon V \times V \to K$ eine symmetrische Bilinearform und $q \colon V \to K$ die zugehörige quadratische Form.
  \begin{enumerate}[leftmargin=*]
    \item
      Zeigen Sie für den Fall $\ringchar K \neq 2$ mithilfe einer Polarisationsformel, dass $\beta$ durch $q$ bereits eindeutig bestimmt ist.
    \item
      Folgern Sie:
      Ist $\ringchar K \neq 2$, $V \neq 0$ und $\beta$ nicht entartet, d.h.\ für jedes $v \in V$ mit $v \neq 0$ gibt es ein $w \in V$ mit $\beta(v, w) \neq 0$, so gibt es ein $v \in V$ mit $\beta(v,v) \neq 0$.
    \item
      Zeigen Sie für den Fall $\ringchar K = 2$, dass es verschieden symmetrische Bilinearformen mit gleicher quadratische Form geben kann.
      Geben Sie auch ein explizites Beispiel hierfür an.
  \end{enumerate}
\end{question}


\begin{question}
  Es sei $V$ eine reeller Vektorraum und $\bil{\cdot, \cdot} \colon V \times V \to \Rbb$ eine symmetrische Bilinearform.
  Zeigen Sie, dass die folgenden Aussagen im Allgemeinen gelten, oder geben Sie jeweils ein Gegenbeispiel an:
  \begin{enumerate}[leftmargin=*]
    \item
      Ist $\bil{v, v} \geq 0$ für alle $v \in V$, so ist $\bil{\cdot, \cdot}$ ein Skalarprodukt.
    \item
      Ist $\mc{B}$ eine Basis von $V$ mit $\bil{b, b} > 0$ für alle $b \in \mc{B}$, so ist $\bil{\cdot, \cdot}$ ein Skalarprodukt.
    \item
      Ist $\mc{B}$ eine Basis von $V$ mit $\bil{b_1, b_2} > 0$ für alle $b_1, b_2 \in \mc{B}$, so ist $\bil{\cdot, \cdot}$ ein Skalarprodukt.
    \item
      Die Teilmenge $U \coloneqq \{ v \in V \mid \text{$\bil{v, w} = 0$ für alle $w \in V$} \}$ ist ein Untervektorraum von $V$.
    \item
      Die Teilmengen
      \[
        U_+ \coloneqq \{ v \in V \mid \bil{v, v} \geq 0 \}
        \quad\text{und}\quad
        U_- \coloneqq \{ v \in V \mid \bil{v, v} \leq 0 \}
      \]
      sind Untervektorräume von $V$.
    \item
      Die Teilmenge $U_0 \coloneqq \{ v \in V \mid \bil{v, v} = 0 \}$ ist ein Untervektorraum von $V$.
    \item
      Für jeden Untervektorraum $U \subseteq V$ ist $\dim V = \dim U + \dim U^\perp$.
    \item
      Ist $U \subseteq V$ ein Untervektorraum mit $(U^\perp)^\perp = V$, so ist $U = V$.
    \item
      Für alle Untervektorräume $U_1 \subseteq U_2$ gilt $(U_1 + U_2)^\perp = U_1^\perp \cap U_2^\perp$.
    \item
      Für $U_0 \coloneqq \{ v \in V \mid \text{$\bil{v, w} = 0$ für alle $w \in V$} \}$ und jeden Untervektorraum $U \subseteq V$ gilt
      \[
        (U^\perp)^\perp = U + U_0.
      \]
  \end{enumerate}
\end{question}


\begin{solution}
  \begin{enumerate}[leftmargin=*]
    \item
      Falsch, siehe Nullbilinearform.
    \item
      Falsch, siehe
      \[
        \begin{pmatrix}
          0 & 1
          1 & 0
        \end{pmatrix}.
      \]
    \item
      Falsch, siehe
      \[
        \begin{pmatrix}
          1 & 2 \\
          2 & 1
        \end{pmatrix}
      \]
    \item
      Wahr.
    \item
      Falsch, für
      \[
        \begin{pmatrix}
          1 & 2 \\
          2 & 1
        \end{pmatrix}
      \]
      sind $e_1$ und $e_2$ in $U_+$, aber $e_1 - e_2$ nicht.
    \item
      Nein, siehe
      \[
        \begin{pmatrix}
          0 & 1 \\
          1 & 0
        \end{pmatrix}
      \]
    \item
      Nein, siehe Nullbilinearform.
    \item
      Falsch, siehe den Span von $e_2$ und
      \[
        \begin{pmatrix}
          0 & 0 \\
          0 & 1
        \end{pmatrix}.
      \]
    \item
      Die Aussage gilt.
    \item
      Falsch: Nehme ein Skalarprodukt auf einem unendlichdimensionalen Raum, und einen dichten, echten Unterraum.
  \end{enumerate}
\end{solution}


\begin{question}
  Ist $\beta \colon V \times W \to K$ eine Bilinearform, so heißen eine Basis $\mc{B} = (v_i)_{i \in I}$ von $V$ und eine Basis $\mc{C} = (w_i)_{i \in I}$ von $W$ \emph{dual bezüglich $\beta$}, falls
  \[
    \beta(v_i, w_j) = \delta_{ij}
    \quad
    \text{für alle $i, j \in I$}.
  \]
  Es sei zunächst $V$ ein $K$-Vektorraum.
  \begin{enumerate}
    \item
      Zeigen Sie, dass die \emph{Evaluation}
      \[
        e \colon V \times V^* \to K
        \quad\text{mit}\quad
        e(v, \varphi) = \varphi(e)
      \]
      eine $K$-bilineare Abbildung ist.
    \item
      Zeigen Sie im Falle der Endlichdimensionalität von $V$, dass es zu jeder Basis $\mc{B} = (b_1, \dotsc, b_n)$ von $V$ genau eine Basis $\mc{C}$ von $V^*$ gibt, die bezüglich $e$ dual zu $\mc{B}$ ist.
      Woher kennen Sie diese Basis?
  \end{enumerate}
  Von nun an sei $V$ ein endlichdimensionaler euklidischer Vektorraum mit Skalarprodukt $\bil{\cdot, \cdot}$.
  \begin{enumerate}[resume]
    \item
      Zeigen Sie, dass die Abbildung
      \[
        \Phi \colon V \to V^*,
        \quad
        v \mapsto \bil{-, v}
      \]
      ein Isomorphismus ist.
    \item
      Folgern Sie, dass es für jede Basis $\mc{B} = (b_1, \dotsc, b_n)$ von $V$ genau eine Basis $\mc{B}^\circ = (b_1^\circ, \dotsc, b_n^\circ)$ von $V$ gibt, die bezüglich $\bil{\cdot, \cdot}$ dual zu $\mc{B}$ ist.
      
      (\emph{Hinweis}:
       Formulieren Sie die Aussage, dass $\mc{C}$ dual zu $\mc{B}$ ist, mithilfe von $\Phi$ um.)
    \item
      Zeigen Sie, dass für jede Basis $\mc{B}$ von $V$ die Gleichheit $(\mc{B}^\circ)^\circ = \mc{B}$ gilt.
      Folgern Sie, dass die Abbildung
      \[
            \left\{ \text{geordnete Basen von $V$} \right\}
        \to \left\{ \text{geordenet Basen von $V$} \right\},
        \quad
        \mc{B} \mapsto \mc{B}^\circ
      \]
      bijektiv ist.
    \item
      Unter welchen Namen kennen Sie Basen von $V$, die bezüglich $(-)^\circ$ selbstdual sind, die also $\mc{B}^\circ = \mc{B}$ erfüllen?
  \end{enumerate}
\end{question}


\begin{question}
  Es sei $V$ ein $K$-Vektorraum und $\beta \colon V \times V \to K$ eine symmetrische Bilinearform.
  \begin{enumerate}[leftmargin=*]
    \item
      Zeigen Sie, dass
      \[
        \rad(\beta) \coloneqq \{ v \in V \mid \text{$\beta(v, w) = 0$ für alle $w \in V$} \}
      \]
      ein Untervektorraum von $V$ ist.
      (Man bezeichnet $\rad(\beta)$ als das \emph{Radikal} von $\beta$.)
    \item
      Zeigen Sie, dass $\beta$ eine symmetrische Bilinearform $\bar{\beta} \colon (V\!/ \rad(\beta)) \times (V\!/\rad(\beta)) \to K$ mit
      \[
        \bar{\beta}([v], [w])
        \coloneqq
        \beta(v,w)
        \quad
        \text{für alle $v, w \in V$}
      \]
      induziert.
    \item
      Zeigen Sie, dass $\bar{\beta}$ nicht entartet ist, d.h.\ dass für das Radikal
      \[
                  \rad(\bar{\beta})
        \coloneqq \{ x \in V\!/U \mid \text{$\bar{\beta}(x,y) = 0$ für alle $y \in V\!/U$} \}
      \]
      bereits $\rad(\bar{\beta}) = 0$ gilt.
    \item
      Inwiefern gelten die obigen Aussagen noch, wenn man $U$ durch
      \[
        W \coloneqq \{v \in V \mid \beta(v,v) = 0\}
      \]
      ersetzt?
  \end{enumerate}
\end{question}


\begin{question}
   Es sei $V$ ein $K$-Vektorraum und $b \colon V \times V \to K$ eine Bilinearform.
  \begin{enumerate}
    \item
      Zeigen Sie für $\ringchar K \neq 2$, dass es eindeutige Bilinearformen $b_s, b_a \colon V \times V \to K$ gibt, so dass
      \begin{itemize}
        \item
          $b = b_s + b_a$
        \item
          $b_s$ ist symmetrisch und $b_a$ ist antisymmetrisch
      \end{itemize}
    \item
      Zeigen Sie durch Angabe eines Gegenbeispiels, dass die Aussage für $\ringchar K = 2$ nicht gilt.
  \end{enumerate}
  Es sei nun $V$ der Vektorraum der Polynomfunktionen $\Rbb \to \Rbb$.
  \begin{enumerate}[resume]
    \item
      Zeigen Sie, dass die Abbildung $b \colon V \times V \to \Rbb$ mit
      \[
        b(p, q) \coloneqq \int_{-1}^1 p(t) q'(t) \dd{t}
      \]
      eine Bilinearform ist.
    \item
      Geben Sie den symmtrischen Anteil $b_s$ in einer Form an, in der kein Integral vorkommt.
%     \item
%       Geben Sie eine Basis von $V_n$ an, bezüglich der $b_s$ durch eine Diagonalmatrix mit Einträgen $0, 1, -1$ beschrieben wird, und bestimmen Sie die Signatur von $b_s$ auf $V_n$.
  \end{enumerate}
\end{question}


\begin{question}
  Für je zwei $K$-Vektorräume $V$ und $W$ sei
  \[
              \Bil(V, W)
    \coloneqq \{b \colon V \times  W \to K \mid \text{$b$ ist bilinear}\}
  \]
  der Raum der Bilinearformen $V \times W \to K$.
  \begin{enumerate}[leftmargin=*]
    \item
      Zeigen Sie, dass die Flipabbildung
      \[
        F \colon \Bil(V, W) \to \Bil(W, V),
        \quad
        b \mapsto F(b)
        \quad\text{mit}\quad
        F(b)(w,v) = b(v,w)
      \]
      ein Isomorphismus von $K$-Vektorräumen ist.
    \item
      Es sei $b \in \Bil(V, W)$ eine Bilinearform.
      Zeigen Sie, dass $b$ ein lineare Abbildung
      \[
        \Phi_{V,W}(b) \colon V \to W^*,
        \quad
        v \mapsto b(v, -)
      \]
      induziert.
      Dabei ist
      \[
        b(v, -) \colon W \to K,
        \quad
        w \mapsto b(v,w).
      \]
    \item
      Zeigen Sie, dass die Abbildung
      \[
        \Phi_{V,W} \colon \Bil(V, W) \to \Hom(V, W^*),
        \quad
        b \mapsto \Phi_{V,W}(b)
      \]
      ein Isomorphismus von $K$-Vektorräumen ist.
    \item
      Geben Sie mithilfe der vorherigen Aufgabenteile explizit einen Isomorphismus
      \[
        \Hom(V, W^*) \to \Hom(W, V^*)
      \]
      an.
  \end{enumerate}
  Wir betrachten nun den Fall $W = V^*$.
  \begin{enumerate}[resume, leftmargin=*]
    \item
      Zeigen Sie, dass die Evaluation
      \[
        e \colon V \times V^* \to K,
        \quad
        (v, \varphi) \mapsto \varphi(v)
      \]
      eine Bilinearform ist.
   \item
      Nach den vorherigen Aufgabenteilen entspricht die Bilinearform $e$ einer linearen Abbildung $V \to V^{**}$, sowie einer linearen Abbildung $V^* \to V^*$.
      Bestimmen Sie diese Abbildungen.
    \item
      Woher kennen Sie diese Abbildung?
  \end{enumerate}
\end{question}


\begin{question}
  Es seien $V$ und $W$ zwei $K$-Vektorräume und $f \colon V \to W$ eine lineare Abbildung.
  \begin{enumerate}[leftmargin=*]
    \item
      Geben Sie die Definition der dualen Abbildung $f^* \colon W^* \to V^*$ an, und zeigen Sie ihre Linearität.
    \item
      Zeigen Sie für jeden $K$-Vektorraum $U$, dass die Abbildung
      \[
        \bil{\cdot, \cdot} \colon U \times U^* \to K
        \quad\text{mit}\quad
        \bil{v, \varphi} = \varphi(v)
        \quad\text{für alle $v \in V$, $\varphi \in V^*$}
      \]
      eine Bilinearform ist.
    \item
      Zeigen Sie, dass
      \[
        \bil{f(v), \psi} = \bil{v, f^*(\psi)}
        \quad
        \text{für alle $v \in V$, $\psi \in W^*$}.
      \]
  \end{enumerate}
\end{question}


\begin{question}
  \begin{enumerate}[leftmargin=*]
    \item
      Zeigen Sie, dass die Abbildung
      \[
        \sigma \colon \Mat_n(K) \times \Mat_n(K) \to K
        \quad\text{mit}\quad
        \sigma(A, B) \coloneqq \tr(AB)
      \]
      eine symmetrische Bilinearform ist.
      Man bezeichnet diese als die \emph{Traceform}.
    \item
      Zeigen Sie, dass $\sigma$ in dem Sinne assoziativ ist, dass
      \[
        \sigma(AB, C) = \sigma(A, BC)
        \quad
        \text{für alle $A, B, C \in \Mat_n(K)$}.
      \]
    \item
      Zeigen sie, dass $\sigma$ nicht entartet ist, d.h.\ dass es für jedes $A \in \Mat_n(K)$ mit $A \neq 0$ ein $B \in \Mat_n(K)$ mit $\sigma(A, B) \neq 0$ gibt.
  \end{enumerate}
  Es sei nun
  \[
    S_+ \coloneqq \{ A \in \Mat_n(K) \mid A^T = A \}
  \]
  der Untervektorraum der symmetrischen Matrizen und
  \[
    S_- \coloneqq \{ A \in \Mat_n(K) \mid A^T = -A \}
  \]
  der Untervektorraum der schiefsymmetrischen Matrizen.
  Zeigen Sie:
  \begin{enumerate}[leftmargin=*, resume]
    \item
      Ist $\ringchar K \neq 2$, so sind $S_+$ und $S_-$ bezüglich $\sigma$ orthogonal zueinander.
    \item
      Ist $K = \Rbb$, so ist die Einschränkung von $\sigma$ auf $S_+$ positiv definit, und die Einschränkung auf $S_-$ negativ definit.
  \end{enumerate}
\end{question}


\begin{question}
  Es sei $V$ ein endlichdimensionaler $K$-Vektorraum, $b \colon V \times V \to K$ eine Bilinearform, $\mc{B} = (b_1, \dotsc, b_n)$ eine Basis von $V$, und $\mc{B}^* = (b_1^*, \dotsc, b_n^*)$ die entsprechende duale Basis von $V^*$.
  \begin{enumerate}[leftmargin=*]
    \item
      Zeigen Sie, dass die Abbildung
      \[
        B \colon V \to V^*,
        \quad
        v \mapsto b(-,v)
      \]
      $K$-linear ist.
    \item
      Zeigen Sie die Gleihheit
      \[
        \Mat_\mc{B}(b) = \Mat_{\mc{B}, \mc{B}^*}(B).
      \]
      (Beachten Sie, dass auf der linken Seite die darstellende Matrix einer Bilinearform steht, und auf der rechten Seite die darstellende Matrix einer linearen Abbildung.)
  \end{enumerate}
\end{question}










% LIE STUFF AND COMMUTATORS


\begin{question}
  Für jede Matrix $X \in \Mat_n(K)$ sei
  \[
    \lambda_X \colon \Mat_n(K) \to \Mat_n(K),
    \quad
    A \mapsto XA
  \]
  die Linksmultiplikation mit $X$,
  \[
    \rho_X \colon \Mat_n(K) \to \Mat_n(K),
    \quad
    A \mapsto AX
  \]
  die Rechtsmultiplikation mit $X$, und
  \[
    \ad_X = [X, -] \colon \Mat_n(K) \to \Mat_n(K),
    \quad
    A \mapsto [X, A] = XA - AX
  \]
  der Kommutator mit $X$.
  \begin{enumerate}[leftmargin=*]
    \item
      Zeigen Sie:
      Ist $X$ nilpotent, so sind auch $\lambda_X$ und $\rho_X$ nilpotent.
    \item
      Folgern Sie:
      Ist $X$ nilpotent, so ist auch $\ad_X$ nilpotent.
      
      (\emph{Hinweis}:
      Nutzen Sie, dass $\ad_X = \lambda_X - \rho_X$.)
    \item
      Zeigen Sie:
      Ist $X$ diagonalisierbar, so sind auch $\lambda_X$ und $\rho_X$ diagonalisierbar.
      
      (\emph{Hinweis}:
      Betrachten Sie zunächst den Fall, dass $X$ eine Diagonalmatrix ist.)
    \item
      Folgern Sie:
      Ist $X$ diagonalisierbar, so ist auch $\ad_X$ diagonalisierbar.
      
      (\emph{Hinweis}:
       Nutzen Sie, dass $\ad_X = \lambda_X - \rho_X$.)
  \end{enumerate}
\end{question}


\begin{question}
  Es sei $K$ ein Körper und
  \[
              \slLie_n(K)
    \coloneqq \{ A \in \Mat_n(K) \mid \tr A = 0 \}.
  \]
  \begin{enumerate}
    \item
      Zeigen Sie, dass $\slLie_n(K)$ ein Untervektorraum von $\Mat_n(K)$ mit $\dim \slLie_n(K) = n^2 - 1$ ist.
    \item
      Zeigen Sie, dass
      \[
        \mc{B}
        \coloneqq
              \{ E_{ij} \mid 1 \leq i \neq j \leq n \}
        \cup  \{ E_{11} - E_{ii} \mid 2 \leq i \leq n \}
      \]
      eine Basis von $\slLie_n(K)$ ist, wobei $E_{ij} \in \Mat_n(K)$ die Matrix ist, deren $(i,j)$-ter Eintrag $1$ ist, und für die alle anderen Einträge $0$ sind.
  \end{enumerate}
  Es sei nun $C \coloneqq \Ell( [A,B] \mid A, B \in \Mat_n(K) )$.
  \begin{enumerate}[resume]
    \item
      Zeigen Sie, dass $\tr([A,B]) = 0$ für alle $A, B \in \Mat_n(K)$.
      Folgern Sie, dass $C \subseteq \slLie_n(K)$.
    \item
      Zeigen Sie, dass $\slLie_n(K) \subseteq C$, indem Sie jedes der Basiselement aus $\mc{B}$ also Kommutator schreiben.
      
      (\emph{Hinweis}:
       Überlegen Sie zunächst, dass $E_{ij} E_{kl} = \delta_{jk} E_{il}$ für alle $1 \leq i, j, k, l \leq n$.)
  \end{enumerate}
  Es ist also $\slLie_n(K) = \Ell( [A,B] \mid A, B \in \Mat_n(K) )$ ein ($n^2 - 1$)-dimensionaler Untervektorraum.
  Es sei nun $f \colon \Mat_n(K) \to K$ eine lineare Abbildung mit $f(AB) = f(BA)$ für alle $A, B \in \Mat_n(K)$.
  \begin{enumerate}[resume]
    \item
      Zeigen Sie, dass $f$ eine eindeutige lineare Abbildung
      \[
        \overline{f} \colon \Mat_n(K) / \slLie_n(K) \to K,
        \quad
        [A] \mapsto f(A)
      \]
      induziert.
      Zeigen Sie, dass $\overline{\tr} \neq 0$.
    \item
      Zeigen Sie, dass $\Mat_n(K) / \slLie(K)$ eindimensional ist.
      Folgern Sie, dass es ein eindeutiges $\lambda \in K$ mit $\overline{f} = \lambda \overline{\tr}$ gibt.
    \item
      Folgern Sie, dass bereits $f = \lambda \tr$ gilt.
      (Die Spur ist also durch die Eigenschaft, dass $\tr(AB) = \tr(BA)$ für alle $A, B \in \Mat_n(K)$, bis auf skalares Vielfaches eindeutig bestimmt.)
  \end{enumerate}
\end{question}


\begin{question}
  Das \emph{Zentrum} eines Rings $R$ ist definiert als
  \[
    Z(R) \coloneqq \{r \in R \mid \text{$rs = sr$ für alle $s \in R$}.
  \]
  Man bemerke, dass $R$ genau dann kommutativ ist, wenn $Z(R) = R$.
  Wir werden $Z(\Mat_n(K))$ bestimmen.
  Hierfür sei
  \[
    D_n(K) \coloneqq K I = \{\lambda I \mid \lambda \in K\}
  \]
  der Untervektorraum der Skalarmatrizen.
  \begin{enumerate}[leftmargin=*]
    \item
      Zeigen Sie, dass $D_n(K) \subseteq Z(\Mat_n(K))$.
    \item
      Zeigen Sie für $A \in Z(\Mat_n(K))$, dass $A$ eine Diagonalmatrix ist.
      
      (\emph{Hinweis}:
       Betrachten Sie die Matrizen $E_{ii}$ für $1 \leq i \leq n$.)
    \item
      Zeigen Sie ferner, dass alle Diagonaleinträge von $A$ bereits gleich sind.
      
      (\emph{Hinweis}:
       Betrachten Sie die Matrizen $E_{ij}$ mit $1 \leq i,j \leq n$.)
    \item
      Folgern Sie, dass $Z(\Mat_n(K)) = D_n(K)$.
  \end{enumerate}
\end{question}


\begin{question}
  Es sei $V$ ein endlichdimensionaler $\Cbb$-Vektorraum, und es seien $K, E \colon V \to V$ zwei Endomorphismen mit
  \[
    \text{$K$ ist invertierbar}
    \quad\text{und}\quad
    KE = 2EK.
  \]
  \begin{enumerate}[leftmargin=*]
    \item
      Zeigen Sie, dass
      \[
        (K - 2 \lambda \id_V)^n E = 2^n E (K - \lambda \id_V)^n
      \]
      für alle $n \in \Nbb$.
    \item
      Folgern Sie, dass $E( V^\sim_\lambda(K) ) \subseteq V^\sim_{2\lambda}(K)$ für alle $\lambda \in \Cbb$.
    \item
      Folgern Sie, dass $E$ nilpotent ist.
  \end{enumerate}
\end{question}


\begin{question}
  Es sei $V$ ein $K$-Vektorrraum und $[-,-] \colon V \times V \to V$ eine alternierend bilineare Abbildung.
  Für jedes $x \in V$ sei
  \[
    \ad_x \coloneqq [x,-] \colon V \to V, \quad y \mapsto [x,y].
  \]
  Zeigen Sie, dass die folgenden beiden Aussagen äquivalent sind:
  \begin{enumerate}
    \item
      $[-,-]$ erfüllt die Jacobi-Identität, d.h.\ es ist
      \[
        [x,[y,z]] + [y,[z,x]] + [z,[x,y]] = 0
        \quad
        \text{für alle $x, y, z \in V$}.
      \]
    \item
      Es gilt
      \[
          \ad_x([y,z])
        = [\ad_x(y), z] + [y, \ad_x(z)]
        \quad
        \text{für alle $x, y, z \in V$}.
      \]
      (Man sagt, dass $\ad_x$ eine Derivation bezüglich $[-,-]$ ist.)
  \end{enumerate}
\end{question}


\begin{question}
  Es seien $E$ und $H$ zwei Endomorphismen eines $\Cbb$-Vektorraums $V$, so dass $[H,E] = 2E$.
  \begin{enumerate}[leftmargin=*]
    \item
      Zeigen Sie, dass $E(V_\lambda(H)) \subseteq V_{\lambda + 2}(H)$ für alle $\lambda \in K$.
    \item
      Folgern Sie: Ist $V$ endlichdimensional und $H$ diagonalisierbar, so ist $E$ nilpotent.
  \end{enumerate}
\end{question}


\begin{question}
  Es sei $B \in \Mat_n(\Kbb)$.
  Es seien
  \[
              \Orthogonal(B)
    \coloneqq \{ S \in \GL_n(\Kbb) \mid S^T B S = B \}
  \]
  und
  \[          \gLie(B)
    \coloneqq \{ A \in \Mat_n(\Kbb) \mid A^T B = - B A \}
  \]
  \begin{enumerate}[leftmargin=*]
    \item
      Zeigen Sie, dass $\Orthogonal(B)$ eine Untergruppe von $\GL_n(\Kbb)$ ist.
    \item
      Zeigen Sie, dass $\gLie(B)$ eine Lie-Unteralgebra von $\gl_n(\Kbb)$ ist, dass also $[A_1, A_2] \in \gLie(B)$ für alle $A_1, A_2 \in \gLie(B)$.
    \item
      Zeigen Sie, dass $\exp(A) \in \Orthogonal(B)$ für alle $A \in \gLie(B)$.
      (\emph{Hinweis}:
       Zeigen Sie zunächst, dass $\exp(A)^T B = B \exp(-A)$.)
    \item
      Geben Sie für $\Kbb = \Rbb$ eine Matrix $B \in \Mat_n(\Rbb)$ an, so dass $\Orthogonal(B) = \Orthogonal(n)$.
      Was sind dann die Elemente von $\gLie(B)$?
  \end{enumerate}
\end{question}


\begin{question}
  Für einen endlichdimensionalen $\Kbb$-Vektorraum $V$ und eine Bilinearform $\beta \colon V \times V \to \Kbb$ sei
  \[
              O(\beta)
    \coloneqq \{ \phi \in \GL(V) \mid \text{$\beta(\phi(x), \phi(y)) = \beta(x,y)$ für alle $x,y \in V$} \}
  \]
  die Isometriegruppe von $\beta$, und
  \[
              \gLie(\beta)
    \coloneqq \{ f \in \End(V) \mid \text{$\beta(f(x),y) = -\beta(x, f(y))$ für alle $x, y \in V$} \}
  \]
  die assoziierte Lie-Algebra.
  \begin{enumerate}[leftmargin=*]
    \item
      Zeigen Sie, dass $O(\beta)$ eine Untergruppen von $\GL(V)$ ist.
    \item
      Zeigen Sie, dass $\gLie(\beta)$ eine Lie-Unteralgebra von $\gl(V)$ ist, d.h.\ dass $[f,g] \in \gLie(\beta)$ für alle $f, g \in \gLie(\beta)$.
    \item
      Zeigen Sie, dass $\exp(f) \in O(\beta)$ für alle $f \in \gLie(\beta)$.
      
      (\emph{Hinweis}:
       Zeigen Sie zunächst, dass $\beta(\exp(f)(x) ,y) = \beta(x, \exp(-f)(y))$ für alle $x, y \in V$.
       Nutzen Sie hierfür, dass die bilineare Abbildung $\beta$ in beiden Argumenten stetig ist.)
    \item
      Es sei $\Kbb = \Rbb$ und $\bil{\cdot, \cdot}$ ein Skalarprodukt auf $V$.
      Unter welchen Begriffen sind die Elemente aus $G(\bil{\cdot, \cdot})$ und $\gLie(\bil{\cdot, \cdot})$ bekannt?
  \end{enumerate}
\end{question}


\begin{question}
  Es sei $V$ ein $K$-Vektorraum und $m \colon V \times V \to V$ eine bilineare Abbildung.
  Eine lineare Abbildung $D \colon V \to V$ heißt \emph{$m$-Derivation}, falls
  \[
    D(m(x,y))
    = m(D(x), y) + m(x, D(y))
    \quad
    \text{für alle $x, y \in V$}.
  \]
  Es sei
  \[
              \Der(m)
    \coloneqq \{ D \colon V \to V \mid \text{$D$ ist eine $m$-Derivation} \}.
  \]
  \begin{enumerate}[leftmargin=*]
    \item
      Zeigen Sie für den Fall $V = K[X]$ und die Multiplikation
      \[
        m(p,q) \coloneqq p \cdot q
        \quad
        \text{für alle $p, q \in K[X]$},
      \]
      dass die Ableitung
      \[
        D \colon K[X] \to K[X],
        \quad
        \sum_{d=0}^n a_d X^d  \mapsto \sum_{d=1}^n a_d d X^{d-1} 
      \]
      eine $m$-Derivation ist.
      Unter welchem Namen ist dieser Umstand für gewöhnlich bekannt?
    \item
      Zeigen Sie, dass $\Der(m)$ ein Untervektorraum von $\End(V)$ ist.
    \item
      Zeigen Sie, dass $\Der(m)$ eine Lie-Unteralgebra von $\End(V)$ ist, d.h.\ dass für alle $D_1, D_2 \in \Der(m)$ auch $[D_1, D_2] \in \Der(m)$.
  \end{enumerate}
\end{question}










% OTHER STUFF


\begin{question}
  Es sei $V$ ein endlichdimensionaler $K$-Vektorraum mit Basis $\mc{B} = (b_1, \dotsc, b_n)$ und $\omega \colon V^n \to K$ eine alternierende Multilinearform.
  \begin{enumerate}
    \item
      Zeigen Sie, dass genau dann $\omega \neq 0$, wenn $\omega(b_1, \dotsc, b_n) \neq 0$.
    \item
      Es sei $f \colon V \to V$ ein Endomorphismus und
      \[
        \omega_f \coloneqq \omega \circ f^{\times n} \colon V^n \to V,
      \]
      Zeigen Sie, dass $\omega_f$ ebenfalls multilinear und alternierend ist.
    \item
      Zeigen Sie, dass $\omega_f = \det(f) \omega$.
  \end{enumerate}
\end{question}


\begin{question}
  Es sei $V$ ein endlichdimensionaler $K$-Vektorraum und $n \coloneqq \dim V$.
  Es sei $\omega \colon V^m \to V$ eine alternierende Multilinearform.
  Zeigen Sie, dass $\omega = 0$.
\end{question}


\begin{question}
  Es sei $n \geq 1$.
  Entscheiden Sie, welche der folgenden Aussagen wahr oder falsch sind:
  \begin{enumerate}[leftmargin=*]
    \item
      Die Wegzusammenhangskomponenten von $\GL_n(\Rbb)$ sind die beiden Untergruppen
      \[
        \GL_n(\Rbb)_+ = \{ S \in \GL_n(\Rbb) \mid \det S > 0 \}
        \quad\text{und}\quad
        \GL_n(\Rbb)_- = \{ S \in \GL_n(\Rbb) \mid \det S < 0 \}.
      \]
    \item
      Die Gruppe $\SOrthogonal(n)$ ist nicht wegzusammenhängend.
    \item
      Die Gruppe $\Orthogonal(n)$ ist eine Wegzusammenhangskomponente von $\GL_n(\Rbb)$.
    \item
      Die schiefsymmetrischen Matrizen
      \[
        \mathfrak{o}_n(\Rbb) = \{ A \in \Mat_n(\Rbb) \mid A^T = -A \}
      \]
      sind wegzusammenhängend.
    \item
      Die Gruppe $\Unitary(n) \cap \GL_n(\Rbb)_+$ besteht aus zwei Wegzusammenhangskompenenten.
    \item
      Ist $G$ eine wegzusammenhängende Untergruppe von $\GL_n(\Kbb)$, so ist
      \[
        G' \coloneqq \{ S \in G \mid \det g = 1 \}
      \]
      ebenfalls eine wegzusammenhängende Untergruppe von $\GL_n(\Kbb)$.
    \item
      Jede Untergruppe von $\GL_n(\Cbb)$ ist wegzusammenhängend.
    \item
      Die Gruppe $\SUnitary(n) \cap \Orthogonal(n)$ ist wegzusammenhängend.
    \item
      Es ist $G = \{A^* = - A\}$ eine wegzusammenhängende Untergruppe von $A$.
    \item
      Die Gruppe der Drehmatrizen
      \[
        \left\{
          \begin{pmatrix*}[r]
            \cos \varphi  & -\sin \varphi \\
            \sin \varphi  &  \cos \varphi 
          \end{pmatrix*}
        \,\middle|\,
        \varphi \in \Rbb
        \right\}
      \]
      ist wegzusammenhängend.
  \end{enumerate}
\end{question}

\begin{solution}
  \begin{enumerate}
    \item
      Nein, $\GL_n(\Rbb)_-$ ist keine Untergruppe.
    \item
      Falsch, $\SOrthogonal(n)$ ist wegzusammenhängend.
    \item
      Falsch, die Wegzusammenhangskomponenten von $\GL_n(\Rbb)$ sind $\GL_n(\Rbb)_+$ und $\GL_n(\Rbb)_-$.
    \item
      Ja, denn normierte Vektorräume sind immer wegzusammenhängend.
    \item
      Nein: Der Schnitt ist $\SO(n)$, also wegzusammenhängend.
    \item
      Ja: Wähle einen Weg in $G$ und teile diesen durch die Determinante.
    \item
      Nein, es ist keine Untergruppe.
    \item
      Falsch, siehe $\{1,-1\}$.
    \item
      Ja, der Schnitt ist $\SOrthogona(n)$.
    \item
      Ja: Ändere den Winkel stetig.
  \end{enumerate}
\end{solution}


\begin{question}
  Es sei $V$ ein euklidischer $n$-dimensionaler Vektorraum und $d$ eine normierte, alternierende $n$-Form auf $d$.
  Es sei $u \in V$ mit $\|u\| = 1$.
  Zeigen Sie für das orthogonale Komplement $U \coloneqq u^\perp = \Ell(u)^\perp$, dass die Einschränkung $d(-, \dotsc, -, u)|_{U^{n-1}}$ eine normierte, alternierende $(n-1)$-Form ist.
\end{question}


\begin{question}
  Es sei $V$ ein euklidischer Vektorraum und $\omega \colon V^3 \to K$ eine Trilinearform.
  \begin{enumerate}[leftmargin=*]
    \item
      Zeigen Sie, dass es für alle $u, v \in V$ ein eindeutiges Element $\times_\omega \in V$ gibt, so dass
      \[
          \omega(u, v, w)
        = \bil{u \times_\omega v, w}
        \quad
        \text{für alle $u, v, w \in V$}.
      \]
    \item
      Zeigen Sie, dass die Abbildung $\times_\omega \colon V \times V \to V$ bilinear ist.
    \item
      Zeigen Sie:
      Ist $\omega$ symmetrisch, bzw.\ alternierend, so ist auch $\times_\omega$ symmetrisch, bzw.\ alternierend.
    \item
      Zeigen Sie, dass die Abbildung
      \[
        \Tril(V, V, V; K) \mapsto \Bil(V, V; V),
        \omega
        \mapsto
        \times_\omega
      \]
      ein Isomorphismus von $K$-Vektorräumen ist.
  \end{enumerate} 
\end{question}


\begin{question}
  Es sei $V$ ein $K$-Vektorraum und $\beta \colon V \times V \to K$ eine nicht-entartete Bilinearform.
  \begin{enumerate}
    \item
      Zeigen Sie, dass es für jede Bilinearform $\gamma \colon V \times V \to K$ eine eindeutige lineare Abbildung $f \colon V \to V$ gibt, so dass
      \[
        \gamma(x,y) = \beta(f(x), y)
        \quad
        \text{für alle $x,y \in V$}.
      \]
    \item
      Zeigen oder widerlegen Sie, dass es für jede nicht-entartete Bilinearform $\gamma \colon V \times V \to K$ einen eindeutigen Skalar $c \in K$ mit $\gamma = c \beta$ gibt.
  \end{enumerate}
\end{question}


\begin{question}
  Zeigen oder widerlegen Sie, dass der Kommutator
  \[
    [-, -] \colon \Mat_n(K) \times \Mat_n(K) \to \Mat_n(K)
  \]
  in dem folgenden Sinne universell ist:
  
  Für jeden $K$-Vektorraum $V$ und jede bilineare Abbildung \mbox{$\beta \colon \Mat_n(K) \times \Mat_n(K) \to V$} gibt es eine eindeutige $K$-lineare Abbildung $\varphi \colon \Mat_n(K) \to V$, so dass das folgende Diagram kommutiert:
  \[
    \begin{tikzcd}[row sep = large, column sep = large, ampersand replacement = \&]
            {}
        \&  \Mat_n(K) \times \Mat_n(K)  \arrow{rd}{\beta}
                                        \arrow[swap]{ld}{[-,-]}
        \&  {}
      \\
            \Mat_n(K)                   \arrow{rr}{\varphi}
        \&  {}
        \&  V
    \end{tikzcd}
  \]
\end{question}


\begin{question}
  Es seien $V$ und $W$ zwei $K$-Vektorräume und $\beta \colon V \times V \to W$ eine bilineare Abbildung.
  Zeigen oder widerlegen Sie, dass $\im \beta$ notwendigerweise ein Untervektorraum von $W$ ist.
\end{question}


\begin{question}
  Es sei $m \colon K^n \times K^n \to \Mat_n(K)$ mit
  \[
    m(x,y) = x y^T
    \quad
    \text{für alle $x,y \in K^n$}.
  \]
  \begin{enumerate}[leftmargin=*]
    \item
      Zeigen Sie, dass $m$ bilinear ist.
    \item
      Zeigen Sie, dass $m$ im folgenden Sinne universell ist:
      Für jeden Vektorraum $V$ und jede bilineare Abbildung $b \colon K^n \times K^n \to V$ gibt es eine eindeutige lineare Abbildung $\varphi \colon \Mat_n(K) \to V$ mit $b = \varphi \circ m$, dass also das folgende Diagram kommutiert:
      \[
        \begin{tikzcd}[row sep = large, column sep = large, ampersand replacement = \&]
              {}
          \&  K^n \times K^n  \arrow{rd}{m}
                              \arrow[swap]{ld}{b}
          \&  {}
          \\
              \Mat_n(K)       \arrow{rr}{\varphi}
          \&  {}
          \&  V
        \end{tikzcd}
      \]
  \end{enumerate}
\end{question}






















\end{document}
