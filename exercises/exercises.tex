%\documentclass[a4paper,10pt]{article}
\documentclass[a4paper,10pt]{scrartcl}

\usepackage{../generalstyle}
\usepackage{specificstyle}

\setromanfont[Mapping=tex-text]{Linux Libertine O}
% \setsansfont[Mapping=tex-text]{DejaVu Sans}
% \setmonofont[Mapping=tex-text]{DejaVu Sans Mono}

\title{Übungen zu Lineare Algebra II}
\author{Jendrik Stelzner}
\date{\today}

\begin{document}
\maketitle



% GENERAL STUFF


\begin{question}
  Ein Endomorphismus $f \colon V \to V$ eines $K$-Vektorraums $V$ heißt \emph{lokal nilpotent}, falls es für jedes $v \in V$ ein $n \in \Nbb$ mit $f^n(v) = 0$ gibt.
  \begin{enumerate}[leftmargin=*]
    \item
      Zeigen Sie, dass jeder nilpotente Endomorphismus auch lokal nilpotent ist.
    \item
      Zeige Sie, dass $0$ der einzige mögliche Eigenwert eines lokal nilpotenten Endomorphismus ist.
    \item
      Geben Sie ein Beispiel für einen Vektorraum $V$ und einen Endomorphismus $f \colon V \to V$ an, so dass $f$ zwar lokal nilpotent, nicht aber nilpotent ist.
    \item
      Zeigen Sie, dass jeder lokal nilpotente Endomorphismus eines endlichdimensionalen Vektorraums bereits nilpotent ist.
  \end{enumerate}
\end{question}


\begin{question}
  Es sei $V$ ein $K$-Vektorraum und $f \colon V \to V$ ein Endomorphismus.
  Zeigen Sie:
  \begin{enumerate}[leftmargin=*]
    \item
      Ist $f^2 = f$, so ist $V = \im f \oplus \ker f$, und es gilt $\im f = V_1(f)$ und $\ker f = V_0(f)$.
    \item
      Ist $f^2 = \id_V$, so ist $f$ diagonalisierbar mit (möglichen) Eigenwerten $1$ und $-1$.
    \item
      Sind $\lambda, \mu \in K$ mit $\lambda \neq \mu$ und $(f-\lambda)(f-\mu) = 0$, so ist $f$ diagonalisierbar mit (möglichen) Eigenwerten $\lambda$ und $\mu$.
      Inwiefern sind die vorherigen beiden Aufgabenteile Sonderfälle hiervon?
  \end{enumerate}
\end{question}


\begin{question}
  Es sei $V$ ein $K$-Vektorraum.
  Zeigen Sie, dass die folgenden Aussagen allgemein gelten, oder geben Sie jeweils ein Gegenbeispiel an.
  \begin{enumerate}[leftmargin=*]
    \item
      Ist $V = V_1 \oplus V_2$ für Untervektorräume $V_1, V_2 \subseteq V$, so gilt für jeden Untervektorraum $U \subseteq V$ die Zerlegung
      \[
        U = (U \cap V_1) \oplus (U \cap V_2).
      \]
    \item
      Ist $V = U_1 \oplus W_1 = U_2 \oplus W_2$ mit $W_1 \supseteq W_2$, so ist
      \[
        W_1 = (U_2 \cap W_1) \oplus W_2.
      \]
    \item
      Ist $f \colon V \to V$ ein Endomorphismus und $U \subseteq V$ ein $f$-invarinter Untervektorraum, so gibt es einen $f$-invarianten Untervektorraum $W \subseteq V$ mit $V = U \oplus W$.
    \item
      Für alle Untervektorräume $W, U_1, U_2 \subseteq V$ mit $U_1 \subseteq U_2$ gilt
      \[
        (U_1 + W) \cap U_2 =  U_1 + (W \cap U_2).
      \]
    \item
      Ist $\mc{E} \subseteq V$ ein Erzeugendensystem und $U \subseteq V$ ein Untervektorraum, so ist die Einschränkung $\mc{E}' \coloneqq \mc{E} \cap U$ ein Erzeugendensystem von $U$.
    \item
      Ist $(U_i)_{i \in I}$ eine Famlie von Untervektorräumen $U_i \subseteq V$ mit $V = \sum_{i \in I} U_i$ und $U_i \cap U_j = 0$ für $i \neq j$, so ist $V = \bigoplus_{i \in I} U_i$.
  \end{enumerate}
\end{question}


\begin{question}
  Ein Endomorphismus $f \colon V \to V$ eines $K$-Vektorraums $V$ heißt \emph{algebraisch (über $K$)}, falls es ein Polynom $P \in K[T]$ mit $P \neq 0$ gibt, so dass $P(f) = 0$ gilt.
  \begin{enumerate}[leftmargin=*]
    \item
      Zeigen Sie, dass jeder Endomorphismus eines endlichdimensionalen Vektorraums algebraisch ist.
    \item
      Geben Sie ein Beispiel für einen $K$-Vektorraum $V$ und einen Endomorphismus $f \colon V \to V$ an, der nicht algebraisch ist.
  \end{enumerate}
\end{question}












% QUOTIENTS


\begin{question}
  Es sei $V$ ein $K$-Vektorraum und $U \subseteq V$ ein $K$-Untervektorraum.
  Konstruieren Sie für den Annihilator
  \[
      U^\circ
    = \{ \varphi \in V^* \mid \varphi|_U = 0 \}
  \]
  einen Isomorphismus $F \colon U^\circ \to (V/U)^*$.
\end{question}



\begin{question}
  Es sei $V$ ein Vektorraum und $f \colon V \to V$ ein Endomorphismus.
  Es sei $(U_i)_{i \in I}$ eine Familie von $f$-invarianten Untervektorräumen, und $U \subseteq V$ ein $f$-invarianter Untervektorraum.
  Zeigen Sie:
  \begin{enumerate}[leftmargin=*]
    \item
      Auch der Schnitt $\bigcap_{i \in I} U_i$ ist $f$-invariant.
    \item
      Auch die Summe $\sum_{i \in i} U_i$ ist $f$-invariant.
    \item
      $f$ induziert eine lineare Abbildung
      \[
        \bar{f} \colon V/U \to V/U,
        \quad
        [x] \mapsto [f(x)].
      \]
  \end{enumerate}
\end{question}


\begin{question}
  Es sei $K$ ein algebraisch abgeschlossener Körper und $f \colon V \to V$ ein Endomorphismus eines endlichdimensionalen $K$-Vektorraums $V$.
  Zeigen Sie, dass die folgenden beiden Aussagen äquivalent sind:
  \begin{enumerate}
    \item
      $f$ ist diagonalisierbar.
    \item
      Für jeden $f$-invarianten Untervektorraum $U \subseteq V$ gibt es einen $f$-invarianten Untervektorraum $W \subseteq V$ mit $V = U \oplus W$.
  \end{enumerate}
\end{question}


\begin{question}
  Es sei $V$ ein $K$-Vektorraum und $U \subseteq V$ ein Untervektorraum.
  Es sei $\pi \colon V \to V/U$, $v \mapsto [v]$ die kanonische Projektion.
  \begin{enumerate}[leftmargin=*]
    \item
      Es sei $(b_i)_{i \in I}$ eine Basis von $V$, und für eine Teilmenge $J \subseteq I$ sei $(b_j)_{j \in J}$ eine Basis von $U$.
      Zeigen Sie, dass $([b_i])_{i \in I \smallsetminus J}$ eine Basis von $V/U$ ist.
    \item
      Es sei $(b_i)_{i \in I}$ eine Basis von $U$ und $(c_j)_{j \in J}$ eine Basis von $V/U$, wobei $I \cap J = \emptyset$.
      Für $j \in J$ sei $b_j \in V$ mit $\pi(b_j) = c_j$.
      Zeigen Sie, dass $(b_l)_{l \in L}$ für $L \coloneqq I \cap J$ ist eine Basis von $V$ ist.
  \end{enumerate}
\end{question}










% QUOTIENTS

\begin{question}
  Es seien $V$ und $W$ zwei $K$-Vektorräume und $f \colon V \to W$ eine lineare Abbildung.
  \begin{enumerate}
    \item
      Es sei $U \subseteq V$ ein Untervektorraum mit $f|_U = 0$.
      Zeigen Sie, dass $V$ eine lineare Abbildung
      \[
        \bar{f} \colon V/U \to W, \quad [v] \mapsto f(u)
      \]
      induziert.
    \item
      Zeigen Sie, dass $\im \bar{f} = \im f$.
      Folgern Sie, dass $\bar{f}$ genau dann surjektiv ist, wenn $f$ surjektiv ist.
    \item
      Zeigen Sie, dass $U \subseteq \ker f$, und dass $\ker \bar{f} = (\ker f)/U$.
      Folgern Sie, dass $\bar{f}$ genau dann injektiv ist, wenn bereits $U = \ker f$ gilt.
    \item
      Folgern Sie, dass $f$ einen Isomorphismus
      \[
        V/(\ker f) \to \im f,
        \quad
        [v] \mapsto f(v)
      \]
      induziert.
  \end{enumerate}
\end{question}


\begin{question}
  Es sei $V$ ein $K$-Vektorraum mit Erzeugendensystem $E \subseteq V$.
  Es sei $W$ ein $K$-Vektorraum mit Basis $\{b_e\}_{e \in E} \subseteq W$.
  Konstruieren Sie einen Isomorphismus $V/U \to W$ für einen passenden Untervektorraum $U \subseteq W$.
\end{question}


\begin{question}
  Es sei $V$ ein $K$-Vektorraum und $U \subseteq V$ ein Untervektorraum.
  Es seien $f \colon V \to V$ ein Endomorphismus, so dass $U$ invariant unter $f$ ist (d.h.\ es ist $f(U) \subseteq U$).
  \begin{enumerate}
    \item
      Zeigen Sie, dass $f$ einen Endomorphismus
      \[
        \bar{f} \colon V/U \to V/U,
        \quad
        [v] \mapsto [f(v)]
      \]
    induziert.
  \end{enumerate}
  Es sei nun $g \colon V \to V$ ein weiterer Endomorphismus, so dass $U$ invariant unter $g$ ist, und es sei $\bar{g} \colon V/U \to V/U$ der induzierte Endomorphismus.
  \begin{enumerate}[resume]
    \item
      Es seien $f|_U = g|_U$ und $\bar{f} = \bar{g}$.
      Beweisen oder widerlegen Sie, dass bereits $f = g$ gelten muss.
  \end{enumerate}
\end{question}













% COMPLEXIFICATION


\begin{question}
  Zeigen Sie, dass die Inklusion $\Rbb \to \Cbb$ einen Isomorphismus $\Rbb_\Cbb \to \Cbb$ von $\Cbb$-Vektorräumen induziert.
\end{question}


\begin{question}
  Es sei $V$ ein $\Rbb$-Vektorraum.
  Konstruieren Sie einen Isomorphismus $(V^*)_\Cbb \to (V_\Cbb)^*$.
  (\emph{Hinweis}: Beachten Sie, dass $V$ ist nicht notwendigerweise endlichdimensional ist.)
\end{question}


\begin{question}
  Es seien $V$ und $W$ zwei $\Rbb$-Vektorräume.
  Konstruieren Sie einen Isomorphismus
  \[
    \Hom_\Cbb(V_\Cbb, W_\Cbb) \to \Hom_\Rbb(V, W)_\Cbb
  \]
  von $\Cbb$-Vektorräumen.
\end{question}


\begin{question}
  Es sei $V$ ein reeller Vektorraum und $(U_i)_{i \in I}$ eine Familie von Untervektorräumen $U_i \subseteq V$.
  Zeigen Sie:
  \begin{enumerate}[leftmargin=*]
    \item
      Es gilt
      \[
          \left( \bigcap_{i \in I} U_i \right)_\Cbb
        = \bigcap_{i \in I} (U_i)_\Cbb
      \]
    \item
      Es gilt
      \[
          \left( \sum_{i \in I} U_i \right)_\Cbb
        = \sum_{i \in I} (U_i)_\Cbb.
      \]
    \item
      Folgern Sie, dass genau dann $V = \bigoplus_{i \in I} U_i$, wenn $V_\Cbb = \bigoplus_{i \in I} (U_i)_{\Cbb}$.
  \end{enumerate}
\end{question}


\begin{question}
  Es seien $V$ und $W$ zwei reelle Vektorräume, und $f \colon V \to W$ sei $\Rbb$-linear.
  \begin{enumerate}[leftmargin=*]
    \item
      Zeigen Sie, dass $\ker (f_\Cbb) = (\ker f)_\Cbb$.
    \item
      Folgern Sie, dass $f_\Cbb$ genau dann injektiv ist, wenn $f$ injektiv ist.
    \item
      Folgern Sie ferner, dass $(V_\Cbb)_\lambda(f_\Cbb) = V_\lambda(f)_\Cbb$ für jedes $\lambda \in \Rbb$.
    \item
      Zeigen Sie, dass $\im (f_\Cbb) = (\im f)_\Cbb$.
    \item
      Folgern Sie, dass $f_\Cbb$ genau dann surjektiv ist, wenn $f$ surjektiv ist.
  \end{enumerate}
\end{question}


\begin{question}
  Es sei $V$ ein reeller Vektorraum und $f \colon V \to V$ ein Endomorphismus.
  Zeigen Sie, dass $f$ genau dann diagonalisierbar ist, wenn $f_\Cbb$ diagonalisierbar mit reellen Eigenwerten ist.
\end{question}


\begin{question}
  Es sei
  \[
    \iota \colon \Rbb[X] \to \Cbb[X],
    \quad
    \sum_{k=0}^n a_k X^k \mapsto \sum_{k=0}^n a_k X^k
  \]
  die Teilmengeninklusion.
  \begin{enumerate}[leftmargin=*]
    \item
      Zeigen Sie, dass $\iota$ $\Rbb$-linear ist.
    \item
      Zeigen Sie, dass $\iota$ einen Isomorphismus $\Rbb[X]_\Cbb \to \Cbb[X]$ von $\Cbb$-Vektorräumen induziert.
  \end{enumerate}
\end{question}











% EIGENSTUFF


\begin{question}
  Es sei $V$ ein $K$-Vektorraum, wobei $V \neq 0$ und $K$ algebraisch abgeschlossen ist.
  Es seien $f_1, \dotsc, f_n \colon V \to V$ paarweise kommutierende Endomorphismen.
  Zeigen Sie, dass die Endomorphismen $f_1, \dotsc, f_n$ einen gemeinsamen Eigenvektor besitzen, d.h.\ dass es ein $v \in V$ gibt, das für jedes $f_i$ eine Eigenvektor ist.
\end{question}


\begin{question}
  Es sei $A \in \Mat_2(\Rbb)$ mit $\tr A = 0$ und $\tr A^2 = -2$.
  Bestimmen Sie $\det A$.
\end{question}


\begin{question}
  Zeigen Sie, dass es für $A \in \GL_n(K)$ ein Polynom $P \in K[T]$ mit $\deg P \leq n-1$ gibt, so dass $A^{-1} = P(A)$.
\end{question}


\begin{question}
  Es sei $K$ ein algebraisch abgeschlossener Körper mit $\ringchar K \notin \{2,3\}$.
  Zeigen Sie, dass
  \[
    \det A = \frac{1}{6} (\tr A)^3 - \frac{1}{2} (\tr A^2)(\tr A) + \frac{1}{3} (\tr A^3)
    \quad
    \text{für jedes $A \in \Mat_3(K)$}.
  \]
  (\emph{Hinweis}: Wenn die Rechnungen zu kompliziert werden, dann macht man es falsch.)
\end{question}



\begin{question}
  Für alle $\lambda_1, \dotsc, \lambda_n \in \Cbb$ sei
  \[
    \diag(\lambda_1, \dotsc, \lambda_n)
    \coloneqq
    \begin{pmatrix}
      \lambda_1 &         &           \\
                & \ddots  &           \\
                &         & \lambda_n
    \end{pmatrix}
    \in \Mat_n(\Cbb).
  \]
  Es sei
  \[
    \Diagonal_n(\Cbb)
    \coloneqq
    \left\{
      S \diag(\lambda_1, \dotsc, \lambda_n) S^{-1}
    \,\middle|\,
      S \in \GL_n(\Cbb),
      \lambda_1, \dotsc, \lambda \in \Cbb
    \right\}
    \subseteq \Mat_n(\Cbb)
  \]
  die Menge der diagonalisierbaren komplexen $n \times n$-Matrizen.
  Wir zeigen, dass $D_n(\Cbb) \subseteq \Mat_n(\Cbb)$ dicht ist, d.h.\ dass es für jede Matrix $A \in \Mat_n(\Cbb)$ und jedes $\varepsilon > 0$ eine diagonalisierbare Matrix $D \in \Diagonal_n(\Cbb)$ mit $\|A-D\| < \varepsilon$ gibt.
  
  Es sei $S \in \GL_n(\Cbb)$, so dass $S A S^{-1}$ eine obere Dreiecksmatrix mit Diagonaleinträgen $\lambda_1, \dotsc, \lambda_n$ ist, also
  \[
    S A S^{-1}
    =
    \begin{pmatrix}
      \lambda_1 & *       & \cdots  & *         \\
                & \ddots  & \ddots  & \vdots    \\
                &         & \ddots  & *         \\
                &         &         & \lambda_n
    \end{pmatrix}.
  \]
  Es seien $z_1, \dotsc, z_n \in \Cbb$ paarweise verschieden und
  \[
    B(t)
    \coloneqq
    A + t S \diag(z_1, \dotsc, z_n) S^{-1}
    \quad
    \text{für alle $t \in \Rbb$}.
  \]
  \begin{enumerate}
    \item
      Zeigen Sie, dass $\mu_1(t), \dotsc, \mu_n(t) \in \Cbb$ mit
      \[
        \mu_i(t) \coloneqq \lambda_i + t z_i
        \quad
        \text{für $i = 1, \dotsc, n$}
      \]
      die Eigenwerte von $B(t)$ ist.
    \item
      Zeigen Sie, dass die Zahlen $\mu_1(t), \dotsc, \mu_n(t)$ für fast alle $t \in \Rbb$ paarweise verschieden sind.
    \item
      Folgern Sie, dass $B(t)$ für fast alle $t \in \Rbb$ diagonalisierbar ist.
    \item
      Folgern Sie, dass es ein $\delta > 0$ gibt, so dass alle $D \in B_\delta(A) \setminus \{A\}$ diagonalisierbar sind.
    \item
      Folgern Sie, dass es für alle $\varepsilon > 0$ ein $D \in \Diagonal_n(\Cbb)$ mit $\|A-D\| < \varepsilon$ gibt.
    \item
      Folgern Sie außerdem, dass $\Diagonal_n(\Cbb) \subseteq \Mat_n(\Cbb)$ eine offene Teilmenge ist.
  \end{enumerate}
  Wir wollen die Dichtheit von $\Diagonal_n(\Cbb) \subseteq \Mat_n(\Cbb)$ nutzen, um den Satz von Cayley-Hamilton zu zeigen:
  \begin{enumerate}[resume]
    \item
      Zeigen Sie, dass die Abbildung
      \[
        F \colon \Mat_n(\Cbb) \to \Mat_n(\Cbb),
        \quad
        A \mapsto \chi_A(A)
      \]
      stetig ist, wobei $\chi_A(T) \in \Cbb[T]$ das charakteristische Polynom von $A$ ist.
    \item
      Zeigen Sie, dass $F(D) = 0$ für jede Diagonalmatrix $D \in \Mat_n(\Cbb)$.
    \item
      Zeigen Sie, dass $P(SAS^{-1}) = S P(A) S^{-1}$ für alle $P \in \Cbb[T]$, $A \in \Mat_n(\Cbb)$ und $S \in \GL_n(\Cbb)$.
      Folgern Sie, dass $F(D) = 0$ für jede Matrix $D \in \Diagonal_n(\Cbb)$.
    \item
      Folgern Sie, dass $F(A) = 0$ für alle $A \in \Mat_n(\Cbb)$.
  \end{enumerate}
\end{question}



















% SCALAR PRODUCTS


\begin{question}
  Es seien $V$ und $W$ zwei endlichdimensionale euklidische Vektorräume.
  Ferner sei $f \colon V \to W$ eine $\Rbb$-lineare Abbildung.
  \begin{enumerate}[leftmargin=*]
    \item
      Zeigen Sie, dass die Abbildung
      \[
        \Phi_V \colon V \to V^*,
        \quad
        v \mapsto \bil{-, v}
      \]
      ein $\Rbb$-linearer Isomorphismus ist.
    \item
      Geben Sie die Definition der dualen Abbildung $f^* \colon W^* \to V^*$ an.
      Zeigen Sie, dass $f^*$ $\Rbb$-linear ist.
    \item
      Zeigen Sie, dass die Abbildung $g \coloneqq \Phi_V^{-1} \circ f^* \circ \Phi_W$ $\Rbb$-linear ist, und dass
      \[
        \bil{f(v), w} = \bil{v, g(w)}
        \quad
        \text{für alle $v \in V$, $w \in W$}.
      \]
    \item
      Inwiefern ändern sich die obigen Resultate für denn fall $\Kbb = \Cbb$, wenn also $V$ und $W$ endlichdimensionale unitäre Vektorräume sind?
  \end{enumerate}
\end{question}


\begin{question}
  Es sei $V \coloneqq \mathcal{C}([0,1], \Rbb)$ der Raum der stetigen Funktionen $[0,1] \to \Rbb$.
  Ferner sei $U \coloneqq \{f \in V \mid f(0) = 0\}$.
  \begin{enumerate}[leftmargin=*]
    \item
      Zeigen Sie, dass $U$ ein Untervektorraum von $V$ ist.
    \item
      Zeigen Sie, dass
      \[
        \bil{f, g} \coloneqq \int_0^1 f(t) g(t) \dd{t}
        \quad
        \text{für alle $f, g \in V$}
      \]
      ein Skalarprodukt auf $V$ definiert.
    \item
      Zeigen Sie, dass $U^\perp = 0$.
      Folgern Sie, dass $V \neq U \oplus U^\perp$.
      (\emph{Hinweis}: Betrachten Sie für $g \in U^\perp$ die Funktion $h \colon [0,1] \to \Rbb$ mit $h(t) = t^2 g(t)$.)
    \item
      Zeigen Sie ferner, dass $V/(U \oplus U^\perp)$ eindimensional ist.
  \end{enumerate}
\end{question}


\begin{question}
  Es sei
  \[
    W = \{(a_n)_{n \in \Zbb} \mid \text{$a_n \in \Rbb$ für alle $n \in \Zbb$}\}
  \]
  der Vektorraum der beidseitigen reellwertigen Folgen.
  Wir betrachten den Untervektorraum
  \[
    V \coloneqq
    \left\{
      (a_n)_{n \in \Zbb} \in W
    \,\middle|\,
      \sum_{n \in \Zbb} |a_n|^2 < \infty
   \right\}
  \]
  der quadratsummierbaren Folgen.
  \begin{enumerate}[leftmargin=*]
    \item
      Zeigen Sie, dass $V$ ein Untervektorraum von $W$ ist.
    \item
      Zeigen Sie für alle $(a_n)_{n \in \Zbb}, (b_n)_{n \in \Zbb} \in V$, dass
      \[
        \sum_{n \in \Zbb} a_n b_n < \infty.
      \]
      (\emph{Hinweis:} Zeigen sie zunächst, dass $ab \leq (a^2 + b^2)/2$ für alle $a, b \in \Rbb$.)
    \item
      Zeigen sie, dass
      \[
                  \bil{ (a_n)_{n \in \Zbb}, (b_n)_{n \in \Zbb} }
        \coloneqq \sum_{n \in \Zbb} a_n b_n
        \quad
        \text{für alle $(a_n)_{n \in \Zbb}, (b_n)_{n \in \Zbb} \in V$}
      \]
      ein Skalarprodukt auf $V$ definiert.
    \item
      Es sei
      \[
        R \colon V \to V,
        \quad
        (a_n)_{n \in \Zbb} \mapsto (a_{n-1})_{n \in \Zbb}
      \]
      der Rechtsshift-Operator.
      Zeigen Sie, dass $R$ ein Adjungiertes besitzt, und entscheiden Sie, ob $R$ selbstadjungiert, orthogonal, bzw.\ normal ist.
    \item
      Zeigen Sie, dass $R$ keine Eigenwerte besitzt.
    \item
      Es sei
      \[
        S \colon V \to V,
        \quad
        (a_n)_{n \in \Nbb} \mapsto (a_{-n})_{n \in \Nbb}.
      \]
      Zeigen Sie, dass $S$ ein Adjungiertes besitzt, und entscheiden Sie, ob $R$ selbstadjungiert, orthogonal, bzw.\ normal ist.
    \item
      Zeigen Sie, dass $S$ diagonalisierbar ist.
    \item
      Es sei
      \[
        U \coloneqq \{(a_n)_{n \in \Zbb} \in V \mid \text{$a_n = 0$ für fast alle $n \in \Zbb$}\}.
      \]
      Bestimmen Sie $U^\perp$ und entscheiden Sie, ob $V = U \oplus U^\perp$.
    \item
      Bestimmen Sie eine Orthonormalbasis von $U$.
  \end{enumerate}
\end{question}


\begin{question}
  \begin{enumerate}[leftmargin=*]
    \item
      Zeigen Sie, dass durch
      \[
        \sigma(A, B) \coloneqq \tr\left( A^T B \right)
        \quad
        \text{für alle $A, B \in \Mat_n(\Rbb)$}
      \]
      ein Skalarprodukt auf $\Mat_n(\Rbb)$ definiert wird.
    \item
      Zeigen Sie, dass die Standardbasis $(E_{ij})_{i,j=1,\dotsc,n}$ von $\Mat_n(\Rbb)$ mit
      \[
        (E_{ij})_{kl} \coloneqq \delta_{ik} \delta_{jl}
        \quad
        \text{für alle $1 \leq i,j,k,l \leq n$}
      \]
      eine Orthonormalbasis von $\Mat_n(\Rbb)$ bezüglich $\sigma$ bilden.
    \item
      Es sei
      \[
        S_+ \coloneqq \{A \in \Mat_n(\Rbb) \mid A^T = A\}
      \]
      der Untervektorraum der symmetrischen Matrizen, und
      \[
        S_- \coloneqq \{A \in \Mat_n(\Rbb) \mid A^T  = -A\}
      \]
      der Untervektorraum der schiefsymmetrischen Matrizen.
      Zeigen Sie, dass
      \[
        \Mat_n(\Rbb) = S_+ \oplus S_-,
      \]
      und dass die Summe orthogonal ist.
  \end{enumerate}
\end{question}


\begin{question}
  Es sei $V$ ein Skalarproduktraum und
  \[
    O(V) \coloneqq \{ f \in \End(V) \mid f f^* = \id \}.
  \]
  Zeigen Sie, dass $O(V)$ eine Untergruppe von $\GL(V)$ bildet.
\end{question}


\begin{question}
Zeigen sie, dass für eine Matrix $A \in \Mat_n(\Kbb)$ die folgenden Bedingungen äquivalent sind:
  \begin{enumerate}
    \item
      $A$ ist invertierbar mit $A^{-1} = A^*$.
    \item
      $A A^* = I$.
    \item
      $A^* A = I$.
    \item
      Die Spalten von $A$ bilden eine Orthonormalbasis des $\Kbb^n$.
    \item
      Die Zeilen von $A$ bilden eine Orthonormalbasis des $\Kbb^n$.
  \end{enumerate}
\end{question}


\begin{question}
  Es sei $A \in \Mat_n(\Cbb)$.
  \begin{enumerate}[leftmargin=*]
    \item
      Zeigen Sie, dass es eindeutige hermitsche Matrizen $B, C \in \Mat_n(\Cbb)$ mit $A = B + i C$ gibt.
    \item
      Zeigen Sie, dass $A$ genau dann normal ist, wenn $B$ und $C$ kommutieren.
  \end{enumerate}
\end{question}











% BILINEAR FORMS


\begin{question}
  Für je zwei $K$-Vektorräume $V$ und $W$ sei
  \[
              \Bil(V, W)
    \coloneqq \{b \colon V \times  W \to K \mid \text{$b$ ist bilinear}\}
  \]
  der Raum der Bilinearformen $V \times W \to K$.
  \begin{enumerate}[leftmargin=*]
    \item
      Zeigen Sie, dass die Flipabbildung
      \[
        F \colon \Bil(V, W) \to \Bil(W, V),
        \quad
        b \mapsto F(b)
        \quad\text{mit}\quad
        F(b)(w,v) = b(v,w)
      \]
      ein Isomorphismus von $K$-Vektorräumen ist.
    \item
      Es sei $b \in \Bil(V, W)$ eine Bilinearform.
      Zeigen Sie, dass $b$ ein lineare Abbildung
      \[
        \Phi_{V,W}(b) \colon V \to W^*,
        \quad
        v \mapsto b(v, -)
      \]
      induziert.
      Dabei ist
      \[
        b(v, -) \colon W \to K,
        \quad
        w \mapsto b(v,w).
      \]
    \item
      Zeigen Sie, dass die Abbildung
      \[
        \Phi_{V,W} \colon \Bil(V, W) \to \Hom(V, W^*),
        \quad
        b \mapsto \Phi_{V,W}(b)
      \]
      ein Isomorphismus von $K$-Vektorräumen ist.
    \item
      Geben Sie mithilfe der vorherigen Aufgabenteile explizit einen Isomorphismus
      \[
        \Hom(V, W^*) \to \Hom(W, V^*)
      \]
      an.
  \end{enumerate}
  Wir betrachten nun den Fall $W = V^*$.
  \begin{enumerate}[resume, leftmargin=*]
    \item
      Zeigen Sie, dass die Evaluation
      \[
        e \colon V \times V^* \to K,
        \quad
        (v, \varphi) \mapsto \varphi(v)
      \]
      eine Bilinearform ist.
   \item
      Nach den vorherigen Aufgabenteilen entspricht die Bilinearform $e$ einer linearen Abbildung $V \to V^{**}$, sowie einer linearen Abbildung $V^* \to V^*$.
      Bestimmen Sie diese Abbildungen.
    \item
      Woher kennen Sie diese Abbildung?
  \end{enumerate}
\end{question}


\begin{question}
  Es seien $V$ und $W$ zwei $K$-Vektorräume und $f \colon V \to W$ eine lineare Abbildung.
  \begin{enumerate}[leftmargin=*]
    \item
      Geben Sie die Definition der dualen Abbildung $f^* \colon W^* \to V^*$ an, und zeigen sie ihre Linearität.
    \item
      Zeigen Sie für jeden $K$-Vektorraum $U$, dass die Abbildung
      \[
        \bil{\cdot, \cdot} \colon U \times U^* \to K
        \quad\text{mit}\quad
        \bil{v, \varphi} = \varphi(v)
        \quad\text{für alle $v \in V$, $\varphi \in V^*$}
      \]
      eine Bilinearform ist.
    \item
      Zeigen Sie, dass
      \[
        \bil{f(v), \psi} = \bil{v, f^*(\psi)}
        \quad
        \text{für alle $v \in V$, $\psi \in W^*$}.
      \]
  \end{enumerate}
\end{question}


\begin{question}
  \begin{enumerate}[leftmargin=*]
    \item
      Zeigen Sie, dass die Abbildung
      \[
        \sigma \colon \Mat_n(K) \times \Mat_n(K) \to K
        \quad\text{mit}\quad
        \sigma(A, B) \coloneqq \tr(AB)
      \]
      eine symmetrische Bilinearform ist.
    \item
      Zeigen Sie, dass $\sigma$ in dem Sinne assoziativ ist, dass
      \[
        \sigma(AB, C) = \sigma(A, BC)
        \quad
        \text{für alle $A, B, C \in \Mat_n(K)$}.
      \]
  \end{enumerate}
\end{question}










% Lie Stuff


\begin{question}
  Es sei $V$ ein $K$-Vektorraum und $m \colon V \times V \to V$ eine bilineare Abbildung.
  Eine lineare Abbildung $D \colon V \to V$ heißt \emph{$m$-Derivation}, falls
  \[
    D(m(x,y))
    = m(D(x), y) + m(x, D(y))
    \quad
    \text{für alle $x, y \in V$}.
  \]
  Es sei
  \[
              \Der(m)
    \coloneqq \{ D \colon V \to V \mid \text{$D$ ist eine $m$-Derivation} \}.
  \]
  \begin{enumerate}[leftmargin=*]
    \item
      Zeigen Sie für den Fall $V = K[X]$ und die Multiplikation
      \[
        m(p,q) \coloneqq p \cdot q
        \quad
        \text{für alle $p, q \in K[X]$},
      \]
      dass die Ableitung
      \[
        D \colon K[X] \to K[X],
        \quad
        \sum_{d=0}^n a_d X^d  \mapsto \sum_{d=1}^n a_d d X^{d-1} 
      \]
      eine $m$-Derivation ist.
      Unter welchem Namen ist dieser Umstand für gewöhnlich bekannt?
    \item
      Zeigen Sie, dass $\Der(m)$ ein Untervektorraum von $\End(V)$ ist.
    \item
      Zeigen Sie, dass $\Der(m)$ eine Lie-Unteralgebra von $\End(V)$ ist, d.h.\ dass für alle $D_1, D_2 \in \Der(m)$ auch $[D_1, D_2] \in \Der(m)$.
  \end{enumerate}
\end{question}


\begin{question}
  Es sei $V$ ein $K$-Vektorrraum und $[-,-] \colon V \times V \to V$ eine alternierend bilineare Abbildung.
  Für jedes $x \in V$ sei
  \[
    \ad_x \coloneqq [x,-] \colon V \to V, \quad y \mapsto [x,y].
  \]
  Zeigen Sie, dass die folgenden beiden Aussagen äquivalent sind:
  \begin{enumerate}
    \item
      $[-,-]$ erfüllt die Jacobi-Identität, d.h.\ es ist
      \[
        [[x,y],z] + [[y,z],x] + [[z,y],x] = 0
        \quad
        \text{für alle $x, y, z \in V$}.
      \]
    \item
      Es gilt
      \[
          \ad_x([y,z])
        = [\ad_x(y), z] + [x, \ad_y(z)]
        \quad
        \text{für alle $x, y, z \in V$}.
      \]
      (Für jedes $x \in V$ ist also $\ad_x$ eine Derivation bezüglich $[-,-]$.)
  \end{enumerate}
\end{question}



\begin{question}
  Es seien $E$ und $H$ zwei Endomorphismen eines $\Cbb$-Vektorraums $V$, so dass $[H,E] = 2E$.
  \begin{enumerate}
    \item
      Zeigen Sie, dass $E(V_\lambda(H)) \subseteq V_{\lambda + 2}(H)$ für alle $\lambda \in K$.
    \item
      Folgern Sie: Ist $V$ endlichdimensional und $H$ diagonalisierbar, so ist $E$ nilpotent.
  \end{enumerate}
\end{question}


\begin{question}
  Für einen endlichdimensionalen $\Kbb$-Vektorraum $V$ und eine Bilinearform $\beta \colon V \times V \to \Kbb$ sei
  \[
              O(\beta)
    \coloneqq \{ \phi \in \GL(V) \mid \text{$\beta(\phi(x), \phi(y)) = \beta(x,y)$ für alle $x,y \in V$} \}
  \]
  die Isometriegruppe von $\beta$, und
  \[
              \gLie(\beta)
    \coloneqq \{ f \in \End(V) \mid \text{$\beta(f(x),y) + \beta(x, f(y)) = 0$ für alle $x, y \in V$} \}
  \]
  die assoziierte Lie-Algebra.
  \begin{enumerate}[leftmargin=*]
    \item
      Zeigen Sie, dass $O(\beta)$ eine Untergruppen von $\GL(V)$ ist.
    \item
      Zeigen Sie, dass $\gLie(\beta)$ eine Lie-Unteralgebra von $\gl(V)$ ist, d.h.\ dass $[f,g] \in \gLie(\beta)$ für alle $f, g \in \gLie(\beta)$.
    \item
      Zeigen Sie, dass $\exp(f) \in O(\beta)$ für alle $f \in \gLie(\beta)$.
    \item
      Es sei $\Kbb = \Rbb$ und $\bil{\cdot, \cdot}$ ein Skalarprodukt auf $V$.
      Unter welchen Begriffen sind die Elemente aus $G(\bil{\cdot, \cdot})$ und $\gLie(\bil{\cdot, \cdot})$ bekannt?
  \end{enumerate}
\end{question}







\end{document}
