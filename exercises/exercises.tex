\documentclass[a4paper,10pt]{article}
%\documentclass[a4paper,10pt]{scrartcl}

\usepackage{../generalstyle}
\usepackage{specificstyle}

\setromanfont[Mapping=tex-text]{Linux Libertine O}
% \setsansfont[Mapping=tex-text]{DejaVu Sans}
% \setmonofont[Mapping=tex-text]{DejaVu Sans Mono}

\title{Übungen zu Lineare Algebra II}
\author{Jendrik Stelzner}
\date{\today}

\begin{document}
\maketitle


% GENERAL STUFF

\begin{question}
  Es sei $V$ ein $K$-Vektorraum und $f \colon V \to V$ ein Endomorphismus.
  Zeigen Sie:
  \begin{enumerate}[leftmargin=*]
    \item
      Ist $f^2 = f$, so ist $V = \im f \oplus \ker f$, und es gilt $\im f = V_1(f)$ und $\ker f = V_0(f)$.
    \item
      Ist $f^2 = \id_V$, so ist $f$ diagonalisierbar mit (möglichen) Eigenwerten $1$ und $-1$.
    \item
      Sind $\lambda, \mu \in K$ mit $\lambda \neq \mu$ und $(f-\lambda)(f-\mu) = 0$, so ist $f$ diagonalisierbar mit (möglichen) Eigenwerten $\lambda$ und $\mu$.
      Inwiefern sind die vorherigen beiden Aufgabenteile Sonderfälle hiervon?
  \end{enumerate}
\end{question}


\begin{question}
  Es sei $V$ ein $K$-Vektorraum.
  Zeigen Sie, dass die folgenden Aussagen allgemein gelten, oder geben Sie jeweils ein Gegenbeispiel an.
  \begin{enumerate}[leftmargin=*]
    \item
      Ist $V = V_1 \oplus V_2$ für Untervektorräume $V_1, V_2 \subseteq V$, so gilt für jeden Untervektorraum $U \subseteq V$ die Zerlegung
      \[
        U = (U \cap V_1) \oplus (U \cap V_2).
      \]
    \item
      Ist $V = U_1 \oplus W_1 = U_2 \oplus W_2$ mit $W_1 \supseteq W_2$, so ist
      \[
        W_1 = (U_2 \cap W_1) \oplus W_2.
      \]
    \item
      Ist $f \colon V \to V$ ein Endomorphismus und $U \subseteq V$ ein $f$-invarinter Untervektorraum, so gibt es einen $f$-invarianten Untervektorraum $W \subseteq V$ mit $V = U \oplus W$.
    \item
      Für alle Untervektorräume $W, U_1, U_2 \subseteq V$ mit $U_1 \subseteq U_2$ gilt
      \[
        (U_1 + W) \cap U_2 =  U_1 + (W \cap U_2).
      \]
    \item
      Ist $\mc{E} \subseteq V$ ein Erzeugendensystem und $U \subseteq V$ ein Untervektorraum, so ist die Einschränkung $\mc{E}' \coloneqq \mc{E} \cap U$ ein Erzeugendensystem von $U$.
    \item
      Ist $(U_i)_{i \in I}$ eine Famlie von Untervektorräumen $U_i \subseteq V$ mit $V = \sum_{i \in I} U_i$ und $U_i \cap U_j = 0$ für $i \neq j$, so ist $V = \bigoplus_{i \in I} U_i$.
  \end{enumerate}
\end{question}


\begin{question}
  Es sei $V$ ein Vektorraum und $f \colon V \to V$ ein Endomorphismus.
  Es sei $(U_i)_{i \in I}$ eine Familie von $f$-invarianten Untervektorräumen, und $U \subseteq V$ ein $f$-invarianter Untervektorraum.
  Zeigen Sie:
  \begin{enumerate}[leftmargin=*]
    \item
      Auch der Schnitt $\bigcap_{i \in I} U_i$ ist $f$-invariant.
    \item
      Auch die Summe $\sum_{i \in i} U_i$ ist $f$-invariant.
    \item
      $f$ induziert eine lineare Abbildung
      \[
        \bar{f} \colon V/U \to V/U,
        \quad
        [x] \mapsto [f(x)].
      \]
  \end{enumerate}
\end{question}


\begin{question}
  Es sei $K$ ein algebraisch abgeschlossener Körper und $f \colon V \to V$ ein Endomorphismus eines endlichdimensionalen $K$-Vektorraums $V$.
  Zeigen Sie, dass die folgenden beiden Aussagen äquivalent sind:
  \begin{enumerate}
    \item
      $f$ ist diagonalisierbar.
    \item
      Für jeden $f$-invarianten Untervektorraum $U \subseteq V$ gibt es einen $f$-invarianten Untervektorraum $W \subseteq V$ mit $V = U \oplus W$.
  \end{enumerate}
\end{question}


\begin{question}
  Es sei $V$ ein $K$-Vektorraum und $U \subseteq V$ ein Untervektorraum.
  Es sei $\pi \colon V \to V/U$, $v \mapsto [v]$ die kanonische Projektion.
  \begin{enumerate}[leftmargin=*]
    \item
      Es sei $(b_i)_{i \in I}$ eine Basis von $V$, und für eine Teilmenge $J \subseteq I$ sei $(b_j)_{j \in J}$ eine Basis von $U$.
      Zeigen Sie, dass $([b_i])_{i \in I \smallsetminus J}$ eine Basis von $V/U$ ist.
    \item
      Es sei $(b_i)_{i \in I}$ eine Basis von $U$ und $(c_j)_{j \in J}$ eine Basis von $V/U$, wobei $I \cap J = \emptyset$.
      Für $j \in J$ sei $b_j \in V$ mit $\pi(b_j) = c_j$.
      Zeigen Sie, dass $(b_l)_{l \in L}$ für $L \coloneqq I \cap J$ ist eine Basis von $V$ ist.
  \end{enumerate}
\end{question}










% COMPLEXIFICATION


\begin{question}
  Es sei $V$ ein reeller Vektorraum und $(U_i)_{i \in I}$ eine Familie von Untervektorräumen $U_i \subseteq V$.
  Zeigen Sie, dass genau dann $V = \bigoplus_{i \in I} U_i$, wenn $V_\Cbb = \bigoplus_{i \in I} (U_i)_{\Cbb}$.
\end{question}


\begin{question}
  Es seien $V$ und $W$ zwei reelle Vektorräume, und $f \colon V \to W$ sei $\Rbb$-linear.
  \begin{enumerate}[leftmargin=*]
    \item
      Zeigen Sie, dass $\ker (f_\Cbb) = (\ker f)_\Cbb$.
    \item
      Folgern Sie, dass $f_\Cbb$ genau dann injektiv ist, wenn $f$ injektiv ist.
    \item
      Folgern Sie ferner, dass $(V_\Cbb)_\lambda(f_\Cbb) = V_\lambda(f)_\Cbb$ für jedes $\lambda \in \Rbb$.
    \item
      Zeigen Sie, dass $\im (f_\Cbb) = (\im f)_\Cbb$.
    \item
      Folgern Sie, dass $f_\Cbb$ genau dann surjektiv ist, wenn $f$ surjektiv ist.
  \end{enumerate}
\end{question}


\begin{question}
  Es sei $V$ ein reeller Vektorraum und $f \colon V \to V$ ein Endomorphismus.
  Zeigen Sie, dass $f$ genau dann diagonalisierbar ist, wenn $f_\Cbb$ diagonalisierbar mit reellen Eigenwerten ist.
\end{question}


\begin{question}
  Es sei
  \[
    \iota \colon \Rbb[X] \to \Cbb[X],
    \quad
    \sum_{k=0}^n a_k X^k \mapsto \sum_{k=0}^n a_k X^k
  \]
  die Teilmengeninklusion.
  \begin{enumerate}[leftmargin=*]
    \item
      Zeigen Sie, dass $\iota$ $\Rbb$-linear ist.
    \item
      Zeigen Sie, dass $\iota$ einen Isomorphismus $\Rbb[X]_\Cbb \to \Cbb[X]$ von $\Cbb$-Vektorräumen induziert.
  \end{enumerate}
\end{question}











% SCALAR PRODUCTS


\begin{question}
  Es seien $V$ und $W$ zwei endlichdimensionale euklidische Vektorräume.
  Ferner sei $f \colon V \to W$ eine $\Rbb$-lineare Abbildung.
  \begin{enumerate}[leftmargin=*]
    \item
      Zeigen Sie, dass die Abbildung
      \[
        \Phi_V \colon V \to V^*,
        \quad
        v \mapsto \bil{-, v}
      \]
      ein $\Rbb$-linearer Isomorphismus ist.
    \item
      Geben Sie die Definition der dualen Abbildung $f^* \colon W^* \to V^*$ an.
      Zeigen Sie, dass $f^*$ $\Rbb$-linear ist.
    \item
      Zeigen Sie, dass die Abbildung $g \coloneqq \Phi_V^{-1} \circ f^* \circ \Phi_W$ $\Rbb$-linear ist, und dass
      \[
        \bil{f(v), w} = \bil{v, g(w)}
        \quad
        \text{für alle $v \in V$, $w \in W$}.
      \]
    \item
      Inwiefern ändern sich die obigen Resultate für denn fall $\Kbb = \Cbb$, wenn also $V$ und $W$ endlichdimensionale unitäre Vektorräume sind?
  \end{enumerate}
\end{question}


\begin{question}
  Es sei $V \coloneqq \mathcal{C}([0,1], \Rbb)$ der Raum der stetigen Funktionen $[0,1] \to \Rbb$.
  Ferner sei $U \coloneqq \{f \in V \mid f(0) = 0\}$.
  \begin{enumerate}[leftmargin=*]
    \item
      Zeigen Sie, dass $U$ ein Untervektorraum von $V$ ist.
    \item
      Zeigen Sie, dass
      \[
        \bil{f, g} \coloneqq \int_0^1 f(t) g(t) \dd{t}
        \quad
        \text{für alle $f, g \in V$}
      \]
      ein Skalarprodukt auf $V$ definiert.
    \item
      Zeigen Sie, dass $U^\perp = 0$.
      Folgern Sie, dass $V \neq U \oplus U^\perp$.
      (\emph{Hinweis}: Betrachten Sie für $g \in U^\perp$ die Funktion $h \colon [0,1] \to \Rbb$ mit $h(t) = t^2 g(t)$.)
    \item
      Zeigen Sie ferner, dass $V/(U \oplus U^\perp)$ eindimensional ist.
  \end{enumerate}
\end{question}


\begin{question}
  Es sei
  \[
    W = \{(a_n)_{n \in \Zbb} \mid \text{$a_n \in \Rbb$ für alle $n \in \Zbb$}\}
  \]
  der Vektorraum der beidseitigen reellwertigen Folgen.
  Ferner sei
  \[
    V \coloneqq
    \left\{
      (a_n)_{n \in \Zbb} \in W
    \,\middle|\,
      \sum_{n \in \Zbb} |a_n|^2 < \infty
   \right\}
  \]
  der Untervektorraum der quadratsummierbaren Folgen.
  \begin{enumerate}[leftmargin=*]
    \item
      Zeigen Sie, dass $V$ ein Untervektorraum von $W$ ist.
    \item
      Zeigen Sie für alle $(a_n)_{n \in \Zbb}, (b_n)_{n \in \Zbb} \in V$, dass
      \[
        \sum_{n \in \Zbb} a_n b_n < \infty.
      \]
      (\emph{Hinweis:} Zeigen sie zunächst, dass $ab \leq (a^2 + b^2)/2$ für alle $a, b \in \Rbb$.)
    \item
      Zeigen sie, dass
      \[
                  \bil{ (a_n)_{n \in \Zbb}, (b_n)_{n \in \Zbb} }
        \coloneqq \sum_{n \in \Zbb} a_n b_n
        \quad
        \text{für alle $(a_n)_{n \in \Zbb}, (b_n)_{n \in \Zbb} \in V$}
      \]
      ein Skalarprodukt auf $V$ definiert.
    \item
      Es sei
      \[
        R \colon V \to V,
        \quad
        (a_n)_{n \in \Zbb} \mapsto (a_{n-1})_{n \in \Zbb}
      \]
      der Rechtsshift-Operator.
      Zeigen Sie, dass $R$ ein Adjungiertes besitzt, und entscheiden Sie, ob $R$ selbstadjungiert, unitär, bzw.\ normal ist.
    \item
      Zeigen Sie, dass $R$ keine Eigenwerte besitzt.
    \item
      Es sei
      \[
        S \colon V \to V,
        \quad
        (a_n)_{n \in \Nbb} \mapsto (a_{-n})_{n \in \Nbb}.
      \]
      Zeigen Sie, dass $S$ ein Adjungiertes besitzt, und entscheiden Sie, ob $R$ selbstadjungiert, unitär, bzw.\ normal ist.
    \item
      Zeigen Sie, dass $S$ diagonalisierbar ist.
    \item
      Es sei
      \[
        U \coloneqq \{(a_n)_{n \in \Zbb} \in V \mid \text{$a_n = 0$ für fast alle $n \in \Zbb$}\}.
      \]
      Bestimmen Sie $U^\perp$ und entscheiden Sie, ob $V = U \oplus U^\perp$.
    \item
      Bestimmen Sie eine Orthonormalbasis von $U$.
  \end{enumerate}
\end{question}


\begin{question}
  \begin{enumerate}[leftmargin=*]
    \item
      Zeigen Sie, dass durch
      \[
        \sigma(A, B) \coloneqq \tr\left( A^T B \right)
        \quad
        \text{für alle $A, B \in \Mat_n(\Rbb)$}
      \]
      ein Skalarprodukt auf $\Mat_n(\Rbb)$ definiert wird.
    \item
      Zeigen Sie, dass die Standardbasis $(E_{ij})_{i,j=1,\dotsc,n}$ von $\Mat_n(\Rbb)$ mit
      \[
        (E_{ij})_{kl} \coloneqq \delta_{ik} \delta_{jl}
        \quad
        \text{für alle $1 \leq i,j,k,l \leq n$}
      \]
      eine Orthonormalbasis von $\Mat_n(\Rbb)$ bezüglich $\sigma$ bilden.
    \item
      Es sei
      \[
        S_+ \coloneqq \{A \in \Mat_n(\Rbb) \mid A^T = A\}
      \]
      der Untervektorraum der symmetrischen Matrizen, und
      \[
        S_- \coloneqq \{A \in \Mat_n(\Rbb) \mid A^T  = -A\}
      \]
      der Untervektorraum der schiefsymmetrischen Matrizen.
      Zeigen Sie, dass
      \[
        \Mat_n(\Rbb) = S_+ \oplus S_-,
      \]
      und dass die Summe orthogonal ist.
  \end{enumerate}
\end{question}


\begin{question}
  Es sei $V$ ein Skalarproduktraum und
  \[
    O(V) \coloneqq \{ f \in \End(V) \mid f f^* = \id \}.
  \]
  Zeigen Sie, dass $O(V)$ eine Untergruppe von $\GL(V)$ bildet.
\end{question}


\begin{question}
Zeigen sie, dass für eine Matrix $A \in \Mat_n(\Kbb)$ die folgenden Bedingungen äquivalent sind:
  \begin{enumerate}
    \item
      $A$ ist invertierbar mit $A^{-1} = A^*$.
    \item
      $A A^* = I$.
    \item
      $A^* A = I$.
    \item
      Die Spalten von $A$ bilden eine Orthonormalbasis des $\Kbb^n$.
    \item
      Die Zeilen von $A$ bilden eine Orthonormalbasis des $\Kbb^n$.
  \end{enumerate}
\end{question}










% BILINEAR FORMS


\begin{question}
  Für je zwei $K$-Vektorräume $V$ und $W$ sei
  \[
              \Bil(V, W)
    \coloneqq \{b \colon V \times  W \to K \mid \text{$b$ ist bilinear}\}
  \]
  der Raum der Bilinearformen $V \times W \to K$.
  \begin{enumerate}[leftmargin=*]
    \item
      Zeigen Sie, dass die Flipabbildung
      \[
        F \colon \Bil(V, W) \to \Bil(W, V),
        \quad
        b \mapsto F(b)
        \quad\text{mit}\quad
        F(b)(w,v) = b(v,w)
      \]
      ein Isomorphismus von $K$-Vektorräumen ist.
    \item
      Es sei $b \in \Bil(V, W)$ eine Bilinearform.
      Zeigen Sie, dass $b$ ein lineare Abbildung
      \[
        \Phi_{V,W}(b) \colon V \to W^*,
        \quad
        v \mapsto b(v, -)
      \]
      induziert.
      Dabei ist
      \[
        b(v, -) \colon W \to K,
        \quad
        w \mapsto b(v,w).
      \]
    \item
      Zeigen Sie, dass die Abbildung
      \[
        \Phi_{V,W} \colon \Bil(V, W) \to \Hom(V, W^*),
        \quad
        b \mapsto \Phi_{V,W}(b)
      \]
      ein Isomorphismus von $K$-Vektorräumen ist.
    \item
      Geben Sie mithilfe der vorherigen Aufgabenteile explizit einen Isomorphismus $\Hom(V, W^*) \to \Hom(W, V^*)$ an.
  \end{enumerate}
  Wir betrachten nun den Fall $W = V^*$.
  \begin{enumerate}[resume, leftmargin=*]
    \item
      Zeigen Sie, dass die Evaluation
      \[
        e \colon V \times V^* \to K,
        \quad
        (v, \varphi) \mapsto \varphi(v)
      \]
      eine Bilinearform ist.
   \item
      Nach den vorherigen Aufgabenteilen entspricht die Bilinearform $e$ einer linearen Abbildung $V \to V^{**}$, sowie einer linearen Abbildung $V^* \to V^*$.
      Bestimmen Sie diese Abbildungen.
    \item
      Woher kennen Sie diese Abbildung?
  \end{enumerate}
\end{question}


\begin{question}
  Es seien $V$ und $W$ zwei $K$-Vektorräume und $f \colon V \to W$ eine lineare Abbildung.
  \begin{enumerate}[leftmargin=*]
    \item
      Geben Sie die Definition der dualen Abbildung $f^* \colon W^* \to V^*$ an, und zeigen sie ihre Linearität.
    \item
      Zeigen Sie für jeden $K$-Vektorraum $U$, dass die Abbildung
      \[
        \bil{\cdot, \cdot} \colon U \times U^* \to K
        \quad\text{mit}\quad
        \bil{v, \varphi} = \varphi(v)
        \quad\text{für alle $v \in V$, $\varphi \in V^*$}
      \]
      eine Bilinearform ist.
    \item
      Zeigen Sie, dass
      \[
        \bil{f(v), \psi} = \bil{v, f^*(\psi)}
        \quad
        \text{für alle $v \in V$, $\psi \in W^*$}.
      \]
  \end{enumerate}
\end{question}


\begin{question}
  \begin{enumerate}[leftmargin=*]
    \item
      Zeigen Sie, dass die Abbildung
      \[
        \sigma \colon \Mat_n(K) \times \Mat_n(K) \to K
        \quad\text{mit}\quad
        \sigma(A, B) \coloneqq \tr(AB)
      \]
      eine symmetrische Bilinearform ist.
    \item
      Zeigen Sie, dass $\sigma$ in dem Sinne assoziativ ist, dass
      \[
        \sigma(AB, C) = \sigma(A, BC)
        \quad
        \text{für alle $A, B, C \in \Mat_n(K)$}.
      \]
  \end{enumerate}
\end{question}




\end{document}
