
% BILINEAR FORMS


\begin{question}
  Bestimmen Sie die Signatur $(n_0, n_+, n_-)$ der folgenden quadratischen Formen auf $\Rbb^n$:
  \begin{enumerate}[leftmargin=*]
    \item
      $
          q(x_1, x_2)
        = 2 x_1^2 - 3 x_2^2 + 2 x_1 x_2
      $
    \item
      $
          q(x_1, x_2)
        = - x^1 + x_2 + a x_1 x_2
      $
      mit $a \in \Rbb$
    \item
      $
          q(x_1, x_2)
        = x_1^2 + 15 x_2^2 + 6 x_1 x_2
      $
    \item
      $
          q(x_1, x_2)
        = 2 x_1 x_2
      $
    \item
      $
          q(x_1, x_2, x_3)
        = x_1^2 + 2 x_1 x_2 - 2 x_1 x_3 + x_2^2 - 2 x_2 x_3 - x_3^2
      $
    \item
      $
          q(x_1, x_2, x_3, x_4)
        = x_1^2 - 7 x_2^2 - x_3^2 - x_4^2 + 2 x_1 x_2 - 6 x_2 x_3 + 6 x_2 x_4 + 2 x_3 x_4.
      $
  \end{enumerate}
\end{question}
\begin{solution}
  \begin{enumerate}[leftmargin=*]
    \item
      Die Signatur ist $(0,1,-1)$.
    \item
      Die Signatur ist $(0,1,-1)$.
    \item
      Die Signatur ist $(0,2,0)$.
    \item
      Die Signatur ist $(0,1,-1)$
    \item
      Die Signatur ist $(1,1,1)$.
    \item
      Die Signatur ist $(1,2,1)$.
  \end{enumerate}
\end{solution}

j


\begin{question}
  Es sei $V$ ein $K$-Vektorraum, $\beta \colon V \times V \to K$ eine symmetrische Bilinearform und $q \colon V \to K$ die zugehörige quadratische Form.
  \begin{enumerate}[leftmargin=*]
    \item
      Zeigen Sie für den Fall $\ringchar K \neq 2$ mithilfe einer Polarisationsformel, dass $\beta$ durch $q$ bereits eindeutig bestimmt ist.
    \item
      Folgern Sie:
      Ist $\ringchar K \neq 2$, $V \neq 0$ und $\beta$ nicht entartet, d.h.\ für jedes $v \in V$ mit $v \neq 0$ gibt es ein $w \in V$ mit $\beta(v, w) \neq 0$, so gibt es ein $v \in V$ mit $\beta(v,v) \neq 0$.
    \item
      Zeigen Sie für den Fall $\ringchar K = 2$, dass es verschieden symmetrische Bilinearformen mit gleicher quadratische Form geben kann.
      Geben Sie auch ein explizites Beispiel hierfür an.
  \end{enumerate}
\end{question}


\begin{question}
  Es sei $V$ eine reeller Vektorraum und $\bil{\cdot, \cdot} \colon V \times V \to \Rbb$ eine symmetrische Bilinearform.
  Zeigen Sie, dass die folgenden Aussagen im Allgemeinen gelten, oder geben Sie jeweils ein Gegenbeispiel an:
  \begin{enumerate}[leftmargin=*]
    \item
      Ist $\bil{v, v} \geq 0$ für alle $v \in V$, so ist $\bil{\cdot, \cdot}$ ein Skalarprodukt.
    \item
      Ist $\mc{B}$ eine Basis von $V$ mit $\bil{b, b} > 0$ für alle $b \in \mc{B}$, so ist $\bil{\cdot, \cdot}$ ein Skalarprodukt.
    \item
      Ist $\mc{B}$ eine Basis von $V$ mit $\bil{b_1, b_2} > 0$ für alle $b_1, b_2 \in \mc{B}$, so ist $\bil{\cdot, \cdot}$ ein Skalarprodukt.
    \item
      Die Teilmenge $U \coloneqq \{ v \in V \mid \text{$\bil{v, w} = 0$ für alle $w \in V$} \}$ ist ein Untervektorraum von $V$.
    \item
      Die Teilmengen
      \[
        U_+ \coloneqq \{ v \in V \mid \bil{v, v} \geq 0 \}
        \quad\text{und}\quad
        U_- \coloneqq \{ v \in V \mid \bil{v, v} \leq 0 \}
      \]
      sind Untervektorräume von $V$.
    \item
      Die Teilmenge $U_0 \coloneqq \{ v \in V \mid \bil{v, v} = 0 \}$ ist ein Untervektorraum von $V$.
    \item
      Für jeden Untervektorraum $U \subseteq V$ ist $\dim V = \dim U + \dim U^\perp$.
    \item
      Ist $U \subseteq V$ ein Untervektorraum mit $(U^\perp)^\perp = V$, so ist $U = V$.
    \item
      Für alle Untervektorräume $U_1 \subseteq U_2$ gilt $(U_1 + U_2)^\perp = U_1^\perp \cap U_2^\perp$.
    \item
      Für $U_0 \coloneqq \{ v \in V \mid \text{$\bil{v, w} = 0$ für alle $w \in V$} \}$ und jeden Untervektorraum $U \subseteq V$ gilt
      \[
        (U^\perp)^\perp = U + U_0.
      \]
  \end{enumerate}
\end{question}


\begin{solution}
  \begin{enumerate}[leftmargin=*]
    \item
      Falsch, siehe Nullbilinearform.
    \item
      Falsch, siehe
      \[
        \begin{pmatrix}
          0 & 1
          1 & 0
        \end{pmatrix}.
      \]
    \item
      Falsch, siehe
      \[
        \begin{pmatrix}
          1 & 2 \\
          2 & 1
        \end{pmatrix}
      \]
    \item
      Wahr.
    \item
      Falsch, für
      \[
        \begin{pmatrix}
          1 & 2 \\
          2 & 1
        \end{pmatrix}
      \]
      sind $e_1$ und $e_2$ in $U_+$, aber $e_1 - e_2$ nicht.
    \item
      Nein, siehe
      \[
        \begin{pmatrix}
          0 & 1 \\
          1 & 0
        \end{pmatrix}
      \]
    \item
      Nein, siehe Nullbilinearform.
    \item
      Falsch, siehe den Span von $e_2$ und
      \[
        \begin{pmatrix}
          0 & 0 \\
          0 & 1
        \end{pmatrix}.
      \]
    \item
      Die Aussage gilt.
    \item
      Falsch: Nehme ein Skalarprodukt auf einem unendlichdimensionalen Raum, und einen dichten, echten Unterraum.
  \end{enumerate}
\end{solution}


\begin{question}
  Ist $\beta \colon V \times W \to K$ eine Bilinearform, so heißen eine Basis $\mc{B} = (v_i)_{i \in I}$ von $V$ und eine Basis $\mc{C} = (w_i)_{i \in I}$ von $W$ \emph{dual bezüglich $\beta$}, falls
  \[
    \beta(v_i, w_j) = \delta_{ij}
    \quad
    \text{für alle $i, j \in I$}.
  \]
  Es sei zunächst $V$ ein $K$-Vektorraum.
  \begin{enumerate}
    \item
      Zeigen Sie, dass die \emph{Evaluation}
      \[
        e \colon V \times V^* \to K
        \quad\text{mit}\quad
        e(v, \varphi) = \varphi(e)
      \]
      eine $K$-bilineare Abbildung ist.
    \item
      Zeigen Sie im Falle der Endlichdimensionalität von $V$, dass es zu jeder Basis $\mc{B} = (b_1, \dotsc, b_n)$ von $V$ genau eine Basis $\mc{C}$ von $V^*$ gibt, die bezüglich $e$ dual zu $\mc{B}$ ist.
      Woher kennen Sie diese Basis?
  \end{enumerate}
  Von nun an sei $V$ ein endlichdimensionaler euklidischer Vektorraum mit Skalarprodukt $\bil{\cdot, \cdot}$.
  \begin{enumerate}[resume]
    \item
      Zeigen Sie, dass die Abbildung
      \[
        \Phi \colon V \to V^*,
        \quad
        v \mapsto \bil{-, v}
      \]
      ein Isomorphismus ist.
    \item
      Folgern Sie, dass es für jede Basis $\mc{B} = (b_1, \dotsc, b_n)$ von $V$ genau eine Basis $\mc{B}^\circ = (b_1^\circ, \dotsc, b_n^\circ)$ von $V$ gibt, die bezüglich $\bil{\cdot, \cdot}$ dual zu $\mc{B}$ ist.
      
      (\emph{Hinweis}:
       Formulieren Sie die Aussage, dass $\mc{C}$ dual zu $\mc{B}$ ist, mithilfe von $\Phi$ um.)
    \item
      Zeigen Sie, dass für jede Basis $\mc{B}$ von $V$ die Gleichheit $(\mc{B}^\circ)^\circ = \mc{B}$ gilt.
      Folgern Sie, dass die Abbildung
      \[
            \left\{ \text{geordnete Basen von $V$} \right\}
        \to \left\{ \text{geordenet Basen von $V$} \right\},
        \quad
        \mc{B} \mapsto \mc{B}^\circ
      \]
      bijektiv ist.
    \item
      Unter welchen Namen kennen Sie Basen von $V$, die bezüglich $(-)^\circ$ selbstdual sind, die also $\mc{B}^\circ = \mc{B}$ erfüllen?
  \end{enumerate}
\end{question}


\begin{question}
  Es sei $V$ ein $K$-Vektorraum und $\beta \colon V \times V \to K$ eine symmetrische Bilinearform.
  \begin{enumerate}[leftmargin=*]
    \item
      Zeigen Sie, dass
      \[
        \rad(\beta) \coloneqq \{ v \in V \mid \text{$\beta(v, w) = 0$ für alle $w \in V$} \}
      \]
      ein Untervektorraum von $V$ ist.
      (Man bezeichnet $\rad(\beta)$ als das \emph{Radikal} von $\beta$.)
    \item
      Zeigen Sie, dass $\beta$ eine symmetrische Bilinearform $\bar{\beta} \colon (V\!/ \rad(\beta)) \times (V\!/\rad(\beta)) \to K$ mit
      \[
        \bar{\beta}([v], [w])
        \coloneqq
        \beta(v,w)
        \quad
        \text{für alle $v, w \in V$}
      \]
      induziert.
    \item
      Zeigen Sie, dass $\bar{\beta}$ nicht entartet ist, d.h.\ dass für das Radikal
      \[
                  \rad(\bar{\beta})
        \coloneqq \{ x \in V\!/U \mid \text{$\bar{\beta}(x,y) = 0$ für alle $y \in V\!/U$} \}
      \]
      bereits $\rad(\bar{\beta}) = 0$ gilt.
    \item
      Inwiefern gelten die obigen Aussagen noch, wenn man $U$ durch
      \[
        W \coloneqq \{v \in V \mid \beta(v,v) = 0\}
      \]
      ersetzt?
  \end{enumerate}
\end{question}


\begin{question}
   Es sei $V$ ein $K$-Vektorraum und $b \colon V \times V \to K$ eine Bilinearform.
  \begin{enumerate}
    \item
      Zeigen Sie für $\ringchar K \neq 2$, dass es eindeutige Bilinearformen $b_s, b_a \colon V \times V \to K$ gibt, so dass
      \begin{itemize}
        \item
          $b = b_s + b_a$
        \item
          $b_s$ ist symmetrisch und $b_a$ ist antisymmetrisch
      \end{itemize}
    \item
      Zeigen Sie durch Angabe eines Gegenbeispiels, dass die Aussage für $\ringchar K = 2$ nicht gilt.
  \end{enumerate}
  Es sei nun $V$ der Vektorraum der Polynomfunktionen $\Rbb \to \Rbb$.
  \begin{enumerate}[resume]
    \item
      Zeigen Sie, dass die Abbildung $b \colon V \times V \to \Rbb$ mit
      \[
        b(p, q) \coloneqq \int_{-1}^1 p(t) q'(t) \dd{t}
      \]
      eine Bilinearform ist.
    \item
      Geben Sie den symmtrischen Anteil $b_s$ in einer Form an, in der kein Integral vorkommt.
%     \item
%       Geben Sie eine Basis von $V_n$ an, bezüglich der $b_s$ durch eine Diagonalmatrix mit Einträgen $0, 1, -1$ beschrieben wird, und bestimmen Sie die Signatur von $b_s$ auf $V_n$.
  \end{enumerate}
\end{question}


\begin{question}
  Für je zwei $K$-Vektorräume $V$ und $W$ sei
  \[
              \Bil(V, W)
    \coloneqq \{b \colon V \times  W \to K \mid \text{$b$ ist bilinear}\}
  \]
  der Raum der Bilinearformen $V \times W \to K$.
  \begin{enumerate}[leftmargin=*]
    \item
      Zeigen Sie, dass die Flipabbildung
      \[
        F \colon \Bil(V, W) \to \Bil(W, V),
        \quad
        b \mapsto F(b)
        \quad\text{mit}\quad
        F(b)(w,v) = b(v,w)
      \]
      ein Isomorphismus von $K$-Vektorräumen ist.
    \item
      Es sei $b \in \Bil(V, W)$ eine Bilinearform.
      Zeigen Sie, dass $b$ ein lineare Abbildung
      \[
        \Phi_{V,W}(b) \colon V \to W^*,
        \quad
        v \mapsto b(v, -)
      \]
      induziert.
      Dabei ist
      \[
        b(v, -) \colon W \to K,
        \quad
        w \mapsto b(v,w).
      \]
    \item
      Zeigen Sie, dass die Abbildung
      \[
        \Phi_{V,W} \colon \Bil(V, W) \to \Hom(V, W^*),
        \quad
        b \mapsto \Phi_{V,W}(b)
      \]
      ein Isomorphismus von $K$-Vektorräumen ist.
    \item
      Geben Sie mithilfe der vorherigen Aufgabenteile explizit einen Isomorphismus
      \[
        \Hom(V, W^*) \to \Hom(W, V^*)
      \]
      an.
  \end{enumerate}
  Wir betrachten nun den Fall $W = V^*$.
  \begin{enumerate}[resume, leftmargin=*]
    \item
      Zeigen Sie, dass die Evaluation
      \[
        e \colon V \times V^* \to K,
        \quad
        (v, \varphi) \mapsto \varphi(v)
      \]
      eine Bilinearform ist.
   \item
      Nach den vorherigen Aufgabenteilen entspricht die Bilinearform $e$ einer linearen Abbildung $V \to V^{**}$, sowie einer linearen Abbildung $V^* \to V^*$.
      Bestimmen Sie diese Abbildungen.
    \item
      Woher kennen Sie diese Abbildung?
  \end{enumerate}
\end{question}


\begin{question}
  Es seien $V$ und $W$ zwei $K$-Vektorräume und $f \colon V \to W$ eine lineare Abbildung.
  \begin{enumerate}[leftmargin=*]
    \item
      Geben Sie die Definition der dualen Abbildung $f^* \colon W^* \to V^*$ an, und zeigen Sie ihre Linearität.
    \item
      Zeigen Sie für jeden $K$-Vektorraum $U$, dass die Abbildung
      \[
        \bil{\cdot, \cdot} \colon U \times U^* \to K
        \quad\text{mit}\quad
        \bil{v, \varphi} = \varphi(v)
        \quad\text{für alle $v \in V$, $\varphi \in V^*$}
      \]
      eine Bilinearform ist.
    \item
      Zeigen Sie, dass
      \[
        \bil{f(v), \psi} = \bil{v, f^*(\psi)}
        \quad
        \text{für alle $v \in V$, $\psi \in W^*$}.
      \]
  \end{enumerate}
\end{question}


\begin{question}
  \begin{enumerate}[leftmargin=*]
    \item
      Zeigen Sie, dass die Abbildung
      \[
        \sigma \colon \Mat_n(K) \times \Mat_n(K) \to K
        \quad\text{mit}\quad
        \sigma(A, B) \coloneqq \tr(AB)
      \]
      eine symmetrische Bilinearform ist.
      Man bezeichnet diese als die \emph{Traceform}.
    \item
      Zeigen Sie, dass $\sigma$ in dem Sinne assoziativ ist, dass
      \[
        \sigma(AB, C) = \sigma(A, BC)
        \quad
        \text{für alle $A, B, C \in \Mat_n(K)$}.
      \]
    \item
      Zeigen sie, dass $\sigma$ nicht entartet ist, d.h.\ dass es für jedes $A \in \Mat_n(K)$ mit $A \neq 0$ ein $B \in \Mat_n(K)$ mit $\sigma(A, B) \neq 0$ gibt.
  \end{enumerate}
  Es sei nun
  \[
    S_+ \coloneqq \{ A \in \Mat_n(K) \mid A^T = A \}
  \]
  der Untervektorraum der symmetrischen Matrizen und
  \[
    S_- \coloneqq \{ A \in \Mat_n(K) \mid A^T = -A \}
  \]
  der Untervektorraum der schiefsymmetrischen Matrizen.
  Zeigen Sie:
  \begin{enumerate}[leftmargin=*, resume]
    \item
      Ist $\ringchar K \neq 2$, so sind $S_+$ und $S_-$ bezüglich $\sigma$ orthogonal zueinander.
    \item
      Ist $K = \Rbb$, so ist die Einschränkung von $\sigma$ auf $S_+$ positiv definit, und die Einschränkung auf $S_-$ negativ definit.
  \end{enumerate}
\end{question}


\begin{question}
  Es sei $V$ ein endlichdimensionaler $K$-Vektorraum, $b \colon V \times V \to K$ eine Bilinearform, $\mc{B} = (b_1, \dotsc, b_n)$ eine Basis von $V$, und $\mc{B}^* = (b_1^*, \dotsc, b_n^*)$ die entsprechende duale Basis von $V^*$.
  \begin{enumerate}[leftmargin=*]
    \item
      Zeigen Sie, dass die Abbildung
      \[
        B \colon V \to V^*,
        \quad
        v \mapsto b(-,v)
      \]
      $K$-linear ist.
    \item
      Zeigen Sie die Gleihheit
      \[
        \Mat_\mc{B}(b) = \Mat_{\mc{B}, \mc{B}^*}(B).
      \]
      (Beachten Sie, dass auf der linken Seite die darstellende Matrix einer Bilinearform steht, und auf der rechten Seite die darstellende Matrix einer linearen Abbildung.)
  \end{enumerate}
\end{question}










% LIE STUFF AND COMMUTATORS


\begin{question}
  Für jede Matrix $X \in \Mat_n(K)$ sei
  \[
    \lambda_X \colon \Mat_n(K) \to \Mat_n(K),
    \quad
    A \mapsto XA
  \]
  die Linksmultiplikation mit $X$,
  \[
    \rho_X \colon \Mat_n(K) \to \Mat_n(K),
    \quad
    A \mapsto AX
  \]
  die Rechtsmultiplikation mit $X$, und
  \[
    \ad_X = [X, -] \colon \Mat_n(K) \to \Mat_n(K),
    \quad
    A \mapsto [X, A] = XA - AX
  \]
  der Kommutator mit $X$.
  \begin{enumerate}[leftmargin=*]
    \item
      Zeigen Sie:
      Ist $X$ nilpotent, so sind auch $\lambda_X$ und $\rho_X$ nilpotent.
    \item
      Folgern Sie:
      Ist $X$ nilpotent, so ist auch $\ad_X$ nilpotent.
      
      (\emph{Hinweis}:
      Nutzen Sie, dass $\ad_X = \lambda_X - \rho_X$.)
    \item
      Zeigen Sie:
      Ist $X$ diagonalisierbar, so sind auch $\lambda_X$ und $\rho_X$ diagonalisierbar.
      
      (\emph{Hinweis}:
      Betrachten Sie zunächst den Fall, dass $X$ eine Diagonalmatrix ist.)
    \item
      Folgern Sie:
      Ist $X$ diagonalisierbar, so ist auch $\ad_X$ diagonalisierbar.
      
      (\emph{Hinweis}:
       Nutzen Sie, dass $\ad_X = \lambda_X - \rho_X$.)
  \end{enumerate}
\end{question}


\begin{question}
  Es sei $K$ ein Körper und
  \[
              \slLie_n(K)
    \coloneqq \{ A \in \Mat_n(K) \mid \tr A = 0 \}.
  \]
  \begin{enumerate}
    \item
      Zeigen Sie, dass $\slLie_n(K)$ ein Untervektorraum von $\Mat_n(K)$ mit $\dim \slLie_n(K) = n^2 - 1$ ist.
    \item
      Zeigen Sie, dass
      \[
        \mc{B}
        \coloneqq
              \{ E_{ij} \mid 1 \leq i \neq j \leq n \}
        \cup  \{ E_{11} - E_{ii} \mid 2 \leq i \leq n \}
      \]
      eine Basis von $\slLie_n(K)$ ist, wobei $E_{ij} \in \Mat_n(K)$ die Matrix ist, deren $(i,j)$-ter Eintrag $1$ ist, und für die alle anderen Einträge $0$ sind.
  \end{enumerate}
  Es sei nun $C \coloneqq \Ell( [A,B] \mid A, B \in \Mat_n(K) )$.
  \begin{enumerate}[resume]
    \item
      Zeigen Sie, dass $\tr([A,B]) = 0$ für alle $A, B \in \Mat_n(K)$.
      Folgern Sie, dass $C \subseteq \slLie_n(K)$.
    \item
      Zeigen Sie, dass $\slLie_n(K) \subseteq C$, indem Sie jedes der Basiselement aus $\mc{B}$ also Kommutator schreiben.
      
      (\emph{Hinweis}:
       Überlegen Sie zunächst, dass $E_{ij} E_{kl} = \delta_{jk} E_{il}$ für alle $1 \leq i, j, k, l \leq n$.)
  \end{enumerate}
  Es ist also $\slLie_n(K) = \Ell( [A,B] \mid A, B \in \Mat_n(K) )$ ein ($n^2 - 1$)-dimensionaler Untervektorraum.
  Es sei nun $f \colon \Mat_n(K) \to K$ eine lineare Abbildung mit $f(AB) = f(BA)$ für alle $A, B \in \Mat_n(K)$.
  \begin{enumerate}[resume]
    \item
      Zeigen Sie, dass $f$ eine eindeutige lineare Abbildung
      \[
        \overline{f} \colon \Mat_n(K) / \slLie_n(K) \to K,
        \quad
        [A] \mapsto f(A)
      \]
      induziert.
      Zeigen Sie, dass $\overline{\tr} \neq 0$.
    \item
      Zeigen Sie, dass $\Mat_n(K) / \slLie(K)$ eindimensional ist.
      Folgern Sie, dass es ein eindeutiges $\lambda \in K$ mit $\overline{f} = \lambda \overline{\tr}$ gibt.
    \item
      Folgern Sie, dass bereits $f = \lambda \tr$ gilt.
      (Die Spur ist also durch die Eigenschaft, dass $\tr(AB) = \tr(BA)$ für alle $A, B \in \Mat_n(K)$, bis auf skalares Vielfaches eindeutig bestimmt.)
  \end{enumerate}
\end{question}


\begin{question}
  Das \emph{Zentrum} eines Rings $R$ ist definiert als
  \[
    Z(R) \coloneqq \{r \in R \mid \text{$rs = sr$ für alle $s \in R$}.
  \]
  Man bemerke, dass $R$ genau dann kommutativ ist, wenn $Z(R) = R$.
  Wir werden $Z(\Mat_n(K))$ bestimmen.
  Hierfür sei
  \[
    D_n(K) \coloneqq K I = \{\lambda I \mid \lambda \in K\}
  \]
  der Untervektorraum der Skalarmatrizen.
  \begin{enumerate}[leftmargin=*]
    \item
      Zeigen Sie, dass $D_n(K) \subseteq Z(\Mat_n(K))$.
    \item
      Zeigen Sie für $A \in Z(\Mat_n(K))$, dass $A$ eine Diagonalmatrix ist.
      
      (\emph{Hinweis}:
       Betrachten Sie die Matrizen $E_{ii}$ für $1 \leq i \leq n$.)
    \item
      Zeigen Sie ferner, dass alle Diagonaleinträge von $A$ bereits gleich sind.
      
      (\emph{Hinweis}:
       Betrachten Sie die Matrizen $E_{ij}$ mit $1 \leq i,j \leq n$.)
    \item
      Folgern Sie, dass $Z(\Mat_n(K)) = D_n(K)$.
  \end{enumerate}
\end{question}


\begin{question}
  Es sei $V$ ein endlichdimensionaler $\Cbb$-Vektorraum, und es seien $K, E \colon V \to V$ zwei Endomorphismen mit
  \[
    \text{$K$ ist invertierbar}
    \quad\text{und}\quad
    KE = 2EK.
  \]
  \begin{enumerate}[leftmargin=*]
    \item
      Zeigen Sie, dass
      \[
        (K - 2 \lambda \id_V)^n E = 2^n E (K - \lambda \id_V)^n
      \]
      für alle $n \in \Nbb$.
    \item
      Folgern Sie, dass $E( V^\sim_\lambda(K) ) \subseteq V^\sim_{2\lambda}(K)$ für alle $\lambda \in \Cbb$.
    \item
      Folgern Sie, dass $E$ nilpotent ist.
  \end{enumerate}
\end{question}


\begin{question}
  Es sei $V$ ein $K$-Vektorrraum und $[-,-] \colon V \times V \to V$ eine alternierend bilineare Abbildung.
  Für jedes $x \in V$ sei
  \[
    \ad_x \coloneqq [x,-] \colon V \to V, \quad y \mapsto [x,y].
  \]
  Zeigen Sie, dass die folgenden beiden Aussagen äquivalent sind:
  \begin{enumerate}
    \item
      $[-,-]$ erfüllt die Jacobi-Identität, d.h.\ es ist
      \[
        [x,[y,z]] + [y,[z,x]] + [z,[x,y]] = 0
        \quad
        \text{für alle $x, y, z \in V$}.
      \]
    \item
      Es gilt
      \[
          \ad_x([y,z])
        = [\ad_x(y), z] + [y, \ad_x(z)]
        \quad
        \text{für alle $x, y, z \in V$}.
      \]
      (Man sagt, dass $\ad_x$ eine Derivation bezüglich $[-,-]$ ist.)
  \end{enumerate}
\end{question}


\begin{question}
  Es seien $E$ und $H$ zwei Endomorphismen eines $\Cbb$-Vektorraums $V$, so dass $[H,E] = 2E$.
  \begin{enumerate}[leftmargin=*]
    \item
      Zeigen Sie, dass $E(V_\lambda(H)) \subseteq V_{\lambda + 2}(H)$ für alle $\lambda \in K$.
    \item
      Folgern Sie: Ist $V$ endlichdimensional und $H$ diagonalisierbar, so ist $E$ nilpotent.
  \end{enumerate}
\end{question}


\begin{question}
  Es sei $B \in \Mat_n(\Kbb)$.
  Es seien
  \[
              \Orthogonal(B)
    \coloneqq \{ S \in \GL_n(\Kbb) \mid S^T B S = B \}
  \]
  und
  \[          \gLie(B)
    \coloneqq \{ A \in \Mat_n(\Kbb) \mid A^T B = - B A \}
  \]
  \begin{enumerate}[leftmargin=*]
    \item
      Zeigen Sie, dass $\Orthogonal(B)$ eine Untergruppe von $\GL_n(\Kbb)$ ist.
    \item
      Zeigen Sie, dass $\gLie(B)$ eine Lie-Unteralgebra von $\gl_n(\Kbb)$ ist, dass also $[A_1, A_2] \in \gLie(B)$ für alle $A_1, A_2 \in \gLie(B)$.
    \item
      Zeigen Sie, dass $\exp(A) \in \Orthogonal(B)$ für alle $A \in \gLie(B)$.
      (\emph{Hinweis}:
       Zeigen Sie zunächst, dass $\exp(A)^T B = B \exp(-A)$.)
    \item
      Geben Sie für $\Kbb = \Rbb$ eine Matrix $B \in \Mat_n(\Rbb)$ an, so dass $\Orthogonal(B) = \Orthogonal(n)$.
      Was sind dann die Elemente von $\gLie(B)$?
  \end{enumerate}
\end{question}


\begin{question}
  Für einen endlichdimensionalen $\Kbb$-Vektorraum $V$ und eine Bilinearform $\beta \colon V \times V \to \Kbb$ sei
  \[
              O(\beta)
    \coloneqq \{ \phi \in \GL(V) \mid \text{$\beta(\phi(x), \phi(y)) = \beta(x,y)$ für alle $x,y \in V$} \}
  \]
  die Isometriegruppe von $\beta$, und
  \[
              \gLie(\beta)
    \coloneqq \{ f \in \End(V) \mid \text{$\beta(f(x),y) = -\beta(x, f(y))$ für alle $x, y \in V$} \}
  \]
  die assoziierte Lie-Algebra.
  \begin{enumerate}[leftmargin=*]
    \item
      Zeigen Sie, dass $O(\beta)$ eine Untergruppen von $\GL(V)$ ist.
    \item
      Zeigen Sie, dass $\gLie(\beta)$ eine Lie-Unteralgebra von $\gl(V)$ ist, d.h.\ dass $[f,g] \in \gLie(\beta)$ für alle $f, g \in \gLie(\beta)$.
    \item
      Zeigen Sie, dass $\exp(f) \in O(\beta)$ für alle $f \in \gLie(\beta)$.
      
      (\emph{Hinweis}:
       Zeigen Sie zunächst, dass $\beta(\exp(f)(x) ,y) = \beta(x, \exp(-f)(y))$ für alle $x, y \in V$.
       Nutzen Sie hierfür, dass die bilineare Abbildung $\beta$ in beiden Argumenten stetig ist.)
    \item
      Es sei $\Kbb = \Rbb$ und $\bil{\cdot, \cdot}$ ein Skalarprodukt auf $V$.
      Unter welchen Begriffen sind die Elemente aus $G(\bil{\cdot, \cdot})$ und $\gLie(\bil{\cdot, \cdot})$ bekannt?
  \end{enumerate}
\end{question}


\begin{question}
  Es sei $V$ ein $K$-Vektorraum und $m \colon V \times V \to V$ eine bilineare Abbildung.
  Eine lineare Abbildung $D \colon V \to V$ heißt \emph{$m$-Derivation}, falls
  \[
    D(m(x,y))
    = m(D(x), y) + m(x, D(y))
    \quad
    \text{für alle $x, y \in V$}.
  \]
  Es sei
  \[
              \Der(m)
    \coloneqq \{ D \colon V \to V \mid \text{$D$ ist eine $m$-Derivation} \}.
  \]
  \begin{enumerate}[leftmargin=*]
    \item
      Zeigen Sie für den Fall $V = K[X]$ und die Multiplikation
      \[
        m(p,q) \coloneqq p \cdot q
        \quad
        \text{für alle $p, q \in K[X]$},
      \]
      dass die Ableitung
      \[
        D \colon K[X] \to K[X],
        \quad
        \sum_{d=0}^n a_d X^d  \mapsto \sum_{d=1}^n a_d d X^{d-1} 
      \]
      eine $m$-Derivation ist.
      Unter welchem Namen ist dieser Umstand für gewöhnlich bekannt?
    \item
      Zeigen Sie, dass $\Der(m)$ ein Untervektorraum von $\End(V)$ ist.
    \item
      Zeigen Sie, dass $\Der(m)$ eine Lie-Unteralgebra von $\End(V)$ ist, d.h.\ dass für alle $D_1, D_2 \in \Der(m)$ auch $[D_1, D_2] \in \Der(m)$.
  \end{enumerate}
\end{question}










% OTHER STUFF


\begin{question}
  Es sei $V$ ein endlichdimensionaler $K$-Vektorraum mit Basis $\mc{B} = (b_1, \dotsc, b_n)$ und $\omega \colon V^n \to K$ eine alternierende Multilinearform.
  \begin{enumerate}
    \item
      Zeigen Sie, dass genau dann $\omega \neq 0$, wenn $\omega(b_1, \dotsc, b_n) \neq 0$.
    \item
      Es sei $f \colon V \to V$ ein Endomorphismus und
      \[
        \omega_f \coloneqq \omega \circ f^{\times n} \colon V^n \to V,
      \]
      Zeigen Sie, dass $\omega_f$ ebenfalls multilinear und alternierend ist.
    \item
      Zeigen Sie, dass $\omega_f = \det(f) \omega$.
  \end{enumerate}
\end{question}


\begin{question}
  Es sei $V$ ein endlichdimensionaler $K$-Vektorraum und $n \coloneqq \dim V$.
  Es sei $\omega \colon V^m \to V$ eine alternierende Multilinearform.
  Zeigen Sie, dass $\omega = 0$.
\end{question}


\begin{question}
  Es sei $n \geq 1$.
  Entscheiden Sie, welche der folgenden Aussagen wahr oder falsch sind:
  \begin{enumerate}[leftmargin=*]
    \item
      Die Wegzusammenhangskomponenten von $\GL_n(\Rbb)$ sind die beiden Untergruppen
      \[
        \GL_n(\Rbb)_+ = \{ S \in \GL_n(\Rbb) \mid \det S > 0 \}
        \quad\text{und}\quad
        \GL_n(\Rbb)_- = \{ S \in \GL_n(\Rbb) \mid \det S < 0 \}.
      \]
    \item
      Die Gruppe $\SOrthogonal(n)$ ist nicht wegzusammenhängend.
    \item
      Die Gruppe $\Orthogonal(n)$ ist eine Wegzusammenhangskomponente von $\GL_n(\Rbb)$.
    \item
      Die schiefsymmetrischen Matrizen
      \[
        \mathfrak{o}_n(\Rbb) = \{ A \in \Mat_n(\Rbb) \mid A^T = -A \}
      \]
      sind wegzusammenhängend.
    \item
      Die Gruppe $\Unitary(n) \cap \GL_n(\Rbb)_+$ besteht aus zwei Wegzusammenhangskompenenten.
    \item
      Ist $G$ eine wegzusammenhängende Untergruppe von $\GL_n(\Kbb)$, so ist
      \[
        G' \coloneqq \{ S \in G \mid \det g = 1 \}
      \]
      ebenfalls eine wegzusammenhängende Untergruppe von $\GL_n(\Kbb)$.
    \item
      Jede Untergruppe von $\GL_n(\Cbb)$ ist wegzusammenhängend.
    \item
      Die Gruppe $\SUnitary(n) \cap \Orthogonal(n)$ ist wegzusammenhängend.
    \item
      Es ist $G = \{A^* = - A\}$ eine wegzusammenhängende Untergruppe von $A$.
    \item
      Die Gruppe der Drehmatrizen
      \[
        \left\{
          \begin{pmatrix*}[r]
            \cos \varphi  & -\sin \varphi \\
            \sin \varphi  &  \cos \varphi 
          \end{pmatrix*}
        \,\middle|\,
        \varphi \in \Rbb
        \right\}
      \]
      ist wegzusammenhängend.
  \end{enumerate}
\end{question}

\begin{solution}
  \begin{enumerate}
    \item
      Nein, $\GL_n(\Rbb)_-$ ist keine Untergruppe.
    \item
      Falsch, $\SOrthogonal(n)$ ist wegzusammenhängend.
    \item
      Falsch, die Wegzusammenhangskomponenten von $\GL_n(\Rbb)$ sind $\GL_n(\Rbb)_+$ und $\GL_n(\Rbb)_-$.
    \item
      Ja, denn normierte Vektorräume sind immer wegzusammenhängend.
    \item
      Nein: Der Schnitt ist $\SO(n)$, also wegzusammenhängend.
    \item
      Ja: Wähle einen Weg in $G$ und teile diesen durch die Determinante.
    \item
      Nein, es ist keine Untergruppe.
    \item
      Falsch, siehe $\{1,-1\}$.
    \item
      Ja, der Schnitt ist $\SOrthogona(n)$.
    \item
      Ja: Ändere den Winkel stetig.
  \end{enumerate}
\end{solution}


\begin{question}
  Es sei $V$ ein euklidischer $n$-dimensionaler Vektorraum und $d$ eine normierte, alternierende $n$-Form auf $d$.
  Es sei $u \in V$ mit $\|u\| = 1$.
  Zeigen Sie für das orthogonale Komplement $U \coloneqq u^\perp = \Ell(u)^\perp$, dass die Einschränkung $d(-, \dotsc, -, u)|_{U^{n-1}}$ eine normierte, alternierende $(n-1)$-Form ist.
\end{question}


\begin{question}
  Es sei $V$ ein euklidischer Vektorraum und $\omega \colon V^3 \to K$ eine Trilinearform.
  \begin{enumerate}[leftmargin=*]
    \item
      Zeigen Sie, dass es für alle $u, v \in V$ ein eindeutiges Element $\times_\omega \in V$ gibt, so dass
      \[
          \omega(u, v, w)
        = \bil{u \times_\omega v, w}
        \quad
        \text{für alle $u, v, w \in V$}.
      \]
    \item
      Zeigen Sie, dass die Abbildung $\times_\omega \colon V \times V \to V$ bilinear ist.
    \item
      Zeigen Sie:
      Ist $\omega$ symmetrisch, bzw.\ alternierend, so ist auch $\times_\omega$ symmetrisch, bzw.\ alternierend.
    \item
      Zeigen Sie, dass die Abbildung
      \[
        \Tril(V, V, V; K) \mapsto \Bil(V, V; V),
        \omega
        \mapsto
        \times_\omega
      \]
      ein Isomorphismus von $K$-Vektorräumen ist.
  \end{enumerate} 
\end{question}


\begin{question}
  Es sei $V$ ein $K$-Vektorraum und $\beta \colon V \times V \to K$ eine nicht-entartete Bilinearform.
  \begin{enumerate}
    \item
      Zeigen Sie, dass es für jede Bilinearform $\gamma \colon V \times V \to K$ eine eindeutige lineare Abbildung $f \colon V \to V$ gibt, so dass
      \[
        \gamma(x,y) = \beta(f(x), y)
        \quad
        \text{für alle $x,y \in V$}.
      \]
    \item
      Zeigen oder widerlegen Sie, dass es für jede nicht-entartete Bilinearform $\gamma \colon V \times V \to K$ einen eindeutigen Skalar $c \in K$ mit $\gamma = c \beta$ gibt.
  \end{enumerate}
\end{question}


\begin{question}
  Zeigen oder widerlegen Sie, dass der Kommutator
  \[
    [-, -] \colon \Mat_n(K) \times \Mat_n(K) \to \Mat_n(K)
  \]
  in dem folgenden Sinne universell ist:
  
  Für jeden $K$-Vektorraum $V$ und jede bilineare Abbildung \mbox{$\beta \colon \Mat_n(K) \times \Mat_n(K) \to V$} gibt es eine eindeutige $K$-lineare Abbildung $\varphi \colon \Mat_n(K) \to V$, so dass das folgende Diagramm kommutiert:
  \[
    \begin{tikzcd}[row sep = large, column sep = large, ampersand replacement = \&]
            {}
        \&  \Mat_n(K) \times \Mat_n(K)  \arrow{rd}{\beta}
                                        \arrow[swap]{ld}{[-,-]}
        \&  {}
      \\
            \Mat_n(K)                   \arrow{rr}{\varphi}
        \&  {}
        \&  V
    \end{tikzcd}
  \]
\end{question}


\begin{question}
  Es seien $V$ und $W$ zwei $K$-Vektorräume und $\beta \colon V \times V \to W$ eine bilineare Abbildung.
  Zeigen oder widerlegen Sie, dass $\im \beta$ notwendigerweise ein Untervektorraum von $W$ ist.
\end{question}


\begin{question}
  Es sei $m \colon K^n \times K^n \to \Mat_n(K)$ mit
  \[
    m(x,y) = x y^T
    \quad
    \text{für alle $x,y \in K^n$}.
  \]
  \begin{enumerate}[leftmargin=*]
    \item
      Zeigen Sie, dass $m$ bilinear ist.
    \item
      Zeigen Sie, dass $m$ im folgenden Sinne universell ist:
      Für jeden Vektorraum $V$ und jede bilineare Abbildung $b \colon K^n \times K^n \to V$ gibt es eine eindeutige lineare Abbildung $\varphi \colon \Mat_n(K) \to V$ mit $b = \varphi \circ m$, dass also das folgende Diagramm kommutiert:
      \[
        \begin{tikzcd}[row sep = large, column sep = large, ampersand replacement = \&]
              {}
          \&  K^n \times K^n  \arrow{rd}{m}
                              \arrow[swap]{ld}{b}
          \&  {}
          \\
              \Mat_n(K)       \arrow{rr}{\varphi}
          \&  {}
          \&  V
        \end{tikzcd}
      \]
  \end{enumerate}
\end{question}
