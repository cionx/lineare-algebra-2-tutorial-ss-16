\section{Skalarprodukträume}


%%% PRIORITY 1


\begin{question}[subtitle = Definitionen]{1}
  Definieren Sie die Begriffe eines reellen, bzw.\ komplexen Skalarprodukts, sowie eines reellen, bzw.\ komplexen Hilbertraums.
\end{question}


\begin{question}[subtitle = Cauchy-Schwarz]{1}
  Formulieren und Beweisen Sie die Cauchy-Schwarz-Ungleichung.
\end{question}


\begin{question}[subtitle = Existenz von Orthonormalbasen]{1}
  Zeigen Sie, dass jeder endlichdimensionale Skalarproduktraum eine Orthonormalbasis besitzt.
\end{question}


\begin{question}[subtitle = Definition und Eindeutigkeit der adjungierten Abbildung]{1}
  Es seien $V$ und $W$ zwei Skalarprodukträume über $\Kbb$ und $f \colon V \to W$ eine lineare Abbildung.
  \begin{enumerate}[leftmargin=*]
    \item
      Definieren Sie die zu $f$ adjungierte Abbildung.
    \item
      Zeigen Sie, dass die adjungierte Abbildung eindeutig ist.
  \end{enumerate}
\end{question}


\begin{question}[subtitle = Darstellende Matrix der adjungiertes Abbildung]{1}
  Es seien $V$ und $W$ zwei $\Kbb$-Skalarprodukträume und $f \colon V \to W$ eine lineare Abbildung.
  Es sei $\mc{B} = (b_1, \dotsc, b_n)$ eine Orthonormalbasis von $V$ und $\mc{C} = (c_1, \dotsc, c_m)$ eine Orthonormalbasis von $W$.
  Zeigen Sie die Gleichheit
  \[
      \Mat_{\mc{C}, \mc{B}}(f^*)
    = \Mat_{\mc{B}, \mc{C}}(f)^*.
  \]
\end{question}


\begin{question}[subtitle = Orthogonalität der Eigenräume normaler Endomorphismen]{1}
  Es sei $V$ ein Skalarproduktraum und $f \colon V \to V$ ein normaler Endomorphismus.
  Zeigen Sie, dass die Eigenräume $V_\lambda(f)$ und $V_\mu(f)$ für alle $\lambda \neq \mu$ orthogonal sind.
\end{question}


\begin{question}[subtitle = Haupt- und Eigenräume normaler Endomorphismen]{1}
  Es sei $V$ ein endlichdimensionaler Skalarproduktraum und $f \colon V \to V$ ein normaler Endomorphismus.
  \begin{enumerate}[leftmargin=*]
    \item
      Zeigen Sie, dass $\|f(v)\| = \|f^*(v)\|$ für alle $v \in V$.
    \item
      Zeigen Sie, dass $V_\lambda(f) = V_{\overline{\lambda}}(f^*)$ und $V^\sim_\lambda(f) = V^\sim_{\overline{\lambda}}(f^*)$.
  \end{enumerate}
\end{question}


\begin{question}[subtitle = Eigenwerte (anti)selbstadjungierter Endomorphismen]{1}
  Es sei $S \colon V \to V$ ein Endomorphismus eines Skalarproduktraums.
  Zeigen Sie:
  \begin{enumerate}[leftmargin=*]
    \item
      Ist $S$ selbstadjungiert, so sind alle Eigenwerte von $S$ rein reell.
    \item
      Ist $S$ antiselbstadjungiert, so sind alle Eigenwerte von $S$ rein imaginär.
  \end{enumerate}
\end{question}


\begin{question}[subtitle = Komposition selbstadjungierter Endomorphismen]{1}
  Es seien $F$ und $G$ zwei selbstadjungierte Endomorphismen eines Skalarproduktraums $V$.
  Zeigen Sie, dass $F \circ G$ genau dann selbstadjungiert ist, wenn $F$ und $G$ kommutieren.
\end{question}


\begin{question}[subtitle = Rechenregeln für das Matrixexponential]{1}
  Es sei $V$ ein endlichdimensionaler Skalarproduktraum und $f \colon V \to V$ ein Endomorphismus.
  Zeigen Sie:
  \begin{enumerate}[leftmargin=*]
    \item
      Es gilt $\exp(f)^* = \exp(f^*)$.
    \item
      Ist $f$ normal, so ist auch $\exp(f)$ normal.
    \item
      Ist $f$ selbstadjungiert, so ist auch $\exp(f)$ selbstadjungiert.
    \item
      Ist $f$ antiselbstadjungiert, so ist $\exp(f)$ orthogonal ($\Kbb = \Rbb$), bzw.\ unitär ($\Kbb = \Cbb$).
    \item
      Es gilt $\det \exp(f) = \exp(\tr f)$.
  \end{enumerate}
\end{question}






%%% PRIORITY 2


\begin{question}[subtitle = Orthogonalität und lineare Unabhängigkeit von Vektoren]{2}
  Es sei $V$ ein Skalarproduktraum und es seien $v_1, \dotsc, v_n \in V$
  \begin{enumerate}[leftmargin=*]
    \item
      Zeigen Sie:
      Sind die Vektoren $v_1, \dotsc, v_n$ paarweise orthogonal und ist $v_1, \dotsc, v_n \neq 0$, so ist die Familie $(v_1, \dotsc, v_n)$ linear unabhängig.
    \item
      Zeigen Sie:
      Ist die Famlie $(v_1, \dotsc, v_n)$ linear unabhängig, so sind die Vektoren $v_1, \dotsc, v_n$ nicht notwendigerweise orthogonal.
  \end{enumerate}
\end{question}


\begin{question}[subtitle = Ein Gegenbeispiel]{2}
  Es sei $V$ ein $\Kbb$-Vektorraum mit abzählbarer Orthonormalbasis $(e_i)_{i \in \Nbb}$.
  Es sei $T \colon V \to V$ die eindeutige lineare Abbildung mit $T(e_i) = e_1$ für alle $i \in \Nbb$.
  Zeigen Sie, dass $T$ kein Adjungiertes besitzt.
\end{question}


\begin{question}[subtitle = Eine Kürzungsregel]{2}
  Es sei $V$ ein Skalarproduktraum.
  Zeigen Sie für Endomorphismen $f, g_1, g_2 \colon V \to V$ Endomorphismen die folgende Kürzungsregel:
  Falls $f^*$ existiert und $f^* f g_1 = f^* f g_2$, dann ist bereits $f g_1 = f g_2$.
\end{question}


\begin{question}[subtitle = Bestimmung von Abbildungen]{2}
  \begin{enumerate}
    \item
      Es sei $V$ ein Skalarproduktraum und $f \colon V \to V$ ein selbstadjungierter, nilpotenter Endomorphismus.
      Zeigen Sie, dass $f = 0$.
    \item
      Es sei $V$ ein endlichdimensionaler euklidischer Vektorraum und $f \colon V \to V$ ein selbstadjungierter, orthogonaler Endomorphismus mit nur positiven Eigenwerten.
      Zeigen Sie, dass bereits $f = \id_V$ gilt.
    \item
        Es sei $V$ ein euklidischer Vektorraum, und die Abbildung $P \colon V \to V$ sei orthogonal und eine Orthogonalprojektion.
        Bestimmen Sie $P$.
  \end{enumerate}
\end{question}


\begin{question}[subtitle = Orthogonalprojektion auf eine Gerade]{2}
  Es sei $x \in \Rbb^n$ ein normierter Spaltenvektor und
  \[
    A \coloneqq x x^T \in \Mat(n \times n, \Rbb).
  \]
  Zeigen Sie, dass die Abbildung
  \[
    P \colon \Rbb^n \to \Rbb^n,
    \quad
    y \mapsto Ay
  \]
  die orthogonale Projektion auf die Gerade $\Rbb x$ ist.
\end{question}


\begin{question}[subtitle = Eine Formel für die Orthogonalprojektion]{2}
  Es sei $V$ ein endlichdimensionaler Skalarproduktraum und $U \subseteq V$ ein Untervektorraum mit Orthonormalbasis $(u_1, \dotsc, u_n)$.
  Zeigen Sie, dass die lineare Abbildung
  \[
    P \colon V \to V,
    \quad
    v \mapsto \sum_{i=1}^n \bil{v, u_i} u_i
  \]
  die orthogonale Projektion auf $U$ ist.
\end{question}


\begin{question}[subtitle = Eine Bedingung für Orthonormalbasen]{2}
  Es sei $V$ ein endlichdimensionaler Skalarproduktraum und $v_1, \dotsc, v_n \in V$ seien Einheitsvektoren.
  Zeigen Sie, dass die folgenden beiden Aussage äquivalent sind:
  \begin{enumerate}
    \item
      $(v_1, \dotsc, v_n)$ ist eine Orthonormalbasis von $V$.
    \item
      Für alle $v \in V$ ist $\| v \|^2 = \sum_{i=1}^n |\bil{v, v_i}|^2$.
  \end{enumerate}
\end{question}


\begin{question}[subtitle = Normale Endomorphismen und ihre Eigenwerte]{2}
  Es sei $V$ ein endlichdimensionaler unitärer Vektorraum und $f \colon V \to V$ ein normaler Endomorphismus.
  Zeigen Sie:
  \begin{enumerate}[leftmargin=*]
    \item
      $f$ ist genau dann unitär, wenn alle Eigenwerte von $f$ Betrag $1$ haben.
    \item
      $f$ ist genau dann selbstadjungiert, wenn alle Eigenwerte von $f$ reell sind.
    \item
      $f$ ist genau dann antiselbstadjungiert, wenn alle Eigenwerte von $f$ rein imaginär sind.
    \item
      $f$ ist genau dann eine Orthogonalprojektion, wenn $0$ und $1$ die einzigen Eigenwerte von $f$ sind.
  \end{enumerate}
\end{question}


\begin{question}[subtitle = Vielfache von Skalarprodukten]{2}
  Es sei $V \neq 0$ ein $\Kbb$-Vektorraum mit Skalarprodukt $\bil{\cdot, \cdot}$.
  Für alle $\lambda \in \Kbb$ sei
  \[
    \bil{x, y}_\lambda \coloneqq \lambda \bil{x, y}
    \quad
    \text{für alle $x, y \in V$}.
  \]
  Bestimmen Sie alle $\lambda \in \Kbb$, für die $\bil{\cdot, \cdot}_\lambda$ ein Skalarprodukt auf $V$ ist.
\end{question}


\begin{question}[subtitle = Skalarprodukte durch Vorgabe von Orthonormalbasen]{2}
  Es sei $V$ ein endlichdimensionaler $\Kbb$-Vektorraum mit $n \coloneqq \dim V$.
  \begin{enumerate}[leftmargin=*]
    \item
      Zeigen Sie, dass es für jede Basis $\mc{B} = \{b_1, \dotsc, b_n\}$ von $V$ genau ein Skalarprodukt $\bil{\cdot, \cdot}_\mc{B}$ auf $V$ gibt, so dass $\mc{B}$ eine Orthonormalbasis von $V$ bezüglich $\bil{\cdot, \cdot}_\mc{B}$ ist.
    \item
      Untersuchen Sie die Abbildung
      \begin{align*}
         \{\mc{B} \subseteq V \mid \text{$\mc{B}$ ist eine Basis von $V$}\}
        &\longrightarrow
         \{\bil{\cdot, \cdot} \colon V \times V \to V \mid \text{$\bil{\cdot, \cdot}$ ist ein Skalarprodukt auf $V$}\},
        \\
         \mc{B}
        &\longmapsto
         \bil{\cdot, \cdot}_\mc{B}
      \end{align*}
      auf Injektivität und Surjektivität.
  \end{enumerate}
\end{question}


\begin{question}[subtitle = Die Spur einer unitären Matrix]{2}
  Es sei $A \in \Unitary(n)$.
  Zeigen Sie, dass $|\tr A| \leq n$.
  Wann gilt Gleichheit?
\end{question}


\begin{question}[subtitle = Ein selbstadjungierter Endomorphismus im Unendlichdimensionalen]{2}
  Es sei $V$ der reelle Vektorraum der Polynomfunktionen $\Rbb \to \Rbb$, und für alle $n \in \Nbb$ sei $V_n \subseteq V$ der Untervektorraum der Polynomfunktionen von Grad $\leq n$.
  \begin{enumerate}[leftmargin=*]
    \item
      Zeigen Sie, dass
      \[
                  \bil{f, g}
        \coloneqq \int_{-1}^1 f(t) g(t) \dd{t}
        \quad
        \text{für alle $f, g \in V$}
      \]
      ein Skalarprodukt auf $V$ definiert.
    \item
      Zeigen Sie, dass die lineare Abbildung $\psi \colon V \to V$ mit
      \[
        \psi(f)(t) \coloneqq (t^2 - 1) f''(t) + 2t f'(t)
        \quad
        \text{für alle $f \in V$ und $t \in \Rbb$}
      \]
      selbstadjungiert bezüglich $\bil{\cdot, \cdot}$ ist.
  \end{enumerate}
  Es sei $\mc{G} \coloneqq (p_n)_{n \geq 0}$ die Orthonormalbasis von $V$, die durch Anwenden des Gram-Schmidt-Verfahrens auf die Standardbasis $\mc{B} \coloneqq (x^n)_{n \geq 0}$ von $V$ ensteht.
  \begin{enumerate}[resume, leftmargin=*]
    \item
      Zeigen Sie für alle $n \geq 0$, dass $V_n$ invariant unter $\psi$ ist.
    \item
      Zeigen Sie für alle $n \geq 0$, dass $\mc{G}_n \coloneqq (p_0, \dotsc, p_n)$ eine Basis von $V_n$ ist.
    \item
      Zeigen Sie für alle $n \geq 0$, dass $\Mat_{\mc{G}_n}(\psi|_{V_n})$ eine obere Dreiecksmatrix ist.
      Betrachten Sie hierfür die Filtration
      \[
        0 \subseteq V_0 \subseteq V_1 \subseteq V_2 \subseteq V_3 \subseteq \dotsb \subseteq V_n,
      \]
      und nutzen Sie, dass $V_i = \Ell(\mc{G}_i)$ für alle $1 \leq i \leq n$ invariant unter $\psi$ ist.
    \item
      Folgen Sie mithilfe der Selbstadjungiertheit von $\psi$, dass $\Mat_{\mc{G}_n}(\psi|_{V_n})$ für alle $n \geq 0$ bereits eine Diagonalmatrix ist.
      Folgern Sie, dass $\mc{G}$ eine Basis aus Eigenvektoren von $\psi$ ist.
    \item
      Bestimmen Sie für alle $n \geq 0$ die Eigenwerte der Einschränkung $\psi|_{V_n}$, indem Sie die darstellende Matrix bezüglich der Basis $\mc{B}_n = (1, x, \dotsc, x^n)$ von $V_n$ bestimmen.
    \item
      Geben Sie den zu $p_n$ gehörigen Eigenwert von $\psi$ an.
    \item
      Berechnen Sie $\mc{G}_4$.
  \end{enumerate}
\end{question}


\begin{question}[subtitle = Ein $L^2$-Skalarprodukt und ein Gegenbeispiel im Unendlichdimensionalen]{2}
  Es sei $V \coloneqq \mathcal{C}([0,1], \Rbb)$ der Raum der stetigen Funktionen $[0,1] \to \Rbb$, und es sei
  \[
    U \coloneqq \{ f \in V \mid f(0) = 0 \}.
  \]
  \begin{enumerate}[leftmargin=*]
    \item
      Zeigen Sie, dass $U$ ein Untervektorraum von $V$ ist.
    \item
      Zeigen Sie, dass
      \[
        \bil{f, g} \coloneqq \int_0^1 f(t) g(t) \dd{t}
        \quad
        \text{für alle $f, g \in V$}
      \]
      ein Skalarprodukt auf $V$ definiert.
    \item
      Zeigen Sie, dass $U^\perp = 0$.
      Folgern Sie, dass $V \neq U \oplus U^\perp$.
      
      (\emph{Hinweis}:
       Betrachten Sie für $g \in U^\perp$ die Funktion $h \colon [0,1] \to \Rbb$ mit $h(t) = t^2 g(t)$.)
    \item
      Zeigen Sie ferner, dass $V\!/U$ eindimensional ist.
  \end{enumerate}
\end{question}


\begin{question}[subtitle = Beispiele und Gegenbeispiele auf $\ell^2(\Zbb)$]{2}
  Es sei
  \[
    \Rbb^\Zbb = \{(a_n)_{n \in \Zbb} \mid \text{$a_n \in \Rbb$ für alle $n \in \Zbb$}\}
  \]
  der Vektorraum der beidseitigen reellwertigen Folgen.
  Wir betrachten die Teilmenge der quadratsummierbaren Folgen
  \[
    \ell^2(\Zbb) \coloneqq
    \left\{
      (a_n)_{n \in \Zbb} \in \Rbb^\Zbb
    \,\middle|\,
      \sum_{n \in \Zbb} |a_n|^2 < \infty
   \right\}.
  \]
  \begin{enumerate}[leftmargin=*]
    \item
      Zeigen Sie für alle $(a_n)_{n \in \Zbb}, (b_n)_{n \in \Zbb} \in V$, dass
      \[
        \sum_{n \in \Zbb} a_n b_n < \infty.
      \]
      
      (\emph{Hinweis}:
       Zeigen sie zunächst, dass $ab \leq (a^2 + b^2)/2$ für alle $a, b \in \Rbb$.)
    \item
      Folgern Sie, dass $\ell^2(\Zbb)$ ein Untervektorraum von $\Rbb^\Zbb$ ist.
    \item
      Zeigen Sie, dass
      \[
                  \bil{ (a_n)_{n \in \Zbb}, (b_n)_{n \in \Zbb} }
        \coloneqq \sum_{n \in \Zbb} a_n b_n
        \quad
        \text{für alle $(a_n)_{n \in \Zbb}, (b_n)_{n \in \Zbb} \in V$}
      \]
      ein Skalarprodukt auf $\ell^2(\Zbb)$ definiert.
    \item
      Es sei
      \[
        R \colon \ell^2(\Zbb) \to \ell^2(\Zbb),
        \quad
        (a_n)_{n \in \Zbb} \mapsto (a_{n-1})_{n \in \Zbb}
      \]
      der Rechtsshift-Operator.
      Zeigen Sie, dass $R$ ein Adjungiertes besitzt, und entscheiden Sie, ob $R$ selbstadjungiert, orthogonal, bzw.\ normal ist.
    \item
      Zeigen Sie, dass $R$ keine Eigenwerte besitzt.
    \item
      Es sei
      \[
        S \colon V \to V,
        \quad
        (a_n)_{n \in \Nbb} \mapsto (a_{-n})_{n \in \Nbb}.
      \]
      Zeigen Sie, dass $S$ ein Adjungiertes besitzt, und entscheiden Sie, ob $S$ selbstadjungiert, orthogonal, bzw.\ normal ist.
    \item
      Zeigen Sie, dass $S$ diagonalisierbar ist.
    \item
      Es sei
      \[
        U \coloneqq \{(a_n)_{n \in \Zbb} \in \ell^2(\Zbb) \mid \text{$a_n = 0$ für fast alle $n \in \Zbb$}\}.
      \]
      Bestimmen Sie $U^\perp$ und entscheiden Sie, ob $V = U \oplus U^\perp$.
    \item
      Bestimmen Sie eine Orthonormalbasis von $U$.
  \end{enumerate}
\end{question}


\begin{question}[subtitle = Ein Skalarprodukt auf den reellen Matrizen]{2}
  \begin{enumerate}[leftmargin=*]
    \item
      Zeigen Sie, dass durch
      \[
        \sigma(A, B) \coloneqq \tr\left( A^T B \right)
        \quad
        \text{für alle $A, B \in \Mat_n(\Rbb)$}
      \]
      ein Skalarprodukt auf $\Mat_n(\Rbb)$ definiert wird.
    \item
      Zeigen Sie, dass die Standardbasis $(E_{ij})_{i,j=1,\dotsc,n}$ von $\Mat_n(\Rbb)$ mit
      \[
        (E_{ij})_{kl} \coloneqq \delta_{ik} \delta_{jl}
        \quad
        \text{für alle $1 \leq i,j,k,l \leq n$}
      \]
      eine Orthonormalbasis von $\Mat_n(\Rbb)$ bezüglich $\sigma$ bilden.
      (Der $(i,j)$-te Eintrag von $E_{ij}$ ist also $1$, und alle anderen Einträge sind $0$.)
      
      (\emph{Hinweis}:
       Überlegen sie sich zunächst, dass $E_{ij} E_{kl} = \delta_{jk} E_{il}$ für alle $1 \leq i,j,k,l \leq n$.)
    \item
      Es sei
      \[
        S_+ \coloneqq \{A \in \Mat_n(\Rbb) \mid A^T = A\}
      \]
      der Untervektorraum der symmetrischen Matrizen, und
      \[
        S_- \coloneqq \{A \in \Mat_n(\Rbb) \mid A^T  = -A\}
      \]
      der Untervektorraum der schiefsymmetrischen Matrizen.
      Zeigen Sie, dass
      \[
        \Mat_n(\Rbb) = S_+ \oplus S_-,
      \]
      und dass die Summe orthogonal ist.
  \end{enumerate}
\end{question}


\begin{question}[subtitle = Diagonalisierbarkeit und Selbstadjungiertheit]{2}
  Es sei $V$ ein endlichdimensionale $\Kbb$-Vektorraum und $f \colon V \to V$ ein Endomorphismus.
  \begin{enumerate}[leftmargin=*]
    \item
      Zeigen Sie für denn Fall $\Kbb = \Rbb$, dass $f$ genau dann diagonalisierbar ist, wenn es ein Skalarprodukt auf $V$ gibt, bezüglich dessen $f$ selbstadjungiert ist.
    \item
      Zeigen oder widerlegen Sie die analoge Aussage für $\Kbb = \Cbb$.
  \end{enumerate}
\end{question}


\begin{question}[subtitle = Die Isometriegruppe]{2}
  Es sei $V$ ein Skalarproduktraum und
  \[
    \Orthogonal(V) \coloneqq \{ f \in \GL(V) \mid f f^* = \id \}.
  \]
  \begin{enumerate}[leftmargin=*]
    \item
      Zeigen Sie, dass $\Orthogonal(V)$ eine Untergruppe von $\GL(V)$ bildet.
    \item
      Zeigen Sie für $f \in \End(V)$, dass genau dann $f \in \Orthogonal(V)$, wenn $f$ ein Isomorphismus ist, so dass $\bil{f(v), f(w)} = \bil{v,w}$ für alle $v, w \in V$.
  \end{enumerate}
\end{question}


\begin{question}[subtitle = Charakterisierung von Matrixexponentialen normaler Endomorphismen über Eigenwerte]{2}
  Es sei $V$ ein endlichdimensionaler Skalarproduktraum und $f \colon V \to V$ ein Endomorphismus.
  \newline
  Zeigen Sie die folgenden Äquivalenzen für den Fall $\Kbb = \Cbb$:
  \begin{enumerate}[leftmargin=*]
    \item
      Es gibt genau dann einen normalen Endomorphismus $g \colon V \to V$ mit $f = \exp(g)$, wenn $f$ normal und invertierbar ist.
    \item
      Es gibt genau dann einen antiselbstadjungierten Endomorphismus $g \colon V \to V$ mit $f = \exp(g)$, wenn $f$ unitär ist.
    \item
      Es gibt genau dann einen selbstadjungierten Endomorphismus $g \colon V \to V$ mit $f = \exp(g)$, wenn $f$ selbstadjungiert mit positiven Eigenwerten ist.
  \end{enumerate}
  Zeigen Sie die folgenden Aussagen für den Fall $\Kbb = \Rbb$:
  \begin{enumerate}[leftmargin=*, resume]
    \item
      Es gibt genau dann einen normalen Endomorphismus $g \colon V \to V$ mit $f = \exp(g)$, wenn $f$ normal und invertierbar ist, und alle negativen reellen Eigenwerte von $g$ gerade Vielfachheit haben.
    \item
      Es gibt genau dann einen antiselbstadjungierten Endomorphismus $g \colon V \to V$ mit $f = \exp(g)$, wenn $f$ orthogonal ist und alle negativen reellen Eigenwerte von $f$ gerade Vielfachheit haben.
    \item
      Es gibt genau dann einen selbstadjungierten Endomorphismus $g \colon V \to V$ mit $f = \exp(g)$, wenn $f$ selbstadjungiert mit positiven reellen Eigenwerten ist.
  \end{enumerate}
\end{question}





%%% PRIORITY 3


\begin{question}[subtitle = Die Definitionen unitäre Matrizen]{3}
Zeigen sie, dass für eine Matrix $A \in \Mat_n(\Kbb)$ die folgenden Bedingungen äquivalent sind:
  \begin{enumerate}
    \item
      $A$ ist invertierbar mit $A^{-1} = A^*$.
    \item
      $A A^* = I$.
    \item
      $A^* A = I$.
    \item
      Die Spalten von $A$ bilden eine Orthonormalbasis des $\Kbb^n$.
    \item
      Die Zeilen von $A$ bilden eine Orthonormalbasis des $\Kbb^n$.
  \end{enumerate}
\end{question}


\begin{question}[subtitle = Die Determinanten einiger Matrixgruppen]{3}
  Es sei $\det \colon \Mat_n(\Cbb) \to \Cbb^\times$ die Determinantenabbildung, wobei $\Cbb^\times$ die multiplikative Gruppe des Körpers $\Cbb$ bezeichnet.
  \begin{enumerate}[leftmargin=*]
    \item
      Zeigen Sie, dass $\det$ ein surjektiver Gruppenhomomorphismus ist.
    \item
     Geben Sie den Kern von $\det$ an.
    \item
      Bestimmen Sie das Bild der Einschränkung $\det|_{\GL_n(\Rbb)}$, und geben Sie den Kern an.
    \item
      Bestimmen Sie das Bild der Einschränkung $\det|_{\Unitary(n)}$, und geben Sie den Kern an.
    \item
      Bestimmen Sie das Bild der Einschränkung $\det|_{\Orthogonal(n)}$, und geben Sie den Kern an.
  \end{enumerate}
\end{question}


\begin{question}[subtitle = Darstellende Matrizen von Gram-Schmidt]{3}
  Es sei $V$ ein endlichdimensionaler Skalarproduktraum und $\mc{B} = (b_1, \dotsc, b_n)$ und $\mc{C} = (c_1, \dotsc, c_n)$ seien zwei geordnete Basen von $V$.
  \begin{enumerate}[leftmargin=*]
    \item
      Die Basis $\mc{C}$ entstehen aus $\mc{B}$ durch Anwendung des Gram-Schmidt-Verfahrens.
      Zeigen Sie, dass die Basiswechselmatrix $T_{\mc{C} \to \mc{B}}$ (von $\mc{C}$ nach $\mc{B}$) eine obere Dreiecksmatrix mit positiven reellen Diagonaleinträgen ist.
    \item
      Zeigen oder widerlegen Sie die umgekehrte Aussage:
      Ist die Basiswechselmatrix $T_{\mc{C} \to \mc{B}}$ eine obere Dreiecksmatrix mit positiven reellen Diagonaleinträgen, so ist $\mc{C}$ notwendigerweise die Orthonormalbasis von $V$, die durch Anwendung des Gram-Schmidt-Verfahrens aus $\mc{B}$ entsteht.
    \item
      Entsteht $\mc{C}$ durch Anwendung des Gram-Schmidt-Verfahrens aus $\mc{B}$, so ist die Basiswechselmatrix $T_{\mc{C} \to \mc{B}}$ unitär.
    \item
      Sind $\mc{B}$ und $\mc{C}$ orthonormal, so ist die Basiswechselmatrix $T_{\mc{C} \to \mc{B}}$ unitär.
  \end{enumerate}
\end{question}


\begin{question}[subtitle = Zerlegungen und Normalität komplexer Matrizen]{3}
  Es sei $A \in \Mat_n(\Cbb)$.
  \begin{enumerate}[leftmargin=*]
    \item
      Zeigen Sie, dass es eindeutige hermitsche Matrizen $B, C \in \Mat_n(\Cbb)$ mit $A = B + i C$ gibt.
    \item
      Zeigen Sie, dass $A$ genau dann normal ist, wenn $B$ und $C$ kommutieren.
    \item
      Zeigen Sie, dass es eine eindeutige hermitsche Matrix $D \in \Mat_n(\Cbb)$ und schiefhermitsche Matrix $E \in \Mat_n(\Cbb)$ gibt, so dass $A = D + E$.
    \item
      Zeigen Sie, dass $A$ genau dann normal ist, wenn $D$ und $E$ kommutieren.
    \item
      Wie hängen die Zerlegungen $A = B + i C$ und $A = D + E$ zusammen?
  \end{enumerate}
\end{question}


\begin{question}[subtitle = Ein Skalarprodukt konstruieren]{3}
  Es seien $v_1, v_2 \in \Rbb^3$ definiert als
  \[
    v_1 \coloneqq \vect{1 \\ 1 \\ 0}
    \quad\text{und}\quad
    v_2 \coloneqq \vect{0 \\ 1 \\ 1}.
  \]
  \begin{enumerate}[leftmargin=*]
    \item
      Zeigen Sie, dass es ein Skalarprodukt auf $\Rbb^3$ gibt, bezüglich dessen $(v_1, v_2)$ orthonormal ist.
    \item
      Geben Sie eine Matrix $B \in \Mat_3(\Rbb)$ an, so dass die Bilinearform $\bil{\cdot, \cdot}_B \colon \Rbb^3 \times \Rbb^3 \to \Rbb$ mit
      \[
        \bil{x, y}_B \coloneqq x^T B y
        \quad
        \text{für alle $x, y \in \Rbb^3$}
      \]
      ein solches Skalarprodukt ist.
  \end{enumerate}
\end{question}



\begin{question}[subtitle = Bilder und Kerne adjungierter und normaler Endomorphismen]{3}
  Es sei $V$ ein endlichdimensionaler Skalarproduktsraum und $f \colon V \to V$ ein Endomorphismus.
  \begin{enumerate}[leftmargin=*]
    \item
      Zeigen Sie, dass $\ker f^* \subseteq (\im f)^\perp$.
    \item
      Folgern Sie daraus, dass $\im f^* \subseteq (\ker f)^\perp$.
    \item
      Folgern Sie aus den beiden Inklusionen $\ker f^* \subseteq (\im f)^\perp$ und $\im f^* \subseteq (\ker f)^\perp$ mithilfe der Endlichdimensionalität von $V$, dass bereits Gleichheiten gelten, dass also
      \[
        \ker f^* = (\im f)^\perp
        \quad\text{und}\quad
        \im f^* = (\ker f)^\perp.
      \]
  \end{enumerate}
  Von nun an sei $f$ normal.
  \begin{enumerate}[leftmargin=*, resume]
    \item
      Zeigen Sie, dass $\|f(x)\| = \|f^*(x)\|$ für alle $x \in V$.
    \item
      Folgern Sie, dass $\ker f = \ker f^*$.
    \item
      Folgern Sie damit aus den obigen Gleichheiten, dass $V = \im f \oplus \ker f$ gilt, und dass die Summe orthogonal ist.
      
      (\emph{Hinweis}:
       Zeigen Sie zuerst, dass $\im f$ und $\ker f$ orthogonal sind, und nutzen Sie dann die Endlichdimensionalität von $V$.)
  \end{enumerate}
\end{question}


\begin{question}[subtitle = Spiegelungen]{3}
  Es sei $V$ ein euklidischer Vektorraum.
  Für jedes $\alpha \in V$ mit $\alpha \neq 0$ sei
  \[
    s_\alpha \colon V \to V,
    \quad\text{mit}\quad
              s_\alpha(x)
    \coloneqq x - 2 \frac{\bil{x, \alpha}}{\|\alpha\|^2} \alpha.
  \]
  Ferner seien $L_\alpha \coloneqq \Rbb \alpha$ und $H_\alpha \coloneqq L_\alpha^\perp$.
  \begin{enumerate}[leftmargin=*]
    \item
      Zeigen Sie, dass $L_\alpha = V_{-1}(s_\alpha)$ und $H_\alpha = V_1(s_\alpha)$.
      Folgern Sie, dass $s_\alpha$ diagonalisierbar ist.
    \item
      Interpretieren Sie $V$ geometrisch anschaulich.
    \item
      Zeigen Sie, dass $s_\alpha^2 = \id_V$, und dass $s_{\lambda \alpha} = s_\alpha$, $L_{\lambda \alpha} = L_\alpha$ und $H_{\lambda \alpha} = H_\alpha$ für alle $\lambda \in \Rbb^\times$.
    \item
      Es sei $s' \colon V \to V$ ein Endomorphismus mit $s'(\alpha) = -\alpha$ und $s'(x) = x$ für alle $x \in H_\alpha$.
      Zeigen Sie, dass bereits $s' = s_\alpha$ gilt.
    \item
      Es sei $t \colon V \to V$ ein orthogonaler Isomorphismus. Zeigen Sie, dass $t s_\alpha t^{-1} = s_{t(\alpha)}$.
  \end{enumerate}
\end{question}


\begin{question}[subtitle = Konstruktion der adjungierten Abbildung]{3}
  Es seien $V$ und $W$ zwei endlichdimensionale euklidische Vektorräume.
  Ferner sei $f \colon V \to W$ eine $\Rbb$-lineare Abbildung.
  \begin{enumerate}[leftmargin=*]
    \item
      Zeigen Sie, dass die Abbildung
      \[
        \Phi_V \colon V \to V^*,
        \quad
        v \mapsto \bil{-, v}
      \]
      ein $\Rbb$-linearer Isomorphismus ist.
    \item
      Geben Sie die Definition der dualen Abbildung $f^* \colon W^* \to V^*$ an.
      Zeigen Sie, dass $f^*$ $\Rbb$-linear ist.
    \item
      Zeigen Sie, dass die Abbildung $g \coloneqq \Phi_V^{-1} \circ f^* \circ \Phi_W$ $\Rbb$-linear ist, und dass
      \[
        \bil{f(v), w} = \bil{v, g(w)}
        \quad
        \text{für alle $v \in V$, $w \in W$}.
      \]
    \item
      Zeigen Sie:
      Eine Basis $\mc{B} = (v_1, \dotsc, v_n)$ von $V$ ist genau dann eine Orthonormalbasis, wenn die Basis $\Phi_V(\mc{B}) = (\Phi_V(v_1), \dotsc, \Phi_V(v_n))$ von $V^*$ die duale Basis $\mc{B}^*$ ist.
    \item
      Inwiefern ändern sich die obigen Resultate für denn Fall $\Kbb = \Cbb$, wenn also $V$ und $W$ endlichdimensionale unitäre Vektorräume sind?
  \end{enumerate}
\end{question}


\begin{question}[subtitle = Zusammenhang zwischen Skalarproduktraum und Dualraum]{3}
  Es seien $V$ und $W$ zwei endlichdimensionale euklidische Vektorräume.
  Für jeden Untervektorraum $U \subseteq V$ sei
  \[
              U^\perp
    \coloneqq \{ v \in V \mid \text{$\bil{u, v} = 0$ für alle $u \in U$} \}
  \]
  das orthogonale Komplement von $U$, und
  \[
              U^\circ
    \coloneqq \{ \varphi \in V^* \mid \text{$\varphi(u) = 0$ für alle $u \in U$} \}
  \]
  der Annihilator von $U$.
  Es sei $f \colon V \to W$ eine lineare Abbildung.
  Es sei $f^* \colon W \to V$ die zu $f$ adjungierte Abbildung, und
  \[
    f^T \colon W^* \to V^*,
    \quad
    \varphi \mapsto \varphi \circ f
  \]
  die zu $f$ duale Abbildung.
  \begin{enumerate}[leftmargin=*]
    \item
      Zeigen Sie, dass die Abbildung
      \[
        \Phi_V \colon V \to V^*,
        \quad
        v \mapsto \bil{-, v}
      \]
      ein Isomorphismus ist.
    \item
      Zeigen Sie, dass für jeden Untervektorraum $U \subseteq V$ die Gleichheit $\Phi_V(U^\perp) = U^\circ$ gilt.
    \item
      Zeigen Sie, dass $f^T \circ \Phi_W = \Phi_V \circ f^*$, dass also das folgende Diagramm kommutiert:
      \[
        \begin{tikzcd}[row sep = large, column sep = large, ampersand replacement = \&]
                V   \arrow[swap]{d}{\Phi_V}
            \&  W   \arrow[swap]{l}{f^*}
                    \arrow{d}{\Phi_W}
          \\
                V^* 
            \&  W^* \arrow{l}{f^T}
        \end{tikzcd}
      \]
      Folgern Sie, dass $f^* = \Phi_V^{-1} \circ f^T \circ \Phi_W$.
  \end{enumerate}
  In Lineare Algebra I wurde gezeigt, dass
  \[
      \ker f^T
    = (\im f)^\circ
    \quad\text{und}\quad
      \im f^T
    = (\ker f )^\circ,
  \]
  und dass für je zwei Untervektorräume $U_1, U_2 \subseteq V$ die Gleichheiten
  \[
      (U_1 + U_2)^\circ
    = U_1^\circ \cap U_2^\circ
    \quad\text{und}\quad
      (U_1 \cap U_2)^\circ
    = U_1^\circ + U_2^\circ
  \]
  gelten.
  \begin{enumerate}[leftmargin=*, resume]
    \item
      Folgern Sie aus den vorherigen Aufgabenteilen und den Aussagen aus Lineare Algebra I für alle Untervektorräume $U_1, U_2 \subseteq V$ die Gleichheiten
      \[
          (U_1 + U_2)^\perp
        = U_1^\perp \cap U_2^\perp
        \quad\text{und}\quad
          (U_1 \cap U_2)^\perp
        = U_1^\perp + U_2^\perp.
      \]
      
      (\emph{Hinweis}:
       Nutzen Sie, dass $\Phi_V$ ein Isomorphismus ist.)
    \item
      Folgern Sie aus den vorherigen Aufgabenteilen und den Aussagen aus Lineare Algebra I, dass
      \[
          \ker f^*
        = (\im f)^\perp
        \quad\text{und}\quad
          \im f^*
        = (\ker f)^\perp.
      \]
      
      (\emph{Hinweis}:
       Nutzen Sie, dass $\Phi_V$ und $\Phi_W$ Isomorphismen sind.)
  \end{enumerate}
\end{question}


\begin{question}[subtitle = Eine Anwendung des Rieszschen Darstellungssatzes]{3}
  Es sei $V$ ein endlichdimensionaler euklidischer Vektorraum.
  \begin{enumerate}[leftmargin=*]
    \item
      Zeigen Sie, dass die Abbildung
      \[
        \Phi \colon V \to V^*,
        \quad
        v \mapsto \bil{-,v}
      \]
      ein Isomorphismus ist.
    \item
      Zeigen Sie:
      Eine Basis $\mc{B} = (v_1, \dotsc, v_n)$ von $V$ ist genau dann orthonormal, wenn
      \[
          \varphi
        = \sum_{i=1}^n \varphi(v_i) \bil{-,v_i}
        \quad
        \text{für alle $\varphi \in V^*$}.
      \]
  \end{enumerate}
  Es sei nun $V$ der unendlichdimensionale Vektorraum der Polynomsfunktionen $\Rbb \to \Rbb$, und für alle $n \in \Nbb$ sei $V_n \subseteq V$ der endlichdimensionale Untervektorraum der Polynomsfunktionen vom Grad $\leq n$.
  Für $a \in \Rbb$ sei
  \[
    \varphi_a \colon V \to \Rbb
    \quad\text{mit}\quad
    \varphi_a(f) = f(a)
    \quad
    \text{für alle $a \in \Rbb$}
  \]
  die Auswertung an $a$.
  \begin{enumerate}[leftmargin=*, resume]
    \item
      Zeigen Sie, dass
      \[
        \bil{f, g} \coloneqq \int_{-1}^1 f(t)g(t) \dd{t}
        \quad
        \text{für alle $f, g \in V$}
      \]
      ein Skalarprodukt auf $V$ definiert.
    \item
      Zeigen Sie, dass es für alle $n \in \Nbb$ und $a \in \Rbb$ und eine eindeutige Funktion $g_{n,a} \in V$ gibt, so dass
      \[
          f(a)
        = \int_{-1}^1 f(t) g_{n,a}(t) \dd{t}
        \quad
        \text{für alle $f \in V_n$}.
      \]
    \item
      Bestimmen Sie eine Orthonormalbasis von $V_2$.
    \item
      Bestimmen Sie $g_{2,a}$ in Abhängigkeit von $a$.
  \end{enumerate}
\end{question}


\begin{question}[subtitle = Invariante Skalarprodukte]{3}
  Es sei $V$ ein endlichdimensionaler euklidischer Vektorraum mit Skalarprodukt $\bil{\cdot, \cdot}$, und $G \subseteq \GL(V)$ sei eine endliche Untergruppe.
  \begin{enumerate}[leftmargin=*]
    \item
      Zeigen Sie, dass
      \[
        \bil{x,y}_G \coloneqq \frac{1}{|G|} \sum_{\phi \in G} \bil{\phi(x), \phi(y)}
        \quad
        \text{für alle $x, y \in V$}
      \]
      ein Skalarprodukt auf $V$ definiert.
    \item
      Zeigen Sie, dass $\bil{\cdot, \cdot}_G$ in dem Sinne $G$-invariant ist, dass
      \[
        \bil{\psi(x), \psi(y)}_G = \bil{x,y}_G
        \quad
        \text{für alle $x, y \in V$ und $\psi \in G$}.
      \]
      
      (\emph{Hinweis}:
       Beachten Sie, dass die Multiplikation $G \to G$, $h \mapsto hg$ für alle $g \in G$ bijektiv ist.)
    \item
      Zeigen Sie auch, dass $\bil{\cdot, \cdot}_G = \bil{\cdot, \cdot}$, wenn $\bil{\cdot, \cdot}$ bereits $G$-invariant ist.
    \item
      Folgern Sie, dass es eine Basis $\mc{B}$ von $V$ gibt, so dass $M_\mc{B}(\psi)$ für alle $\psi \in G$ eine orthogonale Matrix ist.
    \item
      Folgern Sie damit, dass es für $n = \dim V$ einen injektiven Gruppenhomomorphismus $\Phi \colon G \to \Orthogonal(n)$ gibt, $G$ also isomorph zu der Untergruppe $\im \Phi$ von $\Orthogonal(n)$ ist.
  \end{enumerate}
\end{question}


\begin{question}[subtitle = Implikationen zwischen verschiedene Aussagen]{3}
  Es sei $V$ ein endlichdimensionaler euklidischer Vektorraum und $f \colon V \to V$ ein Endomorphismus.
  Entscheiden Sie, welche der folgenden Aussagen sich implizieren.
  \begin{enumerate}
    \item
      Der Endomorphismus $f$ ist selbstadjungiert mit positiven Eigenwerten.
    \item
      Der Endomorphismus $f$ ist orthogonal, und alle Eigenwerte von $f$ sind positiv.
    \item
      Der Endomorphismus $f$ ist normal mit $\det f > 0$.
    \item
      Es gibt einen selbstadjungierten Endomorphismus $g \colon V \to V$ mit $f = \exp(g)$.
    \item
      Der Endomorphismus $f$ ist selbstadjungiert und orthogonal.
  \end{enumerate}
\end{question}


\begin{question}[subtitle = Rotationsgruppen]{3}
  Zeigen Sie, dass die drei Gruppen $\SOrthogonal(2)$, $S^1$ und $\Unitary(1)$ isomorph sind.
  (Dabei ist $S^1 = \{z \in \Cbb \mid |z| = 1\}$ eine Untergruppe von $\Cbb^\times$.)
\end{question}


\begin{question}[subtitle = Permutationsmatrizen sind orthogonal]{3}
  Es sei $\pi \in S_n$ eine Permutation und $P_\pi \colon \Rbb^n \to \Rbb^n$ die eindeutige lineare Abbildung mit
  \[
    P_\pi(e_i) = e_{\pi(i)}
    \quad
    \text{für alle $i = 1, \dotsc, n$},
  \]
  wobei $(e_1, \dotsc, e_n)$ die Standardbasis von $\Rbb^n$ ist.
  \begin{enumerate}[leftmargin=*]
    \item
      Zeigen Sie, dass $P_\pi$ orthogonal ist.
    \item
      Bestimmen Sie die möglichen Eigenwerte von $P_\pi$.
    \item
      Geben Sie ein Beispiel an, bei dem alle möglichen Eigenwerte auftreten.
  \end{enumerate}
\end{question}


\begin{question}[subtitle = Linearität orthogonaler Abbildungen]{3}
  Es seien $V$ und $W$ euklidische Vektorräume, und $f \colon V \to W$ sei eine surjektive Funktion mit
  \[
    \bil{f(v_1), f(v_2)} = \bil{v_1, v_2}
    \quad
    \text{für alle $v_1, v_2 \in V$}.
  \]
  \begin{enumerate}[leftmargin=*]
    \item
      Zeigen Sie, dass $f$ linear ist.
    \item
      Zeigen Sie, dass $f$ bereits ein Isomorphismus ist.
  \end{enumerate}
\end{question}


\begin{question}[subtitle = Die Adjungierte Abblidung als Polynom]{3}
  \begin{enumerate}[leftmargin=*]
    \item
      Es seien $z_1, \dotsc, z_n \in \Cbb$ paarweise verschieden Punkte.
      Zeigen Sie, dass es für beliebige Werte $w_1, \dotsc, w_n \in \Cbb$ ein Polynom $P \in \Cbb[T]$ mit $P(z_j) = w_j$ für alle $j = 1, \dotsc, n$ gibt.
    \item
      Es sei $f \colon V \to V$ ein normaler Endomorphismus eines endlichdimensionalen unitären Vektorraums $V$.
      Zeigen Sie, dass es ein Polynom $P \in \Cbb[T]$ mit $f^* = P(f)$ gibt.
      
      (\emph{Hinweis}:
       Nutzen Sie, dass $f$ diagonalisierbar ist.)
  \end{enumerate}
\end{question}





%%% PRIORITY 4


\begin{question}[subtitle = Ein Skalarprodukt auf dem Endomorphismenraum]{4}
  Es sei $V$ ein endlichdimensionaler Skalarproduktraum über $\Kbb$.
  Für den $\Kbb$-Vektorraum $\End_\Kbb(V)$ sei
  \[
    S \coloneqq \{ f \in \End_\Kbb(V) \mid f^* = f \}
  \]
  der Untervektorraum der selbstadjungierten Endomorphismen und
  \[
    A \coloneqq \{ f \in \End_\Kbb(V) \mid f^* = -f \}
  \]
  der Untervektorraum der antiselbstadjungierten Endomorphismen.
  \begin{enumerate}[leftmargin=*]
    \item
      Zeigen Sie, dass $\langle f, g \rangle \coloneqq \tr(f \circ g^*)$ ein Skalarprodukt auf $\End_\Kbb(V)$ definiert.
    \item
      Folgern Sie, dass $|\tr(f \circ g^*)| \leq \sqrt{|\tr(f \circ f^*) \tr(g \circ g^*)|}$ für alle $f, g \in \End_\Kbb(V)$.
    \item
      Zeigen Sie, dass $\End_\Kbb(V) = S \oplus A$, und dass die Summe orthogonal ist.
  \end{enumerate}
\end{question}


\begin{question}[subtitle = Charakterisierungen normaler Endomorphismen für unitäre Vektorräume]{4}
  Es sei $V$ ein endlichdimensionaler unitärer Vektorraum.
  Zeigen Sie, dass für eine lineare Abbildung $S \colon V \to V$ die folgenden Bedingungen äquivalent sind:
  \begin{enumerate}
    \item
      Der Endomorphismus $S$ ist normal.
    \item
      Der Vektorraum $V$ hat eine Orthonormalbasis aus Eigenvektoren von $S$.
    \item
      Für jeden $S$-invarianten Untervektorraum $U \subseteq V$ ist auch das orthogonale Komplement $U^\perp$ invariant unter $S$.
  \end{enumerate}
\end{question}


\begin{question}[subtitle = Eine Charakterisierung von $\SUnitary(2)$]{4}
  Es sei
  \[
    \Phi \colon \SUnitary(2) \to S^3,
    \quad
    \begin{pmatrix}
      a & b \\
      c & d
    \end{pmatrix}
    \mapsto
    \vect{a \\ c}
  \]
  die Abbildung auf die erste Spalte, wobei
  \[
              S^3
    \coloneqq \left\{ \vect{z_1 \\ z_2} \in \Cbb^2 \,\middle|\, |z_1|^2 + |z_2|^2 = 1 \right\}
    =         \left\{ x \in \Cbb^2 \mid \|x\| = 1 \right\}
  \]
  \begin{enumerate}[leftmargin=*]
    \item
      Zeigen Sie, dass $\Phi$ wohldefiniert ist.
    \item
      Zeigen Sie, dass $\Phi$ bijektiv ist.
  \end{enumerate}
\end{question}
