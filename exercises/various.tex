\section{Verschiedenes}










\subsection{Allgemeines Zeugs}





% PRIORITY 1


\begin{question}[subtitle = Invertierbarkeit im Endlichdimensionalen]{1}
  Es sei $V$ ein $K$-Vektorraum und $f, g \colon V \to V$ seien zwei Endomorphismen.
  \begin{enumerate}[leftmargin=*]
    \item
      Es sei $f \circ g = \id_V$ und $V$ sei endlichdimensional.
      Zeigen Sie, dass auch $g \circ f = \id_V$.
    \item
      Zeigen Sie, dass die Aussage nicht mehr notwendigerweise gilt, wenn $V$ unendlichdimensional ist.
  \end{enumerate}
\end{question}


\begin{question}[subtitle = Dimensionsformel]{1}
  Es seien $V$ und $W$ zwei $K$-Vektorräume, so dass $V$ endlichdimensional ist, und $f \colon V \to W$ sei eine lineare Abbildung.
  Zeigen Sie die Dimensionsformel
  \[
    \dim V = \dim \ker f + \dim \im f.
  \]
\end{question}


\begin{question}[subtitle = Operationen mit invarianten Unterräumen]{1}
  Es sei $V$ ein Vektorraum und $f \colon V \to V$ ein Endomorphismus.
  Es sei $(U_i)_{i \in I}$ eine Familie von $f$-invarianten Untervektorräumen, und $U \subseteq V$ ein $f$-invarianter Untervektorraum.
  Zeigen Sie:
  \begin{enumerate}[leftmargin=*]
    \item
      Auch der Schnitt $\bigcap_{i \in I} U_i$ ist $f$-invariant.
    \item
      Auch die Summe $\sum_{i \in i} U_i$ ist $f$-invariant.
  \end{enumerate}
\end{question}





% PRIORITY 2


\begin{question}[subtitle = Konjugationsinvarianz der Spur]{2}
  Es sei $K$ ein Körper.
  \begin{enumerate}[leftmargin=*]
    \item
      Zeigen Sie, dass für alle $A, B \in \Mat_n(K)$ die Gleichheit $\tr(AB) = \tr(BA)$ gilt.
    \item
      Folgern Sie, dass die Spur invariant unter Konjugation ist, d.h.\ dass
      \[
        \tr(S A S^{-1}) = \tr(A)
        \quad
        \text{für alle $A \in \Mat_n(K)$ und $S \in \GL_n(K)$}.
      \]
  \end{enumerate}
\end{question}


\begin{question}[subtitle = Der Satz von Cayley-Hamilton]{2}
  \begin{enumerate}[leftmargin=*]
    \item
      Formulieren Sie den Satz von Cayley-Hamilton.
    \item
      Zeigen Sie den Satz für ($2 \times 2$)-Matrizen durch explizites Nachrechnen.
    \item
      Zeigen Sie den Satz für Diagonalmatrizen.
    \item
      Folgern Sie den Satz für diagonalisierbare Matrizen.
  \end{enumerate}
\end{question}


\begin{question}[subtitle = Invertieren durch das charakteristische Polynom]{2}
  Es sei $A \in \GL_n(K)$ und $\chi_A(T)$ das charakteristische Polynom von $A$.
  \begin{enumerate}[leftmargin=*]
    \item
      Zeigen Sie, dass der konstante Term von $\chi_A(T)$ nicht verschwindet.
    \item
      Zeigen Sie, dass es ein Polynom $P \in K[T]$ gibt, so dass $A^{-1} = P(A)$.
  \end{enumerate}
\end{question}





% PRIORITY 3


\begin{question}[subtitle = Unterscheidung zwischen nilpotenten und lokal nilpotenten Endomorphismen]{3}
  Ein Endomorphismus $f \colon V \to V$ eines $K$-Vektorraums $V$ heißt \emph{lokal nilpotent}, falls es für jedes $v \in V$ ein $n \in \Nbb$ mit $f^n(v) = 0$ gibt.
  \begin{enumerate}[leftmargin=*]
    \item
      Zeigen Sie, dass jeder nilpotente Endomorphismus auch lokal nilpotent ist.
    \item
      Zeige Sie, dass $0$ der einzige mögliche Eigenwert eines lokal nilpotenten Endomorphismus ist.
    \item
      Geben Sie ein Beispiel für einen Vektorraum $V$ und einen Endomorphismus $f \colon V \to V$ an, so dass $f$ zwar lokal nilpotent, nicht aber nilpotent ist.
    \item
      Zeigen Sie, dass jeder lokal nilpotente Endomorphismus eines endlichdimensionalen Vektorraums bereits nilpotent ist.
  \end{enumerate}
\end{question}


\begin{question}[subtitle = Zur Unterscheidung von Polynomen und Polynomsfunktionen]{3}
  Es sei $K$ ein endlicher Körper.
  \begin{enumerate}[leftmargin=*]
    \item
      Geben Sie ein Polynom $p \in K[X]$ an, so dass zwar $p \neq 0$ aber $p(\lambda) = 0$ für alle $\lambda \in K$.
    \item
      Geben Sie ein Polynom $p \in K[X]$ an, so dass zwar $\deg p \geq 1$, aber $p(\lambda) = 1$ für alle $\lambda \in K$.
    \item
      Folgern Sie, dass es keine algebraisch abgeschlossenen endlichen Körper gibt.
  \end{enumerate}
\end{question}


\begin{question}[subtitle = Auf- und absteigende Ketten von Bild und Kern]{3}
  Es sei $V$ ein $K$-Vektorraum und $f \colon V \to V$ ein Endomorphismus.
  Für alle $k \in \Nbb$ sei
  \[
    R_k \coloneqq \im f^k
    \quad\text{und}\quad
    N_k \coloneqq \ker f^k.
  \]
  \begin{enumerate}[leftmargin=*]
    \item
      Zeigen Sie, dass $R_0 = V$, und dass $R_i \supseteq R_{i+1}$ für alle $i \in \Nbb$.
      Es gibt also eine absteigende Kette
      \[
        V = R_0 \supseteq R_1 \supseteq R_2 \supseteq R_3 \supseteq R_4 \supseteq \dotsb
      \]
      von Untervektorräumen.
    \item
      Zeigen Sie, dass für $i \in \Nbb$ mit $R_i = R_{i+1}$ auch $R_{i+1} = R_{i+2}$ gilt.
    \item
      Folgern Sie:
      Gilt in der obigen absteigenden Kette einmal Gleichheit, also $R_i = R_{i+1}$ für ein $i \in \Nbb$, so stabilisiert die Kette bereits, d.h.\ es gilt $R_j = R_i$ für alle $j \geq i$.
    \item
      Zeigen Sie, dass $N_0 = 0$, und dass $N_i \subseteq N_{i+1}$ für alle $i \in \Nbb$.
      Es gibt also eine aufsteigende Kette
      \[
        0 = N_0 \subseteq N_1 \subseteq N_2 \subseteq N_3 \subseteq N_4 \subseteq \dotsb
      \]
      von Untervektorräumen.
    \item
      Zeigen Sie, dass für $i \in \Nbb$ mit $N_i = N_{i+1}$ auch $N_{i+1} = N_{i+2}$ gilt.
    \item
      Folgern Sie:
      Gilt in der obigen aufsteigende Kette einmal Gleichheit, also $N_i = N_{i+1}$ für ein $i \in \Nbb$, so stabilisiert die Kette bereits, d.h.\ es gilt $N_j = N_i$ für alle $j \geq i$.
    \item
      Folgern Sie:
      Ist $V$ endlichdimensional, so stabilisieren beide Ketten.
  \end{enumerate}
\end{question}


\begin{question}[subtitle = Algebraische Endomorphismen]{3}
  Ein Endomorphismus $f \colon V \to V$ eines $K$-Vektorraums $V$ heißt \emph{algebraisch (über $K$)}, falls es ein Polynom $P \in K[T]$ mit $P \neq 0$ gibt, so dass $P(f) = 0$ gilt.
  \begin{enumerate}[leftmargin=*]
    \item
      Zeigen Sie, dass jeder Endomorphismus eines endlichdimensionalen Vektorraums algebraisch ist.
    \item
      Geben Sie ein Beispiel für einen $K$-Vektorraum $V$ und einen Endomorphismus $f \colon V \to V$, der nicht algebraisch ist.
    \item
      Entscheiden Sie, ob die lineare Abbildung $K[X] \to K[X]$, $p \mapsto X \cdot p$ algebraisch ist.
    \item
      Zeigen Sie, dass ein diagonalisierbarer Endomorphismus genau dann algebraisch ist, wenn er nur endlich viele Eigenwerte hat.
  \end{enumerate}
\end{question}


\begin{question}[subtitle = Einschränkung des Inversen]{3}
  Es sei $V$ ein $K$-Vektorraum, $f \colon V \to V$ ein Automorphismus und $U \subseteq V$ ein $f$-invarianter Untervektorraum.
  \begin{enumerate}[leftmargin=*]
    \item
      Zeigen Sie:
      Ist $U$ endlichdimensional, so ist $U$ auch invariant unter $f^{-1}$.
    \item
      Zeigen Sie, dass die Aussage nicht gelten muss, falls $U$ unendlichdimensional ist.
  \end{enumerate}
\end{question}





% PRIORITY 4


\begin{question}[subtitle = Das Zentrum des Matrizenrings]{4}
  Das \emph{Zentrum} eines Rings $R$ ist
  \[
    Z(R) \coloneqq \{ r \in R \mid \text{$rs = sr$ für alle $s \in R$} \}.
  \]
  Man bemerke, dass $R$ genau dann kommutativ ist, wenn $Z(R) = R$.
  Im Folgenden wird das Zentrum des Matrizenrings $\Mat_n(K)$ bestimmt.
  Hierfür sei
  \[
              \Diagonal_n(K)
    \coloneqq K I
    =         \{\lambda I \mid \lambda \in K\}
  \]
  der Untervektorraum der Skalarmatrizen.
  \begin{enumerate}[leftmargin=*]
    \item
      Zeigen Sie, dass $D_n(K) \subseteq Z(\Mat_n(K))$.
    \item
      Zeigen Sie für $A \in Z(\Mat_n(K))$, dass $A$ eine Diagonalmatrix ist.
      
      (\emph{Hinweis}:
       Betrachten Sie die Matrizen $E_{ii}$ für $1 \leq i \leq n$.)
    \item
      Zeigen Sie ferner, dass alle Diagonaleinträge von $A$ bereits gleich sind.
      
      (\emph{Hinweis}:
       Betrachten Sie die Matrizen $E_{ij}$ mit $1 \leq i,j \leq n$.)
    \item
      Folgern Sie, dass $Z(\Mat_n(K)) = D_n(K)$.
  \end{enumerate}
\end{question}










\subsection{Diagonalisierbarkeit und Eigenzeugs}


\begin{question}[subtitle = Multiple Choice zur Kombination diagonalisierbarer Endomorphismen]{2}
  Es seien $f, g \colon V \to V$ zwei Endomorphismen eines $K$-Vektorraums $V$.
  Entscheiden sie für die folgenden Aussagen jeweils, ob diese allgemein gültig sind.
  Geben Sie, sofern möglich, auch ein Gegenbeispiel an.
  \begin{enumerate}[leftmargin=*]
    \item
      Sind $f$ und $g$ diagonalisierbar, so ist auch $f \circ g$ diagonalisierbar.
    \item
      Kommutieren $f$ und $g$ und ist $f \circ g$ diagonalisierbar, so ist $f$ oder $g$ diagonalisierbar.
    \item
      Sind $f$ und $g$ diagonalisierbar, so ist auch $f + g$ diagonalisierbar.
    \item
      Falls $f$ und $g$ kommutieren und diagonalisierbar sind, so ist $f \circ g$ invertierbar.
    \item
      Falls $f$ und $g$ kommutieren und diagonalisierbar sind, so ist auch $f + g$ diagonalisierbar.
    \item
      Ist $f$ diagonalisierbar, so ist für jedes $p \in K[X]$ auch $p(f)$ diagonalisierbar.
    \item
      Falls $f$ und $g$ kommutieren und diagonalisierbar sind, so folgt, wenn $g$ invertierbar ist, dass $ \circ g^{-1}$ diagonalisierbar ist.
  \end{enumerate}
\end{question}

\begin{solution}
  \begin{enumerate}
    \item
      Nein, braucht etwa simultan diagonalisierbar.
    \item
      Nein.
    \item
      Nein, braucht etwa simultan diagonalisierbar.
    \item
      Ja, da simultan diagonalisierbar.
    \item
      Ja, da simultan diagonalsierbar.
    \item
      Ja.
  \end{enumerate}
\end{solution}


\begin{question}[subtitle = Einschränkung diagonalisierbarer Endomorphismen]{2}
  Es sei $V$ ein $K$-Vektorraum und $f \colon V \to V$ ein diagonalisierbarer Endomorphismus von $V$ (d.h.\ es gilt \mbox{$V = \bigoplus_{\lambda \in K} V_\lambda(f)$}).
  Zeigen Sie, dass für jeden $f$-invarianten Untervektorraum $U \subseteq V$ die Einschränkung $f|_U \colon U \to U$ diagonalisierbar ist, und dass $U_\lambda(f|_U) = U \cap V_\lambda(f)$ für alle $\lambda \in K$.
\end{question}


\begin{question}[subtitle = Zur Existenz gemeinsamer Eigenvektoren]{3}
  Es sei $V \neq 0$ ein $K$-Vektorraum, wobei $K$ algebraisch abgeschlossen ist.
  Es seien $f_1, \dotsc, f_n \colon V \to V$ paarweise kommutierende Endomorphismen.
  \begin{enumerate}[leftmargin=*]
    \item
      Zeigen Sie, dass für alle $I \subseteq \{1, \dotsc, n\}$ und Skalare $\lambda_i \in K$ mit $i \in I$ der \emph{gemeinsame Eigenraum}
      \[
                   V( (f_i, \lambda_i)_{i \in I} )
        \coloneqq  \{ v \in V \mid \text{$f_i(v) = \lambda_i v$ für alle $i \in I$} \}.
      \]
      invariant unter $f_1, \dotsc, f_n$ ist.
     \item
      Folgern Sie, dass die Endomorphismen $f_1, \dotsc, f_n$ einen gemeinsamen Eigenvektor besitzen, d.h.\ dass es einen Vektor $v \in V$ gibt, so dass $v$ für jedes $f_i$ eine Eigenvektor ist.
      
      (\emph{Hinweis}: Konstruieren sie induktiv $\lambda_1, \dotsc, \lambda_n \in K$, so dass $V((f_1, \lambda_1), \dotsc, (f_i, \lambda_i)) \neq 0$ für alle $i = 1, \dotsc, n$.)
  \end{enumerate}
\end{question}


\begin{question}[subtitle = Simultane Diagonalisierbarkeit]{2}
  Es sei $V$ ein $K$-Vektorraum.
  Für alle Endomorphismen $f_1, \dotsc, f_n \colon V  \to V$ und Skalare (Eigenwerte) \mbox{$\lambda_1, \dotsc, \lambda_n \in K$} sei
  \[
              V(f_1, \lambda_1; \dotsc; f_n, \lambda_n)
    \coloneqq \{ v \in V \mid \text{$f_i(v) = \lambda_i v$ für alle $i = 1, \dotsc, n$} \}
  \]
  der \emph{gemeinsame Eigenraum} der Endomorphismen $f_1, \dotsc, f_n$ zu den Eigenwerten $\lambda_1, \dotsc, \lambda_n$.
  \begin{enumerate}[leftmargin=*]
    \item
      Zeigen Sie, dass
      \[
          V(f_1, \lambda_1; \dotsc; f_n, \lambda_n)
        = \bigcap_{i=1}^n V(f_i, \lambda_i)
      \]
      für alle Endomorphismen $f_1, \dotsc, f_n \in \End(V)$ und Eigenwerte $\lambda_1, \dotsc, \lambda_n \in K$.
    \item
      Es seien $f_1, \dotsc, f_n, g \in \End(V)$ Endomorphismen, so dass $g$ mit jedem $f_i$ kommutiert.
      Zeigen sie, dass der gemeinsame Eigenraum $V(f_1, \lambda_1; \dots; f_n, \lambda_n)$ für alle $\lambda_1, \dotsc, \lambda_n \in K$ invariant unter $g$ ist.
    \item
      Zeigen Sie: Sind die Endomorphismen $f_1, \dotsc, f_n \colon V \to V$ diagonalisierbar (d.h.\ für alle $i = 1, \dotsc, n$ ist $V = \bigoplus_{\lambda \in K} V(f_i, \lambda)$ ) und paarweise kommutierend, so sind die Endomorphismen \emph{simultan diagonalisierbar}, d.h.\ es ist
      \[
          V
        = \bigoplus_{\lambda_1, \dotsc, \lambda_n \in K}  V(f_1, \lambda_1; \dotsc; f_n, \lambda_n).
      \]
    \item
      Zeigen Sie, dass auch die Umkehrung gilt:
      Sind Endomorphismen $f_1, \dotsc, f_n \colon V \to V$ simultan diagonalisierbar, so sind $f_1, \dotsc, f_n$ diagonalisierbar und kommutieren.
  \end{enumerate}
  Von nun an sei $V$ endlichdimensional.
  \begin{enumerate}[resume]
    \item
      Zeigen Sie, dass Endomorphismen $f_1, \dotsc, f_n \colon V \to V$ genau dann simultan diagonalisierbar sind, wenn es eine geordnete Basis $\mc{B}$ von $V$ gibt, so dass $\Mat_\mc{B}(f_i)$ für jedes $i = 1, \dotsc, n$ in Diagonalgestalt ist.
    \item
      Es sei nun $H \subseteq \End(V)$ ein Untervektorraum aus diagonalisierbaren und paarweise kommutierenden Endomorphismen.
      Zeigen Sie, dass es eine Basis $\mc{B}$ von $V$ gibt, so dass $\Mat_\mc{B}(f)$ für jedes $f \in H$ eine Diagonalmatrix ist.
      
      (\emph{Hinweis}:
       Nutzen Sie, dass $H$ endlichdimensional ist.)
  \end{enumerate}
\end{question}


\begin{question}[subtitle = Eine Knobelaufgabe]{3}
  Es sei $f \colon V \to V$ ein Endomorphismus eines $n$-dimensionalen $K$-Vektorraums $V$ und $\{ v_1, \dotsc, v_{n+1} \} \subseteq V$ eine Teilmenge aus Eigenvektoren von $f$, so dass jede $n$-elementige Teilmenge linear unabhängig ist.
  Zeigen Sie, dass $f$ bereits ein skalares Vielfaches der Identität ist.
\end{question}


\begin{question}[subtitle = Diagonalisierbarkeit und Nilpotenz von $\ad_X$]{3}
  Für jede Matrix $X \in \Mat_n(K)$ sei
  \[
    \lambda_X \colon \Mat_n(K) \to \Mat_n(K),
    \quad
    A \mapsto XA
  \]
  die Linksmultiplikation mit $X$,
  \[
    \rho_X \colon \Mat_n(K) \to \Mat_n(K),
    \quad
    A \mapsto AX
  \]
  die Rechtsmultiplikation mit $X$, und
  \[
    \ad_X = [X, -] \colon \Mat_n(K) \to \Mat_n(K),
    \quad
    A \mapsto [X, A] = XA - AX
  \]
  der Kommutator mit $X$.
  \begin{enumerate}[leftmargin=*]
    \item
      Zeigen Sie:
      Ist $X$ nilpotent, so sind auch $\lambda_X$ und $\rho_X$ nilpotent.
    \item
      Folgern Sie:
      Ist $X$ nilpotent, so ist auch $\ad_X$ nilpotent.
      
      (\emph{Hinweis}:
      Nutzen Sie, dass $\ad_X = \lambda_X - \rho_X$.)
    \item
      Zeigen Sie:
      Ist $X$ eine Diagonalmatrix, so sind $\lambda_X$ und $\rho_X$ diagonalisierbar.
    \item
      Folgern Sie:
      Ist $X$ diagonalisierbar, so sind auch $\lambda_X$ und $\rho_X$ diagonalisierbar.
    \item
      Folgern Sie:
      Ist $X$ diagonalisierbar, so ist auch $\ad_X$ diagonalisierbar.
      
      (\emph{Hinweis}:
       Nutzen Sie, dass $\ad_X = \lambda_X - \rho_X$.)
  \end{enumerate}
\end{question}


\begin{question}[subtitle = Shiften von Eigenräumen]{3}
  Es seien $E$ und $H$ zwei Endomorphismen eines $\Cbb$-Vektorraums $V$, so dass $[H,E] = 2E$.
  \begin{enumerate}[leftmargin=*]
    \item
      Zeigen Sie, dass $E(V_\lambda(H)) \subseteq V_{\lambda + 2}(H)$ für alle $\lambda \in K$.
    \item
      Folgern Sie: Ist $V$ endlichdimensional und $H$ diagonalisierbar, so ist $E$ nilpotent.
  \end{enumerate}
\end{question}










\subsection{Multilinearität}





% PRIORITY 1





% PRIORITY 2


\begin{question}[subtitle = Das Verschwinden von alternierenden Formen für große Dimensionen]{2}
  Es sei $V$ ein endlichdimensionaler $K$-Vektorraum und $n \coloneqq \dim V$.
  Es sei $\omega \colon V^m \to V$ eine alternierende Multilinearform.
  Zeigen Sie, dass $\omega = 0$.
\end{question}





% PRIORITY 3


\begin{question}[subtitle = Charakterisierungen der Jacobi-Identität]{3}
  Es sei $V$ ein $K$-Vektorrraum und $[-,-] \colon V \times V \to V$ eine alternierend bilineare Abbildung.
  Für jedes $x \in V$ sei
  \[
    \ad_x \coloneqq [x,-] \colon V \to V, \quad y \mapsto [x,y].
  \]
  Zeigen Sie, dass die folgenden beiden Aussagen äquivalent sind:
  \begin{enumerate}
    \item
      Die alternierende Bilinearform $[-,-]$ erfüllt die Jacobi-Identität, d.h.\ es ist
      \[
        [x,[y,z]] + [y,[z,x]] + [z,[x,y]] = 0
        \quad
        \text{für alle $x, y, z \in V$}.
      \]
    \item
      Es gilt $\ad_x([y,z]) = [\ad_x(y), z] + [y, \ad_x(z)]$ für alle $x, y, z \in V$.
      (Man sagt, dass $\ad_x$ eine Derivation bezüglich $[-,-]$ ist.)
  \end{enumerate}
\end{question}


\begin{question}[subtitle = Charakterisierungen nicht verschwindender alternierender Formen]{3}
  Es sei $V$ ein endlichdimensionaler $K$-Vektorraum mit $n \coloneqq \dim V$ und $\omega \colon V^n \to K$ eine alternierende Multilinearform.
  Zeigen Sie, dass die folgenden Bedingungen äquivalent sind:
  \begin{enumerate}
    \item
      Es ist $\omega \neq 0$.
    \item
      Es gibt eine Basis $(b_1, \dotsc, b_n)$ von $V$, so dass $\omega(b_1, \dotsc, b_n) \neq 0$.
    \item
      Für jede Basis $(b_1, \dotsc, b_n)$ von $V$ gilt $\omega(b_1, \dotsc, b_n) \neq 0$.
  \end{enumerate}
\end{question}





% PRIORITY 4


\begin{question}[subtitle = Multilineare Formen und die Determinante]{4}
  Es sei $V$ ein endlichdimensionaler $K$-Vektorraum und $n \coloneqq \dim V$.
  Es sei $\omega \colon V^n \to K$ eine alternierende Multilinearform.
  Es sei $f \colon V \to V$ ein Endomorphismus und
  \[
    \omega_f \coloneqq \omega \circ f^{\times n} \colon V^n \to K,
  \]
  \begin{enumerate}[leftmargin=*]
    \item
      Zeigen Sie, dass $\omega_f$ ebenfalls alternierende Multilinearform ist.
    \item
      Zeigen Sie, dass $\omega_f = \det(f) \omega$.
  \end{enumerate}
\end{question}










\subsection{Wegzusammenhang und Geometrisches}





% PRIORITY 1


\begin{question}[subtitle = Multiple Choice zu Wegzusammenhangskomponenten]{1}
  Es sei $n \geq 1$.
  Entscheiden Sie, welche der folgenden Aussagen gelten.
  \begin{enumerate}[leftmargin=*]
    \item
      Die Wegzusammenhangskomponenten von $\GL_n(\Rbb)$ sind die beiden Untergruppen
      \[
        \GL_n(\Rbb)_+ = \{ S \in \GL_n(\Rbb) \mid \det S > 0 \}
        \quad\text{und}\quad
        \GL_n(\Rbb)_- = \{ S \in \GL_n(\Rbb) \mid \det S < 0 \}.
      \]
    \item
      Von den beiden Wegzusammenhangskomponente von $\GL_n(\Rbb)$ ist $\Orthogonal(n)$ diejenige, die die Einheitsmatrix enthält.
    \item
      Die schiefsymmetrischen Matrizen $\mathfrak{o}_n(\Rbb) = \{ A \in \Mat_n(\Rbb) \mid A^T = -A \}$ sind eine wegzusammenhängende Teilmenge von $\Mat_n(\Rbb)$.
    \item
      Die Gruppe $\Unitary(n) \cap \GL_n(\Rbb)_+$ besteht aus zwei Wegzusammenhangskomponenten.
    \item
      Ist $G$ eine wegzusammenhängende Untergruppe von $\GL_n(\Kbb)$, so ist $G' \coloneqq \{ S \in G \mid \det S = 1 \}$ ebenfalls eine wegzusammenhängende Untergruppe von $\GL_n(\Kbb)$.
    \item
      Jede Untergruppe von $\GL_n(\Cbb)$ ist wegzusammenhängend.
    \item
      Es ist $G = \{ S \in \GL_n(\Rbb) \mid S^{-1} = -S \}$ eine zusammehängende, aber nicht wegzusammenhängende Untergruppe von $\GL_n(\Rbb)$.
    \item
      Die Gruppe $\SUnitary(n) \cap \GL_n(\Rbb)$ ist wegzusammenhängend.
    \item
      Die Menge der Drehmatrizen
      \[
        D
        \coloneqq
        \left\{
          \begin{pmatrix*}[r]
            \cos \varphi  & -\sin \varphi \\
            \sin \varphi  &  \cos \varphi 
          \end{pmatrix*}
        \,\middle|\,
        \varphi \in \Rbb
        \right\}
      \]
      ist eine wegzusammenhängende Untergruppe von $\GL_2(\Rbb)$.
  \end{enumerate}
\end{question}
\begin{solution}
  \begin{enumerate}
    \item
      Nein, $\GL_n(\Rbb)_-$ ist keine Untergruppe.
    \item
      Falsch, die Wegzusammenhangskomponenten von $\GL_n(\Rbb)$ sind $\GL_n(\Rbb)_+$ und $\GL_n(\Rbb)_-$.
    \item
      Ja, denn normierte Vektorräume sind immer wegzusammenhängend.
    \item
      Nein: Der Schnitt ist $\SO(n)$, also wegzusammenhängend.
    \item
      Ja: Wähle einen Weg in $G$ und teile diesen durch die Determinante.
    \item
      Nein, es ist keine Untergruppe.
    \item
      Falsch, siehe $\{1,-1\}$.
    \item
      Ja, der Schnitt ist $\SOrthogona(n)$.
    \item
      Ja: Ändere den Winkel stetig.
      Außerdem ist es $\SOrthogonal(2)$.
  \end{enumerate}
\end{solution}





% PRIORITY 2


\begin{question}[subtitle = Zerschneidung von $\Rbb^n$]{2}
  Es sei $f \colon \Rbb^n \to \Rbb$ eine lineare Abbildung mit $f \neq 0$.
  Zeigen Sie, dass $\Rbb^n \smallsetminus \ker f$ nicht wegzusammenhängend ist.
\end{question}


\begin{question}[subtitle = Definition und Sinus des unorientierten Winkels]{2}
  Es sei $V$ ein euklidischer Vektorraum und es seien $v, w \in V$ mit $v, w \neq 0$.
  \begin{enumerate}[leftmargin=*]
    \item
      Zeigen Sie, dass es genau einen Winkel $\alpha \in [0,\pi]$ gibt, so dass
      \[
          \cos \alpha
        = \frac{\bil{v, w}}{\|v\| \|w\|}.
      \]
    \item
      Zeigen Sie, dass genau dann $\sin \alpha \neq 0$, wenn $v$ und $w$ linear unabhängig sind.
    \item
      Bestimmen Sie, wann $\sin \alpha = 1$.
  \end{enumerate}
\end{question}


\begin{question}[subtitle = Zur Existenz und Eindeutigkeit normierter Vektoren auf der Gerade]{2}
  Es sei $V$ ein eindimensionaler euklidischer Vektorraum.
  \begin{enumerate}[leftmargin=*]
    \item
      Zeigen Sie, dass es genau zwei verschiedene Vektoren $v_1, v_2 \in V$ gibt, so dass $\|v_1\| = 1$ und $\|v_2\| = 1$, und dass $v_2 = \pm v_1$.
    \item
      Entscheiden Sie, ob die Aussage auch für einen eindimensionalen unitären Vektorraum gilt.
  \end{enumerate}
\end{question}


\begin{question}[subtitle = Zur Existenz von Tangentialvektoren]{2}
  Es sei $V$ ein euklidischer Vektorraum, und $A, B \in V$ seien zwei linear unabhängige Punkte.
  Zeigen Sie, dass es genau ein Element $\tangent_{AB} \in \Ell(A,B)$ gibt, so dass $\tangent_{AB} \perp A$, $\|\tangent_{AB}\| = 1$ und $\bil{\tangent_{AB}, B} > 0$.
\end{question}





% PRIORITY 3


\begin{question}[subtitle = Einschränkung der Orientierung auf das orthogonale Komplement]{3}
  Es sei $V$ ein euklidischer $n$-dimensionaler Vektorraum und $d$ eine normierte, alternierende $n$-Form auf $d$.
  Es sei $u \in V$ mit $\|u\| = 1$.
  Zeigen Sie für das orthogonale Komplement $U \coloneqq u^\perp = \Ell(u)^\perp$, dass die Einschränkung $d(-, \dotsc, -, u)|_{U^{n-1}}$ eine normierte, alternierende $(n-1)$-Form ist.
\end{question}


\begin{question}[subtitle = Konstruktion und Eigenschaften des Kreuzprodukts]{3}
  Es sei $V$ ein dreidimensionaler orientierter euklidischer Vektorraum mit normierter alternierender Trilinearform $d$.
  \begin{enumerate}[leftmargin=*]
    \item
      Zeigen Sie, dass es für alle $u, v \in V$ genau ein Element $u \times v \in V$ gibt, so dass $d(u,v,w) = \bil{u \times v, w}$ für alle $w \in V$.
    \item
      Zeigen Sie, dass die Abbildung $V \times V \to V$, $(u,v) \mapsto u \times v$ bilinear und alternierend ist.
    \item
      Zeigen Sie für alle $u, v \in V$, dass $u \times v$ orthogonal zu $u$ und $v$ ist.
    \item
      Zeigen Sie für alle $u, v \in V$, dass $u \times v = 0$ genau dann, wenn $u$ und $v$ linear abhängig sind.
      
      (\emph{Hinweis}:
       Nutzen Sie, dass $d \neq 0$, und deshalb $d(b_1, b_2, b_3) \neq 0$ für jede Basis $\{b_1, b_2, b_3\}$ von $V$.)
    \item
      Zeigen Sie für alle $u, v \in V$, dass $\|u \times v\| = \|u\| \|v\| \sin \alpha$, wobei $\alpha$ der unorientierte Winkel zwischen $u$ und $v$ ist.
    \item
      Zeigen Sie:
      Ist $(e_1, e_2, e_3)$ eine positiv orientierte Orthonormalbasis von $V$, so gilt für $u, v \in V$ mit $u = u_1 e_1 + u_2 e_2 + u_2 e_3$ und $v = v_1 e_1 + v_2 e_2 + v_3 e_3$, dass
      \[
        u \times v = (u_2 v_3 - u_3 v_2) e_1 + (u_3 v_1 - u_1 v_3) e_2 + (u_1 v_2 - u_2 v_1) e_3.
      \]
  \end{enumerate}
\end{question}





% PRIORITY 4












