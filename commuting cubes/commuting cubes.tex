%\documentclass[a4paper,10pt]{article}
\documentclass[a4paper,10pt,numbers=noenddot]{scrartcl}

\usepackage{../generalstyle}
\usepackage{specificstyle}

\setromanfont[Mapping=tex-text]{Linux Libertine O}
% \setsansfont[Mapping=tex-text]{DejaVu Sans}
% \setmonofont[Mapping=tex-text]{DejaVu Sans Mono}

\title{Alternative Lösungen \\ Zettel 3}
\author{Jendrik Stelzner}
\date{\today}

\begin{document}
\maketitle










\begin{abstract}
  Wir geben alternative Lösungen für Aufgaben 3, Teile (v) und (vi) von Zettel 3.
  Dabei ersetzen wir explizite Rechnungen durch kommutative Diagramme.
\end{abstract}










\section{Hilfsaussagen}

Wir nennen hier explizit einige Aussagen die wir im Folgenden nutzen werden.
Beweise werden bewusst nicht angegeben

\begin{lemma}
  Es seien $V$ und $W$ zwei $K$-Vektorräume und $f, g \colon V \to W$ linear
  Ist $E \subseteq V$ ein Erzeugendensystem von $V$, so ist genau dann $f = g$ wenn $f(e) = g(e)$ für alle $e \in E$.
\end{lemma}


\begin{corollary}
  Es sei
  \[
    \begin{tikzcd}
        X_1
        \arrow[r, "f"]
        \arrow[d, "s"']
      & X_2
        \arrow[d, "t"]
      \\
        Y_1
        \arrow[r, "g"]
      & Y_2
    \end{tikzcd}
  \]
  ein Diagram von $K$-Vektorräumen, d.h.\ $X_1$, $X_2$, $Y_1$ und $Y_2$ sind $K$-Vektorräume und $f$, $g$, $s$ und $t$ sind $K$-lineare Abbildungen.
  Ist $E \subseteq X_1$ ein Erzeugendensystem von $X_1$, so kommutiert das obige Diagram genau dann, wenn
  \[
    t(f(e)) = g(s(e))
    \quad
    \text{für alle $e \in E$}.
  \]
  (Die Kommutativität von Diagrammen lässt sich also auf Erzeugendensystemen, inbesondere also auf Basen, nachrechnen.)
\end{corollary}


\begin{definition}
  Es seien $V$ und $W$ zwei $\Cbb$-Vektorräume.
  \begin{enumerate}[leftmargin=*]
    \item 
      Eine Abbildung $f \colon V \to W$ \emph{($\Cbb$-)antilinear}, falls
      \begin{enumerate}
        \item
          $f(v_1 + v_2) = f(v_1) + f(v_2)$ für alle $v_1, v_2 \in V$, und
        \item
          $f(\lambda v) = \overline{\lambda} f(v)$ für alle $\lambda \in \Cbb$ und $v \in V$.
      \end{enumerate}
    \item
      Sind $f, g \colon V \to W$ jeweils $\Cbb$-linear oder $\Cbb$-antilinear, so sagen wir, dass $f$ und $g$ die \emph{gleiche Parität} haben, wenn sie beide $\Cbb$-linear oder beide $\Cbb$-antilinear sind.
  \end{enumerate}
\end{definition}


\begin{lemma}
  Es seien $V$ und $W$ zwei $\Cbb$-Vektorräume und $f, g \colon V \to W$ seien $\Cbb$-antilinear.
  Ist $E \subseteq V$ ein Erzeugendensystem, so ist genau dann $f = g$, wenn $f(e) = g(e)$ für alle $e \in E$.
\end{lemma}


\begin{remark}
  Wenn $f$ und $g$ unterschiedliche Parität haben, so gilt die analoge Aussage nur für den trivalen Fall $f = g = 0$.
\end{remark}


\begin{corollary}
  Es sei
  \[
    \begin{tikzcd}
        X_1
        \arrow[r, "f"]
        \arrow[d, "s"']
      & X_2
        \arrow[d, "t"]
      \\
        Y_1
        \arrow[r, "g"]
      & Y_2
    \end{tikzcd}
  \]
  ein Diagram, so dass $X_1$, $X_2$, $Y_1$ und $Y_2$ jeweils $\Cbb$-Vektorräume sind, und $f$, $g$, $s$ und $t$ jeweils $\Cbb$-linear oder $\Cbb$-antilinear.
  Ist $E \subseteq X_1$ ein Erzeugendensystem, so kommutiert das obige Diagram genau dann, wenn
  \[
    \text{$t(f(e)) = g(s(e))$ für alle $e \in E$, und $f$ und $g$ haben die gleiche Parität}.
  \]
\end{corollary}


\begin{lemma}
  Es sei $V$ ein endlichdimensionaler $K$-Vektorraum und $\mc{B} = (b_1, \dotsc, b_n)$ eine Basis von $V$.
  Dann gibt es genau eine lineare Abbildung $\Phi_{\mc{B}} \colon V \to K^n$ mit $\Phi_{\mc{B}}(b_i) = e_i$ für alle $1 \leq i \leq n$, wobei $(e_1, \dotsc, e_n)$ die Standardbasis des $K^n$ bezeichnet.
  Außerdem ist $\Phi_{\mc{B}}$ ein Isomorphismus.
\end{lemma}


\begin{lemma}
  Es seien $V$ und $W$ zwei endlichdimensionale $K$-Vektorräume.
  Es sei zum einen $\mc{B} = (b_1, \dotsc, b_n)$ eine Basis von $V$ und zum anderen $\mc{C} = (c_1, \dotsc, c_m)$ eine Basis von $W$.
  Ist $f \colon V \to W$ eine lineare Abbildung, so gilt:
  \begin{enumerate}[leftmargin=*]
  
    \item
      Es gibt eine eindeutige Matrix $A \in \Mat(m \times n, K)$, so dass das folgende Diagram kommutiert:
      \[
        \begin{tikzcd}
            V \arrow[r, "f"] \arrow[d, "\Phi_\mc{B}"']
          & W \arrow[d, "\Phi_\mc{C}"]
          \\
            {K^n} \arrow[r, "A \cdot"]
          & {K^m}
        \end{tikzcd}
      \]
      (Hier bezeichnet $A \cdot$ die Multiplikation mit $A$ von links.)
    
    \item
      Es gilt $A = \Mat_{\mc{C} \from \mc{B}}(f)$, d.h.\ $A$ ist die darstellende Matrix von $f$ bezüglich der Basen $\mc{B}$ und $\mc{C}$.
  \end{enumerate}
\end{lemma}



\section{Setup}

Wir erinnern an Notationen und Ergebnisse auf vorherigen Aufgabenteilen, die wir im Folgenden nutzen werden.

Es seien $V$ und $W$ zwei endlichdimensionale $\Rbb$-Vektorräume und
\begin{gather*}
  \iota_V \colon V \to V_\Cbb,
  \quad
  v \mapsto v + i \cdot 0
\shortintertext{und}
  \iota_W \colon W \to W_\Cbb,
  \quad
  w \mapsto w + i \cdot 0
\end{gather*}
die kanonischen Inklusionen. Es sind komplexe Konjugationen
\begin{gather*}
  V_\Cbb \to V_\Cbb,
  \quad
  v_1 + i v_2 \mapsto v_1 - i v_2
  \quad\text{für alle $v_1, v_2 \in V$},
\shortintertext{und}
  W_\Cbb \to W_\Cbb, \quad
  w_1 + i w_2 \mapsto w - i w_2
  \quad
  \text{für alle $w_1, w_2 \in W$}
\end{gather*}
gegeben.
Diese Konjugationen sind $\Cbb$-antilinear.
Insbesondere sind sie $\Rbb$-linear, für $V \neq 0$, bzw.\ $W \neq 0$ aber nicht $\Cbb$-linear.


Es sei $\mc{B} = (b_1, \dotsc, b_n)$ eine $\Rbb$-Basis von $V$ und $\mc{C} = (c_1, \dotsc, c_m)$ eine $\Rbb$-Basis von $W$.
Dann ist $\iota_V(\mc{B})$ eine $\Cbb$-Basis von $V_\Cbb$ und $\iota_w(\mc{C})$ eine $\Cbb$-Basis von $W_\Cbb$.
Die Basen $\iota_V(\mc{B})$ und $\iota_W(\mc{B})$ sind punktweise invariant unter der der Konjugation auf den jeweiligen Komplexifizierungen.











\section{Alternative Lösung zu (v)}

Es sei $f \colon V \to W$ eine $\Rbb$-lineare Abbildung, und $f_\Cbb \colon V_\Cbb \to W_\Cbb$ die induzierte $\Cbb$-lineare Abbildung.
Die Abbildung $f_\Cbb$ bringt also das folgende Diagram von $\Cbb$-Vektorräumen zum kommutieren, und ist eindeutig mit dieser Eigenschaft:
\[
  \begin{tikzcd}
      V       \arrow[r, "f"]      \arrow[d, "\iota_V"']
    & W                           \arrow[d, "\iota_W"]
    \\
      V_\Cbb  \arrow[r, "f_\Cbb"]
    & W_\Cbb
  \end{tikzcd}
\]
Es sei $A \coloneqq M_{\mc{C} \from \mc{B}}(f) \in \Mat(m \times n, \Rbb)$ die darstellende Matrix von $f$ bezüglich der $\Rbb$-Basis $\mc{B}$ von $V$ und $\mc{C}$ von $W$.
Es bringt also $A$ das folgende Diagram zum kommutieren, und ist die eindeutige ($m \times n$)-Matrix über $\Rbb$ mit dieser Eigenschaft:
\[
  \begin{tikzcd}
      V      \arrow[r, "f"]       \arrow[d, "\Phi^\Rbb_\mc{B}"']
    & W                           \arrow[d, "\Phi^\Rbb_\mc{C}"]
    \\
      \Rbb^n \arrow[r, "A \cdot"]
    & \Rbb^m
  \end{tikzcd}
\]
Dabei bezeichnen $\Phi^\Rbb_\mc{B} \colon V \to \Rbb^n$ und $\Phi^\Rbb_\mc{C} \colon W \to \Rbb^m$ die eindeutigen $\Rbb$-linearen Isomorphismen mit
\[
  \text{$\Phi^\Rbb_\mc{B}(b_j) = e_j$ für alle $1 \leq j \leq n$}
  \quad\text{und}\quad
  \text{$\Phi^\Rbb_\mc{C}(c_i) = e_i$ für alle $1 \leq i \leq m$}.
\]

Es gilt zu zeigen, dass $A = \Mat_{\mc{C} \from \mc{B}}(f_\Cbb)$.
Dies bedeutet gerade, dass das folgende Diagram kommutieren soll:
\[
  \begin{tikzcd}
      V_\Cbb  \arrow[r, "f_\Cbb"]   \arrow[d, "\Phi^\Cbb_\mc{B}"']
    & W_\Cbb                        \arrow[d, "\Phi^\Cbb_\mc{C}"]
    \\
      \Cbb^n  \arrow[r, "A \cdot"]
    & \Cbb^m
  \end{tikzcd}
\]
Dabei bezeichnen $\Phi^\Cbb_\mc{B} \colon V_\Cbb \to \Cbb^n$ und $\Phi^\Cbb_\mc{C} \colon W \to \Cbb^m$ die eindeutigen $\Cbb$-linearen Isomorphismen mit
\begin{align*}
  &\text{$\Phi^\Cbb_\mc{B}(b_j + i \cdot 0) = e_j$ für alle $1 \leq j \leq n$}
\shortintertext{und}
  &\text{$\Phi^\Cbb_\mc{C}(c_j + i \cdot 0) = e_j$ für alle $1 \leq j \leq m$}.
\end{align*}

Wir betrachten den folgenden Würfel:
\[
  \begin{tikzcd}[row sep = large, column sep = large]
      {}
    & V
      \arrow[rr, "f"]
      \arrow[ld, hookrightarrow, "\iota_V"']
      \arrow[dd, "\Phi^\Rbb_\mc{B}" near start]
    & 
    & W
      \arrow[ld, hookrightarrow, "\iota_W"']
      \arrow[dd, "\Phi^\Rbb_\mc{C}", near start]
    \\
      V_\Cbb
      \arrow[rr, crossing over, "f_\Cbb" near end]
      \arrow[dd, "\Phi^\Cbb_\mc{B}" near start]
    & 
    & W_\Cbb
    & 
    \\
      {}
    & \Rbb^n
      \arrow[ld, hookrightarrow, "\iota_n"']
      \arrow[rr, "A \cdot" near start]
    & 
    & \Rbb^m
      \arrow[ld, hookrightarrow, "\iota_m"']
    \\
      \Cbb^n
      \arrow[rr, "A \cdot"]
    & 
    & \Cbb^m
      \arrow[from=uu, crossing over, "\Phi^\Cbb_\mc{B}" near start]
    & 
  \end{tikzcd}
\]
Dabei bezeichnen $\iota_k \colon \Rbb^k \hookrightarrow \Cbb^k$, $x \mapsto x$ für $k \in \{n, m\}$ die kanonischen Inklusionen.

Wir wissen bereits, dass die Rückseite des Würfels kommutiert, und wollen zeigen, dass auch die Vorderseite kommutiert.
Hierfür zeigen wir zunächst, dass die übrigen vier Seiten des Würfels kommutieren:

Der Deckel des Würfels kommutiert nach Definition von $f_\Cbb$.
Dass der Boden des Würfels kommutiert ist klar.
Die linke Seites des Würfels ist das folgende Diagram:
\[
  \begin{tikzcd}
      V
      \arrow[r, hookrightarrow, "\iota_V"]
      \arrow[d, "\Phi^\Rbb_\mc{B}"']
    & V_\Cbb
      \arrow[d, "\Phi^\Cbb_\mc{B}"]
    \\
      \Rbb^n
      \arrow[r, hookrightarrow, "\iota_n"]
    & \Cbb^n
  \end{tikzcd}
\]
Da alle Abbildungen in diesem Diagram $\Rbb$-linear sind, genügt es die Kommutativität des Diagrams auf einer $\Rbb$-Basis von $V$ nachzurechnen
Für die $\Rbb$-Basis $\mc{B}$ von $V$ ergibt sich, dass
\[
  \Phi^\Cbb(\iota_V(b_j))
  = \Phi^\Cbb(b_j + i \cdot 0)
  = e_j
  = \Phi^\Rbb_\mc{B}(b_j)
  = \iota_n(\Phi^\Rbb_\mc{B}(b_j))
  \quad
  \text{für alle $1 \leq j \leq n$},
\]
bzw.\ das folgende „kommutative Diagram von Elementen“:
\[
  \begin{tikzcd}
      b_j
      \arrow[r, maps to, "\iota_V"]
      \arrow[d, maps to, "\Phi^\Rbb_\mc{B}"']
    & b_j + i \cdot 0
      \arrow[d, maps to, "\Phi^\Cbb_\mc{B}"]
    \\
      e_j
      \arrow[r, maps to, "\iota_n"]
    & e_j
  \end{tikzcd}
\]
Also kommutiert die linke Seite des Würfels.
Analog ergibt sich, dass auch die rechte Seite kommutiert.

Wir können nun die Kommutativität der Vorderseite des Würfels aus der Kommutativität der anderen Seiten folgern:
Hierfür bemerken wir zunächst, dass die beiden $\Cbb$-linearen Abbildungen
\[
  (A \cdot) \circ \Phi^\Cbb_\mc{B}, \;
  \Phi^\Cbb_\mc{C} \circ f_\Cbb
  \colon
  V_\Cbb \to \Cbb^m
\]
nach Aufgabenteil (iv) genau dann gleich sind, wenn die beiden $\Rbb$-linearen Abbildungen
\[
    (A \cdot) \circ \Phi^\Cbb_\mc{B} \circ \iota_V, \;
    \Phi^\Cbb_\mc{C} \circ f_\Cbb \circ \iota_V
    \colon
    V \to \Cbb^m
\]
übereinstimmen.
Diese zu zeigende Gleichheit lässt sich mithilfe des Würfels sehr einfach ausdrücken:
\[
  \begin{tikzcd}
      {}
    & *
      \arrow[rr]
      \arrow[ld, red]
      \arrow[dd]
    & {}
    & *
      \arrow[ld]
      \arrow[dd]
    \\
      *
      \arrow[rr, crossing over, red]
      \arrow[dd]
    & {}
    & *
      \arrow[dd]
    & {}
    \\
      {}
    & *
      \arrow[rr]
      \arrow[ld]
    & {}
    & *
      \arrow[ld]
    \\
      *
      \arrow[rr]
    & {}
    & *
      \arrow[from=uu, crossing over, red]
    & {}
  \end{tikzcd}
  \quad=\quad
  \begin{tikzcd}
      {}
    & *
      \arrow[rr]
      \arrow[ld, red]
      \arrow[dd]
    & {}
    & *
      \arrow[ld]
      \arrow[dd]
    \\
      *
      \arrow[rr, crossing over]
      \arrow[dd, red]
    & {}
    & *
      \arrow[dd]
    & {}
    \\
      {}
    & *
      \arrow[rr]
      \arrow[ld]
    & {}
    & *
      \arrow[ld]
    \\
      *
      \arrow[rr, red]
    & {}
    & *
      \arrow[from=uu, crossing over]
    & {}
  \end{tikzcd}
\]
(Wir lässen hier zur Übersichtlichkeit die Bennenung der einzelnen Ecken und Kanten des Würfels weg.)
Mithilfe dieser graphischen Notation und der Kommutativität der nicht-Vor\-der\-sei\-ten lässt sich die obige Gleichheit zeigen, indem wir um den Würfel herumwackeln:
\[
  \renewcommand{\arraystretch}{6}
  \begin{array}{cccccccc}
      &
    \begin{tikzcd}[row sep = 10, column sep = 10]
        {}
      & *
        \arrow[rr]
        \arrow[ld, red]
        \arrow[dd]
      & {}
      & *
        \arrow[ld]
        \arrow[dd]
      \\
        *
        \arrow[rr, crossing over, red]
        \arrow[dd]
      & {}
      & *
        \arrow[dd]
      & {}
      \\
        {}
      & *
        \arrow[rr]
        \arrow[ld]
      & {}
      & *
        \arrow[ld]
      \\
        *
        \arrow[rr]
      & {}
      & *
        \arrow[from=uu, crossing over, red]
      & {}
    \end{tikzcd}
    & = &
    \begin{tikzcd}[row sep = 10, column sep = 10]
        {}
      & *
        \arrow[rr, red]
        \arrow[ld]
        \arrow[dd]
      & {}
      & *
        \arrow[ld, red]
        \arrow[dd]
      \\
        *
        \arrow[rr, crossing over]
        \arrow[dd]
      & {}
      & *
        \arrow[dd]
      & {}
      \\
        {}
      & *
        \arrow[rr]
        \arrow[ld]
      & {}
      & *
        \arrow[ld]
      \\
        *
        \arrow[rr]
      & {}
      & *
        \arrow[from=uu, crossing over, red]
      & {}
    \end{tikzcd}
    & = &
    \begin{tikzcd}[row sep = 10, column sep = 10]
        {}
      & *
        \arrow[rr, red]
        \arrow[ld]
        \arrow[dd]
      & {}
      & *
        \arrow[ld]
        \arrow[dd, red]
      \\
        *
        \arrow[rr, crossing over]
        \arrow[dd]
      & {}
      & *
        \arrow[dd]
      & {}
      \\
        {}
      & *
        \arrow[rr]
        \arrow[ld]
      & {}
      & *
        \arrow[ld, red]
      \\
        *
        \arrow[rr]
      & {}
      & *
        \arrow[from=uu, crossing over]
      & {}
    \end{tikzcd}
    \\
      = &
    \begin{tikzcd}[row sep = 10, column sep = 10]
        {}
      & *
        \arrow[rr]
        \arrow[ld]
        \arrow[dd, red]
      & {}
      & *
        \arrow[ld]
        \arrow[dd]
      \\
        *
        \arrow[rr, crossing over]
        \arrow[dd]
      & {}
      & *
        \arrow[dd]
      & {}
      \\
        {}
      & *
        \arrow[rr, red]
        \arrow[ld]
      & {}
      & *
        \arrow[ld, red]
      \\
        *
        \arrow[rr]
      & {}
      & *
        \arrow[from=uu, crossing over]
      & {}
    \end{tikzcd}
    & = &
    \begin{tikzcd}[row sep = 10, column sep = 10]
        {}
      & *
        \arrow[rr]
        \arrow[ld]
        \arrow[dd, red]
      & {}
      & *
        \arrow[ld]
        \arrow[dd]
      \\
        *
        \arrow[rr, crossing over]
        \arrow[dd]
      & {}
      & *
        \arrow[dd]
      & {}
      \\
        {}
      & *
        \arrow[rr]
        \arrow[ld,red]
      & {}
      & *
        \arrow[ld]
      \\
        *
        \arrow[rr,red]
      & {}
      & *
        \arrow[from=uu, crossing over]
      & {}
    \end{tikzcd}
    & = &
    \begin{tikzcd}[row sep = 10, column sep = 10]
        {}
      & *
        \arrow[rr]
        \arrow[ld, red]
        \arrow[dd]
      & {}
      & *
        \arrow[ld]
        \arrow[dd]
      \\
        *
        \arrow[rr, crossing over]
        \arrow[dd, red]
      & {}
      & *
        \arrow[dd]
      & {}
      \\
        {}
      & *
        \arrow[rr]
        \arrow[ld]
      & {}
      & *
        \arrow[ld]
      \\
        *
        \arrow[rr,red]
      & {}
      & *
        \arrow[from=uu, crossing over]
      & {}
    \end{tikzcd}
  \end{array}
\]
In Formeln besagt die obigen Gleichungskette, dass
\[
\begin{array}{clclcl}
    & \Phi^\Cbb_\mc{C} \circ f_\Cbb \circ \iota_V
  &=&  \Phi^\Cbb_\mc{C} \circ \iota_W \circ f
  &=&  \iota_m \circ \Phi^\Rbb_\mc{C} \circ f \\
   =&  \iota_m \circ (A \cdot) \circ \Phi^\Rbb_\mc{B}
  &=&  (A \cdot) \circ \iota_n \circ \Phi^\Rbb_\mc{B}
  &=&  (A \cdot) \circ \Phi^\Cbb_\mc{B} \circ \iota_V.
\end{array}
\]

Damit erhalten wir also, dass auch Vorderseite des Würfels kommutiert.
Ingesamt haben wir also einen kommutierenden Würfel.










\section{Alternative Lösung für (vi)}

Es sei $A \in \Mat(m \times n, \Cbb)$ und $f \colon V_\Cbb \to W_\Cbb$ die $\Cbb$-lineare Abbildung, die bezüglich der $\Cbb$-Basen $\mc{B}$ und $\mc{C}$ von $V_\Cbb$ und $W_\Cbb$ durch $A$ beschrieben wird.
Also bringt $f_\Cbb$ das folgende Diagram zum kommutieren, und ist die eindeutige $\Cbb$-lineare Abbildung mit dieser Eigenschaft:
\[
  \begin{tikzcd}
      V_\Cbb
      \arrow[r, "f"]
      \arrow[d, "\Phi^\Cbb_\mc{B}"']
    & W_\Cbb
      \arrow[d, "\Phi^\Cbb_\mc{C}"]
    \\
      \Cbb^n
      \arrow[r, "A \cdot"]
    & \Cbb^m
  \end{tikzcd}
\]
Es gilt zu zeigen, dass die $\Cbb$-lineare Abbildung
\[
  g \colon V_\Cbb \to W_\Cbb,
  \quad
  x \mapsto \overline{f(\overline{x})}
\]
bezüglich der $\Cbb$-Basen $\mc{B}$ und $\mc{C}$ von $V_\Cbb$ und $W_\Cbb$ durch die Matrix $\overline{A}$ beschrieben wird.
Es gilt also zu zeigen, dass das folgende Diagram kommutiert:
\[
  \begin{tikzcd}
      V_\Cbb
      \arrow[r, "g"]
      \arrow[d, "\Phi^\Cbb_\mc{B}"']
    & W_\Cbb
      \arrow[d, "\Phi^\Cbb_\mc{C}"]
    \\
      \Cbb^n
      \arrow[r, "\overline{A} \cdot"]
    & \Cbb^m
  \end{tikzcd}
\]

Wir betrachten hierfür den folgenden Würfel:
\[
  \begin{tikzcd}[row sep = large, column sep = large]
      {}
    & V_\Cbb
      \arrow[rr, "f"]
      \arrow[dd, "\Phi^\Cbb_\mc{B}" near start]
    & {}
    & W_\Cbb
      \arrow[dl, "\overline{\,\cdot\,}"]
      \arrow[dd, "\Phi^\Cbb_\mc{C}" near start]
    \\
      V_\Cbb
      \arrow[ur, "\overline{\,\cdot\,}"]
      \arrow[rr, crossing over, "g" near end]
      \arrow[dd, "\Phi^\Cbb_\mc{B}" near start]
    & {}
    & W_\Cbb
    & {}
    \\
      {}
    & \Cbb^n
      \arrow[rr, "A \cdot" near start]
    & {}
    & \Cbb^m
      \arrow[dl, "\overline{\,\cdot\,}"]
    \\
      \Cbb^n
      \arrow[ur, "\overline{\,\cdot\,}"]
      \arrow[rr, "\overline{A} \cdot "]
    & {}
    & \Cbb^m
      \arrow[from=uu, crossing over, "\Phi^\Cbb_\mc{C}" near start]
    & {}
  \end{tikzcd}
\]
Dabei bezeichnet $\overline{\,\cdot\,}$ die jeweilige Konjugationsabbildung.
Wir wissen bereits, dass die Rückseite kommutiert, und möchten zeigen, dass auch die Vorderseite kommutiert.
Wir zeigen zunächst, dass alle anderen Seiten kommutieren:

Der Deckel kommutiert nach Definition von $g$.
Der Boden kommutiert, da
\[
    \overline{A \cdot \overline{x}}
  = \overline{A} \cdot \overline{\overline{x}}
  = \overline{A} \cdot x
  \quad
  \text{für alle $x \in \Cbb^n$}.
\]
Die linke Seite des Würfels ist das folgende Diagram:
\[
  \begin{tikzcd}
      V_\Cbb
      \arrow[r, "\overline{\,\cdot\,}"]
      \arrow[d, "\Phi^\Cbb_\mc{B}"']
    & V_\Cbb
      \arrow[d, "\Phi^\Cbb_\mc{B}"]
    \\
      \Cbb^n
      \arrow[r, "\overline{\,\cdot\,}"]
    & \Cbb^n
  \end{tikzcd}
\]
Da alle auftretenden Abbildungen $\Cbb$-linear oder -antilinear sind, lässt sich die Kommutativität des Diagrams auf einer $\Cbb$-Basis von $V_\Cbb$ nachrechnen.
Für die Basis $\iota_V(\mc{B})$ von $V_\Cbb$ erhalten wir das folgende „kommutative Diagram von Elementen“:
\[
  \begin{tikzcd}
      b_j + i \cdot 0
      \arrow[r, maps to, "\overline{\,\cdot\,}"]
      \arrow[d, maps to, "\Phi^\Cbb_\mc{B}"']
    & b_j + i \cdot 0
      \arrow[d, maps to, "\Phi^\Cbb_\mc{B}"]
    \\
      e_j
      \arrow[r, maps to, "\overline{\,\cdot\,}"]
    & e_j
  \end{tikzcd}
\]
In Formeln bedeutet dies, dass
\[
    \overline{\Phi^\Cbb_\mc{B}(b_j + i \cdot 0)}
  = \overline{e_j}
  = e_j
  = \Phi^\Cbb_\mc{B}(b_j + i \cdot 0)
  = \Phi^\Cbb_\mc{B}\left( \overline{b_j + i \cdot 0} \right).
\]

Damit wissen wir von allen Seiten des Würfels, bis auf die Vorderseite, dass sie kommutieren.
Durch Herumwackeln um den Würfel folgt daraus, dass auch die Vorderseite kommutiert:
\[
  \begin{array}{cccccccc}
        &
    \begin{tikzcd}[row sep = 10, column sep = 10]
        {}
      & *
        \arrow[rr]
        \arrow[dd]
      & {}
      & *
        \arrow[dl]
        \arrow[dd]
      \\
        *
        \arrow[ur]
        \arrow[rr, crossing over, red]
        \arrow[dd]
      & {}
      & *
      & {}
      \\
        {}
      & *
        \arrow[rr]
      & {}
      & *
        \arrow[dl]
      \\
        *
        \arrow[ur]
        \arrow[rr]
      & {}
      & *
        \arrow[from=uu, crossing over, red]
      & {}
    \end{tikzcd}
    & = &
   \begin{tikzcd}[row sep = 10, column sep = 10]
        {}
      & *
        \arrow[rr, red]
        \arrow[dd]
      & {}
      & *
        \arrow[dl, red]
        \arrow[dd]
      \\
        *
        \arrow[ur, red]
        \arrow[rr, crossing over]
        \arrow[dd]
      & {}
      & *
      & {}
      \\
        {}
      & *
        \arrow[rr]
      & {}
      & *
        \arrow[dl]
      \\
        *
        \arrow[ur]
        \arrow[rr]
      & {}
      & *
        \arrow[from=uu, crossing over, red]
      & {}
    \end{tikzcd}
    & = &
    \begin{tikzcd}[row sep = 10, column sep = 10]
        {}
      & *
        \arrow[rr, red]
        \arrow[dd]
      & {}
      & *
        \arrow[dl]
        \arrow[dd, red]
      \\
        *
        \arrow[ur, red]
        \arrow[rr, crossing over]
        \arrow[dd]
      & {}
      & *
      & {}
      \\
        {}
      & *
        \arrow[rr]
      & {}
      & *
        \arrow[dl, red]
      \\
        *
        \arrow[ur]
        \arrow[rr]
      & {}
      & *
        \arrow[from=uu, crossing over]
      & {}
    \end{tikzcd}
    \\
      = &
    \begin{tikzcd}[row sep = 10, column sep = 10]
        {}
      & *
        \arrow[rr]
        \arrow[dd, red]
      & {}
      & *
        \arrow[dl]
        \arrow[dd]
      \\
        *
        \arrow[ur, red]
        \arrow[rr, crossing over]
        \arrow[dd]
      & {}
      & *
      & {}
      \\
        {}
      & *
        \arrow[rr, red]
      & {}
      & *
        \arrow[dl, red]
      \\
        *
        \arrow[ur]
        \arrow[rr]
      & {}
      & *
        \arrow[from=uu, crossing over]
      & {}
    \end{tikzcd}
    & = &
    \begin{tikzcd}[row sep = 10, column sep = 10]
        {}
      & *
        \arrow[rr]
        \arrow[dd]
      & {}
      & *
        \arrow[dl]
        \arrow[dd]
      \\
        *
        \arrow[ur]
        \arrow[rr, crossing over]
        \arrow[dd, red]
      & {}
      & *
      & {}
      \\
        {}
      & *
        \arrow[rr, red]
      & {}
      & *
        \arrow[dl, red]
      \\
        *
        \arrow[ur, red]
        \arrow[rr]
      & {}
      & *
        \arrow[from=uu, crossing over]
      & {}
    \end{tikzcd}
    & = &
    \begin{tikzcd}[row sep = 10, column sep = 10]
        {}
      & *
        \arrow[rr]
        \arrow[dd]
      & {}
      & *
        \arrow[dl]
        \arrow[dd]
      \\
        *
        \arrow[ur]
        \arrow[rr, crossing over]
        \arrow[dd, red]
      & {}
      & *
      & {}
      \\
        {}
      & *
        \arrow[rr]
      & {}
      & *
        \arrow[dl]
      \\
        *
        \arrow[ur]
        \arrow[rr, red]
      & {}
      & *
        \arrow[from=uu, crossing over]
      & {}
    \end{tikzcd}
  \end{array}
\]


























\end{document}
