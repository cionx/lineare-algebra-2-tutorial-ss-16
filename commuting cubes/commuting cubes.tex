\documentclass[a4paper,10pt]{article}
%\documentclass[a4paper,10pt]{scrartcl}

\usepackage{../generalstyle}

\setromanfont[Mapping=tex-text]{Linux Libertine O}
% \setsansfont[Mapping=tex-text]{DejaVu Sans}
% \setmonofont[Mapping=tex-text]{DejaVu Sans Mono}

\title{Alternative Lösungen \\ Zettel 3}
\author{Jendrik Stelzner}
\date{\today}

\begin{document}
\maketitle










\begin{abstract}
  Wir geben alternative Lösungen für Zettel 3, Aufgaben 3, Teile (v) und (vi), in Form kommutativer Diagramme.
\end{abstract}










\section{Hilfsaussagen}

Wir nennen hier explizit einige Aussagen die wir im Folgenden nutzen werden.

\begin{lemma}
  Es seien $V$ und $W$ zwei $K$-Vektorräume, und es sei $B = (b_i)_{i \in I}$ eine Basis von $V$.
  Sind $f, g \colon V \to W$ linear mit $f(b_i) = g(b_i)$ für alle $i \in I$, so ist bereits $f = g$.
\end{lemma}


\begin{lemma}
  Es sei $V$ ein endlichdimensionaler $K$-Vektorraum und $\mc{B} = (b_1, \dotsc, b_n)$ eine Basis von $V$.
  Dann gibt es genau eine lineare Abbildung $\Phi_{\mc{B}} \colon V \to K^n$ mit $\Phi_{\mc{B}}(b_i) = e_i$ für alle $1 \leq i \leq n$, wobei $(e_1, \dotsc, e_n)$ die Standardbasis des $K^n$ bezeichnet.
  Außerdem ist $\Phi_{\mc{B}}$ ein Isomorphismus.
\end{lemma}


\begin{lemma}
  Es seien $V$ und $W$ zwei endlichdimensionale $K$-Vektorräume.
  Es sei $\mc{B} = (b_1, \dotsc, b_n)$ eine Basis von $V$ und $\mc{C} = (c_1, \dotsc, c_m)$ eine Basis von $W$.
  Ist $f \colon V \to W$ eine lineare Abbildung, so gilt:
  \begin{enumerate}
  
    \item
      Es gibt eine eindeutige Matrix $A \in \Mat(m \times n, K)$, so dass das folgende Diagram kommutiert:
      \[
        \begin{tikzcd}
            V \arrow[r, "f"] \arrow[d, "\Phi_\mc{B}"']
          & W \arrow[d, "\Phi_\mc{C}"]
          \\
            {K^n} \arrow[r, "A \cdot"]
          & {K^m}
        \end{tikzcd}
      \]
      (Hier bezeichnet $A \cdot$ die Multiplikation mit $A$ von links.)
    
    \item
      Es gilt $A = \Mat_{\mc{C} \from \mc{B}}(f)$, d.h.\ $A$ ist die darstellende Matrix von $f$ bezüglich der Basen $\mc{B}$ und $\mc{C}$.
  \end{enumerate}
\end{lemma}



\section{Setup}

Wir erinnern an Notationen, die wir im Folgenden nutzen werden:

Es seien $V$ und $W$ zwei endlichdimensionale $\Rbb$-Vektorräume.
Es seien
\[
  \iota_V \colon V \to V_\Cbb,
  \quad
  v \mapsto v = v + i \cdot 0
\]
und $\iota_W \colon W \to W_\Cbb$ die kanonischen Inklusionen.
(Hier nutzen wir bereits die Identifikation von $V$ mit dem reellen Untervektorraum $\iota_V(V) \subseteq V_\Cbb$.)
Es sei $\mc{B} = (b_1, \dotsc, b_n)$ eine $\Rbb$-Basis von $V$ und $\mc{C} = (c_1, \dotsc, c_m)$ eine $\Rbb$-Basis von $W$.
Dann ist $\mc{B}$ eine $\Cbb$-Basis von $V_\Cbb$ und $\mc{C}$ eine $\Cbb$-Basis von $W_\Cbb$.
(Hier nutzen wir die Identifikation von $V$ mit dem reellen Untervektorraum $\iota(V) \subseteq V_\Cbb$.
Ohne diese Identifikation müssten wir hier sagen, dass
\[
    \iota_V(\mc{B})
  = (\iota_V(b_1), \dotsc, \iota_V(b_n))
  = ((b_1, 0), \dotsc, (b_n, 0))
\]
eine $\Cbb$-Basis von $V_\Cbb$ ist, und dass
\[
    \iota_W(\mc{C})
  = (\iota_W(c_1), \dotsc, \iota_W(c_m))
  = ((c_1, 0), \dotsc, (c_m, 0))
\]
eine $\Cbb$-Basis von $W_\Cbb$ ist.)






\section{Alternative Lösung zu (v)}

Es sei $f \colon V \to W$ eine $\Rbb$-lineare Abbildung, und $f_\Cbb \colon V_\Cbb \to W_\Cbb$ die induzierte $\Cbb$-lineare Abbildung.
Die Abbildung $f_\Cbb$ bringt also das folgende Diagram von $\Cbb$-Vektorräumen zum kommutieren, und ist eindeutig mit dieser Eigenschaft:
\[
  \begin{tikzcd}
      V       \arrow[r, "f"]      \arrow[d, "\iota_V"']
    & W                           \arrow[d, "\iota_W"]
    \\
      V_\Cbb  \arrow[r, "f_\Cbb"]
    & W_\Cbb
  \end{tikzcd}
\]
Es sei $A \coloneqq M_{\mc{C} \from \mc{B}}(f) \in \Mat(m \times n, \Rbb)$ die darstellende Matrix von $f$ bezüglich der $\Rbb$-Basen $\mc{B}$ von $V$ und $\mc{C}$ von $W$.
Es bringt also $A$ das folgende Diagram zum kommutieren, und ist die eindeutige ($m \times n$)-Matrix über $\Rbb$ mit dieser Eigenschaft:
\[
  \begin{tikzcd}
      V      \arrow[r, "f"]       \arrow[d, "\Phi^\Rbb_\mc{B}"']
    & W                           \arrow[d, "\Phi^\Rbb_\mc{C}"]
    \\
      \Rbb^n \arrow[r, "A \cdot"]
    & \Rbb^m
  \end{tikzcd}
\]
Dabei bezeichnen $\Phi^\Rbb_\mc{B} \colon V \to \Rbb^n$ und $\Phi^\Rbb_\mc{C} \colon W \to \Rbb^m$ die eindeutigen $\Rbb$-linearen Isomorphismen mit
\[
  \text{$\Phi^\Rbb_\mc{B}(b_j) = e_j$ für alle $1 \leq j \leq n$}
  \quad\text{und}\quad
  \text{$\Phi^\Rbb_\mc{C}(c_i) = e_i$ für alle $1 \leq i \leq m$}.
\]

Es gilt zu zeigen, dass $A = \Mat_{\mc{C} \from \mc{B}}(f_\Cbb)$.
Dies bedeutet gerade, dass das folgende Diagram kommutieren soll:
\[
  \begin{tikzcd}
      V_\Cbb  \arrow[r, "f_\Cbb"]   \arrow[d, "\Phi^\Cbb_\mc{B}"']
    & W_\Cbb                        \arrow[d, "\Phi^\Cbb_\mc{C}"]
    \\
      \Cbb^n  \arrow[r, "A \cdot"]
    & \Cbb^m
  \end{tikzcd}
\]
Dabei bezeichnen $\Phi^\Cbb_\mc{B} \colon V_\Cbb \to \Cbb^n$ und $\Phi^\Cbb_\mc{C} \colon W \to \Cbb^m$ die eindeutigen $\Cbb$-linearen Isomorphismen mit
\[
  \text{$\Phi^\Cbb_\mc{B}(b_j) = e_j$ für alle $1 \leq j \leq n$}
  \quad\text{und}\quad
  \text{$\Phi^\Cbb_\mc{C}(c_i) = e_i$ für alle $1 \leq i \leq m$}.
\]

Wir wollen 


\end{document}
